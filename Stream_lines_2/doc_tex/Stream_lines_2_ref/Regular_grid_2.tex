%% +------------------------------------------------------------------------+
% | Reference manual page: Regular_grid_2.tex
% +------------------------------------------------------------------------+
% | 07.12.2004 Abdelkrim Mebarki
% | Package: Stream_lines_2
% | 
\RCSdef{\RCSRegulargridRev}{$Id$}
\RCSdefDate{\RCSRegulargridDate}{$Date$}
% |
%%RefPage: end of header, begin of main body
% +------------------------------------------------------------------------+


\begin{ccRefClass}{Regular_grid_2<StreamLinesTraits_2>}  %% add template arg's if necessary

%% \ccHtmlCrossLink{}     %% add further rules for cross referencing links
%% \ccHtmlIndexC[class]{} %% add further index entries
\ccCreationVariable{rgrid_2}
\ccDefinition


The template parameter \ccc{StreamLinesTraits_2} has to be
instantiated by a model of the concept \ccc{StreamLinesTraits_2}.\\
This class provides a 2D vector field specified by a set of sample
points defined on a regular grid, with a bilinear interpolation scheme
over its cells (i.e. for each point \ccc{p} in a cell \ccc{c}, the
vector value is interpolated from the vertices of \ccc{c}).

\ccTypes
\ccThree{typedef VectorField_2:::Point_iterator_x}{xxx_xxxx_xxx;}{}
\ccTypedef{typedef StreamLinesTraits_2::FT FT;}{the scalar type.}
\ccGlue
\ccTypedef{typedef StreamLinesTraits_2::Point_2 Point_2;}{the point type.}
\ccGlue
\ccTypedef{typedef StreamLinesTraits_2::Vector_2 Vector_2;}{the vector type.}

\ccCreation
\ccCreationVariable{rgrid}  %% choose variable name

\ccThree{Regular_grid_2<StreamLinesTraits_2>}{r = Regular_grid_2 rgrid}{}
\ccThreeToTwo

\ccThree{Regular_grid_2<StreamLinesTraits>}{r = Regular_grid_2 regular_grid_2}{}
\ccThreeToTwo
\ccConstructor{Regular_grid_2(int x_samples, int y_samples, FT x_size, FT y_size);}
{Generate a regular grid \ccVar\ whose size is \ccc{x_size} by \ccc{y_size}, while \ccc{x_samples} and
	\ccc{y_samples} specify the number of samples on \ccc{x} and \ccc{y}.}

\ccHeading{Modifiers}

In addition to the minimum interface required by the concept
definition, the class \ccRefName{} provides the following function to
fill the vector field with the user data.

\ccMethod{void set_xy(int i, int j, Vector_2 v);}{Attribute the vector v
 to the position (i,j) on the regular grid.}

\ccHeading{Access Functions}

\ccMethod{std::pair<int, int> get_dimension();}{returns the dimension of the grid.}

\ccMethod{std::pair<FT, FT> get_size();}{returns the size of the grid.}

%\ccInclude{CGAL/Regular_grid_2.h}

\ccIsModel 

\ccc{VectorField_2} \\

\ccSeeAlso
\ccc{Triangular_field_2<StreamLinesTraits_2>} \\
\end{ccRefClass}

% +------------------------------------------------------------------------+
%%RefPage: end of main body, begin of footer
% EOF
% +------------------------------------------------------------------------+

