\begin{ccRefConcept}{PolynomialTraits_d::PolynomialSubresultants}

\textbf{Note:} This functor is optional!

\ccDefinition

Computes the polynomial subresultant of two polynomials $p$ and $q$ of 
type \ccc{PolynomialTraits_d::Polynomial_d} with respect to outermost variable.
Let 
$p=\ccSum{i=0,\ldots,n}{} p_i x^i$ and 
$q=\ccSum{i=0,\ldots,m}{} q_i x^i$, where $x$
is the outermost variable.
The $i$-th subresultant (with $i=0,\ldots,\min\{n,m\}$) is defined by

\begin{ccTexOnly}
\begin{eqnarray*}
\mathrm{Sres}_i(p,q)&=&\det \left(\begin{array}{ccccccc}
p_n & \ldots &&\ldots& p_{2i-m+2}&x^{m-i-1}p \\
&\ddots&&&\vdots&\vdots\\
&&p_n&\ldots&p_{i+1}&p\\
q_m & \ldots &&\ldots & q_{2i-n+2}&x^{n-i-1}q \\
&\ddots&&&\vdots&\vdots\\
&&q_m&\ldots&q_{i+1}&q
\end{array}\right)
\end{eqnarray*}
\end{ccTexOnly}

\begin{ccHtmlOnly}
<CENTER>
<IMG BORDER=0 SRC="./subresultant_def.gif" ALIGN=middle ALT="Subresultants">
</CENTER>
\end{ccHtmlOnly}

where $p_i$ and $q_i$ are set to zero if $i<0$.
In the case that $n=m$, $\mathrm{Sres_n}$ is set to $q$.

The result is written in an output range, starting with the $0$-th subresultant
$\mathrm{Sres}_0(p,q)$
(aka as the resultant of $p$ and $q$).


\ccRefines 
\ccc{AdaptableBinaryFunction}\\
\ccc{CopyConstructible}\\
\ccc{DefaultConstructible}\\

\ccCreationVariable{fo}

\ccOperations
\ccMethod{template<typename OutputIterator> 
        OutputIterator operator()(Polynomial_d   p,
                                  Polynomial_d   q,
                                  OutputIterator out);}
         { computes the polynomial subresultants of $p$ and $q$, 
           with respect to the outermost variable. Each element is of type
           \ccc{PolynomialTraits_d::Polynomial_d}.}

\ccMethod{template<typename OutputIterator>
        OutputIterator operator()(Polynomial_d   p,
                                  Polynomial_d   q,
                                  OutputIterator out,
                                  int i);}
         { computes the polynomial subresultants of $p$ and $q$, 
           with respect to the variable $x_i$.}

%\ccHasModels

\ccSeeAlso

\ccRefIdfierPage{Polynomial_d}\\
\ccRefIdfierPage{PolynomialTraits_d}\\
\ccRefIdfierPage{PolynomialTraits_d::Resultant}\\
\ccRefIdfierPage{PolynomialTraits_d::PrincipalSubresultants}\\
\ccRefIdfierPage{PolynomialTraits_d::PolynomialSubresultantsWithCofactors}\\
\ccRefIdfierPage{PolynomialTraits_d::SturmHabichtSequence}\\

\end{ccRefConcept}
