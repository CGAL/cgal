
\ccChapterAuthor{Michael Hemmer}

%\section{Introduction}

\section{Short Introduction for the reviewer}

This package introduces a concept \ccc{Polynomial_d}, a concept for multivariate 
polynomials in $d$ variables. 
This concept is accompanied with the concepts \ccc{PolynomialTraits_d} and 
\ccc{PolynomialToolBox_d}, where \ccc{PolynomialToolBox_d} refines 
\ccc{PolynomialTraits_d}. 

In principal all functors provided by the \ccc{PolynomialToolBox_d} could 
be local to the \ccc{PolynomialTraits_d} as well. 
So far the distinction is due to the fact that the \ccc{AlgebraicKernel_1} 
and \ccc{AlgebraicKernel_2}
require a \ccc{PolynomialTraits_d} which should be minimal. Therefore, the more 
advanced functors are moved into the \ccc{PolynomialToolBox_d}.
However, I am expecting some discussions on this issue. Therefore, all functors are 
currently documented as if they are part of \ccc{PolynomialTraits_d}.

Note that, depending on the coefficient type,  
it is sometimes not possible to provided a certain functor. 
For instance it is not possible to provide  a \ccc{SignAt} if the coefficient type
is not \ccc{RealEmbeddable}. A similar argument holds for functors requiring  
a gcd on the innermost coefficient. However, whenever possible the 
\ccc{PolynomialToolBox_d} provides an alternative functor, which computes the 
desired entity up to a constant factor. For an example see 
\ccc{SquareFreeFactorize} and \ccc{SquareFreeFactorizeUpToConstantFactor}.

The main idea of the concepts is, that the \ccc{PolynomialTraits_d} provides two 
different views on a multivariate polynomial. 

\begin{itemize}
\item 
A recursive view, that sees the polynomial as an element of 
$R[x_0,\dots,x_{d-2}][x_{d-1}]$. In this view, the polynomial is handled as
an univariate polynomial over the ring $R[x_0,\dots,x_{d-2}]$. 
\item 
A symmetric view, which is symmetric with respect to all variables,
seeing the polynomials as element of $R[x_0,\dots,x_{d-1}]$.
\end{itemize}

All functors related to the univariate view are written such that 
outermost variable $x_{d-1}$ is the default variable. For instance, 
\ccc{PolynomialTraits_d::Degree}$()(p)$ returns the degree of $p$ with respect to $x_{d-1}$. 
However, it is also possible to call 
\ccc{PolynomialTraits_d::Degree}$()(p,i)$ returning the 
degree with respect to variable $x_{i}$. The corresponding functor for the 
multivariate view is \ccc{PolynomialTraits_d::TotalDegree} which is 
symmetric in all variables. 