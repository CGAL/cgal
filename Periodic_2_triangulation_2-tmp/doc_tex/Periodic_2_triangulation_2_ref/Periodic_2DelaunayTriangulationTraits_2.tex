% +------------------------------------------------------------------------+
% | Reference manual page: Periodic_2DelaunayTriangulationTraits_2.tex
% +------------------------------------------------------------------------+
% | 16.03.2010 Nico Kruithof
% | Package: Periodic_2_triangulation_2
% | 
\RCSdef{\RCSPeriodicDelaunayTriangulationTraitsRev}{$Id$}
\RCSdefDate{\RCSPeriodicDelaunayTriangulationTraitsDate}{$Date$}
% |
%%RefPage: end of header, begin of main body
% +------------------------------------------------------------------------+


\begin{ccRefConcept}{Periodic_2DelaunayTriangulationTraits_2}

\ccDefinition
%
The concept \ccRefName\ is the first template parameter of the class
\ccc{Periodic_2_Delaunay_triangulation_2}. It refines the concept
\ccc{Periodic_2TriangulationTraits_2} and
\ccc{DelaunayTriangulationTraits_2} from the \cgal\ \ccRef[2D
Triangulation]{Pkg:Triangulation2} package.  It redefines the
geometric objects, predicates and constructions to work with
point-offset pairs. In most cases the offsets will be (0,0) and the
predicates from \ccc{DelaunayTriangulationTraits_2} can be used
directly. For efficiency reasons we maintain for each functor the
version without offsets.

\ccRefines
%
\ccc{DelaunayTriangulationTraits_2} and \ccc{Periodic_2TriangulationTraits_2}

In addition to the requirements of the concept
\ccc{Periodic_2TriangulationTraits_2}, the concept \ccRefName\
provides a predicate to check the empty circle property. The
corresponding predicate type is called type
\ccc{Side_of_oriented_circle_2}.

The additional constructor object \ccc{Construct_circumcenter_2} is
used to build the dual Voronoi diagram and are required only if the
dual functions are called.  The additional predicate type
\ccc{Compare_distance_2} is required if calls to
\ccc{nearest_vertex(..)} are issued.


\ccTypes
%\ccTwo{Periodic_2DelaunayTriangulationTraits_2::Tetrahedron_2xx}{}
%
%\ccNestedType{Point_2}
%{The point type. It must be a model of \ccc{Kernel::Point_2}. }
%\ccGlue
%\ccNestedType{Vector_2} {The vector type. It must be a model of
% \ccc{Kernel::Vector_2}.}
%\ccGlue
%\ccNestedType{Periodic_2_offset_2} {The offset type. It must be a
%  model of the concept \ccc{Periodic_2Offset_2}.}
%\ccGlue
%\ccNestedType{Iso_rectangle_2} {A type representing an axis-aligned rectangle. It must be a model of \ccc{Kernel::Iso_rectangle_2}.}
%
%The following two types represent geometric primitives in $\mathbb
%R^2$. They are required to provide functions converting primitives
%from $\mathbb T_c^2$ to $\mathbb R^2$, i.e.\ constructing
%representatives in $\mathbb R^2$.
%\ccNestedType{Segment_2} {A segment type. It must be a model of \ccc{Kernel::Segment_2}.}
%\ccGlue
%\ccNestedType{Triangle_2} {A triangle type. It must be a model of
%  \ccc{Kernel::Triangle_2}. }

\ccTwo{Periodic_2}{}
\ccNestedType{Side_of_oriented_circle_2}{Predicate object. Must
provide the operators\\
\ccc{Oriented_side operator()(Point p, Point q, Point r, Point s)} 
which takes four points $p, q, r, s$ as arguments and returns
\ccc{ON_POSITIVE_SIDE}, \ccc{ON_NEGATIVE_SIDE} or, 
\ccc{ON_ORIENTED_BOUNDARY} according to the position of points \ccc{s}
with respect to the oriented circle through  through $p,q$
and $r$ and\\
%
\ccc{Oriented_side operator()( Point p, Point q, Point r, Point s,
  Periodic_2_offset_2 o_p, Periodic_2_offset_2 o_q Periodic_2_offset_2
  o_r, Periodic_2_offset_2 o_s)}
which takes four points $(p, o_p), (q, o_q), (r, o_r), (s, o_s)$ as arguments and returns
\ccc{ON_POSITIVE_SIDE}, \ccc{ON_NEGATIVE_SIDE} or, 
\ccc{ON_ORIENTED_BOUNDARY} according to the position of points \ccc{(s, o_s)}
with respect to the oriented circle through  through \ccc{(p, o_p), (q, o_q)}
and \ccc{(r, o_r)}.\\
%
This type is required only if the function
\ccc{side_of_oriented_circle(Face_handle f, Point p)} is called.}

\ccNestedType{Construct_circumcenter_2}{Constructor
  object. Provides the operators: \\
  \ccc{Point operator()(Point p, Point q, Point r)} \\
  which returns
  the  circumcenter of the three points  \ccc{p, q} and \ccc{r}.\\
%
  \ccc{Point operator()(Point p, Point q, Point r, Periodic_2_offset_2
    o_p, Periodic_2_offset_2 o_q Periodic_2_offset_2 o_r)} \\
  which returns the circumcenter of the three points \ccc{(p, o_p),
    (q, o_q)} and \ccc{(r, o_r)}.\\
%
  This type is required only if the function \ccc{Point
    circumcenter(Face_handle f)}is called.}
%
\ccNestedType{Compare_distance_2} {Predicate type. Provides
  the operators: \\
  \ccc{Comparison_result operator()(Point_2 p, Point_2 q, Point_2 r)}
  which returns \ccc{SMALLER}, \ccc{EQUAL} or \ccc{LARGER} according
  to the distance between p and q being smaller, equal or larger than
  the distance between p and r. \\
%
  \ccc{Comparison_result operator()(Point_2 p, Point_2 q, Point_2 r,
    Periodic_2_offset_2 o_p, Periodic_2_offset_2 o_q
    Periodic_2_offset_2 o_r)} which returns \ccc{SMALLER}, \ccc{EQUAL}
  or \ccc{LARGER} according to the distance between \ccc{(p, o_p)},
  and \ccc{(q, o_q)} being smaller, equal or larger than
  the distance between \ccc{(p, o_p)} and \ccc{(r, o_r)}. \\
%
 This type is only require if
  \ccc{nearest_vertex} queries are issued.}

\ccHeading{Predicate functions}
\ccCreationVariable{traits}  %% choose variable name
\ccThree{Construct_circumcenter_2}{traits.construct_circumcenter_2_object();}
{}
\ccMethod{Side_of_oriented_circle_2
side_of_oriented_circle_2_object();}
{Required only
if \ccc{side_of_oriented_circle} is called
called.}
\ccGlue
\ccMethod{Construct_circumcenter_2 construct_circumcenter_2_object();}
{Required only if \ccc{circumcenter} is called.}
\ccGlue
\ccMethod{Compare_distance_2 compare_distance_2_object();}
{Required only if \ccc{compare_distance} is called.}

\ccHasModels

\ccc{CGAL::Periodic_2_Delaunay_triangulation_traits_2<Traits, Offset>} and \\
\ccc{CGAL::Periodic_2_triangulation_traits_2<Traits, Offset>}, which
implements additional the Delaunay predicates as well if the template
parameter Traits is a model of \ccc{DelaunayTriangulationTraits_2}.

\ccSeeAlso
\ccc{DelaunayTriangulationTraits_2}

\end{ccRefConcept}
