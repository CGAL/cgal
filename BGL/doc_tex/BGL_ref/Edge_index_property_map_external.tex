
\begin{ccRefClass}{Edge_index_property_map_external<Graph>}

%% add template arg's if necessary

%% \ccHtmlCrossLink{}     %% add further rules for cross referencing links
%% \ccHtmlIndexC[class]{} %% add further index entries
\ccDefinition

The class \ccRefName\ provides a model for the concept 
\ccAnchor{http://www.boost.org/libs/property_map/ReadablePropertyMap.html}{ReadablePropertyMap} 
that maps an edge in a {\sc Bgl}
\ccAnchor{http://www.boost.org/libs/graph/doc/Graph.html}{graph}
to the integer numbers in the range \ccc{[0,boost::num_edges(graph)]}
via a non-intrusive mechanism (so that any graph can be indexed).

The template parameter \ccc{Graph} must be a model of a {\sc Bgl}
\ccAnchor{http://www.boost.org/libs/graph/doc/Graph.html}{graph}

\ccInclude{CGAL/boost/graph/Edge_index_property_map_external.h}

\ccTypes
  \ccTypedef{std::size_type value_type;}
    {The type of the property.}
\ccGlue
  \ccTypedef{std::size_type reference;}
    {The result type of the map operator.}
\ccGlue
  \ccNestedType{Graph}{The Graph template parameter.}
\ccGlue
  \ccTypedef{typename boost::graph_traits<Graph>::edge_descriptor key_type;}
  {The type of {\sc Bgl} edge descriptor used as key.}

\ccCreation
\ccCreationVariable{pm}  %% choose variable name

\ccConstructor{CGAL::Edge_index_property_map_external<Graph>( Graph const& g ); }
{Initializes an internal table which associates a unique integer
number in the range \ccc{[0,boost::num_edges(g)]} to each edge in \ccc{g}.}

\ccOperations

\ccMethod
  {reference operator[]( key_type const& edge ) const;}
  {Returns the unique index for \ccc{edge} as maintained by this map.}
    
\ccIsModel
\ccAnchor{http://www.boost.org/libs/property_map/ReadablePropertyMap.html}{ReadablePropertyMap} 


\end{ccRefClass}

% +------------------------------------------------------------------------+
%%RefPage: end of main body, begin of footer
% EOF
% +------------------------------------------------------------------------+

