\section{Sparse Linear Algebra \label{sec:Sparse-Linear-Algebra}}

Parameterizing triangle meshes requires both efficient representation
of sparse matrices and efficient iterative or direct linear
solvers. We provide links to standard libraries ({\sc Taucs})
and include a separate package devoted to OpenNL sparse linear solver.

\subsection{List of Solvers}

We provide an interface to several sparse linear solvers, as models
of the \ccc{SparseLinearAlgebraTraits_d} concept:

\begin{itemize}

\item
    OpenNL \cite{cgal:l-nmdgp-05} is shipped with \cgal. This is the default solver.

    \ccc{OpenNL::DefaultLinearSolverTraits<COEFFTYPE, MATRIX, VECTOR, SOLVER>} in OpenNL package  \\
    \ccc{OpenNL::SymmetricLinearSolverTraits<COEFFTYPE, MATRIX, VECTOR, SOLVER>} in OpenNL package  \\

    \emph{Usage:}

    OpenNL (in the version shipped with \cgal) is a lightweight sparse linear solver.
    It does not support large systems, but it is highly portable and
    supports exact number types.

\item
    \ccAnchor{http://www.cs.tau.ac.il/~stoledo/taucs}{{\sc Taucs}}
    is a state-of-the-art direct solver for sparse symmetric matrices.
    It also includes an out-of-core general sparse solver.

    \ccc{CGAL::Taucs_solver_traits<T>}  \\
    \ccc{CGAL::Taucs_symmetric_solver_traits<T>}  \\

    \emph{Usage:}

    {\sc Taucs} is very robust and supports large systems.
    On the other hand, it is not available on all platforms
    supported by \cgal\ and does not support exact number types.

    \emph{Install:}

    {\sc Taucs} can be downloaded from
    \ccAnchor{http://www.tau.ac.il/~stoledo/taucs/2.2/taucs_full.zip}
    {\ccc{http://www.tau.ac.il/~stoledo/taucs/2.2/taucs_full.zip}}.

\end{itemize}


\subsection{{\sc Taucs} Solver Example}

\ccc{examples/Surface_mesh_parameterization/Taucs_parameterization.cpp} computes the
default parameterization method (Floater mean value coordinates with a circular border),
but specifically instantiates the {\sc Taucs} solver. Specifying a specific solver
instead of the default one (OpenNL) means using the third parameter of
\ccc{CGAL::Mean_value_coordinates_parameterizer_3<ParameterizationMesh_3,
BorderParameterizer_3, SparseLinearAlgebraTraits_d>}.  The differences with the first
example \ccc{examples/Surface_mesh_parameterization/Simple_parameterization.cpp} are:

\begin{ccExampleCode}

#include <CGAL/Taucs_solver_traits.h>

...

//***************************************
// Floater Mean Value Coordinates parameterization
// (circular border) with TAUCS solver
//***************************************

// Circular border parameterizer (the default)
typedef CGAL::Circular_border_arc_length_parameterizer_3<Parameterization_polyhedron_adaptor>
                                                    Border_parameterizer;
// TAUCS solver
typedef CGAL::Taucs_solver_traits<double>           Solver;

// Floater Mean Value Coordinates parameterization
// (circular border) with TAUCS solver
typedef CGAL::Mean_value_coordinates_parameterizer_3<Parameterization_polyhedron_adaptor,
                                                        Border_parameterizer,
                                                        Solver>
                                                    Parameterizer;

Parameterizer::Error_code err = CGAL::parameterize(mesh_adaptor, Parameterizer());

...

\end{ccExampleCode}

