% +------------------------------------------------------------------------+
% | Reference manual page: make_weighted_point_range.tex
% +------------------------------------------------------------------------+
% | 09.04.2009   Pierre Alliez, Laurent Saboret, Gael Guennebaud
% | Package: PDB
% |
\RCSdef{\RCSmakeweightedpointrangeRev}{$Id$}
\RCSdefDate{\RCSmakeweightedpointrangeDate}{$Date$}
% |
\ccRefPageBegin
%%RefPage: end of header, begin of main body
% +------------------------------------------------------------------------+


\begin{ccRefFunction}[PDB::]{make_weighted_point_range}  %% add template arg's if necessary

%% \ccHtmlCrossLink{}     %% add further rules for cross referencing links
%% \ccHtmlIndexC[function]{} %% add further index entries

\ccDefinition

% The section below is automatically generated. Do not edit!
%START-AUTO(\ccDefinition)

\ccFunction{template<class Range, class K> boost::iterator_range<boost::transform_iterator<Weighted_point_from_atom< K>, typename Range::iterator> > make_weighted_point_range(Range r, K);}
{
Return an interator which takes an Atom and return the a \ccc{K::Weighted_point}.
The iterator \ccc{value_type} should be an Atom.
}
\ccGlue

%END-AUTO(\ccDefinition)
  
\ccInclude{PDB/make_weighted_point_range.h}

\ccIsModel

\ccSeeAlso

\ccExample

A short example program.
Instead of a short program fragment, a full running program can be
included using the 
\verb|\ccIncludeExampleCode{PDB/make_weighted_point_range.C}| 
macro. The program example would be part of the source code distribution and
also part of the automatic test suite.

\begin{ccExampleCode}
void your_example_code() {
}
\end{ccExampleCode}

%% \ccIncludeExampleCode{PDB/make_weighted_point_range.C}

\end{ccRefFunction}

% +------------------------------------------------------------------------+
%%RefPage: end of main body, begin of footer
\ccRefPageEnd
% EOF
% +------------------------------------------------------------------------+

