% +------------------------------------------------------------------------+
% | Reference manual page: Atom.tex
% +------------------------------------------------------------------------+
% | 10.04.2009   Author
% | Package: PDB
% |
\RCSdef{\RCSAtomRev}{$Id: header.tex 40270 2007-09-07 15:29:10Z lsaboret $}
\RCSdefDate{\RCSAtomDate}{$Date: 2007-09-07 08:29:10 -0700 (Fri, 07 Sep 2007) $}
% |
\ccRefPageBegin
%%RefPage: end of header, begin of main body
% +------------------------------------------------------------------------+


\begin{ccRefClass}[PDB::]{Atom}  %% add template arg's if necessary

%% \ccHtmlCrossLink{}     %% add further rules for cross referencing links
%% \ccHtmlIndexC[class]{} %% add further index entries

\ccDefinition

% The section below is automatically generated. Do not edit!
%START-AUTO(\ccDefinition)

A class repesenting an atom.

%END-AUTO(\ccDefinition)
  
% The section below is automatically generated. Do not edit!
%START-AUTO(\ccInclude)

\ccInclude{CGAL/PDB/Atom.h}

%END-AUTO(\ccInclude)

\ccIsModel

\ccTypes

% The section below is automatically generated. Do not edit!
%START-AUTO(\ccTypes)

\paragraph[{Index}]{typedef Label$<$Atom$>$ Index}

%END-AUTO(\ccTypes)

\ccConstants

% The section below is automatically generated. Do not edit!
%START-AUTO(\ccConstants)

\paragraph[{Type}]{enum Type}
The type (element) of an atom. The currently supported types are C,N,H,O,S, INVALID.
\ccCommentHeading{Values}
\begin{description}
\item[INVALID
]\item[C
]\item[N
]\item[H
]\item[O
]\item[S
]\item[P
]\item[FE
]\item[PT
]\end{description}

%END-AUTO(\ccConstants)

\ccCreation
\ccCreationVariable{a}  %% choose variable name

% The section below is automatically generated. Do not edit!
%START-AUTO(\ccCreation)

\paragraph[{Atom}]{Atom ()}
Construct and invalid atom.

%END-AUTO(\ccCreation)

\ccOperations

% The section below is automatically generated. Do not edit!
%START-AUTO(\ccOperations)

\paragraph[{point}]{const Point\& point () const}
Cartesian coordinates (x,y,z) for the atom.
\paragraph[{\ccc{set_point}}]{void \ccc{set_point} (const Point\& pt)}
\paragraph[{operator ==}]{bool operator == (const Atom\& al) const}
\paragraph[{operator!=}]{bool operator!= (const Atom\& al) const}
\paragraph[{occupancy}]{const float\& occupancy () const}
The PDB occupancy field.
\paragraph[{\ccc{set_occupancy}}]{void \ccc{set_occupancy} (const float\& k)}
\paragraph[{\ccc{temperature_factor}}]{const float\& \ccc{temperature_factor} () const}
The PDB temperature factor field.
\paragraph[{\ccc{set_temperature_factor}}]{void \ccc{set_temperature_factor} (const float\& k)}
\paragraph[{\ccc{segment_id}}]{const \ccc{std::string}\& \ccc{segment_id} () const}
The PDB segment ID char.
\paragraph[{\ccc{set_segment_id}}]{void \ccc{set_segment_id} (const \ccc{std::string}\& k)}
\paragraph[{element}]{const \ccc{std::string}\& element () const}
The PDB element field.
\paragraph[{\ccc{set_element}}]{void \ccc{set_element} (const \ccc{std::string}\& k)}
\paragraph[{charge}]{const \ccc{std::string}\& charge () const}
The PDB charge field.
\paragraph[{\ccc{set_charge}}]{void \ccc{set_charge} (const \ccc{std::string}\& k)}
\paragraph[{type}]{const Type\& type () const}
The type of the atoms (basically what element).
\paragraph[{\ccc{set_type}}]{void \ccc{set_type} (const Type\& k)}
\paragraph[{radius}]{const double\& radius () const}
Returns the van der Waals radius of the atom.
Values take from the wikipedia so beware.
\paragraph[{index}]{const Index\& index () const}
This is a label which identifies an Atom uniquely within some scale.
The uniqueness is only valid if working within the object which assigned the indices, and if nothing has changed since the corresponding \ccc{index_atoms}() function was called.
\paragraph[{\ccc{set_index}}]{void \ccc{set_index} (Index i) const}
note const
\paragraph[{\ccc{string_to_type}}]{static Type \ccc{string_to_type} (const char $\ast$ c)[static]}
\paragraph[{\ccc{swap_with}}]{void \ccc{swap_with} (Atom\& o)}

%END-AUTO(\ccOperations)

\ccSeeAlso

\ccExample

A short example program.
Instead of a short program fragment, a full running program can be
included using the 
\verb|\ccIncludeExampleCode{PDB/Atom.C}| 
macro. The program example would be part of the source code distribution and
also part of the automatic test suite.

\begin{ccExampleCode}
void your_example_code() {
}
\end{ccExampleCode}

%% \ccIncludeExampleCode{PDB/Atom.C}

\end{ccRefClass}

% +------------------------------------------------------------------------+
%%RefPage: end of main body, begin of footer
\ccRefPageEnd
% EOF
% +------------------------------------------------------------------------+

