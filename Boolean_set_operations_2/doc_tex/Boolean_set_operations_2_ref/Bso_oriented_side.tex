\ccRefPageBegin
\label{ref_bso_oriented_side}

\begin{ccRefFunction}{oriented_side}

\ccThree{Oriented_side}{oriented_side}{}
\ccThreeToTwo

\ccDefinition

\ccInclude{CGAL/Boolean_set_operations_2.h}

\ccGlobalFunction{Oriented_side oriented_side(const Type1 & p1, const Type2 & p2);}
{Each one of these functions returns \ccc{ON_POSITIVE_SIDE} if the two
  given polygons \ccc{p1} and \ccc{p2} intersect in their interior,
  \ccc{ON_NEGATIVE_SIDE} if \ccc{p1} and \ccc{p2} do not intersect at
  all, and \ccc{ON_ORIENTED_BOUNDARY} if \ccc{p1} and \ccc{p2} intersect
  only in their boundaries.}

\begin{ccTexOnly}
\begin{longtable}[c]{|l|l|}
\multicolumn{2}{l}{\sl \ \ }
\endfirsthead
\multicolumn{2}{l}{\sl continued}
\endhead
\hline
\textbf{Arg 1 Type} & \textbf{Arg 2 Type}\\
\hline
\hline
\ccc{Polygon_2} & \ccc{Polygon_2}\\
\hline
\ccc{Polygon_2} & \ccc{Polygon_with_holes_2}\\
\hline
\ccc{Polygon_with_holes_2} & \ccc{Polygon_2}\\
\hline
\ccc{Polygon_with_holes_2} & \ccc{Polygon_with_holes_2}\\
\hline
\ccc{General_polygon_2} & \ccc{General_polygon_2}\\
\hline
\ccc{General_polygon_2} & \ccc{General_polygon_with_holes_2}\\
\hline
\ccc{General_polygon_with_holes_2} & \ccc{General_polygon_2}\\
\hline
\ccc{General_polygon_with_holes_2} & \ccc{General_polygon_with_holes_2}\\
\hline
\end{longtable}
\end{ccTexOnly}

\begin{ccHtmlOnly}
<div align="center">
<table cellpadding=3 border="1">
<tr><th> Arg 1 type</th><th>Arg 2 type</th></tr>
<tr><td valign="center">Polygon_2</td><td valign="center">Polygon_2</td></tr>
<tr><td valign="center">Polygon_2</td><td valign="center">Polygon_with_holes_2</td></tr> 
<tr><td valign="center">Polygon_with_holes_2</td><td valign="center">Polygon_2</td></tr>
<tr><td valign="center">Polygon_with_holes_2</td><td valign="center">Polygon_with_holes_2</td></tr>
<tr><td valign="center">General_polygon_2</td><td valign="center">General_polygon_2</td></tr>
<tr><td valign="center">General_polygon_2</td><td valign="center">General_polygon_with_holes_2</td></tr>
<tr><td valign="center">General_polygon_with_holes_2</td><td valign="center">General_polygon_2</td></tr>
<tr><td valign="center">General_polygon_with_holes_2</td><td valign="center">General_polygon_with_holes_2</td></tr>
</table>
</div>
\end{ccHtmlOnly}

\ccGlobalFunction{template <class Kernel, class Container>
Oriented_side oriented_side(const Polygon_2<Kernel, Container> & p1,
                            const Polygon_2<Kernel, Container> & p2);}
\ccGlue
\ccGlobalFunction{template <class Kernel, class Container>
Oriented_side oriented_side(const Polygon_2<Kernel, Container> & p1,
                            const Polygon_with_holes_2<Kernel, Container> & p2);}
\ccGlue
\ccGlobalFunction{template <class Kernel, class Container>
Oriented_side oriented_side(const Polygon_with_holes_2<Kernel, Container> & p1,
                            const Polygon_2<Kernel, Container> & p2);}
\ccGlue
\ccGlobalFunction{template <class Kernel, class Container>
Oriented_side oriented_side(const Polygon_with_holes_2<Kernel, Container> & p1,
                            const Polygon_with_holes_2<Kernel, Container> & p2);}
\ccGlue
\ccGlobalFunction{template <class Traits>
Oriented_side oriented_side(const General_polygon_2<Traits> & p1,
                            const General_polygon_2<Traits> & p2);}
\ccGlue
\ccGlobalFunction{template <class Traits>
Oriented_side oriented_side(const General_polygon_2<Traits> & p1,
                            const General_polygon_with_holes_2<General_polygon_2<Traits> > & p2);}
\ccGlue
\ccGlobalFunction{template <class Traits>
Oriented_side oriented_side(const General_polygon_with_holes_2<General_polygon_2<Traits> > & p1,
                            const General_polygon_2<Traits> & p2);}
\ccGlue
\ccGlobalFunction{template <class Polygon>
Oriented_side oriented_side(const General_polygon_with_holes_2<Polygon> & p1,
                            const General_polygon_with_holes_2<Polygon> & p2);}

%% \ccGlobalFunction{template <class InputIterator>
%% Oriented_side oriented_side(InputIterator begin, InputIterator end);}
%% {Returns \ccc{ON_POSITIVE_SIDE} if the set of general polygons (or
%%   general polygons with holes) in the given range intersect in their
%%   interior, \ccc{ON_NEGATIVE_SIDE} if they do not intersect at all,
%%   and \ccc{ON_ORIENTED_BOUNDARY} if they intersect only in their
%%   boundaries. (The value type of the input iterator is used to
%%   distinguish between the two).}

%% \ccGlobalFunction{template <class InputIterator1, class InputIterator2>
%% Oriented_side oriented_side(InputIterator1 pgn_begin1,
%%                             InputIterator1 pgn_end1,
%%                             InputIterator2 pgn_begin2,
%% 	                       InputIterator2 pgn_end2);}
%% {Returns \ccc{ON_POSITIVE_SIDE} if the set of general polygons and
%%   general polygons with holes in the given two ranges, respectively,
%%   intersect in their interior, \ccc{ON_NEGATIVE_SIDE} if they do not
%%   intersect at all, and \ccc{ON_ORIENTED_BOUNDARY} if they intersect
%%   only in their boundaries.}

\ccSeeAlso
\ccRefIdfierPage{CGAL::do_intersect}
\end{ccRefFunction}

\ccRefPageEnd
