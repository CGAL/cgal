% +------------------------------------------------------------------------+
% | Reference manual page: ArrangementDirectionalXMonotoneTraits.tex
% +------------------------------------------------------------------------+

\ccRefPageBegin

\begin{ccRefConcept}{ArrangementDirectionalXMonotoneTraits_2}

\ccDefinition

This concept refines the basic arrangement $x$-monotone traits concept.
A model of this concept is able to handle \emph{directed} $x$-monotone curves
that intersect in their interior. Namely, an instance of the
\ccc{X_monotone_curve_2} type defined by a model of the concept
\ccc{ArrangementXMonotoneTraits_2} is only required to have a \emph{left}
(lexicographically smaller) endpoint and a \emph{right} endpoint.
If the traits class is also a model of
\ccc{ArrangementDirectionalXMonotoneTraits_2}, the $x$-monotone curve is
also required to have a direction, namely one of these two endpoint is
viewed as its \emph{source} and the other as its \emph{target}.

\ccRefines
\ccc{ArrangementXMonotoneTraits_2}

\ccHeading{Functor Types}
%========================

\ccThree{Compare_y_at_x_2~~~}{}{\hspace*{12cm}}
\ccThreeToTwo

\ccNestedType{Compare_endpoints_xy_2}
{provides the operator~: \\
 \ccc{Comparison_result operator() (X_monotone_curve_2 c)} \\
 which accepts an input curve \ccc{c} and compare its source and target
 point. It returns \ccc{SMALLER} if the curve is directed from left to right
 (lexicographically --- i.e., in case of a vertical line segment, this means
 it is directed upward), and \ccc{LARGER} if it is directed from right to
 left.}

\ccNestedType{Construct_opposite_2}
{provides the operator~: \\
 \ccc{X_monotone_curve_2 operator() (X_monotone_curve_2 c)} \\
 which accepts an $x$-monotone curve \ccc{c} and returns its opposite curve,
 namely a curve whose graph is the same as \ccc{c}'s, and whose source and
 target are swapped with respect to \ccc{c}'s source and target.}

In addition, the two following functors, required by the concept
\ccc{ArrangementXMonotoneTraits_2} should operate as follows:

\ccNestedType{Intersect_2}
{provides the operator (templated by the \ccc{OutputIterator} type)~: \\
 \ccc{OutputIterator operator() (X_monotone_curve_2 c1, X_monotone_curve_2 c2,
                                 OutputIterator oi)} \\
 which computes the intersections of \ccc{c1} and \ccc{c2} and inserts them
 {\sl in an ascending lexicographic $xy$-order} into the output iterator.
 The value-type of \ccc{OutputIterator} is \ccc{CGAL::Object}, where each
 \ccc{Object} either wraps a \ccc{pair<Point_2,Multiplicity>} instance, which
 represents an intersection point with its multiplicity (in case the
 multiplicity is undefined or not known, it should be set to $0$) or an
 \ccc{X_monotone_curve_2} instance, representing an overlapping subcurve of
 \ccc{c1} and \ccc{c2}. In the latter case, if \ccc{c1} and \ccc{c2} have the
 same direction, then the overlapping subcurves should also be directed the
 same way; otherwise, they can be associated with an arbitrary direction.
 The operator returns a past-the-end iterator for the output sequence.}

\ccNestedType{Split_2}
{provides the operator~: \\
 \ccc{void operator() (X_monotone_curve_2 c, Point_2 p,
                       X_monotone_curve_2& c1, X_monotone_curve_2& c2)} \\
 which accepts an input curve \ccc{c} and a split point \ccc{p} in its
 interior. It splits \ccc{c} at the split point into two subcurves \ccc{c1}
 and \ccc{c2}, such that \ccc{p} is \ccc{c1}'s {\sl right} endpoint and
 \ccc{c2}'s {\sl left} endpoint. The direction of \ccc{c} should be preserved:
 in case \ccc{c} is directed from left to right then \ccc{p} becomes \ccc{c1}'s
 target and \ccc{c2}'s source; otherwise, \ccc{p} becomes \ccc{c2}'s
 target and \ccc{c1}'s source.} 

\ccCreation
\ccCreationVariable{traits}
%==========================

\ccThree{Construct_x_monotone_curve_2~~~}{}{\hspace*{7cm}}
\ccThreeToTwo

\ccConstructor{ArrangementDirectionalXMonotoneTraits_2();}{default constructor.}
\ccGlue
\ccConstructor{ArrangementDirectionalXMonotoneTraits_2(ArrangementDirectionalXMonotoneTraits_2 other);}
{copy constructor}
\ccGlue
\ccMethod{ArrangementDirectionalXMonotoneTraits_2  operator=(other);}{assignment operator.}


\ccHeading{Accessing Functor Objects}
%====================================

\ccMethod{Compare_endpoints_xy_2 compare_endpoints_xy_2_object();} {}
\ccGlue
\ccMethod{Construct_opposite_2 construct_opposite_2_object();} {}

\ccHasModels
%===========

\ccc{CGAL::Arr_segment_traits_2<Kernel>} \\
\ccc{CGAL::Arr_non_caching_segment_traits_2<Kernel>} \\
\ccc{CGAL::Arr_circle_segment_traits_2<Kernel>} \\
\ccc{CGAL::Arr_conic_traits_2<RatKernel,AlgKernel,NtTraits>} \\
\ccc{CGAL::Arr_rational_arc_traits_2<AlgKernel,NtTraits>}

\ccSeeAlso
%=========

\ccc{ArrangementXMonotoneTraits_2}\lcTex{ 
      (\ccRefPage{ArrangementXMonotoneTraits_2})}

\end{ccRefConcept}

\ccRefPageEnd
