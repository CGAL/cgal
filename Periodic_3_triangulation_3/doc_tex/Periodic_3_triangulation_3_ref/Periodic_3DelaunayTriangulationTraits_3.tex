% +------------------------------------------------------------------------+
% | Reference manual page: Periodic_3DelaunayTriangulationTraits_3.tex
% +------------------------------------------------------------------------+
% | 19.2.2009   Monique Teillaud, Manuel Caroli
% | Package: Periodic_3_triangulation_3
% | 
\RCSdef{\RCSPeriodicDelaunayTriangulationTraitsRev}{$Id$}
\RCSdefDate{\RCSPeriodicDelaunayTriangulationTraitsDate}{$Date$}
% |
%%RefPage: end of header, begin of main body
% +------------------------------------------------------------------------+


\begin{ccRefConcept}{Periodic_3DelaunayTriangulationTraits_3}

\ccDefinition
The concept \ccRefName\ is the first template parameter of the classes
\ccc{Periodic_3_Delaunay_triangulation_3} and
\ccc{Periodic_3_triangulation_3}. It refines the concept
\ccc{DelaunayTriangulationTraits_3} from the \cgal\ \ccRef[3D
Triangulation]{Pkg:Triangulation3} package.
It redefines the geometric objects, predicates and constructions to
work with point-offset pairs. In most cases the offsets will be
(0,0,0) and the predicates from \ccc{DelaunayTriangulationTraits_3}
can be used directly. For efficiency reasons we maintain for each
functor the version without offsets.

\ccRefines \ccc{DelaunayTriangulationTraits_3}

In addition to the requirements described for the traits class
\ccc{DelaunayTriangulationTraits_3}, the geometric traits class of a
Periodic Delaunay triangulation must fulfill the following
requirements:


\ccTypes
\ccTwo{Periodic_3DelaunayTriangulationTraits_3::Tetrahedron_3xx}{}

\ccNestedType{Point_3}
{The point type. It must be a model of \ccc{Kernel::Point_3}. }
\ccGlue
\ccNestedType{Vector_3} {The vector type. It must be a model of
  \ccc{Kernel::Vector_3}.}
\ccGlue
\ccNestedType{Periodic_3_offset_3} {The offset type. It must be a
  model of the concept \ccc{Periodic_3Offset_3}.}
\ccGlue
\ccNestedType{Iso_cuboid_3} {A type representing an axis-aligned
  cuboid. It must be a model of \ccc{Kernel::Iso_cuboid_3}.}

The following three types represent geometric primitives in $\mathbb
R^3$. They are required to provide functions converting primitives
from $\mathbb T_c^3$ to $\mathbb R^3$, i.e.\ constructing
representatives in $\mathbb R^3$.
\ccNestedType{Segment_3} {A segment type. It must be a model of \ccc{Kernel::Segment_3}.}
\ccGlue
\ccNestedType{Triangle_3} {A triangle type. It must be a model of
  \ccc{Kernel::Triangle_3}. }
\ccGlue
\ccNestedType{Tetrahedron_3} {A tetrahedron type. It must be a model
of \ccc{Kernel::Tetrahedron_3}. }


\ccTwo{Periodic_3}{}

\ccNestedType{Compare_xyz_3}
{A predicate object that must provide the function operators\\ 
\ccc{Comparison_result operator()(Point_3 p, Point_3 q)},\\
which returns \ccc{EQUAL} if the two points are equal and\\
\ccc{Comparison_result operator()(Point_3 p, Point_3 q,
  Periodic_3_offset_3 o_p, Periodic_3_offset_3 o_q)},\\
which returns \ccc{EQUAL} if the two point-offset pairs are equal.
Otherwise it must return a consistent order for any two points chosen
in a same line.
\ccPrecond{\ccc{p}, \ccc{q} lie inside the domain.}}
\ccGlue
\ccNestedType{Orientation_3}
{A predicate object that must provide the function operators\\
\ccc{Orientation operator()(Point_3 p, Point_3 q, Point_3 r, Point_3 s)},\\
which returns \ccc{POSITIVE}, if \ccc{s} lies on the positive side of
the oriented plane \ccc{h} defined by \ccc{p}, \ccc{q}, and \ccc{r},
returns \ccc{NEGATIVE} if \ccc{s} lies on the negative side of
\ccc{h}, and returns \ccc{COPLANAR} if \ccc{s} lies on \ccc{h} and\\
\ccc{Orientation operator()(Point_3 p, Point_3 q, Point_3 r, Point_3 s,
Periodic_3_offset_3 o_p, Periodic_3_offset_3 o_q,
Periodic_3_offset_3 o_r, Periodic_3_offset_3 o_s)},\\
which returns \ccc{POSITIVE}, if the point-offset pair \ccc{(s,o_s)}
lies on the positive side of the oriented plane \ccc{h} defined by
\ccc{(p,o_p)}, \ccc{(q,o_q)}, and \ccc{(r,o_r)}, 
returns \ccc{NEGATIVE} if \ccc{(s,o_s)} lies on the negative side of
\ccc{h}, and returns \ccc{COPLANAR} if \ccc{(s,o_s)} lies on \ccc{h}.
\ccPrecond{\ccc{p}, \ccc{q}, \ccc{r}, \ccc{s} lie inside the domain.}}

\ccNestedType{Side_of_oriented_sphere_3}
{A predicate object that must provide the function operators\\
\ccc{Oriented_side operator()(Point_3 p, Point_3 q, Point_3 r, Point_3 s, Point_3 t)},\\
which determines on which side of the oriented sphere circumscribing 
\ccc{p, q, r, s} the point \ccc{t} lies and\\
\ccc{Oriented_side operator()(Point_3 p, Point_3 q, Point_3 r, Point_3 s,
  Point_3 t, Periodic_3_offset_3 o_p, Periodic_3_offset_3 o_q,
Periodic_3_offset_3 o_r, Periodic_3_offset_3 o_s,
Periodic_3_offset_3 o_t)},\\
which determines on which side of the oriented sphere circumscribing 
\ccc{(p,o_p), (q,o_q), (r,o_r), (s,o_s)} the point-offset pair
\ccc{(t,o_t)} lies.
\ccPrecond{\ccc{p}, \ccc{q}, \ccc{r}, \ccc{s}, \ccc{t} lie inside the domain.} } 
\ccGlue
\ccNestedType{Compare_distance_3}
{A predicate object that must provide the function operators\\
\ccc{Comparison_result operator()(Point_3 p, Point_3 q, Point_3 r)},\\
which compares the distance between \ccc{p} and \ccc{q} to the distance
between \ccc{p} and \ccc{r} and\\
\ccc{Comparison_result operator()(Point_3 p, Point_3 q, Point_3 r,
Periodic_3_offset_3 o_p, Periodic_3_offset_3 o_q, Periodic_3_offset_3 o_r)},\\
which compares the distance between \ccc{(p,o_p)} and \ccc{(q,o_q)} to
the distance between \ccc{(p,o_p)} and \ccc{(r,o_r)}.
\ccPrecond{\ccc{p}, \ccc{q}, \ccc{r} lie inside the domain.}}

In addition, only when vertex removal is used, the traits class must
provide the following predicate objects:
\ccNestedType{Coplanar_orientation_3}
{A predicate object that must provide the function operators\\
\ccc{Orientation operator()(Point_3 p, Point_3 q, Point_3 r)},\\
which returns \ccc{COLLINEAR}, if the points are collinear; otherwise
it must return a consistent orientation for any three points chosen in
a same plane and\\
\ccc{Orientation operator()(Point_3 p, Point_3 q, Point_3 r
Periodic_3_offset_3 o_p, Periodic_3_offset_3 o_q,
Periodic_3_offset_3 o_r)},\\
which returns \ccc{COLLINEAR}, if the point-offset pairs are
collinear; otherwise it must return a consistent orientation for any
three point-offset pairs chosen in a same plane.
\ccPrecond{\ccc{p}, \ccc{q}, \ccc{r} lie inside the domain.}}
\ccGlue
\ccNestedType{Coplanar_side_of_bounded_circle_3}
{A predicate object that must provide the function operators\\
\ccc{Bounded_side operator()(Point_3 p, Point_3 q, Point_3 r, Point_3 s)},\\
which determines the bounded side of the circle defined by \ccc{p, q},
and \ccc{r} on which the point \ccc{s} lies and\\
\ccc{Bounded_side operator()(Point_3 p, Point_3 q, Point_3 r, Point_3 s,
Periodic_3_offset_3 o_p, Periodic_3_offset_3 o_q,
Periodic_3_offset_3 o_r, Periodic_3_offset_3 o_s)},\\
which determines the bounded side of the circle defined by
\ccc{(p,o_p), (q,o_q)}, and \ccc{(r,o_r)} on which the point-offset pair
\ccc{(s,o_s)} lies.
\ccPrecond{\ccc{p,q,r}, and \ccc{s} are coplanar and \ccc{p,q}, and
  \ccc{r} are not collinear, \ccc{(p,o_p),(q,o_q),(r,o_r)}, and
  \ccc{(s,o_s)} are coplanar and \ccc{(p,o_p),(q,o_q)}, and
  \ccc{(r,o_r)} are not collinear, respectively, and 
\ccc{p}, \ccc{q}, \ccc{r}, \ccc{s}, \ccc{t} lie inside the domain.} } 

In addition, only when \ccc{is_Gabriel} is used, the traits class must
provide the following predicate object:
\ccNestedType{Side_of_bounded_sphere_3}
{A predicate object that must provide the function operators\\
\ccc{Bounded_side operator()(Point_3 p, Point_3 q, Point_3 t)},\\
which returns the position of the point \ccc{t} relative to the sphere
that has \ccc{pq} as its diameter,\\
\ccc{Bounded_side operator()(Point_3 p, Point_3 q, Point_3 t,
Periodic_3_offset_3 o_p, Periodic_3_offset_3 o_q,
Periodic_3_offset_3 o_t)},\\
which returns the position of the point-offset pair \ccc{(t,o_t)}
relative to the sphere that has \ccc{(p,o_p)(q,o_q)} as its diameter,\\
\ccc{Bounded_side operator()(Point_3 p, Point_3 q, Point_3 r, Point_3 t)},\\
which returns the position of the point \ccc{t} relative to the sphere
passing through \ccc{p, q}, and \ccc{r} and whose center is in the
plane defined by these three points,\\
\ccc{Bounded_side operator()(Point_3 p, Point_3 q, Point_3 r,
  Point_3 t, Periodic_3_offset_3 o_p, Periodic_3_offset_3 o_q,
Periodic_3_offset_3 o_r, Periodic_3_offset_3 o_q)},\\
which returns the position of the point-offset pair \ccc{(t,o_t)}
relative to the sphere passing through \ccc{(p,o_p), (q,o_q)}, and
\ccc{(r,o_r)} and whose center is in the plane defined by these three
point-offset pairs,\\
\ccc{Bounded_side operator()(Point_3 p, Point_3 q, Point_3 r, Point_3 s, Point_3 t)},\\
which returns the relative position of point \ccc{t} to the sphere
defined by \ccc{p, q, r}, and \ccc{s}; the order of the points \ccc{p,
  q, r}, and \ccc{s} does not matter, and\\
\ccc{Bounded_side operator()(Point_3 p, Point_3 q, Point_3 r, Point_3 s,
  Point_3 t, Periodic_3_offset_3 o_p, Periodic_3_offset_3 o_q,
Periodic_3_offset_3 o_r, Periodic_3_offset_3 o_s,
Periodic_3_offset_3 o_q)},\\
which returns the relative position of the point-offset pair
\ccc{(t,o_t)} to the sphere defined by \ccc{(p,o_p), (q,o_q),
  (r,o_r)}, and \ccc{(s,o_s)}; the order of the point-offset pairs
\ccc{(p,o_p), (q,o_q), (r,o_r)}, and \ccc{(s,o_s)} does not matter.
\ccPrecond{
\ccc{p, q, r}, and \ccc{s} are not coplanar,
\ccc{(p,o_p), (q,o_q), (r,o_r)}, and \ccc{(s,o_s)} are not coplanar,
\ccc{p}, \ccc{q}, \ccc{r}, \ccc{s}, \ccc{t} lie inside the domain.} } 


Note that the traits must provide exact constructions in order to
guarantee exactness of the following construction functors.

\ccNestedType{Construct_point_3}
{A constructor object that must provide the function operator\\
\ccc{Point_3 operator()(Point_3 p, Periodic_3_offset_3 o_p)},\\
which constructs a point from a point-offset pair.
\ccPrecond{\ccc{p} lies inside the domain.}}
\ccGlue
\ccNestedType{Construct_segment_3}
{A constructor object that must provide the function operators\\
\ccc{Segment_3 operator()(Point_3 p, Point_3 q)},\\
which constructs a segment from two points and\\
\ccc{Segment_3 operator()(Point_3 p, Point_3 q,
Periodic_3_offset_3 o_p, Periodic_3_offset_3 o_q)},\\
which constructs a segment from two point-offset pairs.
\ccPrecond{\ccc{p}, \ccc{q} lie inside the domain.}}
\ccGlue
\ccNestedType{Construct_triangle_3}
{A constructor object that must provide the function operators\\
\ccc{Triangle_3 operator()(Point_3 p, Point_3 q, Point_3 r )},\\
which constructs a triangle from three points and\\
\ccc{Triangle_3 operator()(Point_3 p, Point_3 q, Point_3 r,
  Periodic_3_offset_3 o_q,
  Periodic_3_offset_3 o_q,
  Periodic_3_offset_3
o_r)},\\
which constructs a triangle from three point-offset pairs.
\ccPrecond{\ccc{p}, \ccc{q}, \ccc{r} lie inside the domain.}}
\ccGlue
\ccNestedType{Construct_tetrahedron_3}
{A constructor object that must provide the function operators\\
\ccc{Tetrahedron_3 operator()(Point_3 p, Point_3 q, Point_3 r, Point_3 s)},\\
which constructs a tetrahedron from four points and\\
\ccc{Tetrahedron_3 operator()(Point_3 p, Point_3 q, Point_3 r, Point_3 s,
  Periodic_3_offset_3 o_q, Periodic_3_offset_3 o_q,
  Periodic_3_offset_3 o_r, Periodic_3_offset_3 o_s)},\\
which constructs a tetrahedron from four point-offset pairs.
\ccPrecond{\ccc{p}, \ccc{q}, \ccc{r}, \ccc{s} lie inside the domain.}}

In addition, only when the dual operations are used, the traits class
must provide the following constructor object: 

\ccNestedType{Construct_circumcenter_3}
{A constructor object that must provide the function operators\\
\ccc{Point_3 operator()(Point_3 p, Point_3 q, Point_3 r, Point_3 s)},\\
which constructs the circumcenter of four points and\\
\ccc{Point_3 operator()(Point_3 p, Point_3 q, Point_3 r, Point_3 s,
Periodic_3_offset_3 o_p, Periodic_3_offset_3 o_q,
Periodic_3_offset_3 o_r, Periodic_3_offset_3 o_s)},\\
which constructs the circumcenter of four point-offset pairs.
\ccPrecond{\ccc{p}, \ccc{q}, \ccc{r} and \ccc{s} as well as
\ccc{(p,o_p)}, \ccc{(q,o_q)}, \ccc{(r,o_r)} and \ccc{(s,o_s)} must be
non coplanar.
\ccc{p}, \ccc{q}, \ccc{r}, \ccc{s} lie inside the domain.}}

The optional types must be provided in any case, however they can be
replaced by dummy types if the respective functions are not used.

\ccCreation
\ccCreationVariable{traits}
\ccThree{Periodic_3Triangulation_traits_3xxx();}{Periodic_3Triangulation_traits_3 & tr}{}
\ccThreeToTwo

\ccConstructor{Periodic_3_triangulation_traits_3();}{Default constructor.}
\ccGlue
\ccConstructor{Periodic_3_triangulation_traits_3(const Periodic_triangulation_traits_3 & tr);}
{Copy constructor.}

\ccAccessFunctions
\ccThree{void}{set_domain(Iso-cuboid_3 domain);}{}
\ccMethod{void set_domain(Iso_cuboid_3 domain);}{Set the size of the
  fundamental domain. This is necessary to evaluate predicates
  correctly.
\ccPrecond{\ccc{domain} represents a cube.}}

\ccOperations

The following functions give access to the predicate and construction objects:

\ccThree{coplanar_side_of_bounded_circle_3xxx}{gt.xxxxxxxxx(Point p0, Point p1)x}{}


\ccMethod{Compare_xyz_3 compare_xyz_3_object();}{}
\ccGlue
\ccMethod{Orientation_3 orientation_3_object();}{}

\ccMethod{Side_of_oriented_sphere_3 side_of_oriented_sphere_3_object();}{}
\ccGlue
\ccMethod{Compare_distance_3 compare_distance_3_object();}{}

The following functions must be provided if vertex removal is
used; otherwise dummy functions can be provided.

\ccMethod{Coplanar_orientation_3 coplanar_3_orientation_3_object();}{}
\ccGlue
\ccMethod{Coplanar_side_of_bounded_circle_3
coplanar_side_of_bounded_circle_3_object();}{}

The following function must be provided only if the \ccc{is_Gabriel}
methods of \ccc{Periodic_3_Delaunay_triangulation_3} are used;
otherwise a dummy function can be provided.

\ccMethod{Side_of_bounded_sphere_3 side_of_bounded_sphere_3_object();}{}


\ccMethod{Construct_segment_3 construct_segment_3_object();}{}
\ccGlue
\ccMethod{Construct_triangle_3 construct_triangle_3_object();}{}
\ccGlue
\ccMethod{Construct_tetrahedron_3 construct_tetrahedron_3_object();}{}

The following function must be provided only if the methods of 
\ccc{Periodic_3_Delaunay_triangulation_3} returning elements of the
Voronoi diagram are used; otherwise a dummy function can be provided:

\ccMethod{Construct_circumcenter_3 construct_circumcenter_3_object();}{}

\ccHasModels

\ccc{CGAL::Periodic_3_triangulation_traits_3}

\ccSeeAlso

\ccc{DelaunayTriangulationTraits_3}

\end{ccRefConcept}
