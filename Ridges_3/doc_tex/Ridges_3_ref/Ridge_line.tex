% +------------------------------------------------------------------------+
% | Reference manual page: Ridge_line.tex
% +------------------------------------------------------------------------+
% | 15.09.2006   Marc Pouget and Fr�d�ric Cazals
% | Package: Ridges_3
% | 
\RCSdef{\RCSRidgelineRev}{$Id$}
\RCSdefDate{\RCSRidgelineDate}{$Date$}
% |
%%RefPage: end of header, begin of main body
% +------------------------------------------------------------------------+


\begin{ccRefClass}{Ridge_line<TriangularPolyhedralSurface>}  %% add template arg's if necessary

%% \ccHtmlCrossLink{}     %% add further rules for cross referencing links
%% \ccHtmlIndexC[class]{} %% add further index entries

\ccDefinition
  
The class \ccRefName\ stores the description of a ridge line.

\ccInclude{CGAL/Ridges.h}


\ccTypes

\ccTypedef{typedef typename TriangularPolyhedralSurface::Traits::FT  FT;}{}
\ccGlue
\ccTypedef{typedef typename TriangularPolyhedralSurface::Halfedge_handle  Halfedge_handle;}{}
\ccGlue
\ccTypedef{ typedef std::pair< Halfedge_handle, FT>           ridge_he;}{}


\ccCreation
\ccCreationVariable{ridge_line}  %% choose variable name

\ccConstructor{Ridge_line();}{default constructor.}

\ccAccessFunctions
% +--------------------------------------------------------------
\ccMethod{const Ridge_type line_type();}{}
\ccGlue
\ccMethod{Ridge_type& line_type();}{}
\ccMethod{const FT strength();}{}
\ccGlue
\ccMethod{FT& strength();}{}
\ccMethod{const FT sharpness();}{}
\ccGlue
\ccMethod{FT& sharpness(); }{}
\ccMethod{const std::list<ridge_he>* line();}{}
\ccGlue
\ccMethod{std::list<ridge_he>* line();}{}

The opertor $<<$ is overloaded for this class and returns the line
type, strength, sharpness and coordinates of the points of the
polyline.

\ccSeeAlso

\ccc{Ridge_approximation},

\end{ccRefClass}

% +------------------------------------------------------------------------+
%%RefPage: end of main body, begin of footer
% EOF
% +------------------------------------------------------------------------+

