% +------------------------------------------------------------------------+
% | Reference manual page: Ridge_approximation.tex
% +------------------------------------------------------------------------+
% | 15.09.2006   Marc Pouget and Fr�d�ric Cazals
% | Package: Ridges_3
% | 
\RCSdef{\RCSRidgeapproximationRev}{$Id$}
\RCSdefDate{\RCSRidgeapproximationDate}{$Date$}
% |
%%RefPage: end of header, begin of main body
% +------------------------------------------------------------------------+


\begin{ccRefClass}{Ridge_approximation<TriangularPolyhedralSurface,OutputIt,Vertex2FTPropertyMap,Vertex2VectorPropertyMap>} 
%% add templatearg's if necessary

%% \ccHtmlCrossLink{}     %% add further rules for cross referencing links
%% \ccHtmlIndexC[class]{} %% add further index entries

\ccDefinition
  
The class \ccRefName\ is designed to perform the approximation of
ridges of a triangular polyhedral surface. 

\ccInclude{CGAL/Ridges.h}

\ccParameters
The class \ccRefName\ has four template parameters. 
\ccc{TriangularPolyhedralSurface} provides  the surface. 
Parameter \ccc{OutputIt} is a stl Output Iterator whose
\ccc{value_type} is \ccc{Ridge_line*}. Parameters  \ccc{Vertex2FTPropertyMap}
and \ccc{Vertex2VectorPropertyMap} provide the differential properties of
the surface associated to its vertices.

Requirements (checked at compile time) : the types
\ccc{TriangularPolyhedralSurface::Traits::FT} and
\ccc{Vertex2FTPropertyMap::value_type} must coincide; the types
\ccc{TriangularPolyhedralSurface::Traits::Vector_3} and
\ccc{Vertex2VectorPropertyMap::value_type} must coincide; the types
\ccc{TriangularPolyhedralSurface::Vertex_handle},
\ccc{Vertex2FTPropertyMap::key_type} and
\ccc{Vertex2VectorPropertyMap::key_type} must coincide;

\ccTypes

%\ccNestedType{TYPE}{some nested types}
\ccEnum{   enum Tag_order {Tag_3, Tag_4};}
{Order of differential quantities used to distinguish elliptic and
hyperbolic ridges.}


\ccCreation
\ccCreationVariable{ridge_approximation}  %% choose variable name, given by \ccVar

\ccConstructor{Ridge_approximation(TriangularPolyhedralSurface &P,     
			Vertex2FTPropertyMap& vertex2k1_pm,
		      Vertex2FTPropertyMap& vertex2k2_pm,
		      Vertex2FTPropertyMap& vertex2b0_pm,
		      Vertex2FTPropertyMap& vertex2b3_pm,
		      Vertex2FTPropertyMap& vertex2P1_pm,
		      Vertex2FTPropertyMap& vertex2P2_pm,
		      Vertex2VectorPropertyMap& vertex2d1_pm,
		      Vertex2VectorPropertyMap& vertex2d2_pm);}
		      {Precondition : all faces of P may be  triangular. }

%\ccOperations

\ccMethod{  OutputIt compute_all_ridges(OutputIt it, Tag_order ord = Tag_3);}
{Computes blue, red and crest ridges.}
\ccGlue
\ccMethod{   OutputIt compute_ridges(Ridge_interrogation_type r_type, 
			  OutputIt ridge_lines_it,
			  Tag_order ord = Tag_3);}
{Computes blue, red or crest ridges.}

\ccSeeAlso

\ccc{Ridge_line}

%\ccExample

%A short example program.
%Instead of a short program fragment, a full running program can be
%included using the 
%\verb|\ccIncludeExampleCode{Jet_fitting_3/Ridge_approximation.cpp}| 
%macro. The program example would be part of the source code distribution and
%also part of the automatic test suite.

%\begin{ccExampleCode}
%void your_example_code() {
%}
%\end{ccExampleCode}

%% \ccIncludeExampleCode{Jet_fitting_3/Ridge_approximation.cpp}

\end{ccRefClass}

% +------------------------------------------------------------------------+
%%RefPage: end of main body, begin of footer
% EOF
% +------------------------------------------------------------------------+

