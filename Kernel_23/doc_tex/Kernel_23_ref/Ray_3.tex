\begin{ccRefClass} {Ray_3<Kernel>}

\ccDefinition
An object \ccStyle{r} of the data type \ccRefName\ is a directed
straight ray in the three-dimensional Euclidean space $\E^3$. It starts
in a  point called the {\em source} of  \ccStyle{r} and it goes to infinity.


\ccCreation
\ccCreationVariable{r}


\ccHidden \ccConstructor{Ray_3();}
             {introduces an uninitialized variable \ccVar.}

\ccHidden \ccConstructor{Ray_3(const Ray_3<Kernel> &s);}
 	    {copy constructor.}

\ccConstructor{Ray_3(const Point_3<Kernel> &p, const Point_3<Kernel> &q);}
            {introduces a ray \ccVar\ 
             with source $p$ and passing through point $q$.}

\ccConstructor{Ray_3(const Point_3<Kernel> &p, const Direction_3<Kernel> &d)}
            {introduces a ray \ccVar\ with source $p$ and with 
             direction $d$.}

\ccConstructor{Ray_3(const Point_3<Kernel> &p, const Vector_3<Kernel> &v)}
            {introduces a ray \ccVar\ with source $p$ and with 
             a direction given by $v$.}

\ccConstructor{Ray_3(const Point_3<Kernel> &p, const Line_3<Kernel> &l)}
            {introduces a ray \ccVar\ starting at source $p$ with 
             the same direction as $l$.}

\ccOperations

\ccHidden \ccMethod{Ray_3<Kernel> &operator=(const Ray_3<Kernel> &s);}
        {Assignment.}

\ccMethod{bool operator==(const Ray_3<Kernel> &h) const;}
       {Test for equality: two rays are equal, iff they have the same 
        source and the same direction.}

\ccMethod{bool operator!=(const Ray_3<Kernel> &h) const;}
       {Test for inequality.}



\ccMethod{Point_3<Kernel> source() const;}
       {returns the source of \ccVar}

\ccMethod{Point_3<Kernel> point(int i) const;}
       {returns a point on \ccVar. \ccStyle{point(0)} is the source.
        \ccStyle{point(i)}, with $i>0$, is different from the 
        source. \ccPrecond $i \geq 0$.}

\ccMethod{Direction_3<Kernel> direction() const;}
       {returns the direction of \ccVar.}

\ccMethod{Vector_3<Kernel> to_vector() const;}
       {returns a vector giving the direction of \ccVar.}

\ccMethod{Line_3<Kernel>      supporting_line() const;}
       {returns the line supporting \ccVar\ which has the same direction.}

\ccMethod{Ray_3<Kernel>       opposite() const;}
       {returns the ray with the same source and the opposite direction.}

\ccMethod{bool is_degenerate() const;}
       {ray \ccVar\ is degenerate, if the source and the second defining
        point fall together (that is if the direction is degenerate).}


\ccMethod{bool has_on(const Point_3<Kernel> &p) const;}
       {A point is on \ccVar, iff it is equal to the source 
        of \ccVar, or if it is in the interior of \ccVar.}
% 
% \ccMethod{bool collinear_has_on(const Point_3<Kernel> &p) const;}
%        {checks if point $p$ is on ray \ccVar. This function is faster
%         than function \ccStyle{has_on()}.
%         \ccPrecond{$p$ is collinear to \ccVar.}}
% 
\ccMethod{Ray_3<Kernel> transform(const Aff_transformation_3<Kernel> &t) const;}
       {returns the ray obtained by applying $t$ on the source 
        and on the direction of \ccVar.}

\ccSeeAlso
\ccRefConceptPage{Kernel::Ray_3}

\end{ccRefClass} 
