\begin{ccRefFunctionObjectConcept}{Kernel::ConstructBisector_2}
A model for this must provide:

\ccCreationVariable{fo}

\ccMemberFunction{Kernel::Line_2 operator()(const Kernel::Point_2&p, 
                                            const Kernel::Point_2&q );}
{constructs the bisector of $p$ and $q$.
The bisector is oriented in such a way that \ccc{p} lies on its
positive side. \ccPrecond{\ccc{p} and \ccc{q} are not equal.}}

\ccMemberFunction{Kernel::Line_2 operator()(const Kernel::Line_2&l1, 
                                            const Kernel::Line_2&l2);}
{constructs the bisector of the two lines $l1$ and $l2$.
In the general case, the bisector has the direction of the vector which
is the sum of the normalized directions of the two lines, and which passes
through the intersection of \ccc{l1} and \ccc{l2}.
If \ccc{l1} and \ccc{l2} are parallel, then the bisector is defined as the line
which has the same direction as \ccc{l1}, and which is at the same distance
from \ccc{l1} and \ccc{l2}.
This function requires that \ccc{Kernel::RT} supports the \ccc{sqrt()}
operation.}

\ccRefines
\ccc{AdaptableFunctor} (with two arguments)

\ccSeeAlso
\ccRefIdfierPage{CGAL::bisector}

\end{ccRefFunctionObjectConcept}
