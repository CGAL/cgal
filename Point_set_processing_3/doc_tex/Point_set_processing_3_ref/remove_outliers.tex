% +------------------------------------------------------------------------+
% | Reference manual page: remove_outliers.tex
% +------------------------------------------------------------------------+
% | 02.06.2008   Pierre Alliez, Laurent Saboret, Gael Guennebaud
% | Package: Point_set_processing_3
% |
\RCSdef{\RCSoutlierremovalRev}{$Id$}
\RCSdefDate{\RCSoutlierremovalDate}{$Date$}
% |
\ccRefPageBegin
%%RefPage: end of header, begin of main body
% +------------------------------------------------------------------------+


\begin{ccRefFunction}{remove_outliers}  %% add template arg's if necessary

%% \ccHtmlCrossLink{}     %% add further rules for cross referencing links
%% \ccHtmlIndexC[function]{} %% add further index entries

\ccDefinition


\ccc{CGAL::remove_outliers()} deletes a user-specified fraction of outliers from the input point set. More specifically, it sorts the input points in increasing order of average squared distances to the $k$ nearest neighbors and deletes the points with largest value.

\ccInclude{CGAL/remove_outliers.h}

% The section below is automatically generated. Do not edit!
%START-AUTO(\ccDefinition)

\ccFunction{template<typename ForwardIterator, typename Kernel> ForwardIterator remove_outliers(ForwardIterator first, ForwardIterator beyond, unsigned int k, const Kernel& kernel, double threshold_percent);}
{
Remove outliers:\begin{itemize}
\item compute average squared distance to the K nearest neighbors,\item sort the points in increasing order of average distance.\end{itemize}
This method modifies the order of input points, and returns an iterator over the first point to remove (see erase-remove idiom). Warning: this method should not be called on sorted containers.
\ccPrecond k $>$= 2.
\ccCommentHeading{Template Parameters}
\begin{description}
\item \ccc{ForwardIterator}: \ccc{value_type} must be convertible to \ccc{Point_3<Kernel>}. \item \ccc{Kernel}: Geometric traits class. It can be omitted and deduced automatically from the iterator type.\end{description}
\ccCommentHeading{Returns} iterator over the first point to remove.
\ccCommentHeading{Parameters}
\begin{description}
\item \ccc{first}: iterator over the first input/output point. \item \ccc{beyond}: past-the-end iterator. \item \ccc{k}: number of neighbors. \item \ccc{kernel}: geometric traits. \item \ccc{threshold_percent}: percentage of points to remove. \end{description}
}
\ccGlue

%END-AUTO(\ccDefinition)

\ccExample

\ccIncludeExampleCode{Point_set_processing_3/remove_outliers_example.cpp}

\end{ccRefFunction}

% +------------------------------------------------------------------------+
%%RefPage: end of main body, begin of footer
\ccRefPageEnd
% EOF
% +------------------------------------------------------------------------+

