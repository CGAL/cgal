% +------------------------------------------------------------------------+
% | Reference manual page: pca_estimate_normals.tex
% +------------------------------------------------------------------------+
% | 14.02.2008   Pierre Alliez, Laurent Saboret, Gael Guennebaud
% | Package: Point_set_processing_3
% |
\RCSdef{\RCSpcanormalestimationRev}{$Id$}
\RCSdefDate{\RCSpcanormalestimationDate}{$Date$}
% |
\ccRefPageBegin
%%RefPage: end of header, begin of main body
% +------------------------------------------------------------------------+


\begin{ccRefFunction}{pca_estimate_normals}  %% add template arg's if necessary

%% \ccHtmlCrossLink{}     %% add further rules for cross referencing links
%% \ccHtmlIndexC[function]{} %% add further index entries

\ccDefinition

\ccc{CGAL::pca_estimate_normals()} estimates normal directions at all points from an input point set by linear least squares fitting of a plane over their $k$ nearest neighbors. The result is an unoriented normal vector for each input point.

\ccInclude{CGAL/pca_estimate_normals.h}

% The section below is automatically generated. Do not edit!
%START-AUTO(\ccDefinition)

\ccFunction{template<typename InputIterator, typename PointPMap, typename NormalPMap, typename Kernel> void pca_estimate_normals(InputIterator first, InputIterator beyond, PointPMap point_pmap, NormalPMap normal_pmap, unsigned int k, const Kernel& kernel);}
{
Estimates normal directions of the [first, beyond) range of points by linear least squares fitting of a plane over the k nearest neighbors. The output normals are randomly oriented.
\ccPrecond k $>$= 2.
\ccCommentHeading{Template Parameters}
\begin{description}
\item \ccc{InputIterator}: iterator over input points. \item \ccc{PointPMap}: is a model of \ccc{boost::ReadablePropertyMap} with a \ccc{value_type} = \ccc{Point_3<Kernel>}. It can be omitted if InputIterator \ccc{value_type} is convertible to \ccc{Point_3<Kernel>}. \item \ccc{NormalPMap}: is a model of \ccc{boost::WritablePropertyMap} with a \ccc{value_type} = \ccc{Vector_3<Kernel>}. \item \ccc{Kernel}: Geometric traits class. It can be omitted and deduced automatically from PointPMap \ccc{value_type}. \end{description}
\ccCommentHeading{Parameters}
\begin{description}
\item \ccc{first}: iterator over the first input point. \item \ccc{beyond}: past-the-end iterator over the input points. \item \ccc{point_pmap}: property map InputIterator -$>$ \ccc{Point_3}. \item \ccc{normal_pmap}: property map InputIterator -$>$ \ccc{Vector_3}. \item \ccc{k}: number of neighbors. \item \ccc{kernel}: geometric traits. \end{description}
}
\ccGlue

%END-AUTO(\ccDefinition)

\ccSeeAlso

\ccRefIdfierPage{CGAL::jet_estimate_normals}  \\
\ccRefIdfierPage{CGAL::mst_orient_normals}  \\

\ccExample

See \ccc{normals_example.cpp}.

\end{ccRefFunction}

% +------------------------------------------------------------------------+
%%RefPage: end of main body, begin of footer
\ccRefPageEnd
% EOF
% +------------------------------------------------------------------------+

