\section{Input/Output}


\subsection{Property Maps}

The algorithms of this component take as input parameters iterator ranges of 3D points, or points with normals.
As the latter might be represented in various ways, e.g., as a class derived from the \cgal\ point class, or as a
\ccc{std::pair<Point_3<K>, Vector_3<K> >}, or as a \ccc{boost::tuple<..,Point_3<K>, ..., Vector_3<K> >}, the
algorithms use property maps to access the point or vector information in the input data.  \\
\\
An another component provides property maps to support these cases:  \\
\ccc{CGAL::Dereference_property_map<T>}  \\
\ccc{CGAL::First_of_pair_property_map<Pair>} and \ccc{CGAL::Second_of_pair_property_map<Pair>}  \\
\ccc{CGAL::Nth_of_tuple_property_map<N, Tuple>}  \\
\\
Note that Dereference_property_map is the default value of the position property map expected by all functions in this component.
Users of this package may use other types to represent positions and normals if they implement the corresponding property maps.


\subsection{Streams}

This component provides functions to read and write sets of points or points with normals from the following ASCII file formats:
\begin{itemize}
\item XYZ (three point coordinates \ccc{x y z} per line or three point coordinates and three normal vector coordinates \ccc{x y z nx ny nz} per line), and \\
\item Object File Format (OFF)~\cite{cgal:p-gmgv16-96}.
\end{itemize}

\ccc{CGAL::read_xyz_points}  \\
\ccc{CGAL::read_off_points}  \\
\ccc{CGAL::write_off_points}  \\
\ccc{CGAL::write_xyz_points}  \\


\subsection{Example}

The following example reads a point set from an input file and writes it to a file, both in the xyz format. Position and normal are stored in pairs and accessed through property maps.
\ccIncludeExampleCode{Point_set_processing_3/read_write_xyz_point_set_example.cpp}