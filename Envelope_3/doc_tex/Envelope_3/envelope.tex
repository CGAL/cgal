
A continuous surface $S$ in $\reals^3$ is called {\em $xy$-monotone} if
every line parallel to the $z$-axis intersects it at a single point
at most. For example, the sphere $x^2 + y^2 + z^2 = 1$ is {\em not}
$xy$-monotone as the $z$-axis intersects it at $(0, 0, -1)$ and at
$(0, 0, 1)$; however, if we use the $xy$-plane to split it to an
upper hemisphere and a lower hemisphere, these two hemispheres are
$xy$-monotone.

An $xy$-monotone surface can therefore be represented as a
bivariate function $z = S(x,y)$, defined over some continuous range
$R_S \subseteq \reals^2$. Given a set $\calS = \{ S_1, S_2, \ldots,
S_n \}$ of $xy$-monotone surfaces, their {\em lower envelope} is defined
as the point-wise minimum of all surfaces. Namely, the lower envelope
of the set $\calS$ can be defined as the following function:
\begin{eqnarray*}
\calL_{\calS} (x,y) = \min_{1 \leq k \leq n}{\overline{S}_k (x,y)} \ ,
\end{eqnarray*}
where we define $\overline{S}_k(x,y) = S_k(x,y)$ for $(x,y) \in
R_{S_k}$, and $\overline{S}_k(x,y) = \infty$ otherwise.

Similarly, the {\em upper envelope} of $\calS$ is the point-wise maximum of
the $xy$-monotone surfaces in the set:
\begin{eqnarray*}
\calU_{\calS} (x,y) = \max_{1 \leq k \leq n}{\underline{S}_k (x,y)} \ ,
\end{eqnarray*}
where in this case $\underline{S}_k(x,y) = -\infty$ for $(x,y) \nin
R_{S_k}$.

Given a set of $xy$-monotone surfaces $\calS$, the {\em minimization
diagram} of $\calS$ is a subdivision of the $xy$-plane into cells,
such that the identity of the surfaces that induce the lower diagram
over a specific cell of the subdivision (be it a face, an edge or
a vertex) is the same. In non-degenerate situatuion, a face is
induced by a single surface (or by no surfaces at all, if there are
no $xy$-monotone surfaces defined over it), and an edge is induced
by a single surface and corresponds to its projected boundary, or by
two surfaces and corresponds to their projetced intersection curve.
The {\em maximization diagram} is symmetrically defined for upper envelopes.
In the rest of this chapter, we refer to both these digrams as
{\em envelope diagrams}.

It is easy to see that an envelope diagram is no more than a planar
arrangement (see Chapter~\ref{chapterArrangement_2}), represented
using an extended \dcel\ structure, such that every \dcel\ record
(namely each face, halfedge and vertex) stores an additional container
of it originators: the $xy$-monotone surfaces that induce this feature.

Lower and upper envelopes can be efficiently computed using a
divide-and-conquer approach. First note that the envelope diagram for
a single $xy$-monotone curve $S_k$ is trivial to compute: we porject
the boundary of its range of definition $R_{S_k}$ onto the $xy$-plane
and label the faces it induces accordingly. Given a set $\hat{\calS}$
of (non necessarily $xy$-monotone) surfaces in $\reals^3$, we start by
subdividing each surface into a finite number of weakly $xy$-monotone
surfaces\footnote{To handle degenerate inputs, we consider {\em vertical}
surfaces, namely pathces of planes that are perpendicular to the
$xy$-plane, as {\em weakly} $xy$-monotone.}, obtaining the set $\calS$.
We continue by splitting the set into two disjoint subsets $\calS_1$
and $\calS_2$, and we compute their envelope diagrams recursively.
We finally have tow merge the diagrams, and we do this by overlaying
them and then applying some post-processing on the resulting diagram.
The post-processing stage is non-trivial and invloves the projection
of intersection curves onto the $xy$-plane --- 
see~\cite{Michals_thesis} for more details.

\section{The Envelope-Triats Concept}
%====================================

\section{Examples}
%=================

\begin{figure}[t]
\begin{ccTexOnly}
  \begin{center}
  \begin{tabular}{ccc}
    \epsfig{figure=Envelope_3/fig/ex_triangles.eps,height=1.8in,silent=} &
    \epsfig{figure=Envelope_3/fig/ex_tri_le.eps,height=1.8in,silent=} &
    \epsfig{figure=Envelope_3/fig/ex_tri_ue.eps,height=1.8in,silent=} \\
    {\small (a)} & {\small (b)} & {\small (c)}
  \end{tabular}
  \end{center}
\end{ccTexOnly}
\begin{ccHtmlOnly}
  <p><center>
  <img src="./fig/ex_triangles.gif" border=0 alt="ex_triangles">
  </center>
\end{ccHtmlOnly}
\caption{(a)~Two triangles in $\reals^3$, as given in
\ccc{ex_envelope_triangles.cpp}. (b)~Their lower envelope.
(c)~Their upper envelope.\label{env3_fig:ex_tri}}
\end{figure}

The following example shows how to use the envelope-triats class
for 3D triangles and how to traverse the envelope diagram. It
constructs the lower and upper envelopes of the two triangles,
as depicted in Figure~\ref{env3_fig:ex_tri}(a) and prints the
triangles that induce each face in the output diagrams:

\ccIncludeExampleCode{../examples/Envelope_3/ex_envelope_triangles.cpp}

The next example demonstrates how to instantiate and use the
envelope-traits class for spheres. It reads a set of spheres
from an input file and constructs their lower envelope:

\ccIncludeExampleCode{../examples/Envelope_3/ex_envelope_spheres.cpp}

