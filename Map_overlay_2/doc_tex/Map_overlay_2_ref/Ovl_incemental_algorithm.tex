% +------------------------------------------------------------------------+
% | Reference manual page: Ovl_incremental_algorithm.tex (Map_overlay)
% +------------------------------------------------------------------------+
% | 
% | Package: ovl (Map_overlay_2)
% | 
% +------------------------------------------------------------------------+

\ccRefPageBegin

%%RefPage: end of header, begin of main body
% +------------------------------------------------------------------------+

%**************************************************************************
\renewcommand{\ccRefPageBegin}{\begin{ccAdvanced}}
\renewcommand{\ccRefPageEnd}{\end{ccAdvanced}}

\begin{ccRefClass}{Map_overlay_incremental<Subdivision,Notifier>}
\label{OVL_sec:incremental}

The \ccRefName\ class implements the incremental algorithm,
in which we insert all curves of the two creators into the overlay 
one by one in a random order. 
Theoretically, when maintaining a trapezoidal decomposition of the 
arrangement of the curves in the two creators $R$ and $B$, 
which were added so far, 
the construction of the overlay takes randomized expected time 
$O(N\log{N} + k)$~\cite{m-cgitr-93},
where $N$ is the number of curves the two creators have in total, 
and $k$ is the number of intersections between curves 
from $R$ and $B$. 
However, the usage of the \ccc{Planar_Map_With_Intersections_2}
package employing trapezoidal decomposition is too slow in practice. 
Hence, we construct the arrangement presenting the overlay 
using the ``walk along a line'' strategy, 
implying that the construction running time is quaratic in the input size.
The incremental algorithm can not use the 
\ccc{Planar_Map_2} package, since it has to deal all possibly 
intersections of each inserted curve during the overlay construction.


\ccInclude{CGAL/Map_overlay_incremental.h}

\ccIsModel
  \ccc{MapOverlayAlgorithm_2}

\ccInheritsFrom
  \ccc{Map_overlay_base<Subdivision,Notifier>}

\ccSeeAlso
   Discussion of the different overlay construction strategies in the introduction
of \ccc{Map_overlay_2} reference pages\lcTex{ (\ccRefPage{Ovl_map_overlay})}.

\end{ccRefClass}
\renewcommand{\ccRefPageBegin}{}
\renewcommand{\ccRefPageEnd}{}


% +------------------------------------------------------------------------+
%%RefPage: end of main body, begin of footer
\ccRefPageEnd
% EOF
% +------------------------------------------------------------------------+
