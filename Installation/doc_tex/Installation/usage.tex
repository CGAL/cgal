\section{Configuring and Building Programs Using \cgal}

\cmake\ can be used to configure and build user programs as well via a \cmake\ 
script ({\tt CMakeLists.txt}).  All the examples and
demos contain such \cmake\ scripts.

During configuration of the \cgal\ libraries a file named {\tt
CGALConfig.cmake} is generated in the binary directory. This file
contains the definitions of several \cmake\ variable that summarize the
configuration of \cgal. In order to configure a user program, you need
to indicate the location of that config file in the \cmake\ variable
\texttt{CGAL\_DIR}:

{\ccTexHtml{\scriptsize}{}
\begin{alltt}

cd \cgalrel/examples/Straight_skeleton_2
cmake -DCGAL_DIR=$HOME/\cgalrel . 
make

\end{alltt}
}

%$ % <- added to close non-correct "math" enviroment ("HOME"), DO NOT DELETE

\texttt{CGAL\_DIR} can also be an environment variable.

If you have installed CGAL, \texttt{CGAL\_DIR} must afterward be set to
\texttt{\$CMAKE\_INSTALLED\_PREFIX/lib/CGAL}.

\subsection{Custom flags in the programs using \cgal}

% TODO EBEB: Update this

Normally, programs linked with \cgal\ must be compiled with the same flags
used by the compilation of \cgal\
libraries. For this reason, the {\em very first} time
a program is configured, all the flags given by the \cmake\ variables \texttt{CMAKE\_*\_FLAGS}
are {\em locked} in the sense that the values recorded in \texttt{CGALConfig.cmake} 
are used to override any values given by \cmake\ itself or yourself.

This does not apply to the additional flags that can be given via \texttt{CGAL\_*\_FLAGS}.

Such {\em inherited} values are then recorded in the current \cmake\ cache for the program.
The flags are then {\em unlocked} in the sense that at any subsequent configuration you can
provide your own flags and this time they will not be overridden.

When using the interactive \texttt{cmake-gui} the first press on \texttt{Configure} unlocks
the flags, so that you can edit them as needed. 

\begin{ccAdvanced}
The locking of flags is controlled by the variable {\tt CGAL\_DONT\_OVERRIDE\_CMAKE\_FLAGS}
which starts out FALSE and is toggled right after the flags have been loaded from
\texttt{CGALConfig.cmake}.

If you use the command line tool you can specify flags {\em directly} by setting the
controlling variable right up front:

{\ccTexHtml{\scriptsize}{}
\begin{alltt}

cd \cgalrel

cmake -DCMAKE_BUILD_TYPE=Release -DCMAKE_CXX_FLAGS=-g .

cd \cgalrel/examples/Straight_skeleton_2

cmake -DCGAL_DIR=\cgalrel -DCMAKE_BUILD_TYPE=Debug -DCMAKE_CXX_FLAGS=-O2 -DCGAL_DONT_OVERRIDE_CMAKE_FLAGS=TRUE . 

\end{alltt}
}
\end{ccAdvanced}


\subsection{Creating a cmake script for a program using \cgal}

For relatively simple programs the Bourne-shell script \texttt{cgal\_create\_cmake\_script} can be used
(a short description of this file is provided in Appendix \ref{sec:create_cgal_cmake_script}).
More generally, within a cmake script, once \cgal\ has been found using \texttt{find\_package}
the variable \texttt{CGAL\_USE\_FILE} is set to a compilation environment CMake file. Including
this file within a cmake script sets up include paths and libraries to link with of \cgal\ and third party libraries.

% TODO EBEB: -c Core:GMP:RS3:MPFI

% TODO move appendices in here?



\section{Scripts}

\subsection{\texttt{cgal\_create\_cmake\_script\label{sec:create_cgal_cmake_script}}}
\TTindex{cgal\_create\_cmake\_script}\index{scripts!\texttt{cgal\_create\_cmake\_script}}

The Bourne-shell script \texttt{cgal\_create\_cmake\_script} is contained in the
\texttt{\cgalrel/scripts} directory. It can be used to create
\texttt{CmakeLists.txt} files for compiling \cgal\ applications. Executing
\texttt{cgal\_create\_cmake\_script} in an application directory creates a
\texttt{CMakeLists.txt} containing rules for every \texttt{*.cpp} file
there. Currently, that script only works for applications that only need
the \cgal\ and CGALCore libraries.


%%
%% EOF
%%


