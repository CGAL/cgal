% +------------------------------------------------------------------------+
% | Reference manual page: Tds_3.tex
% +------------------------------------------------------------------------+
% | 27.3.2000   Monique Teillaud
% | Package: Triangulation3
% | 
\RCSdef{\RCSTdsRev}{$Revision$}
\RCSdefDate{\RCSTdsDate}{$Date$}
% |
%%RefPage: end of header, begin of main body
% +------------------------------------------------------------------------+


\begin{ccRefConcept}{Tds_3}

%% \ccHtmlCrossLink{}     %% add further rules for cross referencing links
%% \ccHtmlIndexC[concept]{} %% add further index entries

\ccDefinition
  
The second template parameter of the basic triangulation class
\ccc{Triangulation_3<Triangulation_traits_3,Tds_3>} is a triangulation 
data structure class.  This class can be seen as a container for the
cells and v ertices maintaining incidence and adjacency relations (see 
Section~\ref{TDS3-sec-concept} of Chapter~\ref{chapter-TDS3}). 

Its optional arguments related to geometry are compulsory for this use as a
template parameter of \ccc{Triangulation_3<Triangulation_traits_3,Tds_3>}.

Since \cgal\ now provides only one model of this triangulation data
structure, it is not described in detail here. The description of the
model \ccc{Triangulation_data_structure_3} (see also
Section~\ref{TDS3-sec-class}) will be considered as the description of
the concept.

\ccHasModels

\ccc{Triangulation_data_structure_3}

\ccSeeAlso

\ccc{Triangulation_3}


\end{ccRefConcept}

% +------------------------------------------------------------------------+
%%RefPage: end of main body, begin of footer
% EOF
% +------------------------------------------------------------------------+

