% +------------------------------------------------------------------------+
% | Reference manual page: Triangulation_data_structure_3.tex
% +------------------------------------------------------------------------+
% | 29.3.2000   Monique Teillaud
% | Package: Triangulation3
% | 
\RCSdef{\RCSTriangulationdatastructureRev}{$Revision$}
\RCSdefDate{\RCSTriangulationdatastructureDate}{$Date$}
% |
%%RefPage: end of header, begin of main body
% +------------------------------------------------------------------------+


\begin{ccRefClass}{Triangulation_data_structure_3<Vb,Cb>}  %% add template arg's if necessary

%% \ccHtmlCrossLink{}     %% add further rules for cross referencing links
%% \ccHtmlIndexC[class]{} %% add further index entries

\ccDefinition
The class \ccRefName\ stores a 3D-triangulation data structure
(see Section~\ref{TDS3-sec-concept}) and provides the optional
geometric functionalities to be used as a parameter for a 
3D-geometric triangulation (see Chapter~\ref{chapter-Triangulation3}. 
It is templated by base classes for
vertices and cells (see Section~\ref{TDS3-sec-concept-Base}).

\ccInheritsFrom{\ccc{Triangulation_utils_3}}

\ccc{Triangulation_utils_3} class defines basic computations on
indices of vertices and neighbors of cells. (See
Section~\ref{Triangulation3-sec-class-Utils} of
Chapter~\ref{chapter-Triangulation3}.)

\ccInclude{CGAL/Triangulation_data_structure_3.h}

\ccIsModel

\ccc{Tds_3}

\ccSeeAlso

\ccc{Some_other_class},
\ccc{some_other_function}.

%% \ccExample


%% \ccIncludeExampleCode{examples/Triangulation3/Triangulation_data_structure_3_prog.C}

\end{ccRefClass}

% +------------------------------------------------------------------------+
%%RefPage: end of main body, begin of footer
% EOF
% +------------------------------------------------------------------------+

