\cleardoublepage
\chapter{Number Type Support}\label{Numbertype}

\cgal\ kernel classes are parameterized by number types.  
Depending on the problem and the input data that have to be handled,
one has to make a trade-off between efficiency and accuracy in 
order to select an appropriate number type and kernel class.

In homogeneous representation, two number types are involved,
although only one of them appears as a template parameter in
the homogeneous kernel classes.
This type, for the sake of simplicity and readability called ring type, is
used for the representation of homogeneous coordinates and all 
internal computations. 
If it is assured that the second operand divides the first one, these 
internal computations are basically division-free.
The ring type is a placeholder for an integer type (or an integral 
domain type) rather than for elements of arbitrary rings. 
The name should remind you that the division operation is not
needed for this number type.
Of course, also more general number types can be used as a ring type 
in a homogeneous kernel class. In some computations, e.g.\ accessing 
Cartesian coordinates, divisions cannot be avoided. In these computations a 
second number type, the field type, is used. \cgal\ automatically generates 
this number type as a \ccStyle{Quotient}\ccTexHtml{, 
cf.\ Subsection~\ref{Quotient}}{}. For the Cartesian kernels 
there is only one number type that is used for all calculations.

The kernel classes provide access to the number types 
involved in the representation, although it is not expected that
such access is needed at this level, since low-level geometric 
operations are wrapped in geometric primitives provided by \cgal.
This access can be useful if appropriate primitives are missing.
In a homogeneous kernel class \ccStyle{K}, ring type and field
type can be accessed as \ccStyle{K::RT} and \ccStyle{K::FT}, respectively.
The number type used in Cartesian kernels is considered as 
ring type or as field type depending on the context.
If can be accessed as \ccStyle{K::RT} and \ccStyle{K::FT}, according
to the use of number types used in the homogeneous counterpart.

\section{Required Functionality of Number Types\label{nt-requirements}}

Number types must fulfill certain requirements, such that they can
be successfully used in \cgal\ code.  The syntacitical requirements of
number types are described in
the concepts RingNumberType and FieldNumberType included in the 
\ccAnchor{fsKernel.html}{kernel reference manual}.
Of course, number types also
have evident semantic constraints. They should be meaningful in the
sense that they approximate the integers or the rationals 
or some other subfield of the real numbers.

\section{Utility Routines}

The number type concepts mentioned in the previous section list all 
the required functionality.
For the user of a number type it is handy to have a larger set of
operations available.

\subsection{Utility Functions}
\ccSetThreeColumns{Comparison_result}{}{\hspace*{6cm}}

\ccInclude{CGAL/number_utils.h}

\ccFunction{NT min(const NT& ntval1, const NT& ntval2);}
{returns the smaller of the two values.}
\ccGlue
\ccFunction{NT max(const NT& ntval1, const NT& ntval2);}
{returns the larger of the two values.}
\ccGlue
\ccFunction{NT abs(const NT& ntval);}
{returns the absolute value.}
\ccGlue
\ccFunction{NT square(const NT& ntval);}
{returns the square of \ccc{ntval}.}
\ccGlue
\ccFunction{Sign sign(const NT& ntval);}
{returns the sign: \ccc{POSITIVE}, \ccc{ZERO}, or \ccc{NEGATIVE}.}
\ccGlue
\ccFunction{bool is_negative(const NT& ntval);}
{}
\ccGlue
\ccFunction{bool is_positive(const NT& ntval);}
{}
\ccGlue
\ccFunction{bool is_zero(const NT& ntval);}
{}
\ccGlue
\ccFunction{bool is_one(const NT& ntval);}
{}
\ccGlue
\ccFunction{Comparison_result compare(const NT &n1, const NT &n2);}
{returns \ccc{GREATER} iff $n1>n2$, \ccc{EQUAL} iff $n1=n2$, and \ccc{SMALLER} iff $n1<n2$.}

Those routines are implemented using the required operations from the
number type concepts. They are defined by means of templates, so you do not
have to supply all those operations when you write  a new number type.
But if you have a better implementation for any of them, you can provide a 
corresponding overloading function with the same name for your number types,
which will get preference over the template functions listed above.

For the number types \ccc{int}, and \ccc{double} there is
the random numbers generator \ccc{Random}.

%\ccInclude{CGAL/basic.h}

%\ccFunction{template <class NT> bool is_even( NT x);}
%    {returns $(i \;\&\; 1) == 0$.}

%\ccFunction{template <class NT> bool is_odd( NT x);}
%    {returns $(i \;\&\; 1) == 1$.}


\subsection{Utility Function Classes}
In addition to the utility routines listed above, 
there are function object class templates corresponding to these functions.
Note that the function object class corresponding to \ccc{sign} is
named \ccc{Sgn} in order to avoid a conflict with the type 
\ccc{Sign}.

\cgal\ provides the following function object classes:

\ccInclude{CGAL/number_utils_classes.h}

\ccc{class Min<NT,Compare=std::less>;}\\
\ccc{class Max<NT,Compare=std::less>;}\\
\ccc{class Abs<NT>;}\\
\ccc{class Sgn<NT>;}\\
\ccc{class Is_negative<NT>;}\\
\ccc{class Is_positive<NT>;}\\
\ccc{class Is_zero<NT>;}\\
\ccc{class Is_one<NT>;}\\
\ccc{class Compare<NT>;} \\
\ccc{class Div<NT>;} \\
\ccc{class Gcd<NT>;} \\
\ccc{class Sqrt<NT>;} \\
\ccc{class Square<NT>;}

\section{Built-in Number Types}

The built-in number types \ccc{float} and \ccc{double} have the
required arithmetic and comparison operators. They lack some required
routines though which are automatically included by \cgal.
\ccTexHtml{\footnote{ The functions can be found in the header files 
\ccc{<CGAL/double.h>} and \ccc{<CGAL/float.h>}.}}{}

All built-in number types of \CC\ can represent a discrete (bounded)
subset of the rational numbers only.  We assume that the
floating-point arithmetic of your machine follows {\sc Ieee}
floating-point standard.  Since the floating-point culture has much
more infrastructural support (hardware, language definition and
compiler) than exact computation, it is very efficient.
Like with all number types with finite precision representation
which are used as approximations to the infinite ranges of 
integers or real numbers, the built-in number types are inherently
potentially inexact.
Be aware of this if you decide to use the efficient built-in 
number types: you have to cope with numerical problems.  
For example, you can compute the intersection point of two lines and 
then check whether this point lies on the two lines. 
%With exact arithmetic, the check will always return \ccc{true}. 
With floating point arithmetic,
roundoff errors may cause the answer of the check to be \ccc{false}. 
With the built-in integer types overflow might occur.

%\newpage
\section{Number Types Provided by \cgal}


\begin{ccRefClass} {Quotient<NT>}
\label{Quotient}
%\subsection{Quotient}

\ccDefinition
An object of the class \ccStyle{Quotient<NT>} is an element of the 
field of quotients of the integral domain type \ccStyle{NT}.
If \ccStyle{NT} behaves like an integer, \ccStyle{Quotient<NT>}
behaves like a rational number. 
{\leda}'s class \ccStyle{rational} (see Section~\ref{leda-nt})
has been the basis for \ccStyle{Quotient<NT>}.
A \ccStyle{Quotient<NT>} \ccStyle{q} is represented as a pair of 
\ccStyle{NT}s, representing numerator and denominator.

\ccc{NT} must be at least model of concept \ccc{IntegralDomainWithoutDivision}.\\
\ccc{NT} must be a model of concept \ccc{RealEmbeddable}. \\


\ccInclude{CGAL/Quotient.h}

\ccIsModel

\ccc{Field}\\
\ccc{RealEmbeddable}\\
\ccc{Fraction}

\ccCreation
\ccCreationVariable{q}

\ccConstructor{Quotient();}
             {introduces an uninitialized variable \ccVar.}

\ccHidden \ccConstructor{Quotient(const Quotient<NT> &q);}
 	    {copy constructor.}
\ccGlue
\ccConstructor{template <class T> Quotient<NT>(const T& t);}
{introduces the quotient \ccStyle{t/1}. NT needs to have a constructor from T.}
\ccGlue
\ccConstructor{template <class T> Quotient<NT>(const Quotient<T>& t);}
{introduces the quotient \ccStyle{NT(t.numerator())/NT(t.denominator())}.
NT needs to have a constructor from T.}
\ccGlue
\ccConstructor{Quotient(const NT& n, const NT& d)}
            {introduces the quotient \ccStyle{n/d}.\\
            \ccPrecond{$d \neq 0$.}         }


\ccOperations

%\ccSetTwoOfThreeColumns{5cm}{4cm}
%SetThreeColumns{std::ostream& }{}{\hspace*{8cm}}

There are two access functions, namely to the
numerator and the denominator of a quotient.
Note that these values are not uniquely defined. 
It is guaranteed that \ccStyle{q.numerator()} and 
\ccStyle{q.denominator()} return values \ccStyle{nt_num} and
\ccStyle{nt_den} such that \ccStyle{q = nt_num/nt_den}, only
if  \ccStyle{q.numerator()} and \ccStyle{q.denominator()} are called
consecutively wrt \ccStyle{q}, i.e.~\ccStyle{q} is not involved in 
any other operation between these calls.

\ccMethod{NT numerator() const;}
       {returns a numerator of \ccStyle{q}.}
\ccGlue
\ccMethod{NT denominator() const;}
       {returns a denominator of \ccStyle{q}.}

\ccHidden \ccMethod{Quotient<NT>& normalize();}
{}

The stream operations are available as well. 
They assume that corresponding stream operators for type \ccc{NT} exist.

\ccFunction{std::ostream& operator<<(std::ostream& out, const Quotient<NT>& q);}
       {writes \ccc{q} to ostream \ccc{out} in format ``{\tt n/d}'', where
       {\tt n}$==$\ccc{q.numerator()} and {\tt d}$==$\ccc{q.denominator()}.}

\ccFunction{std::istream& operator>>(std::istream& in, Quotient<NT>& q);}
       {reads \ccc{q} from istream \ccc{in}. Expected format is
        ``{\tt n/d}'', where {\tt n} and {\tt d} are of type \ccc{NT}.
        A single {\tt n} which is not followed by a {\tt /}\  is also
        accepted and interpreted as {\tt n/1}.}

The following functions are added to fulfill the \cgal\ requirements
on number types.

\ccFunction{double to_double(const Quotient<NT>& q);}
       {returns some double approximation to \ccStyle{q}.}
\ccGlue
\ccFunction{bool  is_valid(const Quotient<NT>& q);}
       {returns true, if numerator and denominator are valid.}
\ccGlue
\ccFunction{bool  is_finite(const Quotient<NT>& q);}
       {returns true, if numerator and denominator are finite.}
\ccGlue
\ccFunction{Quotient<NT>  sqrt(const Quotient<NT>& q);}
       {returns the square root of \ccc{q}.  This is supported if and only if
        \ccc{NT} supports the square root as well.}

\end{ccRefClass} 


% $Id$
% $Date$
% author : Sylvain Pion

\begin{ccRefClass} {MP_Float}
%\subsection{Multi Precision Floats}
%\label{mpfloat}

\ccDefinition
An object of the class \ccc{MP_Float} is able to represent a floating point
value with arbitrary precision.  This number type has the property that
additions, subtractions and multiplications are computed exactly, as well as
the construction from \ccc{float}, \ccc{double} and \ccc{long double}.

Division and square root are not enabled by default since \cgal\ release 3.2,
since they are computed approximately.  We suggest that you use
rationals like \ccc{Quotient<MP_Float>} when you need exact divisions.

Note on the implementation : although the mantissa length is basically only
limited by the available memory, the exponent is currently represented by a
(integral valued) \ccc{double}, which can overflow in some circumstances.  We
plan to also have a multiprecision exponent to fix this issue.

\ccInclude{CGAL/MP_Float.h}

\ccIsModel
\ccc{EuclideanRing}.\\
\\
\ccc{RealEmbeddable}

\ccCreation
\ccCreationVariable{m}

\ccConstructor{MP_Float();}
{introduces an uninitialized variable \ccVar.}
\ccGlue
\ccConstructor{MP_Float(const MP_Float &);}
{copy constructor.}
\ccGlue
\ccConstructor{MP_Float(int i)}
{introduces the integral value i.}
\ccGlue
\ccConstructor{MP_Float(float d)}
{introduces the floating point value d (exact conversion).}
\ccGlue
\ccConstructor{MP_Float(double d)}
{introduces the floating point value d (exact conversion).}
\ccGlue
\ccConstructor{MP_Float(long double d)}
{introduces the floating point value d (exact conversion).}

\ccOperations

\ccFunction{std::ostream& operator<<(std::ostream& out, const MP_Float& m);}
{writes a double approximation of \ccc{m} to the ostream \ccc{out}.}

\ccFunction{std::istream& operator>>(std::istream& in, MP_Float& m);}
{reads a \ccc{double} from \ccc{in}, then converts it to an \ccc{MP_Float}.}

\ccFunction{MP_Float approximate_division(const MP_Float &a, const MP_Float &b);}
{computes an approximation of the division by converting the operands to
\ccc{double}, performing the division on \ccc{double}, and converting back to
\ccc{MP_Float}.}

\ccFunction{MP_Float approximate_sqrt(const MP_Float &a);}
{computes an approximation of the square root by converting the operand to
\ccc{double}, performing the square root on \ccc{double}, and converting back
to \ccc{MP_Float}.}


\ccImplementation 
The implementation of \ccc{MP_Float} is simple but provides a quadratic
complexity for multiplications.  This can be a problem for large operands.
For faster implementations of the same functionality with large integral
values, you may want to consider using \ccc{GMP} or \ccc{LEDA} instead.

\end{ccRefClass} 


% $Id$
% $Date$
% author : Sylvain Pion

\begin{ccRefClass} {Lazy_exact_nt<NT>}
%\subsection{Lazy wrapper for exact number types\label{lazy_exact_nt}}

\ccDefinition
An object of the class \ccc{Lazy_exact_nt<NT>} is able to represent any 
real embeddable number which \ccc{NT} is able to represent.
The idea is that \ccc{Lazy_exact_nt<NT>} works exactly like \ccc{NT}, except
that it is expected to be faster because it tries to only compute an 
approximation of the value, and only refers to \ccc{NT} when needed.  
The goal is to speed up exact computations done by any exact but slow 
number type \ccc{NT}.\\

\ccc{NT} must be a model of concept \ccc{RealEmbeddable}. \\
\ccc{NT} must be at least model of concept \ccc{IntegralDomainWithoutDivision}.\\


Note that some filtering mechanism is available at the predicate level
using \ccc{Filtered_predicate} and \ccc{Filtered_kernel}.

\ccInclude{CGAL/Lazy_exact_nt.h}

\ccIsModel
\begin{tabular}{ll}
\ccc{IntegralDomainWithoutDivision} & same as \ccc{NT}\\
\ccc{RealEmbeddable}&\\
\ccc{Fraction} & if \ccc{NT} is a \ccc{Fraction}\\
\end{tabular}       
        

\ccCreation
\ccCreationVariable{m}

\ccConstructor{Lazy_exact_nt();}
{introduces an uninitialized variable \ccVar.}
\ccGlue
%\ccConstructor{Lazy_exact_nt(const Lazy_exact_nt &);}
%{copy constructor.}
%\ccGlue
\ccConstructor{Lazy_exact_nt(int i)}
{introduces the integral value \ccc{i}.}
\ccGlue
\ccConstructor{Lazy_exact_nt(double d)}
{introduces the floating point value \ccc{d} (works only if \ccc{NT} has a
constructor from a double too).}
\ccGlue
\ccConstructor{Lazy_exact_nt(NT n)}
{introduces the value \ccc{n}.}
\ccGlue
\ccConstructor{template <class NT1> Lazy_exact_nt(Lazy_exact_nt<NT1> n)}
{introduces the value \ccc{n}. \ccc{NT1} needs to be convertible to \ccc{NT}
(and this conversion will only be done if necessary).}

\ccOperations

\ccMethod{NT exact();} {returns the corresponding NT value.}

\ccMethod{Interval_nt<true> approx();} {returns an interval containing the
exact value.}

\ccMethod{Interval_nt<false> interval();} {returns an interval containing the 
exact value.}

\ccMethod{static void set_relative_precision_of_to_double(double d);}
{specifies the relative precision that \ccc{to_double()} has to fulfill.
The default value is $10^{-5}$.  \ccPrecond{d>0 and d<1.}}

\ccMethod{static double get_relative_precision_of_to_double();}
{returns the relative precision that \ccc{to_double()} currently fulfills.}

\ccFunction{std::ostream& operator<<(std::ostream& out,
                                     const Lazy_exact_nt<NT>& m);}
{writes \ccc{m} to ostream \ccc{out} in an interval format.}

\ccFunction{std::istream& operator>>(std::istream& in, Lazy_exact_nt<NT>& m);}
{reads a \ccc{NT} from \ccc{in}, then converts it to a \ccc{Lazy_exact_nt<NT>}.}

\ccExample

\begin{verbatim}
#include <CGAL/Cartesian.h>
#include <CGAL/MP_Float.h>
#include <CGAL/Lazy_exact_nt.h>
#include <CGAL/Quotient.h>

typedef CGAL::Lazy_exact_nt<CGAL::Quotient<CGAL::MP_Float> > NT;
typedef CGAL::Cartesian<NT> K;
\end{verbatim}

\end{ccRefClass} 



% author : Olivier Devillers <Olivier.Devillers@sophia.inria.fr>

%\marginparwidth 1.5cm
%\def\ccTagChapterRelease{\ccTrue}
%\def\ccTagChapterAuthor{\ccTrue}
%
%\chapter{Fixed precision numbers} \label{I1_Chapter_Fixed_precision_nt}
%
%\ccChapterRelease{Revision: , Date: }
%
%\ccChapterAuthor{Olivier Devillers}
%
%\section{Introduction} 

\begin{ccClass}{Fixed_precision_nt}

\ccSubsection{Fixed precision numbers}
\label{I1_Chapter_Fixed_precision_nt}

The class \ccStyle{Fixed_precision_nt} provides 24 bits numbers in 
fixed point representation.
Basically these numbers are integers in the range
$[-2^{24},2^{24}]$ with a multiplying factor $2^b$.
The multiplying factor $2^b$ has to be initialized by the user
before the construction of the first \ccStyle{Fixed_precision_nt}
and is common to all variables.

The interest of such a number type is that geometric predicates
can be overloaded to get exact and very efficient predicates.
The drawback is that any \ccStyle{Fixed_precision_nt} is rounded to 
the nearest multiple of $2^b$, which yields to a very poor arithmetic.
The idea is to not use the arithmetic on \ccStyle{Fixed_precision_nt}
but only the specialized predicates.

Note: you must call \ccc{CGAL::force_ieee_double_precision()} in order for the
\ccStyle{Fixed_precision_nt} to work properly on Intel platforms.  This
initializes the FPU to an IEEE compliant rounding mode which is not the
default.

\ccSetTwoColumns{}{\hspace*{8.5cm}}

\ccCreation
\ccCreationVariable{fvar}

\ccInclude{CGAL/Fixed_precision_nt.h}

\ccConstructor{Fixed_precision_nt();}
            {Declaration.}

\ccConstructor{Fixed_precision_nt(double d);}
            {Initialization of a variable. The variable is rounded to the
                nearest legal fixed number (i.e. a multiple of $2^b=$
                \ccStyle{Fixed_precision_nt::unit_value()}}


\ccConstructor{Fixed_precision_nt(const Fixed_precision_nt &fval);}
            {Declaration and initialization.}


\ccConstructor{Fixed_precision_nt(int i)}
            {Declaration and initialization with an integer.}

\ccSetThreeColumns{XXXXXX}{}{\hspace*{8.5cm}}
\ccOperations


\ccMethod{Fixed_precision_nt & operator=(const Fixed_precision_nt &fval);}
        {Assignment. 
}

\ccFunction{bool is_valid(const Fixed_precision_nt &fval);}
{In case of overflow or division by 0, numbers becomes invalid.
If the precision is changed by usage of 
\ccStyle{Fixed_precision_nt::init()}, already existing numbers may become invalid
if they are no longer multiple of $2^b$.}

\ccFunction{bool is_finite(const Fixed_precision_nt &fval);}
{ \ccClassName\ do not implement infinite numbers. 
  \ccStyle{is_finite} is identical to \ccStyle{is_valid}.}

The comparison operations $==$, $!=$, $<$, $>$, $<=$, and $>=$ are all
available.

The arithmetic operators $+$, $-$, $*$, $/$, $+=$, $-=$, $*=$ and $/=$
are all available. The result of the computation is rounded to the
nearest legal \ccClassName. Overflow is possible, and even probable in case
of multiplication or division. \ccClassName\ are designed to use
specialized predicates, not to use arithmetic.


\ccFunction{double to_double(const Fixed_precision_nt &fval);}
         {casts to \ccStyle{double}.}


%\section{Parametrization routines}
%\ccHeading{Parameterization routines}

\ccHeading{Precision initialization}
As mentioned before, the \ccClassName\ numbers takes their values in an
interval $[-2^{24+b},2^{24+b}]$ of multiples of $2^b$,
this number $b$ as to be defined before any
 use of \ccStyle{Fixed_precision_nt}.

\ccSetThreeColumns{Comparison_result}{}{\hspace*{6cm}}

\ccFunction{static bool init(float B);}
{$B$ is an upper bound on the data, $b$ is the smallest integer such that 
$|B|\leq 2^b$. The result of the function is false if initialization was
already done, in that case already existing \ccClassName\ may become invalid.}

\ccFunction{static float unit_value();}
{returns $2^b$.}

\ccFunction{static float upper_bound();}
{returns $2^{24+b}$.}

%\subsection{Perturbation scheme}
\ccHeading{Perturbation scheme}

\ccClassName\ implements perturbation scheme as described by
Alliez, Devillers and Snoeyink \cite{ads-rdppw-98}.
The perturbation mode can be activated or deactivated for different kinds
of perturbations. The default mode is no perturbation.

\ccFunction{static void perturb_incircle();}
{Activate. \ccStyle{side_of_oriented_circle} predicate
 of 4 cocircular points answers degenerate only if
the 4 points are colinear.}

\ccFunction{static void unperturb_incircle();}
{Deactivate}

\ccFunction{static bool is_perturbed_incircle();}
{returns current mode}

\ccFunction{static void perturb_insphere();}
{Activate. \ccStyle{side_of_oriented_sphere} predicate
 of 5 cospherical points answers degenerate only if
the 5 points are coplanar.}

\ccFunction{static void unperturb_insphere();}
{Deactivate}

\ccFunction{static bool is_perturbed_insphere();}
{returns current mode}

%\section{Geometric predicates}
\ccHeading{Geometric predicates}

Through overloading mechanisms, functions such that
\ccStyle{orientation} for 
\ccStyle{Point_2<Cartesian< Fixed_precision_nt> >}
will correctly call the function below.

\ccSetThreeColumns{Oriented_side}{}{\hspace*{6cm}}

%\subsection{Two dimensional}

\ccFunction{Orientation orientationC2(
  Fixed_precision_nt x0, Fixed_precision_nt y0,
  Fixed_precision_nt x1, Fixed_precision_nt y1,
  Fixed_precision_nt x2, Fixed_precision_nt y2);}{}
\ccGlue
\ccFunction{Oriented_side side_of_oriented_circleC2 (
  Fixed_precision_nt x0, Fixed_precision_nt y0,
  Fixed_precision_nt x1, Fixed_precision_nt y1,
  Fixed_precision_nt x2, Fixed_precision_nt y2,
  Fixed_precision_nt x3, Fixed_precision_nt y3);}{Perturbation mode can be activated.}

%\subsection{Three dimensional}

\ccFunction{Orientation orientationC3(  
  Fixed_precision_nt x0, Fixed_precision_nt y0, Fixed_precision_nt z0,
  Fixed_precision_nt x1, Fixed_precision_nt y1, Fixed_precision_nt z1,
  Fixed_precision_nt x2, Fixed_precision_nt y2, Fixed_precision_nt z2,
  Fixed_precision_nt x3, Fixed_precision_nt y3, Fixed_precision_nt z3);}{}
%
%\ccSetThreeColumns{x}{}{\hspace*{3.5cm}}
\ccGlue
\ccFunction{Oriented_side side_of_oriented_sphereC3 (
  Fixed_precision_nt x0, Fixed_precision_nt y0, Fixed_precision_nt z0,
  Fixed_precision_nt x1, Fixed_precision_nt y1, Fixed_precision_nt z1,
  Fixed_precision_nt x2, Fixed_precision_nt y2, Fixed_precision_nt z2,
  Fixed_precision_nt x3, Fixed_precision_nt y3, Fixed_precision_nt z3,
  Fixed_precision_nt x4, Fixed_precision_nt y4, Fixed_precision_nt z4);}
{Perturbation mode can be activated.}

\end{ccClass} 


% +------------------------------------------------------------------------+
% | Reference manual page: Interval.tex
% +------------------------------------------------------------------------+
% | 11.04.2000   Author
% | Package: Package
% | 
\RCSdef{\RCSIntervalRev}{$Id$}
\RCSdefDate{\RCSIntervalDate}{$Date$}
% |
%%RefPage: end of header, begin of main body
% +------------------------------------------------------------------------+


\begin{ccRefConcept}{Interval}

%% \ccHtmlCrossLink{}     %% add further rules for cross referencing links
%% \ccHtmlIndexC[concept]{} %% add further index entries

\ccDefinition
  
The concept \ccRefName\ describes the requirements for the
template argument \ccc{Interval} of a \ccc{Interval_skip_list<Interval>}.

The concept does not specify, whether the interval is open or
closed. It is up to the implementer of a model for this concept
to define that.


\ccCreation
\ccCreationVariable{in}  %% choose variable name

\ccConstructor{Interval();}{default constructor.}
\ccTypes

\ccNestedType{Value}{The type of the lower and upper bound of the interval.}


\ccAccessFunctions
\ccMethod{Value inf() const;}
{returns  the lower bound.}
\ccGlue
\ccMethod{Value sup() const;}{returns the upper bound.}


\ccMethod{bool contains(const Value& v) const;}
{returns \ccc{true}, iff \ccc{in} contains \ccc{v}.}

\ccMethod{bool contains_interval(const Value& i, const Value& s) const;}
{returns \ccc{true}, iff \ccc{in} contains \ccc{(i,s)}.}

\ccMethod{bool operator==(const Interval& I) const;}
{Equality test.}

\ccMethod{bool operator!=(const Interval& I) const;}
{Unequality test.}

\ccHasModels

\ccc{CGAL::Interval_skip_list_interval<Value>}\\
\ccc{CGAL::Face_interval}


\ccSeeAlso

\ccc{Interval_skip_list} 


\end{ccRefConcept}

% +------------------------------------------------------------------------+
%%RefPage: end of main body, begin of footer
% EOF
% +------------------------------------------------------------------------+



\section{Number Types Provided by {\sc Cln}}
\label{CLN}

% \ccChapterAuthor{Sylvain Pion}

\cgal\ defines the functions needed to use the number types \ccc{cl_integer}
and \ccc{cl_rational} provided by {\sc Cln}, the Class Library for
Numbers~\cite{cln}.  To use {\sc Cln} with \cgal, just install \cgal\ with
{\sc Cln} support, and include the corresponding {\sc Cln} file but with the
prefix \ccc{CGAL/CLN/}.

\ccInclude{CGAL/CLN/cl_integer.h}

The class \ccc{cl_I}~: exact multiprecision integers.

\ccInclude{CGAL/CLN/cl_rational.h}

The class \ccc{cl_RA}~: exact multiprecision rationals.

See the {\sc Cln} documentation for additionnal details.


\section{Number Types Provided by \leda}
\label{leda-nt}

\leda\ provides number types that can be used for exact computation 
with both Cartesian and homogeneous representations.  If you are using
homogeneous representation with the built-in integer types
\ccc{short}, \ccc{int}, and \ccc{long} as ring type, exactness of
computations can be guaranteed only if your input data come from a
sufficiently small integral range and the depth of the computations is
sufficiently small.  \leda\ provides the number type \ccc{leda_integer} for
integers of arbitrary length. (Of course the length is
somehow bounded by the resources of your computer.)  It can be used as
ring type in homogeneous kernels and leads to exact
computation as long as all intermediate results are rational.  For the
same kind of problems, Cartesian representation with number type
\ccc{leda_rational} leads to exact computation as well.
The number type \ccc{leda_bigfloat} in \leda\ is a variable precision
floating-point type. Rounding mode and precision (i.e.\ mantissa length) of
\ccc{leda_bigfloat} can be set. 

The most sophisticated number type in \leda\ is the number type called
\ccc{leda_real}. Like in Pascal, where the name \ccc{real} is used for
floating-point numbers, the name \ccc{leda_real} does not describe the
number type precisely, but intentionally.  
\ccc{leda_reals} are a subset of real algebraic
numbers.  Any integer is \ccc{leda_real} and \ccc{leda_real}s are closed under
the operations $+,-,*,/$ and $k$-th root computation. 
%\ccTexHtml{$\sqrt[k]{\ }$}{k-th root computation}. 
\ccc{leda_real}s guarantee that
all comparisons between expressions involving \ccc{leda_real}s produce the
exact result.

In the include files \ccc{<CGAL/leda_integer.h>}, \ccc{<CGAL/leda_rational.h>}, 
\ccc{<CGAL/leda_bigfloat.h>}, and \ccc{<CGAL/leda_real.h>}, 
the \leda\ types \ccc{leda_integer}, \ccc{leda_rational},
\ccc{leda_bigfloat}, 
and \ccc{leda_real} are made conform to the requirements presented in
Section \ref{nt-requirements}. 
Also, in these files the \leda\ number types are included.
For more details on the number types of \leda\ we refer to the \leda\ 
manual~\cite{mnsu-lum}.

\section{Number Types Provided by GNU}

\cgal\ provides wrapper classes for number types defined in the 
{\sc Gnu} Multiple Precision Arithmetic Library~\cite{g-gmpal-96}.
The file {\tt  CGAL/Gmpz.h} provides the class \ccStyle{Gmpz}, 
a wrapper class for the integer type \ccc{mpz_t}, that is compliant to the 
CGAL number type requirements.
To use this, the {\sc Gnu} Multiple Precision Arithmetic Library must be 
installed.

\begin{ccClass} {Gmpz}
\label{Gmpz}
\ccSubsection{GNU Multiple Precision Integer}

\ccDefinition

An object of the class \ccStyle{Gmpz} is an arbitrary precision integer 
based on the {\sc Gnu} Multiple Precision Arithmetic Library. 

\ccInclude{CGAL/Gmpz.h}

\ccCreation
\ccCreationVariable{q}

\ccConstructor{Gmpz();}
             {creates an uninitialized multiple precision integer \ccVar.}

\ccHidden \ccConstructor{Gmpz(const Gmpz& q);}
 	    {copy constructor.}

\ccConstructor{Gmpz(int i)}
            {creates an multiple precision integer initialized with
             \ccStyle{i}.}

\ccConstructor{Gmpz(double d)}
            {creates an multiple precision integer initialized with
             the integral part of \ccStyle{d}.}

\ccOperations

The arithmetic operations $+,\ -,\ *,\ /,\ +=,\ -=,\ *=,\ /=,\ -$(unary),
the comparison operations $<,\ <=,\ >,\  >=,\ ==,\ !=$ and the stream 
operations are all available.

\ccSetThreeColumns{io_Operator }{}{\hspace*{8cm}}

The following functions are added to fulfill the {\cgal} requirements
on number types.

\ccFunction{double to_double(const Gmpz& g);}
       {returns some double approximation to \ccStyle{g}.}

\ccFunction{bool  is_valid(const Gmpz& g);}
       {returns true}

\ccFunction{bool  is_finite(const Gmpz& g);}
       {returns true}

\ccFunction{io_Operator  io_tag(const Gmpz& g);}
       {}

\ccImplementation
\ccc{Gmpz}s are reference counted.
\end{ccClass} 




\section{User-supplied Number Types}

You can also use your own number type with the \cgal\ kernel
classes, e.g.\  the {\sc BigNum} package \cite{svh-bpepa-89}.
Depending on the arithmetic operations carried out by the algorithms
that you are going to use, the number types must fulfill the
corresponding requirements from Section \ref{nt-requirements}. 

\begin{ccRefClass}{Number_type_traits<NT>}

\ccDefinition

The class \ccRefName\ can be used to determine if a certain number type
supports certain operations and thus to determine which number type concept
it is a model of.  

\ccInclude{CGAL/Number_type_traits.h}

\ccTypes

The following three types are each set to either \ccc{CGAL::Tag_true} or
\ccc{CGAL::Tag_false} depending on whether the indicated operation is
supported or not by the number type \ccc{NT}.

\ccNestedType{Has_gcd}{indicates if the number type \ccc{NT} supports the
                       \ccc{gcd} operation or not.}
\ccNestedType{Has_division}{indicates if the number type \ccc{NT} has 
                            \ccc{operator/} defined or not.}
\ccNestedType{Has_sqrt}{indicates if the number type \ccc{NT} supports the
                        operatio \ccc{sqrt} or not.} 

\end{ccRefClass}

