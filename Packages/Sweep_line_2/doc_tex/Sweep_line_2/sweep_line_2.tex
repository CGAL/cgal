\section{Sweep Line}

% Describe the functionality of the package
A \ccc{Planar Map with Intersections} can be built incrementally by
inserting one curve after the other into the map. 
Given a collection $C$ of possibly intersecting and not necessarily 
$x$-monotone curves in the plane, the sweep line algorithm can construct 
the planar map induced by $C$, or rather produce the collections of their 
intersections and the disjoint-interior subcurves induced by $C$.
 
In general, it is much faster to perform the sweep line algorithm on
the collection of input curves in order to produce their planar map 
rather using the \ccc{Planar_map_with_intersections_2<Planar_map>} 
or \ccc{Arrangement_2<Dcel,Traits,Base_node>} packages, due to an 
efficient calculation of intersections and avoiding the usage of 
point location when inserting a curve to the planar map. 
The utility can either build the induced planar
map, collect the pairwise interior disjoint subcurves
computed in a container, or rather collect the intersections of all 
curves to a container of points. 

%If no additional insertions of intersecting curves are planned following the building of the map it is possible to perform a sweep
%line operation that will build the simpler class \ccc{Planar_map_2}.
%This is possible since the output of the sweep line operation is a
%collection of $x$-monotone pairwise interior disjoint, which are
%supported by the \ccc{Planar Map} package. 
When constructing a planar map by the sweep line process the user 
provides (possibly an empty) planar map as the resulting parameter. 
The result of applying the sweep line algorithm to a 
collection of curves and a non-empty map is equal to that 
of applying the algorithm to the union of the
planar map curves and the collection of input curves. Simply put,
the intersections of input curves and planar map curves are also
calculated.

% Put examples here
\section{Example using Sweep Line to construct a Planar map}
\label{ssec:example1_sweep}
The following example demonstrates the usage of the 
\ccc {Sweep line} algorithm. 
It constructs a planar map out of four segments
--- $(0,0)-(1,1)$ , $(0,1)-(1,0)$ , $(0,0)-(1,0)$ and $(0,1)-(1,1)$
(an hourglass shape), two of them are intersecting in their
interior. The resulting planar map will contain all the disjoint
interior sub-segments obtained by the calculation of the sweep
line algorithm. For clarity, we printed all the halfedges of the
resulting planar map to the standard output using the I/O functions
for the \ccc{Planar map} package.

\ccIncludeExampleCode{Sweep_line/example1.C}

The output of the program looks like this:

\ccIncludeExampleCode{Sweep_line/example1.cout}

% EOF ------------------------------------------------------------------------80


