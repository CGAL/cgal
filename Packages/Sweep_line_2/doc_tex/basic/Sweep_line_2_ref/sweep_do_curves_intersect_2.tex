% +------------------------------------------------------------------------+
% | Reference manual page: sweep_do_curves_intersect_2.tex
% +------------------------------------------------------------------------+
% | 17.02.2002   Author
% | Package: Sweep_line_2
% | 
\RCSdef{\RCSsweeptoproducepointsRev}{$Revision$}
\RCSdefDate{\RCSsweeptoproducepointsDate}{$Date$}
% |
%%RefPage: end of header, begin of main body
% +------------------------------------------------------------------------+


\begin{ccRefFunction}{sweep_do_curves_intersect_2} %% add template arg's if necessary

%% \ccHtmlCrossLink{}     %% add further rules for cross referencing links
%% \ccHtmlIndexC[function]{} %% add further index entries

\ccDefinition
  
The function \ccRefName\ checks whether a given set of input curves 
contain intersections. An intersection is considered only when 
two curves intersect in their interior. 

\ccInclude{sweep_do_curves_intersect_2.h}

\ccFunction{
   template<class InputIterator, class SweepLineTraits_2>
   bool  
   sweep_do_curves_intersect_2(InputIterator curves_begin, 
                               InputIteratorcurves_end,  
                               SweepLineTraits_2& traits);}
   {Checks whether the curves given in the range 
   \ccStyle{[curves_begin, curves_end)} induce any intersections.
   \ccc{sweep_do_curves_intersect_2} supports
   curves which are interior intersect, overlap or non x-monotone, and
   hence users may provide their input curves without any
   restriction on the kind mentioned above. 
   Notice that if the input curves are not $x$-monotone, their splitting points 
   to $x$-monotone curves are treated as new endpoints points, and hence 
   \ccc{sweep_do_curves_intersect_2} does not consider them as intersection points.}

\ccHeading{Requirements}
\begin{enumerate}
   \item    \ccc{InputIterator::value_type} is equivalent to \ccc{Traits::Curve_2}.
   \item    \ccc{SweepLineTraits_2} is a model of the \ccc{SweepLineTraits_2} concept
\end{enumerate}


\ccSeeAlso
\ccc{CGAL::sweep_to_construct_planar_map_2} \\
\ccc{CGAL::sweep_to_produce_subcurves_2} \\
\ccc{CGAL::sweep_to_produce_points_2} \\
\ccc{CGAL::sweep_to_report_intersecting_curves_2}

% \ccImplementation
% The implementation uses the container $map$ defined in STL for
% implementing the event queue and the status line.  The implementation
% also keeps per each input curve an associated container of all its
% intersection points ordered from left to right, and also keeps per
% each event point all its outcoming curves.  The complexity of this
% algorithm is $O(n\log{n} + k\log{n})$ where $n$ is the number of the 
% input curves and $k$ is the number of intersection points 
% induced by these curves.

\ccIncludeExampleCode{Sweep_line_2/example6.C}

\end{ccRefFunction}

% +------------------------------------------------------------------------+
%%RefPage: end of main body, begin of footer
% EOF
% +------------------------------------------------------------------------+









