\section{Introduction}

% Put here the definition of the algorithm (theory)
Let $C:=\{c_1, \ldots c_n\}$ be the set of curves for which we want 
to compute all intersections. We want to avoid testing pairs of segments 
that are far apart. To find the intersecting pairs we imagine sweeping  
a line $l$ from left to right over the plane, starting from a position 
left to all curves. While we sweep the imaginary line, we keep track of all 
curves intersecting it.

This type of algorithm is called a {\em plane sweep algorithm} and the line 
$l$ is called the {\em sweep line}. The {\em status} of the sweep line is 
the set of curves intersecting it. The status changes while the sweep line 
moves to right, but not continuously. The sweep line status is updated at 
specific points called the {\em event points}. These points are actually
the endpoints of all curves and their intersection points. The initial set of 
{\em event points} are only the endpoints of the curves. More {\em event points}
are added as intersection points are calculated.

% Put here a description of what's in this chapter

This chapter describes the functions provided in
\cgal\ for producing the corresponding disjoint-interior 
subcurves and intersection points of a given collection 
$C:=\{c_1, \ldots c_n\}$ of planar curves.

We take advantage of the sweep line algorithm implemented here to provide
an efficient construction of a Planar Map with Intersection. See 
\ccc{CGAL::Pm_with_intersection} for more deatils.
