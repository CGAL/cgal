%% Copyright (c) 2005  Foundation for Research and Technology-Hellas (Greece).
%% All rights reserved.
%%
%% This file is part of CGAL (www.cgal.org); you may redistribute it under
%% the terms of the Q Public License version 1.0.
%% See the file LICENSE.QPL distributed with CGAL.
%%
%% Licensees holding a valid commercial license may use this file in
%% accordance with the commercial license agreement provided with the software.
%%
%% This file is provided AS IS with NO WARRANTY OF ANY KIND, INCLUDING THE
%% WARRANTY OF DESIGN, MERCHANTABILITY AND FITNESS FOR A PARTICULAR PURPOSE.
%%
%% $Source$
%% $Revision$ $Date$
%% $Name$
%%
%% Author(s)     : Menelaos Karavelas <mkaravel@tem.uoc.gr>


\begin{ccRefFunctionObjectConcept}{SiteInserter}

The functor \ccc{SiteInserter} is responsible for inserting a site in
the Delaunay graph. The result type of the functor is a vertex handle
of the Delaunay graph which either points to the vertex of the
Delaunay graph corresponding to the site just inserted or the
default constructed vertex handle. The latter case can happen if the
site inserted is \textit{hidden}, i.e., it has an empty Voronoi cell.

\ccRefines
\ccc{DefaultConstructible}, \ccc{CopyConstructible}, \ccc{Assignable},
\ccc{AdaptableFunctor} (with two arguments)

\ccTypes

In addition to the types required by the \ccc{AdaptableFunctor}
concept, a model of this concept must provide the following types:

\ccTwo{SiteInserter::Delaunay_graph+}{}
\ccNestedType{Delaunay_handle}{A type for the Delaunay graph.}
\ccGlue
\ccNestedType{Site_2}{A type for the sites of the Delaunay graph.}

\ccCreationVariable{si}

\ccOperations
%It must also provide the following operations:

\ccThree{result_type}{si(Delaunay_graph& dg, Site_2 t)+}{}
\ccThreeToTwo
\ccMemberFunction{result_type operator()(Delaunay_graph& dg, Site_2 t);}
{Inserts the site \ccc{t} in the Delaunay graph \ccc{dg} and returns a
  vertex handle. The vertex handle either points to the vertex
  corresponding to the site \ccc{t} or is the default constructed
  vertex handle. The latter case happens if the Voronoi cell of
  \ccc{t} is empty.}
%

\ccHasModels
\ccc{CGAL::Apollonius_graph_Voronoi_traits_2<AG2>::Site_inserter}\\
\ccc{CGAL::Apollonius_graph_caching_Voronoi_traits_2<AG2>::Site_inserter}\\
\ccc{CGAL::Apollonius_graph_identity_Voronoi_traits_2<AG2>::Site_inserter}\\
\ccc{CGAL::Delaunay_triangulation_Voronoi_traits_2<DT2>::Site_inserter}\\
\ccc{CGAL::Delaunay_triangulation_caching_Voronoi_traits_2<DT2>::Site_inserter}\\
\ccc{CGAL::Delaunay_triangulation_identity_Voronoi_traits_2<DT2>::Site_inserter}\\
\ccc{CGAL::Regular_triangulation_Voronoi_traits_2<RT2>::Site_inserter}\\
\ccc{CGAL::Regular_triangulation_caching_Voronoi_traits_2<RT2>::Site_inserter}\\
\ccc{CGAL::Regular_triangulation_identity_Voronoi_traits_2<RT2>::Site_inserter}\\
\ccc{CGAL::Segment_Voronoi_diagram_Voronoi_traits_2<SVD2>::Site_inserter}\\
\ccc{CGAL::Segment_Voronoi_diagram_identity_Voronoi_traits_2<SVD2>::Site_inserter}


\ccSeeAlso
\ccc{VoronoiTraits_2}\\
\ccc{CGAL::Apollonius_graph_Voronoi_traits_2<AG2>}\\
\ccc{CGAL::Apollonius_graph_caching_Voronoi_traits_2<AG2>}\\
\ccc{CGAL::Apollonius_graph_identity_Voronoi_traits_2<AG2>}\\
\ccc{CGAL::Delaunay_triangulation_Voronoi_traits_2<DT2>}\\
\ccc{CGAL::Delaunay_triangulation_caching_Voronoi_traits_2<DT2>}\\
\ccc{CGAL::Delaunay_triangulation_identity_Voronoi_traits_2<DT2>}\\
\ccc{CGAL::Regular_triangulation_Voronoi_traits_2<RT2>}\\
\ccc{CGAL::Regular_triangulation_caching_Voronoi_traits_2<RT2>}\\
\ccc{CGAL::Regular_triangulation_identity_Voronoi_traits_2<RT2>}\\
\ccc{CGAL::Segment_Voronoi_diagram_Voronoi_traits_2<SVD2>}\\
\ccc{CGAL::Segment_Voronoi_diagram_identity_Voronoi_traits_2<SVD2>}

\end{ccRefFunctionObjectConcept}
