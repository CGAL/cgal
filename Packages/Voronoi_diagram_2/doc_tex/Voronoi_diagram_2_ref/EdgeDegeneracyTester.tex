%% Copyright (c) 2005  Foundation for Research and Technology-Hellas (Greece).
%% All rights reserved.
%%
%% This file is part of CGAL (www.cgal.org); you may redistribute it under
%% the terms of the Q Public License version 1.0.
%% See the file LICENSE.QPL distributed with CGAL.
%%
%% Licensees holding a valid commercial license may use this file in
%% accordance with the commercial license agreement provided with the software.
%%
%% This file is provided AS IS with NO WARRANTY OF ANY KIND, INCLUDING THE
%% WARRANTY OF DESIGN, MERCHANTABILITY AND FITNESS FOR A PARTICULAR PURPOSE.
%%
%% $Source$
%% $Revision$ $Date$
%% $Name$
%%
%% Author(s)     : Menelaos Karavelas <mkaravel@tem.uoc.gr>

\begin{ccRefFunctionObjectConcept}{EdgeDegeneracyTester}

\ccRefines
\ccc{DefaultConstructible}\\
\ccc{CopyConstructible}\\
\ccc{Assignable}\\
\ccc{AdaptableFunctor} (with two arguments)

\ccTypes

In addition to the types required by the \ccc{AdaptableFunctor}
concept, a model of this concept must provide the following types:


%\ccTwo{EdgeDegeneractTester::Dual_graph+}{}
\ccThree{typedef typename Dual_graph::Finite_edges_iterator}
{Finite_edges_iterator}{}
\ccThreeToTwo
\ccNestedType{Dual_graph}{A type for the Delaunay graph. It must be
  model of the \ccc{DelaunayGraph_2} concept.}
\ccGlue
\ccTypedef{typedef typename Dual_graph::Edge Edge;}{}
\ccGlue
\ccTypedef{typedef typename Dual_graph::Face_handle Face_handle;}{}
\ccGlue
\ccTypedef{typedef typename Dual_graph::Edge_circulator
  Edge_circulator;}{}
\ccGlue
\ccTypedef{typedef typename Dual_graph::All_edge_iterator
  All_edges_iterator;}{}
\ccGlue
\ccTypedef{typedef typename Dual_graph::Finite_edges_iterator
  Finite_edges_iterator;}{}

\ccCreationVariable{edt}

%\ccCreation

%In addition to the default and copy constructors and the assignment
%operator, the following constructor must be defined:

%\ccThree{EdgeDegeneracyTester et(Dual_graph* dg)+}{}
%\ccThreeToTwo
%\ccConstructor{EdgeDegeneracyTester(Dual_graph* dg);}{A constructor
%  that is initialized by a pointer to a \ccc{Dual_graph} object.}

\ccOperations
It must also provide the following operations:

\ccThree{bool}{fdt(Dual_graph dg, Finite_edges_iterator feit)+}{}
\ccThreeToTwo
\ccMemberFunction{bool operator()(Dual_graph dg, Edge e);}
{returns \ccStyle{true}, if the dual Voronoi edge of \ccc{e} (that
  belongs to the dual Delaunay graph \ccc{dg}) is
  degenerate, i.e., it has zero length.}
%
\ccMemberFunction{bool operator()(Dual_graph dg, Face_hanfle f, int i);}
{returns \ccStyle{true}, if the dual Voronoi edge of the edge
  \ccc{(f,i)} (that belongs to the dual Delaunay graph \ccc{dg}) is
  degenerate, i.e., it has zero length.}
%
\ccGlue
\ccMemberFunction{bool operator()(Dual_graph dg, Edge_circulator ec);}
{returns \ccStyle{true}, if the dual Voronoi edge of \ccc{*ec} (that
  belongs to the dual Delaunay graph \ccc{dg}) is
  degenerate, i.e., it has zero length.}
%
\ccGlue
\ccMemberFunction{bool operator()(Dual_graph dg, All_edges_iterator aeit);}
{returns \ccStyle{true}, if the dual Voronoi edge of \ccc{*aeit} (that
  belongs to the dual Delaunay graph \ccc{dg}) is
  degenerate, i.e., it has zero length.}
%
\ccGlue
\ccMemberFunction{bool operator()(Dual_graph dg, Finite_edges_iterator feit);}
{returns \ccStyle{true}, if the dual Voronoi edge of \ccc{*feit}  (that
  belongs to the dual Delaunay graph \ccc{dg}) is
  degenerate, i.e., it has zero length.}


\ccSeeAlso
\ccc{VoronoiTraits_2}\\
\ccc{FaceDegeneracyTester}

\end{ccRefFunctionObjectConcept}
