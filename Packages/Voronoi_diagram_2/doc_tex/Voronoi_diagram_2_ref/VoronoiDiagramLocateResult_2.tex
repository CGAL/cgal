%% Copyright (c) 2005  Foundation for Research and Technology-Hellas (Greece).
%% All rights reserved.
%%
%% This file is part of CGAL (www.cgal.org); you may redistribute it under
%% the terms of the Q Public License version 1.0.
%% See the file LICENSE.QPL distributed with CGAL.
%%
%% Licensees holding a valid commercial license may use this file in
%% accordance with the commercial license agreement provided with the software.
%%
%% This file is provided AS IS with NO WARRANTY OF ANY KIND, INCLUDING THE
%% WARRANTY OF DESIGN, MERCHANTABILITY AND FITNESS FOR A PARTICULAR PURPOSE.
%%
%% $Source$
%% $Revision$ $Date$
%% $Name$
%%
%% Author(s)     : Menelaos Karavelas <mkaravel@tem.uoc.gr>



\begin{ccRefConcept}{VoronoiDiagramLocateResult_2}

%% \ccHtmlCrossLink{}     %% add further rules for cross referencing links
%% \ccHtmlIndexC[concept]{} %% add further index entries
\ccDefinition

The concept \ccc{VoronoiDiagramLocateResult_2} defines the requirements of the
\ccc{Locate_result} type defined in the \ccc{Voronoi_diagram_adaptor_2<DG,VT>}
class. Essentially it provides the requirements for a class
representing the return type of a point location query performed on a
Voronoi diagram. Semantically, the result of the query is either a
vertex, edge or face of the Voronoi diagram. It is a vertex if the
query point is equidistant to at least three generators of the Voronoi
diagram. It is an edge if the query point is equidistant to exactly
two generators of the Voronoi diagram. In all other cases, the result
of the point location is a face, namely, the unique face of the
Voronoi diagram that contains the query point in its interior; in this
case there exists a unique generator closest to the query point. The
\ccc{VoronoiDiagramLocateResult_2} concept encapsulates the 
semantics described above.

\ccRefines
\ccc{DefaultConstructible}\\
\ccc{CopyConstructible}\\
\ccc{Assignable}\\
\ccc{EqualityComparable}

\ccTypes
\ccThree{typedef Dual_graph::Vertex_handle}{Vertex_handle+}{}
\ccThreeToTwo
%
\ccNestedType{Vertex_handle}{Handle for the vertices of the Voronoi diagram.}
\ccGlue
\ccNestedType{Face_handle}{Handle for the faces of the Voronoi diagram.}
\ccGlue
\ccNestedType{Halfedge_handle}{Handle for the halfedges of the Voronoi
  diagram.}

\ccCreationVariable{lr}  %% choose variable name

\ccHeading{Predicate Methods}
\ccThree{bool}{lr.is_vertex()+}{}
\ccThreeToTwo
%
\ccMethod{bool is_vertex();}{Returns \ccStyle{true} iff the result of
  the point location corresponds to a vertex of the Voronoi diagram.
  i.e., iff the query point lies on a vertex of the Voronoi diagram.}
\ccGlue
\ccMethod{bool is_edge();}{Returns \ccStyle{true} iff the result of
  the point location corresponds to an edge of the Voronoi diagram,
  i.e., iff the query point lies on the interior of an edge of the
  Voronoi diagram.}
\ccGlue
\ccMethod{bool is_face();}{Returns \ccStyle{true} iff the result of
  the point location corresponds to a face of the Voronoi diagram,
  i.e., iff the query point lies in the interior of a face of the
  Voronoi diagram.}


\ccHeading{Conversion Operators}
\ccThree{Halfedge_handle}{Halfedge_handle(lr)+}{}
\ccThreeToTwo
%
\ccMethod{operator Vertex_handle();}{Converts to a vertex handle
  representing the vertex of the Voronoi diagram which is the result
  of the point location query.
\ccPrecond{\ccc{is_vertex()} must be \ccStyle{true}.}}
%
\ccGlue
\ccMethod{operator Halfedge_handle();}{Converts to a halfedge handle
  representing one of the two halfedges of the Voronoi edge that is
  the result of the point location query.
\ccPrecond{\ccc{is_edge()} must be \ccStyle{true}.}}
%
\ccGlue
\ccMethod{operator Face_handle();}{Converts to a face handle that
  represents the face of the Voronoi diagram which is the result of
  the point location query.
\ccPrecond{\ccc{is_face()} must be \ccStyle{true}.}}


\ccHasModels
\ccc{CGAL::Voronoi_diagram_adaptor_2<DG,VT>::Locate_result}

\ccSeeAlso
\ccc{CGAL::Voronoi_diagram_adaptor_2<DG,VT>}
\end{ccRefConcept}

% +------------------------------------------------------------------------+
%%RefPage: end of main body, begin of footer
% EOF
% +------------------------------------------------------------------------+

