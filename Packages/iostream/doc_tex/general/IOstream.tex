
\cleardoublepage
\chapter{Operators for IO streams}\label{ChapterIO}

All classes in the \cgal\ kernel provide input and output operators for
IOStreams.  The basic task of such an operator is to produce a
representation of an object that can be written as a sequence of
characters on devices as a console, a file, or a pipe. In \cgal\ 
we distinguish between  a raw {\sc Ascii}, a raw binary and a 
pretty printing format. 

\ccEnum{CGAL_IO::Mode = {ASCII = 0, BINARY, PRETTY};}{}


The first one just writes number, e.g.\ the coordinates of a point or
the coefficients of a line in a machine independent format. The second
one writes the data in a binary format, e.g.\ a double is represented
as a sequence of four byte. The format depends on the machine. The
last one serves mainly for debugging as the type of the geometric
object is written, as well as the data defining the object. For example
for a point at the origin with Cartesian double coordinates, the output
would be \ccc{CGAL_PointC2(0.0, 0.0)}.  At the moment \cgal\ does not
provide input operations for pretty printed data. By default a stream
is in {\sc Ascii} mode.

\ccThree{CGAL_IO::Mode}{CGAL_set_mode(ios& s, CGAL_IO::Mode m);}{}

\cgal\ provides the following functions to modify the mode of an IO stream.

\ccFunction{CGAL_IO::Mode  CGAL_set_mode(ios& s, CGAL_IO::Mode m);}{}

\ccFunction{CGAL_IO::Mode  CGAL_set_ascii_mode(ios& s);}{}
\ccGlue
\ccFunction{CGAL_IO::Mode  CGAL_set_binary_mode(ios& s);}{}
\ccGlue
\ccFunction{CGAL_IO::Mode  CGAL_set_pretty_mode(ios& s);}{}

The following functions allow to test whether a stream is in a certain mode.

\ccFunction{CGAL_IO::Mode  CGAL_get_mode(ios& s);}{}

\ccFunction{bool CGAL_is_ascii(ios& s);}{}
\ccGlue
\ccFunction{bool CGAL_is_binary(ios& s);}{}
\ccGlue
\ccFunction{bool CGAL_is_pretty(ios& s);}{}


\newpage

\ccThree{ostream&}{operator<<( const CGAL_Class &c);}{}

\begin{ccClass} {ostream}
\ccSection{Output Streams}
\ccCreationVariable{os}

\ccDefinition
\cgal\  defines output operators for classes that are derived
from the class \ccStyle{ostream}. This allows to write to ostreams
as \ccStyle{cout} or \ccStyle{cerr}, as well as to strstreams
and fstreams. Let \ccVar\  be an output stream.

\ccOperations
The output operator is defined for all classes in the \cgal\ kernel and for the class \ccc{CGAL_Color} as well.

\ccMethod{ostream& operator<<( const CGAL_Class &c);}
{Inserts object \ccStyle{c} in the stream. }


\ccExample

\begin{cprog}

#include <iostream.h>
#include <fstream.h>

#include <CGAL/Cartesian.h>
#include <CGAL/Segment_2.h>

typedef CGAL_Point_2< CGAL_Cartesian<double> >     Point;
typedef CGAL_Segment_2< CGAL_Cartesian<double> >   Segment;

int main()
{
    Point p(0,1), q(2,2);
    Segment s(p,q);

    CGAL_set_pretty_mode(cout);
    cout << p << endl << q  << endl;

    ofstream f("data.txt");
    CGAL_set_binary_mode(f);
    f << s << p ;

    return 1;
}
\end{cprog} 
\end{ccClass} 


\newpage
\begin{ccClass} {istream}
\ccSection{Input Streams}
\ccCreationVariable{is}

\ccDefinition
\cgal\  defines input operators for classes that are derived
from the class \ccStyle{istream}. This allows to read from istreams
as \ccStyle{cin}, as well as from strstreams and fstreams.
Let \ccVar\ be an input stream.

\ccOperations

The input operator is defined for all classes in the \cgal\ kernel.

\ccMethod{istream& operator>>(CGAL_Class c);}
{Extracts object \ccStyle{c} from the stream.}


\ccExample

\begin{cprog}

#include <iostream.h>
#include <fstream.h>

#include <CGAL/Cartesian.h>
#include <CGAL/Segment_2.h>

typedef CGAL_Point_2< CGAL_Cartesian<double> >     Point;
typedef CGAL_Segment_2< CGAL_Cartesian<double> >   Segment;

int
main()
{
    Point p, q;
    Segment s;

    CGAL_set_ascii_mode(cin);
    cout >> p >> q;

    ofstream f("data.txt");
    CGAL_set_binary_mode(f);
    f >> s >> p;

    return 1;
}
\end{cprog} 
\end{ccClass} 
