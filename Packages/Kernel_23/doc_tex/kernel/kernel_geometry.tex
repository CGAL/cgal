\chapter{Kernel Geometry}

\section{Points and Vectors}
In \cgal, we strictly distinguish between points, vectors and directions.
A {\em point} is a point in the Euclidean space
$\E^d$, a {\em vector} is the difference of two points $p_2$, $p_1$
and denotes the direction and the distance from $p_1$ to $p_2$ in the
vector space $\R^d$, and a {\em direction} is a vector where we forget
about its length.
They are different mathematical concepts. For example, they behave
different under affine transformations and an addition of two
points is meaningless in affine geometry.  By putting them in different
classes we not only get cleaner code, but also type checking by the
compiler which avoids ambiguous expressions. Hence, it pays twice to
make this distinction.

\cgal\ defines a symbolic constant \ccStyle{ORIGIN} of type \ccc{Origin}
which denotes the point at the origin. This constant is used in the conversion 
between points and vectors. Subtracting it from a point $p$ results in the 
locus vector of $p$. 

\begin{cprog}
  Point_2< Cartesian<double> >  p(1.0, 1.0), q;
  Vector_2< Cartesian<double> >  v;
  v = p - ORIGIN;
  q = ORIGIN + v;  
  assert( p == q );
\end{cprog} 

In order to obtain the point corresponding to a vector $v$ you simply
have to add $v$ to \ccStyle{ORIGIN}. If you want to determine 
the point $q$ in the middle between two points $p_1$ and $p_2$, you can write%
\footnote{you might call \ccc{midpoint(p_1,p_2)} instead}

\begin{cprog}
  q = p_1 + (p_2 - p_1) / 2.0;
\end{cprog}  

Note that these constructions do not involve any performance overhead for 
the conversion with the currently available representation classes. 

\section{Kernel Objects}
Besides points (\ccc{Point_2<R>}, \ccc{Point_3<R>}, \ccc{Point_d<R>}), 
vectors (\ccc{Vector_2<R>}, \ccc{Vector_3<R>}), and 
directions (\ccc{Direction_2<R>}, \ccc{Direction_3<R>}), 
\cgal\ provides lines, rays, segments, planes,
triangles, tetrahedra, iso-rectangles, iso-cuboids, circles and spheres.

Lines  (\ccc{Line_2<R>}, \ccc{Line_3<R>}) in {\cgal} are oriented. In 
two-dimensional space, they induce a partition of the plane
into a positive side and a negative side. 
Any two points on a line induce an \ccHtmlNoLinksFrom{orientation}
of this line. 
A ray (\ccc{Ray_2<R>}, \ccc{Ray_3<R>}) is semi-infinite interval on a line, 
and this line is oriented from the finite endpoint of this interval towards 
any other point in this interval. A segment (\ccc{Segment_2<R>}, 
\ccc{Segment_3<R>}) is a bounded interval on a directed line,
and the endpoints are ordered so that they induce the same direction 
as that of the line.  

Planes are affine subspaces of dimension two in $\E^3$, passing through 
three points, or a point and a line, ray, or segment. 
{\cgal} provides a correspondence between any plane in the ambient 
space $\E^3$ and the embedding of $\E^2$ in that space.
Just like lines, planes are oriented and partition space into a positive side 
and a negative side.
In \cgal, there are no special classes for halfspaces. Halfspaces in 2D and
3D are supposed to be represented by oriented lines and planes, respectively.

Concerning polygons and polyhedra, the kernel provides triangles,
iso-oriented rectangles, iso-oriented cuboids and tetrahedra. 
More complex polygons\footnote{Any sequence of points can be seen as
a (not necessary simple) polygon or polyline. This view is used 
frequently in the basic library as well.}
and polyhedra or polyhedral surfaces can be obtained 
from the basic library (\ccc{Polygon_2}, \ccc{Polyhedron_3}), 
so they are not part of the kernel. 
As with any Jordan curves, triangles, iso-oriented rectangles and circles
separate the plane into two regions, one bounded and one unbounded.  

\section{Orientation and Relative Position}
Geometric objects in \cgal\ have member functions that test the
position of a point relative to the object.  Full dimensional objects
and their boundaries are represented by the same type, 
e.g.\ halfspaces and hyperplanes are not distinguished, neither are balls and
spheres and discs and circles. Such objects split the ambient space into two
full-dimensional parts, a bounded part and an unbounded part 
(e.g.\ circles), or two unbounded parts (e.g.\ hyperplanes). By default these
objects are oriented, i.e., one of the resulting parts is called the
positive side, the other one is called the negative side. Both of
these may be unbounded.

For these objects there is a function \ccStyle{oriented_side()} that
determines whether a test point is on the positive side, the negative
side, or on the oriented boundary. These function returns a value of type
\ccc{Oriented_side}.

Those objects that split the space in a bounded and an unbounded part, have
a member function \ccStyle{bounded_side()} with return type
\ccc{Bounded_side}.

If an object is lower dimensional, e.g.\ a triangle in three-dimensional
space or a segment in two-dimensional space, there is only a test whether a
point belongs to the object or not. This member function, which takes a 
point as an argument and returns a boolean value, is called \ccStyle{has_on()}
