\begin{ccRefFunctionObjectConcept}{Kernel::ConstructLine_2}
A model for this must provide:

\ccCreationVariable{fo}

\ccHidden \ccMemberFunction{Kernel::Line_2 operator()();}
             {introduces an uninitialized variable .}

\ccHidden \ccMemberFunction{Kernel::Line_2 operator()(const Kernel::Line_2 &h);}
 	    {copy constructor.}

\ccMemberFunction{Kernel::Line_2 operator()(const Kernel::RT &a, const Kernel::RT &b, const Kernel::RT &c);}
            {introduces a line  with the line equation in Cartesian
	      coordinates $ax +by +c = 0$.}

\ccMemberFunction{Kernel::Line_2 operator()(const Kernel::Point_2 &p, const Kernel::Point_2 &q);}
            {introduces a line  passing through the points $p$ and $q$. 
             Line  is directed from $p$ to $q$.}

\ccMemberFunction{Kernel::Line_2 operator()(const Kernel::Point_2 &p, const Kernel::Direction_2&d);}
            {introduces a line  passing through point $p$ with 
             direction $d$.}

\ccMemberFunction{Kernel::Line_2 operator()(const Kernel::Segment_2 &s);}
            {introduces a line  supporting the segment $s$,
	    oriented from source to target.}

\ccMemberFunction{Kernel::Line_2 operator()(const Kernel::Ray_2 &r);}
            {introduces a line  supporting the ray $r$,
	    with same orientation.}

\ccSeeAlso
\ccRefIdfierPage{CGAL::Line_2<R>}  \\

\end{ccRefFunctionObjectConcept}
