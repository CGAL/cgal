\begin{ccRefFunction}{do_intersect}
\ccInclude{CGAL/intersections.h}\footnote{%
The drawback of \ccc{<CGAL/intersections.h>} is that a lot is included, 
which results in long compilation times.
It is also possible to include only the intersections that are of interest.
The naming scheme of the header files is as follows.
Intersections of types \ccStyle{Type1<Kernel>} and \ccStyle{Type2<Kernel>}
are declared in header file \ccc{CGAL/Type1_Type2_intersection.h}.
So, \ccHtmlNoLinksFrom{intersection} routines of segments and lines in 2D are declared in
\ccc{<CGAL/Segment_2_Line_2_intersection.h>}. The order of the type names does not 
matter. It is also possible to include \ccc{<CGAL/Line_2_Segment_2_intersection.h>}.
For intersections of two objects of the same type, the type name should be
mentioned twice: \ccc{<CGAL/Segment_2_Segment_2_intersection.h>} 
}

\ccUnchecked{
\ccFunction{bool do_intersect(Type1<Kernel> obj1, Type2<Kernel> obj2);}
{checks whether \ccc{obj1} and \ccc{obj2} intersect.
Two objects \ccStyle{obj1} and \ccStyle{obj2} intersect if there is a point
\ccStyle{p} that is part of both \ccStyle{obj1} and \ccStyle{obj2}.
The \ccHtmlNoLinksFrom{intersection} region of those two objects is defined as the set of all
points \ccStyle{p} that are part of both \ccStyle{obj1} and \ccStyle{obj2}.
Note that for objects like triangles and polygons that enclose a
bounded region, this region is part of the object.
}}

The types \ccStyle{Type1} and \ccStyle{Type2} can be any of the following:
\begin{itemize}\ccTexHtml{\itemsep0pt\topsep0pt\partopsep0pt\parskip0pt\parsep0pt}{}
\item \ccStyle{Point_2<Kernel>}
\item \ccStyle{Line_2<Kernel>}
\item \ccStyle{Ray_2<Kernel>}
\item \ccStyle{Segment_2<Kernel>}
\item \ccStyle{Triangle_2<Kernel>}
\item \ccStyle{Iso_rectangle_2<Kernel>}
\end{itemize}

Also, in three-dimensional space \ccc{Type1} can be \ccc{Plane_3<Kernel>} and 
\ccc{Type2} any of 
\begin{itemize}\ccTexHtml{\itemsep0pt\topsep0pt\partopsep0pt\parskip0pt\parsep0pt}{}
\item \ccStyle{Plane_3<Kernel>}
\item \ccStyle{Line_3<Kernel>}
\item \ccStyle{Ray_3<Kernel>}
\item \ccStyle{Segment_3<Kernel>}
\item \ccStyle{Triangle_3<Kernel>}
\end{itemize}

\ccSeeAlso
\ccRefIdfierPage{CGAL::intersection}

\end{ccRefFunction}
