\begin{ccRefFunctionObjectConcept}{Kernel::ConstructDirection_2}
A model for this must provide:

\ccCreationVariable{fo}


\ccHidden \ccMemberFunction{Kernel::Direction_2 operator()();}
             {introduces an uninitialized direction .}

\ccHidden \ccMemberFunction{Kernel::Direction_2 operator()(const Kernel::Direction_2 &d);}
            {copy constructor.}

\ccMemberFunction{Kernel::Direction_2 operator()(const Kernel::Vector_2 &v);}
            {introduces the direction of vector $v$.}

\ccMemberFunction{Kernel::Direction_2 operator()(const Kernel::Line_2 &l);}
            {introduces the direction of line $l$.}

\ccMemberFunction{Kernel::Direction_2 operator()(const Kernel::Ray_2 &r);}
            {introduces the direction of ray $r$.}

\ccMemberFunction{Kernel::Direction_2 operator()(const Kernel::Segment_2 &s);}
            {introduces the direction of segment $s$.}

\ccHidden\ccMemberFunction{Kernel::Direction_2 operator()(const Kernel::RT &x, const Kernel::RT &y);}
            {introduces a direction  passing through the origin
             and the point with Cartesian coordinates $(x, y)$.}

\ccSeeAlso

\ccRefIdfierPage{CGAL::Direction_2<R>}  \\

\end{ccRefFunctionObjectConcept}
