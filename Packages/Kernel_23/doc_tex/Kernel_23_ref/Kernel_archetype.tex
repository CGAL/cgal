\begin{ccRefClass}{Kernel_archetype}

%\KernelRefLayout\gdef\ccTagOperatorLayout{\ccFalse}

\ccDefinition

\ccc{CGAL::Kernel_archetype} is a concept archetype (minimal model) for the CGAL kernel concept.
It provides all functionality required by the CGAL kernel concept, but nothing more.
It can be used for testing successful compilation of packages of the basic library with a minimal
model. Deprecated kernel functionality is not supported. All geometrical types (like the 2d/3d point or segment types)
of \ccc{CGAL::Kernel_archetype} have
copy constructors, default constructors and an assignment operator, and nothing else.
Comparison operators are by default not supported, but can be switched on by defining the macro
\ccc{CGAL_CONCEPT_ARCHETYPE_ALLOW_COMPARISONS}.

The geometrical types of the concept archetype encapsulate no data members, so runtime checks with the archetype
are not very useful (\ccc{CGAL::Kernel_archetype} is only meant for compilation checks with a minimal model in the testsuites
of CGAL packages).

The package supports the two- and three-dimensional part of the CGAL kernel concept. The d-dimensional
part is not supported.

\ccc{CGAL::Kernel_archetype} normally offers the full functionality (all types, functors and constructions
of a CGAL kernel model), but it is possible to restrict the interface. This can be useful
for testing packages that require only a very small subset of the functionality offered
by CGAL kernel models.
If you want to do this, you have to define the macro \ccc{CGAL_CA_LIMITED_INTERFACE} 
(before the inclusion of {\em CGAL/Kernel\_archetype.h}) to switch on the interface limitation. 
Now you have to tell the kernel archetype the types it has to provide for the specific package.

For every type you have to define a macro.
The name of the macro is \ccc{CGAL_CA_NAME_OF_KERNEL_TYPE}, where \ccc{NAME_OF_KERNEL_TYPE} is the name
of the kernel type (written in capitals) that has to be provided by the kernel archetype for a specific package.
If, for example, a package only needs type definitions for \ccc{Point_2} and \ccc{Orientation_2}, you would
define \ccc{CGAL_CA_LIMITED_INTERFACE}, \ccc{CGAL_CA_POINT_2} and \ccc{CGAL_CA_ORIENTATION_2} .

\ccIsModel
\ccRefConceptPage{Kernel}

\ccInclude{CGAL/Kernel_archetype.h}

\ccCreation
\ccCreationVariable{ka}

\ccConstructor{Kernel_archetype();}{Default constructor.}

\ccTypes

We provide all type definitions that must be provided by a CGAL kernel model.
See the CGAL kernel concept manual pages for details.
Deprecated functionality is not supported. You can restrict the interface
by defining macros (see the definition for details).

\ccOperations

For each of the function objects of the kernel archetype, there is a member function
that requires no arguments and returns an instance of that function object.
The name of the member function is the uncapitalized name of the type
returned with the suffix \ccc{_object} appended.


\ccSeeAlso
\ccRefIdfierPage{CGAL::Cartesian<FieldNumberType>} \\
\ccRefIdfierPage{CGAL::Homogeneous<RingNumberType>} \\
\ccRefIdfierPage{CGAL::Simple_cartesian<FieldNumberType>} \\

\end{ccRefClass}
