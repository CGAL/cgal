% +------------------------------------------------------------------------+
% | Reference manual page: Fixed_border_parametizer_3.tex
% +------------------------------------------------------------------------+
% | 23.08.2005   Author
% | Package: Parameterization
% |
\RCSdef{\RCSFixedborderparametizerRev}{$Revision$}
\RCSdefDate{\RCSFixedborderparametizerDate}{$Date$}
% |
%%RefPage: end of header, begin of main body
% +------------------------------------------------------------------------+


\begin{ccRefClass}{Fixed_border_parametizer_3}  %% add template arg's if necessary

%% \ccHtmlCrossLink{}     %% add further rules for cross referencing links
%% \ccHtmlIndexC[class]{} %% add further index entries

\ccDefinition


Class Fixed\_border\_parametizer\_3 is the base class of fixed border parameterization methods (Tutte, Floater, ...). 1 to 1 mapping is guaranteed if surface's border is mapped onto a convex polygon.

Implementation notes:\begin{itemize}
\item Subclasses must at least implement compute\_wij().\item The current implementation does not remove border vertices from the linear systems =$>$ A cannot be symmetric.\end{itemize}

\ccInclude{CGAL/Fixed_border_parametizer_3.h}



\ccIsModel

Fixed\_border\_parametizer\_3 is a model of the ParametizerTraits\_3 concept.



\ccParameters

The full template declaration is:

template$<$
class MeshAdaptor\_3, 
class BorderParametizer\_3 = Circular\_border\_arc\_length\_parametizer\_3$<$MeshAdaptor\_3$>$, 
class SparseLinearAlgebraTraits\_d = OpenNL::DefaultLinearSolverTraits$<$typename MeshAdaptor\_3::NT$>$$>$ 
class Fixed\_border\_parametizer\_3;


\ccTypes


\ccNestedType{Adaptor}
{
}
\ccNestedType{Error_code}
{
The various errors detected by this package.
}
\ccNestedType{NT}
{
}
\ccNestedType{Facet_handle}
{
}
\ccNestedType{Facet_const_handle}
{
}
\ccNestedType{Vertex_handle}
{
}
\ccNestedType{Vertex_const_handle}
{
}
\ccNestedType{Point_3}
{
}
\ccNestedType{Point_2}
{
}
\ccNestedType{Vector_3}
{
}
\ccNestedType{Vector_2}
{
}
\ccNestedType{Facet_iterator}
{
}
\ccNestedType{Facet_const_iterator}
{
}
\ccNestedType{Vertex_iterator}
{
}
\ccNestedType{Vertex_const_iterator}
{
}
\ccNestedType{Border_vertex_iterator}
{
}
\ccNestedType{Border_vertex_const_iterator}
{
}
\ccNestedType{Vertex_around_facet_circulator}
{
}
\ccNestedType{Vertex_around_facet_const_circulator}
{
}
\ccNestedType{Vertex_around_vertex_circulator}
{
}
\ccNestedType{Vertex_around_vertex_const_circulator}
{
}
\ccNestedType{Border_param}
{
}
\ccNestedType{Sparse_LA}
{
}
\ccNestedType{Vector}
{
}
\ccNestedType{Matrix}
{
}


\ccCreation
\ccCreationVariable{p}  %% variable name used by \ccMethod below

\ccConstructor{Fixed_border_parametizer_3 (Border_param border_param = Border_param(), Sparse_LA sparse_la = Sparse_LA());}
{
Constructor.
}


\ccOperations

\ccMethod{Parametizer_traits_3< Adaptor >::Error_code parameterize (Adaptor * mesh);}
{
Compute a 1 to 1 mapping from a triangular 3D surface 'mesh' to a piece of the 2D space. The mapping is linear by pieces (linear in each triangle). The result is the (u,v) pair image of each vertex of the 3D surface.
Preconditions:\begin{itemize}
\item 'mesh' must be a surface with 1 connected component.\item 'mesh' must be a triangular mesh.\item the mesh border must be mapped onto a convex polygon. \end{itemize}
}
\ccMethod{Parametizer_traits_3< Adaptor >::Error_code check_parameterize_preconditions (Adaptor * mesh);}
{
Check parameterize() preconditions:\begin{itemize}
\item 'mesh' must be a surface with 1 connected component.\item 'mesh' must be a triangular mesh.\item the mesh border must be mapped onto a convex polygon. \end{itemize}
}
\ccMethod{void initialize_system_from_mesh_border (Matrix * A, Vector * Bu, Vector * Bv, const Adaptor & mesh);}
{
Initialize A, Bu and Bv after boundary parameterization. Fill the border vertices' lines in both linear systems: "u = constant" and "v = constant".
Preconditions:\begin{itemize}
\item vertices must be indexed.\item A, Bu and Bv must be allocated.\item border vertices must be parameterized. \end{itemize}
}
\ccMethod{virtual NT compute_wij (const Adaptor & mesh, Vertex_const_handle main_vertex_Vi, Vertex_around_vertex_const_circulator neighbor_vertex_Vj);}
{
Compute wij = (i,j) coefficient of matrix A for j neighbor vertex of i. Implementation note: Subclasses must at least implement compute\_wij().
Implemented in Barycentric\_mapping\_parametizer\_3 , Discrete\_authalic\_parametizer\_3 , Discrete\_conformal\_map\_parametizer\_3 , and Mean\_value\_coordinates\_parametizer\_3
}
\ccMethod{Parametizer_traits_3< Adaptor >::Error_code setup_inner_vertex_relations (Matrix * A, Vector * Bu, Vector * Bv, const Adaptor & mesh, Vertex_const_handle vertex);}
{
Compute the line i of matrix A for i inner vertex:\begin{itemize}
\item call compute\_wij() to compute the A coefficient Wij for each neighbor Vj.\item compute Wii = - sum of Wij.\end{itemize}
Preconditions:\begin{itemize}
\item vertices must be indexed.\item vertex i musn't be already parameterized.\item line i of A must contain only zeros. \end{itemize}
}
\ccMethod{void set_mesh_uv_from_system (Adaptor * mesh, const Vector & Xu, const Vector & Xv);}
{
Copy Xu and Xv coordinates into the (u,v) pair of each surface vertex.
}
\ccMethod{Parametizer_traits_3< Adaptor >::Error_code check_parameterize_postconditions (const Adaptor & mesh, const Matrix & A, const Vector & Bu, const Vector & Bv);}
{
Check parameterize() postconditions:\begin{itemize}
\item "A$\ast$Xu = Bu" and "A$\ast$Xv = Bv" systems are solvable with a good conditioning.\item 3D -$>$ 2D mapping is 1 to 1. \end{itemize}
}
\ccMethod{bool is_one_to_one_mapping (const Adaptor & mesh, const Matrix & A, const Vector & Bu, const Vector & Bv);}
{
Check if 3D -$>$ 2D mapping is 1 to 1. The default implementation checks each normal.
}
\ccMethod{Border_param& get_border_parametizer ();}
{
Get the object that maps the surface's border onto a 2D space.
}
\ccMethod{Sparse_LA& get_linear_algebra_traits ();}
{
Get the sparse linear algebra (traits object to access the linear system).
}


\ccSeeAlso

\ccc{Some_other_class},
\ccc{some_other_function}.

\ccExample

A short example program.
Instead of a short program fragment, a full running program can be
included using the
\verb|\ccIncludeExampleCode{Parameterization/Fixed_border_parametizer_3.C}|
macro. The program example would be part of the source code distribution and
also part of the automatic test suite.

\begin{ccExampleCode}
void your_example_code() {
}
\end{ccExampleCode}

%% \ccIncludeExampleCode{Parameterization/Fixed_border_parametizer_3.C}

\end{ccRefClass}

% +------------------------------------------------------------------------+
%%RefPage: end of main body, begin of footer
% EOF
% +------------------------------------------------------------------------+

