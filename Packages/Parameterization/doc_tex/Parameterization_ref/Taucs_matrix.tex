% +------------------------------------------------------------------------+
% | Reference manual page: Taucs_matrix.tex
% +------------------------------------------------------------------------+
% | 21.09.2005   Laurent Saboret, Pierre Alliez
% | Package: Parameterization
% |
\RCSdef{\RCSTaucsmatrixRev}{$Revision$}
\RCSdefDate{\RCSTaucsmatrixDate}{$Date$}
% |
%%RefPage: end of header, begin of main body
% +------------------------------------------------------------------------+


\begin{ccRefClass}{Taucs_matrix}  %% add template arg's if necessary

%% \ccHtmlCrossLink{}     %% add further rules for cross referencing links
%% \ccHtmlIndexC[class]{} %% add further index entries


\ccDefinition

% The section below is automatically generated. Do not edit!
%START-AUTO(\ccDefinition)

Class Taucs\_matrix is a C++ wrapper around TAUCS' matrix type taucs\_ccs\_matrix.

TODO: reallocate the array of non null elements when it's full

%END-AUTO(\ccDefinition)

\ccInclude{CGAL/Taucs_matrix.h}


\ccIsModel

% The section below is automatically generated. Do not edit!
%START-AUTO(\ccIsModel)

Model of the SparseLinearAlgebraTraits\_d::Matrix concept.

%END-AUTO(\ccIsModel)


\ccParameters

The full template declaration is:

% The section below is automatically generated. Do not edit!
%START-AUTO(\ccParameters)

template$<$
class T$>$ 
class Taucs\_matrix;

%END-AUTO(\ccParameters)


\ccTypes

% The section below is automatically generated. Do not edit!
%START-AUTO(\ccTypes)

\ccNestedType{NT}
{
}

%END-AUTO(\ccTypes)


\ccConstants

% The section below is automatically generated. Do not edit!
%START-AUTO(\ccConstants)

None.

%END-AUTO(\ccConstants)


\ccCreation
\ccCreationVariable{m}  % choose variable name for \ccMethod

% The section below is automatically generated. Do not edit!
%START-AUTO(\ccCreation)

\ccConstructor{Taucs_matrix (int dim, bool is_symmetric = false, int nb_max_elements = 0);}
{
Create a square matrix initialized with zeros.
}
\begin{description}
\item[Parameters: ]
\begin{description}
\item[dim]Matrix dimension \item[is\_symmetric]Symmetric/hermitian? \item[nb\_max\_elements]Max number of non 0 elements in the matrix (automatically computed if 0) \end{description}
\end{description}
\ccConstructor{Taucs_matrix (int rows, int columns, bool is_symmetric = false, int nb_max_elements = 0);}
{
Create a rectangular matrix initialized with zeros.
}
\begin{description}
\item[Parameters: ]
\begin{description}
\item[rows]Matrix dimensions \item[is\_symmetric]Symmetric/hermitian? \item[nb\_max\_elements]Max number of non 0 elements in the matrix (automatically computed if 0) \end{description}
\end{description}
\ccConstructor{~Taucs_matrix ();}
{
Delete TAUCS matrix wrapped by this object
}

%END-AUTO(\ccCreation)


\ccOperations

% The section below is automatically generated. Do not edit!
%START-AUTO(\ccOperations)

\ccMethod{int row_dimension () const;}
{
Return the matrix number of rows.
}
\ccMethod{int column_dimension () const;}
{
Return the matrix number of columns.
}
\ccMethod{T get_coef (int i, int j) const;}
{
Read access to 1 matrix coefficient
Preconditions:\begin{itemize}
\item 0 $<$= i $<$ row\_dimension()\item 0 $<$= j $<$ column\_dimension() \end{itemize}
}
\ccMethod{void set_coef (int i, int j, T val);}
{
Write access to 1 matrix coefficient: aij $<$- val
Optimization: For symmetric matrices, Taucs\_matrix stores only the lower triangle set\_coef() does nothing if (i,j) belongs to the upper triangle
Preconditions:\begin{itemize}
\item 0 $<$= i $<$ row\_dimension()\item 0 $<$= j $<$ column\_dimension() \end{itemize}
}
\ccMethod{void add_coef (int i, int j, T val);}
{
Write access to 1 matrix coefficient: aij $<$- aij + val
Optimization: For symmetric matrices, Taucs\_matrix stores only the lower triangle add\_coef() does nothing if (i,j) belongs to the upper triangle
Preconditions:\begin{itemize}
\item 0 $<$= i $<$ row\_dimension()\item 0 $<$= j $<$ column\_dimension() \end{itemize}
}
\ccMethod{const taucs_ccs_matrix* get_taucs_matrix () const;}
{
Get TAUCS matrix wrapped by this object.
}
\ccMethod{taucs_ccs_matrix* get_taucs_matrix ();}
{
}

%END-AUTO(\ccOperations)


\ccSeeAlso

\ccc{Some_other_class},
\ccc{some_other_function}.


\ccExample

A short example program.
Instead of a short program fragment, a full running program can be
included using the
\verb|\ccIncludeExampleCode{Parameterization/Taucs_matrix.C}|
macro. The program example would be part of the source code distribution and
also part of the automatic test suite.

\begin{ccExampleCode}
void your_example_code() {
}
\end{ccExampleCode}

%% \ccIncludeExampleCode{Parameterization/Taucs_matrix.C}


\end{ccRefClass}

% +------------------------------------------------------------------------+
%%RefPage: end of main body, begin of footer
% EOF
% +------------------------------------------------------------------------+

