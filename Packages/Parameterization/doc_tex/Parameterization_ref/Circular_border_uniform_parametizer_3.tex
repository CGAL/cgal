% +------------------------------------------------------------------------+
% | Reference manual page: Circular_border_uniform_parametizer_3.tex
% +------------------------------------------------------------------------+
% | 23.08.2005   Author
% | Package: Parameterization
% | 
\RCSdef{\RCSCircularborderuniformparametizerRev}{$Revision$}
\RCSdefDate{\RCSCircularborderuniformparametizerDate}{$Date$}
% |
%%RefPage: end of header, begin of main body
% +------------------------------------------------------------------------+


\begin{ccRefClass}{Circular_border_uniform_parametizer_3}  %% add template arg's if necessary

%% \ccHtmlCrossLink{}     %% add further rules for cross referencing links
%% \ccHtmlIndexC[class]{} %% add further index entries

\ccDefinition


This class parameterizes the border of a 3D surface onto a circle on an uniform manner: points are equally spaced.

\ccInclude{Circular_border_uniform_parametizer_3.h}


\ccIsModel

Model of the BorderParametizer\_3 concept.



\ccParameters

The full template declaration is:

template$<$
class MeshAdaptor\_3$>$ 
class Circular\_border\_uniform\_parametizer\_3;


\ccTypes


\ccNestedType{Adaptor}
{
}
\ccNestedType{Error_code}
{
}
\ccNestedType{NT}
{
}
\ccNestedType{Facet_handle}
{
}
\ccNestedType{Facet_const_handle}
{
}
\ccNestedType{Vertex_handle}
{
}
\ccNestedType{Vertex_const_handle}
{
}
\ccNestedType{Point_3}
{
}
\ccNestedType{Point_2}
{
}
\ccNestedType{Vector_3}
{
}
\ccNestedType{Vector_2}
{
}
\ccNestedType{Facet_iterator}
{
}
\ccNestedType{Facet_const_iterator}
{
}
\ccNestedType{Vertex_iterator}
{
}
\ccNestedType{Vertex_const_iterator}
{
}
\ccNestedType{Border_vertex_iterator}
{
}
\ccNestedType{Border_vertex_const_iterator}
{
}
\ccNestedType{Vertex_around_facet_circulator}
{
}
\ccNestedType{Vertex_around_facet_const_circulator}
{
}
\ccNestedType{Vertex_around_vertex_circulator}
{
}
\ccNestedType{Vertex_around_vertex_const_circulator}
{
}


\ccCreation
\ccCreationVariable{a}  %% choose variable name

\ccConstructor{Circular_border_uniform_parametizer_3();}{default constructor.}


\ccOperations

\ccMethod{virtual double compute_edge_length (const Adaptor & mesh, Vertex_const_handle source, Vertex_const_handle target);}
{
Compute length of an edge:.
Uniform boundary parameterization: points are equally spaced.
}


\ccSeeAlso

\ccc{Some_other_class},
\ccc{some_other_function}.

\ccExample

A short example program.
Instead of a short program fragment, a full running program can be
included using the 
\verb|\ccIncludeExampleCode{Parameterization/Circular_border_uniform_parametizer_3.C}| 
macro. The program example would be part of the source code distribution and
also part of the automatic test suite.

\begin{ccExampleCode}
void your_example_code() {
}
\end{ccExampleCode}

%% \ccIncludeExampleCode{Parameterization/Circular_border_uniform_parametizer_3.C}

\end{ccRefClass}

% +------------------------------------------------------------------------+
%%RefPage: end of main body, begin of footer
% EOF
% +------------------------------------------------------------------------+

