% +------------------------------------------------------------------------+
% | Reference manual page: Parameterization/intro.tex
% +------------------------------------------------------------------------+
% | 08.22.2005   Laurent Saboret, Pierre Alliez
% | Package: Parameterization
% |
% |
% +------------------------------------------------------------------------+

%\clearpage
%\section{Reference Pages for Planar Parameterization of Triangulated Surface Meshes}
\chapter{Planar Parameterization of Triangulated Surface Meshes}
\label{chap:parameterization_ref}
\ccChapterAuthor{Laurent Saboret \and Pierre Alliez}


Parameterizing a surface amounts to finding a one-to-one mapping from
a suitable domain to the surface. A good mapping is the one which
minimizes either angle or area distortions in some sense. In this
package, we focus on triangulated surfaces that are homeomorphic to a
disk and on piecewise linear mappings into a planar domain.

\ccHeading{Parameterization methods}

This \cgal\ package implements some of
the state-of-the-art parameterization methods:
\begin{itemize}

\item Fixed boundary:

    \begin{itemize}

    \item Tutte uniform weights (guaranteed one-to-one mapping for
    convex boundary).

    \item Mean coordinate values (ditto).

    \item Discrete conformal maps (conditionally guaranteed if all
    weights positive and convex boundary).

    \item Authalic (ditto).

    \end{itemize}

\item Free boundary:

    \begin{itemize}

    \item Least squares conformal maps.

    \item Discrete conformal maps.

    \end{itemize}

\end{itemize}


\ccHeading{Boundary parameterization methods}

Boundary parameterization methods define a
set of constraints (a constraint specifies two u,v coordinates for
each instance of a vertex along the boundary).

This package implements classic boundary parameterization methods:
\begin{itemize}

\item For free boundary methods: only two constraints (the pinned
vertices). They have to be on the specified boundary.

\item For fixed boundary methods:

    \begin{itemize}

    \item the user can select a boundary
        parameterization among two common methods: uniform or
        arc-length parameterization.

    \item one convex shape specified by:

        \begin{itemize}

        \item one shape among a set of standard ones (circle, square).

        \item a convex polygon.

        \end{itemize}

    \end{itemize}

\end{itemize}


\ccHeading{Mesh}

The general definition of input meshes handled by the package is:

\begin{itemize}

\item Triangulated

\item 2-manifold

\item Oriented

\item One connected component. The input mesh can be of any genus,
-but- it has to come with a description of a boundary (a list of
vertices) which is the boundary of
a topological disc. If no boundary is given, we assume that it
coincides with the longest boundary already in the input mesh.  Note
that this way the user is responsible for cutting a closed mesh of
arbitrary genus (even a topological disc with an intricate seam
cut), as long as this condition is verified.

\end{itemize}

The package proposes
an interface with both the 2D Triangulation Data Structure enriched
with 3D points (not yet implemented) and the Polyhedron.

The \cgal\ parameterization package is loosely linked to the
mesh data structure. Replacing it is relatively easy.


\ccHeading{Output}

One uv coordinate for each interior vertex, and one uv coordinate for
each instance of a vertex along the input boundary.


\ccHeading{Sparse Linear Algebra}

Since parameterizing meshes requires
efficient representation of sparse matrices and efficient iterative or
direct linear solvers, we provide an interface to several state-of-the-art
sparse linear solvers:
\begin{itemize}
\item OpenNL (Bruno L{\'e}vy) is shipped with \cgal. This is the default solver.
\item TAUCS is a reference direct solver for sparse symmetric matrices.
\item SuperLU is a reference direct solver for sparse unsymmetric matrices (not yet implemented).
\end{itemize}

The \cgal\ parameterization package is loosely linked to the
solver. Replacing it is easy.


\ccHeading{Assertions}

The assertion flags for the package
use \ccc{PARAMETERIZATION} in their names (\textit{e.g.},
\ccc{CGAL_PARAMETERIZATION_NO_ASSERTIONS}).

For \emph{fixed} boundary parameterizations:
\begin{itemize}
\item Preconditions:
    \begin{itemize}
    \item check that the boundary is mapped onto a convex polygon.
    \item check that the input mesh is triangular (expensive check).
    \item check that the input mesh is a surface with 1 connected component (expensive check).
    \end{itemize}
\item Postconditions:
    \begin{itemize}
    \item check one-to-one mapping.
    \item check if the linear system was solved with a good conditioning (expensive check).
    \end{itemize}
\end{itemize}

For \emph{free} boundary parameterizations:
\begin{itemize}
\item Preconditions:
    \begin{itemize}
    \item check that the input mesh is triangular (expensive check).
    \item check that the input mesh is a surface with 1 connected component (expensive check).
    \end{itemize}
\item Postconditions:
    \begin{itemize}
    \item check one-to-one mapping.
    \item check if the linear system was solved with a good conditioning (expensive check, not yet implemented).
    \end{itemize}
\end{itemize}

Expensive checking is off by default. It can be enabled by
defining \ccc{CGAL_PARAMETERIZATION_CHECK_EXPENSIVE}.


\ccHeading{Concepts}

\ccRefConceptPage{ParametizerTraits_3}  \\
\ccRefConceptPage{BorderParametizer_3}  \\
\ccRefConceptPage{MeshAdaptor_3}  \\
\ccRefConceptPage{PatchableMeshAdaptor_3}  \\
\ccRefConceptPage{SparseLinearAlgebraTraits_d}  \\


\ccHeading{Parameterization Methods Traits Classes}

\ccRefIdfierPage{CGAL::Barycentric_mapping_parametizer_3<Traits>}  \\
\ccRefIdfierPage{CGAL::Discrete_authalic_parametizer_3<Traits>}  \\
\ccRefIdfierPage{CGAL::Discrete_conformal_map_parametizer_3<Traits>}  \\
\ccRefIdfierPage{CGAL::LSCM_parametizer_3<Traits>}  \\
\ccRefIdfierPage{CGAL::Mean_value_coordinates_parametizer_3<Traits>}  \\


\ccHeading{Boundary Parameterization Methods Traits Classes}

\ccRefIdfierPage{CGAL::Circular_border_arc_length_parametizer_3<Traits>}  \\
\ccRefIdfierPage{CGAL::Circular_border_uniform_parametizer_3<Traits>}  \\
\ccRefIdfierPage{CGAL::Square_border_arc_length_parametizer_3<Traits>}  \\
\ccRefIdfierPage{CGAL::Square_border_uniform_parametizer_3<Traits>}  \\
\ccRefIdfierPage{CGAL::Two_vertices_parametizer_3<Traits>}  \\


\ccHeading{MeshAdaptor\_3 and PatchableMeshAdaptor\_3 Traits Classes}

\ccRefIdfierPage{CGAL::Mesh_adaptor_polyhedron_3<Traits>}  \\
\ccRefIdfierPage{CGAL::Mesh_adaptor_patch_3<Traits>}  \\


\ccHeading{Sparse Linear Algebra Traits Classes}

\ccRefIdfierPage{CGAL::Taucs_solver_traits<Traits>}  \\
\ccRefIdfierPage{CGAL::Taucs_symmetric_solver_traits<Traits>}  \\
\ccRefIdfierPage{OpenNL::DefaultLinearSolverTraits<Traits>}  \\
\ccRefIdfierPage{OpenNL::SymmetricLinearSolverTraits<Traits>}  \\


\ccHeading{Helper Classes}

\ccRefIdfierPage{CGAL::Mesh_adaptor_feature_extractor<Traits>}  \\


\ccHeading{Functions}

\ccRefIdfierPage{CGAL::parameterize<Traits>}  \\


\clearpage

\lcHtml{\ccHeading{Alphabetical Listing of Reference Pages}}
