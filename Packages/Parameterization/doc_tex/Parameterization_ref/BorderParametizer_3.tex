% +------------------------------------------------------------------------+
% | Reference manual page: BorderParametizer_3.tex
% +------------------------------------------------------------------------+
% | 23.08.2005   Laurent Saboret, Pierre Alliez
% | Package: Parameterization
% |
\RCSdef{\RCSBorderParametizerRev}{$Revision$}
\RCSdefDate{\RCSBorderParametizerDate}{$Date$}
% |
%%RefPage: end of header, begin of main body
% +------------------------------------------------------------------------+


\begin{ccRefConcept}{BorderParametizer_3}

%% \ccHtmlCrossLink{}     %% add further rules for cross referencing links
%% \ccHtmlIndexC[concept]{} %% add further index entries


\ccDefinition

% The section below is automatically generated. Do not edit!
%START-AUTO(\ccDefinition)

Concept BorderParametizer\_3 of a class that parametrizes a given type of mesh, 'Adaptor', which is a model of the MeshAdaptor\_3 concept.

Implementation note: To simplify the implementation, BorderParametizer\_3 models know only the MeshAdaptor\_3 class. They don't know the parameterization algorithm requirements nor the kind of sparse linear system used.

%END-AUTO(\ccDefinition)


\ccHeading{Design pattern}

% The section below is automatically generated. Do not edit!
%START-AUTO(\ccHeading\{Design pattern\})

BorderParametizer\_3 models are Strategies (see [GOF95]): they implement a strategy of boundary parameterization for models of MeshAdaptor\_3.

%END-AUTO(\ccHeading\{Design pattern\})


\ccRefines

% The section below is automatically generated. Do not edit!
%START-AUTO(\ccRefines)

None.

%END-AUTO(\ccRefines)


\ccTypes

% The section below is automatically generated. Do not edit!
%START-AUTO(\ccTypes)

\ccNestedType{Adaptor}
{
}
\ccNestedType{ErrorCode}
{
}

%END-AUTO(\ccTypes)


\ccConstants

% The section below is automatically generated. Do not edit!
%START-AUTO(\ccConstants)

None.

%END-AUTO(\ccConstants)


\ccCreation
\ccCreationVariable{bp}  %% variable name for \ccMethod below

Construction and destruction are undefined.

% The section below is automatically generated. Do not edit!
%START-AUTO(\ccCreation)
%END-AUTO(\ccCreation)


\ccOperations

% The section below is automatically generated. Do not edit!
%START-AUTO(\ccOperations)

\ccMethod{ErrorCode parameterize_border (Adaptor * mesh);}
{
Assign to mesh's border vertices a 2D position (ie a (u,v) pair) on border's shape. Mark them as "parameterized". Return false on error.
}
\ccMethod{bool is_border_convex ();}
{
Indicate if border's shape is convex.
}

%END-AUTO(\ccOperations)


\ccHasModels

% The section below is automatically generated. Do not edit!
%START-AUTO(\ccHasModels)
%END-AUTO(\ccHasModels)


\ccSeeAlso

Some\_other\_concept,
\ccc{some_other_function}.


\ccExample

A short example program.
Instead of a short program fragment, a full running program can be
included using the
\verb|\ccIncludeExampleCode{Parameterization/BorderParametizer_3.C}|
macro. The program example would be part of the source code distribution and
also part of the automatic test suite.

\begin{ccExampleCode}
void your_example_code() {
}
\end{ccExampleCode}

%% \ccIncludeExampleCode{Parameterization/BorderParametizer_3.C}


\end{ccRefConcept}

% +------------------------------------------------------------------------+
%%RefPage: end of main body, begin of footer
% EOF
% +------------------------------------------------------------------------+

