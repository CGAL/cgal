% +------------------------------------------------------------------------+
% | Reference manual page: Taucs_solver_traits.tex
% +------------------------------------------------------------------------+
% | 21.09.2005   Laurent Saboret, Pierre Alliez
% | Package: Parameterization
% |
\RCSdef{\RCSTaucssolvertraitsRev}{$Revision$}
\RCSdefDate{\RCSTaucssolvertraitsDate}{$Date$}
% |
%%RefPage: end of header, begin of main body
% +------------------------------------------------------------------------+


\begin{ccRefClass}{Taucs_solver_traits}  %% add template arg's if necessary

%% \ccHtmlCrossLink{}     %% add further rules for cross referencing links
%% \ccHtmlIndexC[class]{} %% add further index entries


\ccDefinition

% The section below is automatically generated. Do not edit!
%START-AUTO(\ccDefinition)

Class Taucs\_solver\_traits is a traits class for solving GENERAL (aka unsymmetric) sparse linear systems using TAUCS out-of-core LU factorization.

%END-AUTO(\ccDefinition)

\ccInclude{CGAL/Taucs_solver_traits.h}


\ccIsModel

% The section below is automatically generated. Do not edit!
%START-AUTO(\ccIsModel)

Model of the SparseLinearAlgebraTraits\_d concept.

%END-AUTO(\ccIsModel)


\ccHeading{Design pattern}

% The section below is automatically generated. Do not edit!
%START-AUTO(\ccHeading\{Design pattern\})

None.

%END-AUTO(\ccHeading\{Design pattern\})


\ccParameters

The full template declaration is:

% The section below is automatically generated. Do not edit!
%START-AUTO(\ccParameters)

template$<$
class T$>$ 
class Taucs\_solver\_traits;

%END-AUTO(\ccParameters)


\ccTypes

% The section below is automatically generated. Do not edit!
%START-AUTO(\ccTypes)

\ccNestedType{Matrix}
{
}
\ccNestedType{Vector}
{
}
\ccNestedType{NT}
{
}

%END-AUTO(\ccTypes)


\ccConstants

% The section below is automatically generated. Do not edit!
%START-AUTO(\ccConstants)

None.

%END-AUTO(\ccConstants)


\ccCreation
\ccCreationVariable{a}  % choose variable name for \ccMethod

\ccConstructor{Taucs_solver_traits();}{default constructor.}

% The section below is automatically generated. Do not edit!
%START-AUTO(\ccCreation)
%END-AUTO(\ccCreation)


\ccOperations

% The section below is automatically generated. Do not edit!
%START-AUTO(\ccOperations)

\ccMethod{bool linear_solver (const Matrix & A, const Vector & B, Vector & X, NT & D);}
{
Solve the sparse linear system "A$\ast$X = B". Return true on success. The solution is then (1/D) $\ast$ X.
Preconditions:\begin{itemize}
\item A.row\_dimension() == B.dimension().\item A.column\_dimension() == X.dimension(). \end{itemize}
}
\ccMethod{bool is_solvable (const Matrix & A, const Vector & B);}
{
Indicate if the linear system can be solved and if the matrix conditioning is good.
Preconditions:\begin{itemize}
\item A.row\_dimension() == B.dimension(). \end{itemize}
}

%END-AUTO(\ccOperations)


\ccSeeAlso

\ccRefIdfierPage{CGAL::Taucs_symmetric_solver_traits<Traits>}  \\
\ccRefIdfierPage{CGAL::Taucs_matrix<Traits>}  \\
\ccRefIdfierPage{CGAL::Taucs_symmetric_matrix<Traits>}  \\
\ccRefIdfierPage{CGAL::Taucs_vector<Traits>}  \\
\ccRefIdfierPage{OpenNL::DefaultLinearSolverTraits<Traits>}  \\
\ccRefIdfierPage{OpenNL::SymmetricLinearSolverTraits<Traits>}  \\


\ccExample

This example program instantiates a polyhedron and a polyhedron adaptor
and parameterizes the polyhedron with Floater's mean value coordinates algorithm
using the TAUCS solver.

\ccIncludeExampleCode{Parameterization/Polyhedron_parameterization4.C}


\end{ccRefClass}

% +------------------------------------------------------------------------+
%%RefPage: end of main body, begin of footer
% EOF
% +------------------------------------------------------------------------+

