% +------------------------------------------------------------------------+
% | Reference manual page: Matrix.tex
% +------------------------------------------------------------------------+
% | 21.09.2005   Laurent Saboret, Pierre Alliez
% | Package: Parameterization
% |
\RCSdef{\RCSMatrixRev}{$Revision$}
\RCSdefDate{\RCSMatrixDate}{$Date$}
% |
%%RefPage: end of header, begin of main body
% +------------------------------------------------------------------------+


\begin{ccRefConcept}[SparseLinearAlgebra_d::]{Matrix}

%% \ccHtmlCrossLink{}     %% add further rules for cross referencing links
%% \ccHtmlIndexC[concept]{} %% add further index entries


\ccDefinition

% The section below is automatically generated. Do not edit!
%START-AUTO(\ccDefinition)

Concept SparseLinearAlgebraTraits\_d::Matrix is a concept of a sparse matrix class.

%END-AUTO(\ccDefinition)


\ccRefines

% The section below is automatically generated. Do not edit!
%START-AUTO(\ccRefines)

This is a sub-concept of LinearAlgebraTraits\_d::Matrix.

%END-AUTO(\ccRefines)


\ccTypes

% The section below is automatically generated. Do not edit!
%START-AUTO(\ccTypes)

\ccNestedType{NT}
{
}

%END-AUTO(\ccTypes)


\ccConstants

% The section below is automatically generated. Do not edit!
%START-AUTO(\ccConstants)

None.

%END-AUTO(\ccConstants)


\ccCreation
\ccCreationVariable{m}  %% choose variable name

% The section below is automatically generated. Do not edit!
%START-AUTO(\ccCreation)

\ccConstructor{Matrix (int dimension);}
{
Create a square matrix initialized with zeros.
}
\ccConstructor{Matrix (int rows, int columns);}
{
Create a rectangular matrix initialized with zeros.
}

%END-AUTO(\ccCreation)


\ccOperations

% The section below is automatically generated. Do not edit!
%START-AUTO(\ccOperations)

\ccMethod{int row_dimension () const;}
{
Return the matrix number of rows.
}
\ccMethod{int column_dimension () const;}
{
Return the matrix number of columns.
}
\ccMethod{NT get_coef (int row, int column) const;}
{
Read access to 1 matrix coefficient.
Preconditions:\begin{itemize}
\item 0 $<$= row $<$ row\_dimension().\item 0 $<$= column $<$ column\_dimension(). \end{itemize}
}
\ccMethod{void add_coef (int row, int column, NT value);}
{
Write access to 1 matrix coefficient: aij $<$- aij + val.
Preconditions:\begin{itemize}
\item 0 $<$= row $<$ row\_dimension().\item 0 $<$= column $<$ column\_dimension(). \end{itemize}
}
\ccMethod{void set_coef (int row, int column, NT value);}
{
Write access to 1 matrix coefficient.
Preconditions:\begin{itemize}
\item 0 $<$= row $<$ row\_dimension().\item 0 $<$= column $<$ column\_dimension(). \end{itemize}
}

%END-AUTO(\ccOperations)


\ccHasModels

% The section below is automatically generated. Do not edit!
%START-AUTO(\ccHasModels)

\begin{itemize}
\item Taucs\_matrix\item OpenNL::SparseMatrix \end{itemize}

%END-AUTO(\ccHasModels)


\ccSeeAlso

Some\_other\_concept,
\ccc{some_other_function}.


\ccExample

A short example program.
Instead of a short program fragment, a full running program can be
included using the
\verb|\ccIncludeExampleCode{Parameterization/Matrix.C}|
macro. The program example would be part of the source code distribution and
also part of the automatic test suite.

\begin{ccExampleCode}
void your_example_code() {
}
\end{ccExampleCode}

%% \ccIncludeExampleCode{Parameterization/Matrix.C}


\end{ccRefConcept}

% +------------------------------------------------------------------------+
%%RefPage: end of main body, begin of footer
% EOF
% +------------------------------------------------------------------------+

