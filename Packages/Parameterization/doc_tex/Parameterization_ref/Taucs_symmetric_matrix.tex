% +------------------------------------------------------------------------+
% | Reference manual page: Taucs_symmetric_matrix.tex
% +------------------------------------------------------------------------+
% | 21.09.2005   Laurent Saboret, Pierre Alliez
% | Package: Parameterization
% | 
\RCSdef{\RCSTaucssymmetricmatrixRev}{$Revision$}
\RCSdefDate{\RCSTaucssymmetricmatrixDate}{$Date$}
% |
%%RefPage: end of header, begin of main body
% +------------------------------------------------------------------------+


\begin{ccRefClass}{Taucs_symmetric_matrix}  %% add template arg's if necessary

%% \ccHtmlCrossLink{}     %% add further rules for cross referencing links
%% \ccHtmlIndexC[class]{} %% add further index entries


\ccDefinition

% The section below is automatically generated. Do not edit!
%START-AUTO(\ccDefinition)

Class Taucs\_symmetric\_matrix This class is a C++ wrapper around a TAUCS SYMMETRIC matrix (type taucs\_ccs\_matrix)

%END-AUTO(\ccDefinition)

\ccInclude{CGAL/Taucs_symmetric_matrix.h}


\ccIsModel

% The section below is automatically generated. Do not edit!
%START-AUTO(\ccIsModel)

Model of the SparseLinearAlgebraTraits\_d::Matrix concept.

%END-AUTO(\ccIsModel)


\ccParameters

The full template declaration is:

% The section below is automatically generated. Do not edit!
%START-AUTO(\ccParameters)

template$<$
class T$>$ 
class Taucs\_symmetric\_matrix;

%END-AUTO(\ccParameters)


\ccTypes

% The section below is automatically generated. Do not edit!
%START-AUTO(\ccTypes)

\ccNestedType{NT}
{
}

%END-AUTO(\ccTypes)


\ccConstants

% The section below is automatically generated. Do not edit!
%START-AUTO(\ccConstants)

None.

%END-AUTO(\ccConstants)


\ccCreation
\ccCreationVariable{a}  % choose variable name for \ccMethod

% The section below is automatically generated. Do not edit!
%START-AUTO(\ccCreation)

\ccConstructor{Taucs_symmetric_matrix (int dim, int nb_max_elements = 0);}
{
Create a square SYMMETRIC matrix initialized with zeros.
}
\begin{description}
\item[Parameters: ]
\begin{description}
\item[dim]Matrix dimension \item[nb\_max\_elements]Max number of non 0 elements in the matrix (automatically computed if 0) \end{description}
\end{description}
\ccConstructor{Taucs_symmetric_matrix (int rows, int columns, int nb_max_elements = 0);}
{
Create a square SYMMETRIC matrix initialized with zeros.
}
\begin{description}
\item[Parameters: ]
\begin{description}
\item[rows]Matrix dimensions \item[nb\_max\_elements]Max number of non 0 elements in the matrix (automatically computed if 0) \end{description}
\end{description}

%END-AUTO(\ccCreation)


\ccOperations

% The section below is automatically generated. Do not edit!
%START-AUTO(\ccOperations)

None.

%END-AUTO(\ccOperations)


\ccSeeAlso

\ccc{Some_other_class},
\ccc{some_other_function}.


\ccExample

A short example program.
Instead of a short program fragment, a full running program can be
included using the
\verb|\ccIncludeExampleCode{Parameterization/Taucs_symmetric_matrix.C}|
macro. The program example would be part of the source code distribution and
also part of the automatic test suite.

\begin{ccExampleCode}
void your_example_code() {
}
\end{ccExampleCode}

%% \ccIncludeExampleCode{Parameterization/Taucs_symmetric_matrix.C}


\end{ccRefClass}

% +------------------------------------------------------------------------+
%%RefPage: end of main body, begin of footer
% EOF
% +------------------------------------------------------------------------+

