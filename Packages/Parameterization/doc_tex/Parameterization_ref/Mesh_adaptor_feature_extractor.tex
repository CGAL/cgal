% +------------------------------------------------------------------------+
% | Reference manual page: Mesh_adaptor_feature_extractor.tex
% +------------------------------------------------------------------------+
% | 23.08.2005   Author
% | Package: Parameterization
% | 
\RCSdef{\RCSMeshadaptorfeatureextractorRev}{$Revision$}
\RCSdefDate{\RCSMeshadaptorfeatureextractorDate}{$Date$}
% |
%%RefPage: end of header, begin of main body
% +------------------------------------------------------------------------+


\begin{ccRefClass}{Mesh_adaptor_feature_extractor}  %% add template arg's if necessary

%% \ccHtmlCrossLink{}     %% add further rules for cross referencing links
%% \ccHtmlIndexC[class]{} %% add further index entries

\ccDefinition


Class Mesh\_adaptor\_feature\_extractor

This class computes features (genus, boundaries, ...) of a 3D surface model of the MeshAdaptor\_3 concept.

\ccInclude{Mesh_adaptor_feature_extractor.h}



\ccParameters

The full template declaration is:

template$<$
class MeshAdaptor\_3$>$ 
class Mesh\_adaptor\_feature\_extractor;


\ccTypes


\ccNestedType{Adaptor}
{
}
\ccNestedType{NT}
{
}
\ccNestedType{Facet_handle}
{
}
\ccNestedType{Facet_const_handle}
{
}
\ccNestedType{Vertex_handle}
{
}
\ccNestedType{Vertex_const_handle}
{
}
\ccNestedType{Point_3}
{
}
\ccNestedType{Point_2}
{
}
\ccNestedType{Vector_3}
{
}
\ccNestedType{Vector_2}
{
}
\ccNestedType{Facet_iterator}
{
}
\ccNestedType{Facet_const_iterator}
{
}
\ccNestedType{Vertex_iterator}
{
}
\ccNestedType{Vertex_const_iterator}
{
}
\ccNestedType{Border_vertex_iterator}
{
}
\ccNestedType{Border_vertex_const_iterator}
{
}
\ccNestedType{Vertex_around_facet_circulator}
{
}
\ccNestedType{Vertex_around_facet_const_circulator}
{
}
\ccNestedType{Vertex_around_vertex_circulator}
{
}
\ccNestedType{Vertex_around_vertex_const_circulator}
{
}
\ccNestedType{Boundary}
{
Mesh boundary.
}
\ccNestedType{Skeleton}
{
List of all boundaries of a mesh.
}


\ccCreation
\ccCreationVariable{fe}  %% choose variable name

\ccConstructor{Mesh_adaptor_feature_extractor (Adaptor * mesh);}
{
Constructor
CAUTION: Caller must NOT modify 'mesh' during the Mesh\_adaptor\_feature\_extractor life cycle.
}
\ccMethod{virtual ~Mesh_adaptor_feature_extractor ();}
{
}


\ccOperations

\ccMethod{int get_nb_boundaries ();}
{
Get number of boundaries.
}
\ccMethod{const Skeleton& get_boundaries ();}
{
Get extracted boundaries The longest boundary is the first one
}
\ccMethod{const Boundary* get_longest_boundary ();}
{
Get longest boundary.
}
\ccMethod{int get_nb_connex_components ();}
{
Get \# of connected components.
}
\ccMethod{int get_genus ();}
{
Get the genus.
}


\ccSeeAlso

\ccc{Some_other_class},
\ccc{some_other_function}.

\ccExample

A short example program.
Instead of a short program fragment, a full running program can be
included using the 
\verb|\ccIncludeExampleCode{Parameterization/Mesh_adaptor_feature_extractor.C}| 
macro. The program example would be part of the source code distribution and
also part of the automatic test suite.

\begin{ccExampleCode}
void your_example_code() {
}
\end{ccExampleCode}

%% \ccIncludeExampleCode{Parameterization/Mesh_adaptor_feature_extractor.C}

\end{ccRefClass}

% +------------------------------------------------------------------------+
%%RefPage: end of main body, begin of footer
% EOF
% +------------------------------------------------------------------------+

