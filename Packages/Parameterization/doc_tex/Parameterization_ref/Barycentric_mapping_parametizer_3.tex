% +------------------------------------------------------------------------+
% | Reference manual page: Barycentric_mapping_parametizer_3.tex
% +------------------------------------------------------------------------+
% | 23.08.2005   Laurent Saboret, Pierre Alliez
% | Package: Parameterization
% | 
\RCSdef{\RCSBarycentricmappingparametizerRev}{$Revision$}
\RCSdefDate{\RCSBarycentricmappingparametizerDate}{$Date$}
% |
%%RefPage: end of header, begin of main body
% +------------------------------------------------------------------------+


\begin{ccRefClass}{Barycentric_mapping_parametizer_3}  %% add template arg's if necessary

%% \ccHtmlCrossLink{}     %% add further rules for cross referencing links
%% \ccHtmlIndexC[class]{} %% add further index entries

\ccDefinition
  
The class \ccRefName\ serves as a traits class for all the two-dimensional
convex hull and extreme point calculation function.   This class can be
used to compute the convex hull of a set of 3D points projected onto the
$xy$ plane (\textit{i.e.}, by ignoring the $z$ coordinate).

\ccInclude{CGAL/Barycentric_mapping_parametizer_3.h}

\ccIsModel

\ccRefConceptPage{MeshAdaptor_3}%
\ccIndexSubitem[c]{MeshAdaptor_3}{model} \\

\ccTypes

\ccAutoIndexingOff
\ccSetThreeColumns{typedef R::Less_dist_to_line_plane_xy_2<Point_3>}{Less_signed_distance_to_line_2}{}
\ccThreeToTwo

\ccTypedef{typedef Point_3                        Point_2;}{}
\ccGlue
\ccTypedef{typedef Less_xy_plane_xy_2<Point_3>    Less_xy_2;}{}
\ccGlue
\ccTypedef{typedef Less_yx_plane_xy_2<Point_3>    Less_yx_2;}{}
\ccGlue
\ccTypedef{typedef Less_dist_to_line_plane_xy_2<Point_3> 
                                      Less_signed_distance_to_line_2;}{}
\ccGlue
\ccTypedef{typedef Less_rotate_ccw_plane_xy_2<Point_3> Less_rotate_ccw_2;}{}
\ccGlue
\ccTypedef{typedef Left_turn_plane_xy_2<Point_3>        Left_turn_2;}{}
\ccGlue
\ccTypedef{typedef Equal_xy_plane_xy_2<Point_3>         Equal_2;}{}

\ccCreation
\ccCreationVariable{traits}  %% choose variable name

\ccConstructor{Barycentric_mapping_parametizer_3();}{default constructor.}

\ccOperations

\ccMemberFunction{Less_xy_2 less_xy_2_object(); }{}
\ccGlue
\ccMemberFunction{Less_yx_2 less_yx_2_object(); }{}
\ccGlue
\ccMemberFunction{Less_signed_distance_to_line_2 
                  less_signed_distance_to_line_2_object();}{}
\ccGlue
\ccMemberFunction{Less_rotate_ccw_2 
                  less_rotate_ccw_2_object(); }{}
\ccGlue
\ccMemberFunction{Left_turn_2 left_turn_2_object(); }{}
\ccGlue
\ccMemberFunction{Equal_2 equal_2_object(); }{}


\ccSeeAlso

\ccc{Some_other_class},
\ccc{some_other_function}.

\ccExample

A short example program.
Instead of a short program fragment, a full running program can be
included using the 
\verb|\ccIncludeExampleCode{Parameterization/Barycentric_mapping_parametizer_3.C}| 
macro. The program example would be part of the source code distribution and
also part of the automatic test suite.

\begin{ccExampleCode}
void your_example_code() {
}
\end{ccExampleCode}

%% \ccIncludeExampleCode{Parameterization/Barycentric_mapping_parametizer_3.C}

\end{ccRefClass}

% +------------------------------------------------------------------------+
%%RefPage: end of main body, begin of footer
% EOF
% +------------------------------------------------------------------------+

