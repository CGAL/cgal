% +------------------------------------------------------------------------+
% | Reference manual page: ParametizerTraits_3.tex
% +------------------------------------------------------------------------+
% | 21.09.2005   Laurent Saboret, Pierre Alliez
% | Package: Parameterization
% | 
\RCSdef{\RCSParametizerTraitsRev}{$Revision$}
\RCSdefDate{\RCSParametizerTraitsDate}{$Date$}
% |
%%RefPage: end of header, begin of main body
% +------------------------------------------------------------------------+


\begin{ccRefConcept}{ParametizerTraits_3}

%% \ccHtmlCrossLink{}     %% add further rules for cross referencing links
%% \ccHtmlIndexC[concept]{} %% add further index entries


\ccDefinition

% The section below is automatically generated. Do not edit!
%START-AUTO(\ccDefinition)

Concept of parameterization objects for a given type of mesh, 'Adaptor', which is a model of the MeshAdaptor\_3 concept.

%END-AUTO(\ccDefinition)


\ccHeading{Design pattern}

% The section below is automatically generated. Do not edit!
%START-AUTO(\ccHeading\{Design pattern\})

None.

%END-AUTO(\ccHeading\{Design pattern\})


\ccRefines

% The section below is automatically generated. Do not edit!
%START-AUTO(\ccRefines)

None.

%END-AUTO(\ccRefines)


\ccTypes

% The section below is automatically generated. Do not edit!
%START-AUTO(\ccTypes)

\ccNestedType{Adaptor}
{
}
\ccNestedType{NT}
{
}

%END-AUTO(\ccTypes)


\ccConstants

% The section below is automatically generated. Do not edit!
%START-AUTO(\ccConstants)

\ccEnum{enum ErrorCode {...}}
{
The various errors detected by this package.
}
\begin{description}
\item[Enumerator: ]
\begin{description}
\item[OK
]\item[ERROR\_EMPTY\_MESH
]input mesh is empty \item[ERROR\_NON\_TRIANGULAR\_MESH
]input mesh is not triangular \item[ERROR\_NO\_SURFACE\_MESH
]input mesh is not a surface \item[ERROR\_INVALID\_BOUNDARY
]parameterization requires a convex border \item[ERROR\_BAD\_MATRIX\_CONDITIONING
]result is mathematically unstable \item[ERROR\_CANNOT\_SOLVE\_LINEAR\_SYSTEM
]cannot solve linear system \item[ERROR\_NO\_1\_TO\_1\_MAPPING
]parameterization does not ensure 1 to 1 mapping \item[ERROR\_NOT\_ENOUGH\_MEMORY
]it's time to buy some RAM :-) \item[ERROR\_WRONG\_PARAMETER
]a method received an unexpected parameter \end{description}
\end{description}

%END-AUTO(\ccConstants)


\ccCreation
\ccCreationVariable{p}  % variable name for \ccMethod

Construction and destruction are undefined.

% The section below is automatically generated. Do not edit!
%START-AUTO(\ccCreation)
%END-AUTO(\ccCreation)


\ccOperations

% The section below is automatically generated. Do not edit!
%START-AUTO(\ccOperations)

\ccMethod{ErrorCode parameterize (Adaptor * mesh);}
{
Compute a 1 to 1 mapping from a triangular 3D surface 'mesh' to a piece of the 2D space. The mapping is linear by pieces (linear in each triangle). The result is the (u,v) pair image of each vertex of the 3D surface.
Preconditions:\begin{itemize}
\item 'mesh' must be a surface with 1 connected component and no hole\item 'mesh' must be a triangular mesh \end{itemize}
}

%END-AUTO(\ccOperations)


\ccHasModels

% The section below is automatically generated. Do not edit!
%START-AUTO(\ccHasModels)
%END-AUTO(\ccHasModels)


\ccSeeAlso

Some\_other\_concept,
\ccc{some_other_function}.


\ccExample

A short example program.
Instead of a short program fragment, a full running program can be
included using the
\verb|\ccIncludeExampleCode{Parameterization/ParametizerTraits_3.C}|
macro. The program example would be part of the source code distribution and
also part of the automatic test suite.

\begin{ccExampleCode}
void your_example_code() {
}
\end{ccExampleCode}

%% \ccIncludeExampleCode{Parameterization/ParametizerTraits_3.C}


\end{ccRefConcept}

% +------------------------------------------------------------------------+
%%RefPage: end of main body, begin of footer
% EOF
% +------------------------------------------------------------------------+

