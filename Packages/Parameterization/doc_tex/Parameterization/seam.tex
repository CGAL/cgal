\section{Cutting a Mesh}

\subsection{Computing a Cut}

Surface parameterization methods deal only with topological discs.
The input mesh can be of any genus and have any number of connected components.
If it is not a topological disc, it has to come with a description of a border
(a list of vertices) which is the border of a topological disc.
If no border is given, we assume that the surface border
is the longest border already in the input mesh (the other borders will
be considered as holes).

This package does not provide algorithms that transform
a closed mesh of arbitrary genus into a topological disk.
The package user is responsible for computing such a cut.

We provide in \ccc{polyhedron_ex_parameterization.C} a simple cutting algorithm as example.
Serious users will have to use a more robust method.


\subsection{Applying a Cut}

Parameterization methods in this package support only
triangulated surfaces that are homeomorphic to a
disk (models of \ccc{ParameterizationMesh_3}). This software design simplifies the implementation of new
parameterization methods.

\ccc{Parameterization_mesh_patch_3} class is responsible of virtually {\em cutting}
a patch in a \ccc{ParameterizationPatchableMesh_3} mesh, to make it appear as a topological disk
with a \ccc{ParameterizationMesh_3} interface.

\ccc{ParameterizationPatchableMesh_3} inherits from concept \ccc{ParameterizationMesh_3}, thus is a concept for a 3D surface mesh.
\ccc{ParameterizationPatchableMesh_3} adds the ability to support patches and virtual seams. Patches are a subset of a 3D mesh. Virtual seams are the ability to behave exactly as if the surface was {\em cut} following a certain path.

The \ccc{ParameterizationMesh_3} interfaces with both the 2D Triangulation Data Structure enriched
with 3D points (not yet implemented) and the Polyhedron are also models of \ccc{ParameterizationPatchableMesh_3}:

\ccc{CGAL::Parameterization_polyhedron_adaptor_3}  \\


\subsection{Cutting a Mesh Example}

The code below virtually {\em cuts} a \ccc{Polyhedron_3} mesh to make it a topological disk,
then applies the default parameterization:

\begin{ccExampleCode}

// CGAL kernel
typedef CGAL::Cartesian<double>                             Kernel;

// Mesh true type and parameterization adaptors
typedef CGAL::Polyhedron_3<Kernel>                          Polyhedron;
typedef CGAL::Parameterization_polyhedron_adaptor_3<Polyhedron>         
                                                            Parameterization_polyhedron_adaptor;
typedef CGAL::Parameterization_mesh_patch_3<Parameterization_polyhedron_adaptor> 
                                                            Mesh_patch_polyhedron;

// Parameterizers base class for this kind of mesh
typedef CGAL::Parameterizer_traits_3<Mesh_patch_polyhedron> Parameterizer;

// Type describing a border or seam as a vertex list
typedef std::list<Parameterization_polyhedron_adaptor::Vertex_handle>   
                                                            Seam;

// If the mesh is a topological disk, extract its longest border,
// else compute a very simple cut to make it homeomorphic to a disk.
// Return the border/seam (empty on error)
static Seam cut_mesh(Parameterization_polyhedron_adaptor* mesh_adaptor)
{
    // To be implemented by package user
    ...
}

int main(int argc,char * argv[])
{
    Polyhedron mesh;
    ...

    // The parameterization package needs an adaptor to handle Polyhedron_3 meshes
    Parameterization_polyhedron_adaptor mesh_adaptor(&mesh);

    // The parameterization methods support only meshes that
    // are topological disks => we need to compute a "cutting" of the mesh
    // that makes it it homeomorphic to a disk
    Seam seam = cut_mesh(&mesh_adaptor);

    // Create adaptor that virtually "cuts" the mesh following the 'seam' path
    Mesh_patch_polyhedron   mesh_patch(&mesh_adaptor,
                                       seam.begin(),
                                       seam.end());

    // Floater Mean Value Coordinates parameterization
    Parameterizer::Error_code err = CGAL::parameterize(&mesh_patch);
    ...
}

\end{ccExampleCode}

See the complete code in \ccc{Mesh_cutting_parameterization.C} example.


