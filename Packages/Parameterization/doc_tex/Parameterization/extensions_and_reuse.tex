\section{Extending the Package and Reusing Code}

\subsection{Reusing Mesh Adaptors}

\ccc{ParameterizationMesh_3} defines a concept to access to a general polyhedral mesh.
The current interface is optimized for the parameterization package,
but may be easily generalized.

The package proposes
a \ccc{ParameterizationMesh_3} interface with both the 2D Triangulation Data Structure enriched
with 3D points and the Polyhedron.

Any algorithm that must work on both \ccc{Triangulation_data_structure_2} with 3D points
and \ccc{Polyhedron_3} may take advantage of these adaptors.


\subsection{Reusing Sparse Linear Algebra}

The \ccc{SparseLinearAlgebraTraits_d} concept and the traits classes for OpenNL,
{\sc Taucs} and SuperLU are totaly independent of the rest of the package,
and may be reused directly by CGAL developers.


\subsection{Adding New Parameterization Methods}

Implementing a new fixed border linear parameterization is very easy.
Most of the code of the fixed border methods
is factorized in the \ccc{Fixed_border_parameterizer_3} class.
Subclasses must mainly implement a \ccc{compute_w_ij}() method
that computes $w_{ij}$ = (i,j) coefficient of matrix A for $v_j$ neighbor vertex of $v_i$.

Implementing a new free border linear parameterization is more complex.
Anyway, Least Squares Conformal Maps and Natural Conformal Map
are good starting points.

Implementing \ccc{non} linear parameterizations would be a nice improvement
of this package. In the other hand, most of the work remains to do.
Only the mesh adaptors can be directly reused.


\subsection{Adding New Border Parameterization Methods}

Implementing a new border parameterization method is easy.
Square, Circular and 2-Points border parameterizations are good starting points.


\subsection{Mesh Cutting}

Obviously, this package would benefit of robust algorithms that transform
a closed mesh of arbitrary genus into a topological disk.
