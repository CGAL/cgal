\section{Extending the Package and Reusing Code}

\subsection{Reusing Mesh Adaptors}

MeshAdaptor\_3 defines a concept to access to a general polyhedral mesh.
The current interface is optimized for the parameterization package,
but may be easily generalized.

The package proposes
a MeshAdaptor\_3 interface with both the 2D Triangulation Data Structure enriched
with 3D points and the Polyhedron.

Any algorithm that must work on both TDS\_2 with 3D points and the Polyhedron
may take advantage of these adaptors.


\subsection{Reusing Sparse Linear Algebra}

The SparseLinearAlgebraTraits\_d concept and the traits classes for OpenNL,
TAUCS and SuperLU are totaly independent of the rest of the package,
and may be reused directly by CGAL developers.


\subsection{Adding New Parameterization Methods}

Implementing a new fixed border linear parameterization is very easy.
Most of the code of the fixed border methods
is factorized in the Fixed\_border\_parametizer\_3 class.
Subclasses must mainly implement a compute\_wij() method
that computes $wij$ = (i,j) coefficient of matrix A for $vj$ neighbor vertex of $vi$.

Implementing a new free border linear parameterization is more complex.
Anyway, LSCM and Natural Conformal parameterizations are good starting points.

Implementing \ccc{non} linear parameterizations would be a nice improvment
of this package. In the other hand, most of the work remains to do.
Only the mesh adaptors can be directly reused.


\subsection{Adding New Boundary Parameterization Methods}

Implementing a new boundary parameterization method is easy.
Square, Circular and 2-Points boundary parameterizations are good starting points.


\subsection{Mesh Cutting}

Obviously, this package would benefit of robust algorithms that transform
a closed mesh of arbitrary genus into a topological disk.
