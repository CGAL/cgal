% +------------------------------------------------------------------------+
% | CGAL Reference Manual:  generators.tex
% +------------------------------------------------------------------------+
% | Random sources and geometric object generators.
% |
% | 09.06.1997   Lutz Kettner
% | 
\RCSdef{\generatorsRev}{$Revision$}
\RCSdefDate{\generatorsDate}{$Date$}
% +------------------------------------------------------------------------+

\beforecprogskip\medskipamount
\aftercprogskip\medskipamount
\ccParDims

\chapter{Random Sources and Geometric Object Generators}
\label{chapterGenerators}
\ccChapterSubTitle{\generatorsRev. \ \generatorsDate}\\
\ccChapterAuthor{Lutz Kettner}\\
\ccChapterAuthor{Sven Sch\"onherr}


A variety of generators for random numbers and geometric objects is
provided in \cgal.  They are useful as synthetic test data sets,
e.g.~for testing algorithms on degenerate object sets and for
performance analysis.

The first section describes the random number source used for random
generators. The second section documents generators for point sets,
the third section for segments. Note that the \stl\ algorithm
\ccc{random_shuffle} is useful in this context to achieve random
permutations (e.g.~for points on a grid).


% +------------------------------------------------------------------------+
% =============================================================================
% The CGAL Reference Manual
% Chapter: Generators
% Section: Random Numbers Generator
% -----------------------------------------------------------------------------
% file   : doc_tex/support/Generator/Random.tex
% package: Random_numbers
% author : Sven Sch�nherr <sven@inf.ethz.ch>
% -----------------------------------------------------------------------------
% $Revision$
% $Date$
% =============================================================================

\begin{ccClass}{Random}
\section{Random Numbers Generator}
\label{sec:random_numbers_generator}

% -----------------------------------------------------------------------------
\ccDefinition

An instance of type \ccClassName\ is a random numbers generator. It
allows to generate uniformly distributed random \ccc{bool}s,
\ccc{int}s and \ccc{double}s. It can be used as the random number
generating function object in the STL algorithm \ccc{random_shuffle}.

Instances of \ccClassName\ can be seen as input streams. Different
streams are \emph{independent} of each other, i.e.\ the sequence of
numbers from one stream does \emph{not} depend upon how many numbers
were extracted from the other streams.

It can be very useful, e.g.\ for debugging, to reproduce a sequence of
random numbers.  This can be done by either initialising
deterministically or using the state functions as described below.

\ccInclude{CGAL/Random.h}

% -----------------------------------------------------------------------------
\ccHeading{Global Variables}

\ccVariable{ Random default_random;}{
          This global variable is used as the
          default random numbers generator.}

% -----------------------------------------------------------------------------
\ccTypes

\ccUnchecked
\ccNestedType{State}{State type.}

% -----------------------------------------------------------------------------
\ccCreation
\ccCreationVariable{random}

\ccConstructor{ Random( );}{
        introduces a variable \ccVar\ of type \ccClassTemplateName.}

\ccConstructor{ Random( long seed);}{
        introduces a variable \ccVar\ of type \ccClassTemplateName\
        and initializes its internal state using \ccc{seed}. Equal
        values for \ccc{seed} result in equal sequences of random
        numbers.}

\ccConstructor{ Random( State state);}{
        introduces a variable \ccVar\ of type \ccClassTemplateName\
        and initializes its internal state with \ccc{state}.}

% -----------------------------------------------------------------------------
\ccOperations

\ccMemberFunction{ bool get_bool( );}{
        returns a random \ccc{bool}.}

\ccMemberFunction{ int get_int( int lower, int upper);}{
        returns a random \ccc{int} from the interval
        $[\mbox{\ccc{lower},\ccc{upper}})$.}

\ccMemberFunction{ double get_double( double lower = 0.0,
                                      double upper = 1.0);}{
        returns a random \ccc{double} from the interval
        $[\mbox{\ccc{lower},\ccc{upper}})$.}

\ccMemberFunction{ int operator() ( int upper);}{
        returns \ccVar\ccc{.get_int( 0, upper)}.}

% -----------------------------------------------------------------------------
\ccHeading{State Functions}

\ccMemberFunction{ void save_state( State& state) const;}{
        saves the current internal state in \ccc{state}.}

\ccMemberFunction{ void restore_state( State const& state);}{
        restores the internal state from \ccc{state}.}

% -----------------------------------------------------------------------------
\ccHeading{Equality Test}

\ccMemberFunction{ bool  operator == ( Random const& random2) const;}{
        returns \ccc{true}, iff \ccVar\ and \ccc{random2} have equal
        internal states.}

% -----------------------------------------------------------------------------
\ccImplementation

We use the C library function \ccc{erand48} to generate the random
numbers, i.e.\ the sequence of numbers depends on the implementation
of \ccc{erand48} on your specific platform.

\end{ccClass}

% ===== EOF ===================================================================


% +------------------------------------------------------------------------+
%\newpage
\section{Support Functions for Generators}
\ccThree{OutputIterator}{rand}{}

Two support functions are provided. \ccc{CGAL_copy_n()} copies $n$
items from an input iterator to an output iterator which is useful for
possibly infinite sequences of random geometric objects\footnote{%
The STL release June 13, 1997, from SGI has a new function \ccc{copy_n} which is equivalent with \ccc{CGAL_copy_n}.}.
\ccc{CGAL_random_selection} chooses $n$ items at random from a random
access iterator range which is useful to produce degenerate input data
sets with multiple entries of identical items.

\subsection{{\it CGAL\_copy\_n()}}
\label{sectionCopyN}

\ccInclude{CGAL/copy_n.h}

\ccFunction{template <class InputIterator, class Size, class OutputIterator>
  OutputIterator CGAL_copy_n( InputIterator first, Size n, 
  OutputIterator result);}
{copies the first $n$ items from \ccc{first} to \ccc{result}.
    Returns the value of \ccc{result} after inserting the $n$ items.}

\subsection{{\it CGAL\_random\_selection()}}
\label{sectionRandomSelection}

\ccInclude{CGAL/random_selection.h}

\ccFunction{template <class RandomAccessIterator, class Size, 
                      class OutputIterator, class Random>
    OutputIterator CGAL_random_selection( RandomAccessIterator first,
        RandomAccessIterator last, 
        Size n, OutputIterator result, Random& rnd = CGAL_random);}
{ choose a random item from the range $[\ccc{first},\ccc{last})$ and
    write it to \ccc{result}, each item from the range with equal
    probability. Repeat this $n$ times, thus writing $n$ items to
    \ccc{result}.
    A single random number is needed from \ccc{rnd} for each item.
    Returns the value of \ccc{result} after inserting the $n$ items.
    \ccPrecond \ccc{Random} is a random number generator type as provided 
    by the STL or by \ccc{CGAL_Random}.
}



% +------------------------------------------------------------------------+
\newpage
\section{2D Point Generators}

\ccDefinition

Point generators are provided for random points uniformly distributed
over a two-dimensional domain (square or disc) or a one-dimensional
domain (boundary of a square, circle, or segment). Other point
generators create two-dimensional grids or equally spaced points on a
segment. A perturbation function adds random noise to a given set of
points. Several functions add degeneracies: the duplication of randomly
chosen points and the construction of a collinear point between two randomly
chosen points from a set of points.


\ccInclude{CGAL/point_generators_2.h}

\ccCreation

The random point generators build two-dimensional points from a pair
of \ccc{double}'s. Depending on the point representation and
arithmetic, a different building process is necessary. It is
encapsulated in the global function \ccc{CGAL_build_point()}.
Implementations exist for \ccc{CGAL_Cartesian<double>} and
\ccc{CGAL_Cartesian<float>}. They are automatically included if the
representation type \ccc{CGAL_Cartesian} has been included beforehand.
For other representations and arithmetic types the function can be
overloaded.

\ccThree{void}{random}{}
\ccFunction{void CGAL_build_point( double x, double y, Point&
  p);}{builds a point $(x,y)$ in $p$. \ccc{Point} is the point type in
  question.}

\ccHeading{Random Points}

The random point generators are implemented as classes that satisfies
the requirements for input iterators. They represent the possibly
infinite sequence of randomly generated points. Each call to the
\ccc{operator*} returns a new point. To create a finite sequence in a
container, the function \ccc{CGAL_copy_n()} could be used, see
Section~\ref{sectionCopyN}.

\ccHtmlNoClassFile
\begin{ccClassTemplate}{CGAL_Random_points_in_disc_2<P>}
\ccCreationVariable{g}
\ccConstructor{CGAL_Random_points_in_disc_2( double r, CGAL_Random& rnd =
  CGAL_random);}{%
  $g$ is an input iterator creating points of type \ccc{P} uniformly
  distributed in the open disc with radius $r$,
  i.e.~$|\ccc{*g}| < r$~. Two random numbers are needed from
  \ccc{rnd} for each point.
  \ccPrecond a function \ccc{CGAL_build_point()} for the point type
  \ccc{P} exists.} 
\end{ccClassTemplate}

\ccHtmlNoClassFile
\begin{ccClassTemplate}{CGAL_Random_points_on_circle_2<P>}
\ccCreationVariable{g}
\ccConstructor{CGAL_Random_points_on_circle_2( double r, CGAL_Random& rnd =
  CGAL_random);}{%
  $g$ is an input iterator creating points of type \ccc{P} uniformly
  distributed on the circle with radius $r$,
  i.e.~$|\ccc{*g}| == r$~. A single random number is needed from
  \ccc{rnd} for each point.
  \ccPrecond a function \ccc{CGAL_build_point()} for the point type
  \ccc{P} exists.} 
\end{ccClassTemplate}

\ccHtmlNoClassFile
\begin{ccClassTemplate}{CGAL_Random_points_in_square_2<P>}
\ccCreationVariable{g}
\ccConstructor{CGAL_Random_points_in_square_2( double a, CGAL_Random& rnd =
  CGAL_random);}{%
  $g$ is an input iterator creating points of type \ccc{P} uniformly
  distributed in the half-open square with side length $a$, centered
  at the origin, i.e.~$\forall p = \ccc{*g}:  -\frac{a}{2} \le
  p.x() < \frac{a}{2}$ and $-\frac{a}{2} \le p.y() < \frac{a}{2}$~. 
  Two random numbers are needed from \ccc{rnd} for each point.
  \ccPrecond a function \ccc{CGAL_build_point()} for the point type
  \ccc{P} exists.} 
\end{ccClassTemplate}

\ccHtmlNoClassFile
\begin{ccClassTemplate}{CGAL_Random_points_on_square_2<P>}
\ccCreationVariable{g}
\ccConstructor{CGAL_Random_points_on_square_2( double a, CGAL_Random& rnd =
  CGAL_random);}{%
  $g$ is an input iterator creating points of type \ccc{P} uniformly
  distributed on the boundary of the square with side length $a$,
  centered at the origin, i.e.~$\forall p = \ccc{*g}:$ one
  coordinate is either $\frac{a}{2}$ or $-\frac{a}{2}$ and for the 
  other coordinate $c$ holds $-\frac{a}{2} \le c < \frac{a}{2}$~.
  A single random number is needed from \ccc{rnd} for each point.
  \ccPrecond a function \ccc{CGAL_build_point()} for the point type
  \ccc{P} exists.} 
\end{ccClassTemplate}

\ccHtmlNoClassFile
\begin{ccClassTemplate}{CGAL_Random_points_on_segment_2<P>}
\ccCreationVariable{g}
\ccConstructor{CGAL_Random_points_on_segment_2( const P& p, const P& q,
  CGAL_Random& rnd = CGAL_random);}{%
  $g$ is an input iterator creating points of type \ccc{P} uniformly
  distributed on the segment from $p$ to $q$ except $q$,
  i.e.~$\ccc{*g} == (1-\lambda)\, p + \lambda q$ where $0 \le \lambda < 1$~.
  A single random number is needed from \ccc{rnd} for each point.
  \ccPrecond a function \ccc{CGAL_build_point()} for the point type \ccc{P}
    exists. The expressions \ccc{CGAL_to_double(p.x())} and
    \ccc{CGAL_to_double(p.y())} must  result in the respective
    \ccc{double} representation of the coordinates and similar for $q$.}
\end{ccClassTemplate}

\ccHeading{Grid Points}
\ccThree{OutputIterator}{rand}{}

Grid points are produced by generating functions writing to an output
iterator.

\ccFunction{template <class OutputIterator>
    OutputIterator
    CGAL_points_on_square_grid_2( double a, size_t n, OutputIterator o,
                                  const P*);}
{ creates the $n$ first points on the regular $\lceil\sqrt{n}\,\rceil
    \times \lceil  \sqrt{n}\,\rceil$ grid within the square
  $[-\frac{a}{2},\frac{a}{2}]\times [-\frac{a}{2},\frac{a}{2}]$.
  Returns the value of $o$ after inserting the $n$ points.
  \ccPrecond a function \ccc{CGAL_build_point()} for the point type
  $P$ and $P$ must be assignable to the value type of 
  \ccc{OutputIterator}.} 

\ccFunction{template <class P, class OutputIterator>
    OutputIterator CGAL_points_on_segment_2( const P& p, const P& q, size_t n,
    OutputIterator o);}
{ creates $n$ points regular spaced on the segment from $p$ to $q$,
    i.e.~$\forall i: 0 \le i < n: o[i] := \frac{n-i-1}{n-1}\, p +
    \frac{i}{n-1}\, q$. Returns the value of $o$ after inserting
    the $n$ points.}

\ccHeading{Random Perturbations}

Degenerate input sets like grid points can be randomly perturbed by a
small amount to produce {\em quasi}-degenerate test sets. This
challenges numerical stability of algorithms using inexact arithmetic and
exact predicates to compute the sign of expressions slightly off from zero.

\ccFunction{template <class ForwardIterator>
    void CGAL_perturb_points_2( ForwardIterator first, ForwardIterator last, 
        double xeps, double yeps = xeps, CGAL_Random& rnd = CGAL_random);}
{ perturbs the points in the range $[\ccc{first},\ccc{last})$ by
  replacing each point with a random point from the rectangle
  \ccc{xeps} $\times$ \ccc{yeps} centered at the original point.
  Two random numbers are needed from \ccc{rnd} for each point.
  \ccPrecond a function \ccc{CGAL_build_point()} for the value type of
    the \ccc{ForwardIterator} exists. 
    The expressions \ccc{CGAL_to_double((*first).x())} and
    \ccc{CGAL_to_double((*first).y())} must result in the respective
    coordinate values.
}

\ccHeading{Adding Degeneracies}

For a given point set certain kinds of degeneracies can be produced
adding new points. The \ccc{CGAL_random_selection()} function is
useful to generate multiple entries of identical points, see
Section~\ref{sectionRandomSelection}. The
\ccc{CGAL_random_collinear_points_2()} function adds collinearities to
a point set.


\ccFunction{template <class RandomAccessIterator, class OutputIterator>
    OutputIterator CGAL_random_collinear_points_2( RandomAccessIterator first,
        RandomAccessIterator last, 
        size_t n, OutputIterator first2, CGAL_Random& rnd = CGAL_random);}
{ randomly chooses two points from the range $[\ccc{first},\ccc{last})$,
    creates a random third point on the segment connecting this two
    points, and writes it to \ccc{first2}. Repeats this $n$ times, thus
    writing $n$ points to \ccc{first2} that are collinear with points
    in the range $[\ccc{first},\ccc{last})$.
    Three random numbers are needed from \ccc{rnd} for each point.
    Returns the value of \ccc{first2} after inserting the $n$ points.
  \ccPrecond a function \ccc{CGAL_build_point()} for the value type of
    the \ccc{ForwardIterator} exists. 
    The expressions \ccc{CGAL_to_double((*first).x())} and
    \ccc{CGAL_to_double((*first).y())} must result in the respective
    coordinate values.
}

\ccExample

We want to generate a test set of 1000 points, where 60\% are chosen
randomly in a small disc, 20\% are from a larger grid, 10\% duplicates
are added, and 10\% collinearities added. A random shuffle removes the
construction order from the test set. See \ccTexHtml{%
Figure~\ref{figurePointGenerator}}{Figure <A HREF="#PointGenerators">
  <IMG SRC="cc_ref_up_arrow.gif" ALT="reference arrow" WIDTH="10"
  HEIGHT="10"></A>} for the example output.

\cprogfile{generators_prog1.C}

\begin{ccTexOnly}
  \begin{figure}
    \noindent
    \hspace*{0.025\textwidth}%
    \begin{minipage}{0.45\textwidth}%
      \epsfig{figure=generators_prog1.ps,width=\textwidth}
      \caption{Output of example program for point generators.}
      \label{figurePointGenerator}
    \end{minipage}%
    \hspace*{0.05\textwidth}%
    \begin{minipage}{0.45\textwidth}%
      \epsfig{figure=generators_prog2.ps,width=\textwidth}%      
      \caption{Output of example program for point generators working
        on integer points.}
      \label{figureIntegerPointGenerator}
    \end{minipage}%
  \end{figure}
\end{ccTexOnly}

\begin{ccHtmlOnly}
  <A NAME="PointGenerators">
  <TABLE><TR><TD ALIGN=LEFT VALIGN=TOP WIDTH=60%>
    <A HREF="./generators_prog1.gif">Figure:</A>
    Output of example program for point generators.
  </TD><TD ALIGN=LEFT VALIGN=TOP WIDTH=5% NOWRAP>
  </TD><TD ALIGN=LEFT VALIGN=TOP WIDTH=35% NOWRAP>
    <A HREF="./generators_prog1.gif">
        <img src="./generators_prog1_small.gif" 
             alt="Point Generator Example Output"></A>
  </TD></TR></TABLE>
\end{ccHtmlOnly}


The second example demonstrates the point generators with integer
points. Arithmetic with \ccc{double}'s is sufficient to produce
regular integer grids. See \ccTexHtml{%
Figure~\ref{figureIntegerPointGenerator}}{Figure 
  <A HREF="#IntegerPointGenerators">
  <IMG SRC="cc_ref_up_arrow.gif" ALT="reference arrow" WIDTH="10"
  HEIGHT="10"></A>}
for the example output.

\cprogfile{generators_prog2.C}

\begin{ccHtmlOnly}
  <A NAME="IntegerPointGenerators">
  <TABLE><TR><TD ALIGN=LEFT VALIGN=TOP WIDTH=60%>
    <A HREF="./generators_prog2.gif">Figure:</A>
        Output of example program for point generators working
        on integer points.
  </TD><TD ALIGN=LEFT VALIGN=TOP WIDTH=5% NOWRAP>
  </TD><TD ALIGN=LEFT VALIGN=TOP WIDTH=35% NOWRAP>
    <A HREF="./generators_prog2.gif">
        <img src="./generators_prog2_small.gif" 
             alt="Integer Point Generator Example Output"></A>
  </TD></TR></TABLE>
\end{ccHtmlOnly}


% +------------------------------------------------------------------------+
\newpage
\section{2D Segment Generators}

The following generic segment generator uses two point generators to
create a segment from two endpoints. This is a design example how
further generators could look like -- for segments and for other
higher level objects.


% +------------------------------------------------------------------------+
\begin{ccClassTemplate}{CGAL_Segment_generator<S,P1,P2>}
\ccCreationVariable{g}
\subsection{A Generic Segment Generator from Two Points}

\ccDefinition

The generic segment generator \ccClassTemplateName\ uses two point
generators \ccc{P1} and \ccc{P2} to create a segment of type $S$ from
two endpoints.  \ccClassTemplateName\ satisfies the requirements for
an input iterator. It represents the possibly infinite sequence of
generated segments. Each call to the \ccc{operator*} returns a new
segment. To create a finite sequence in a container, the function
\ccc{CGAL_copy_n()} could be used, see Section~\ref{sectionCopyN}.

\ccInclude{CGAL/Segment_generator_2.h}

\ccCreation

\ccConstructor{CGAL_Segment_generator( P1& p1, P2& p2);}{%
  $g$ is an input iterator creating segments of type \ccc{S} from two 
  input points, one chosen from \ccc{p1}, the other chosen from
  \ccc{p2}. \ccc{p1} and \ccc{p2} are allowed be be same point
  generator.\ccPrecond $S$ must provide a constructor with two
  arguments, such that the value type of \ccc{P1} and the
  value type of \ccc{P2} can be used to construct a segment.}

\ccOperations

$g$ satisfies the requirements of an input iterator. Each call to the
\ccc{operator*} returns a new segment. 

\ccExample

We want to generate a test set of 200 segments, where one endpoint is
chosen randomly from a horizontal segment of length 200, and the other
endpoint is chosen randomly from a circle of radius 250. See
\ccTexHtml{Figure~\ref{figureSegmentGenerator}}{Figure <A
  HREF="#SegmentGenerator"> <IMG SRC="cc_ref_up_arrow.gif"
  ALT="reference arrow" WIDTH="10" HEIGHT="10"></A>} for the example
output.

\begin{ccTexOnly}
  \begin{figure}
    \noindent
    \hspace*{0.025\textwidth}%
    \begin{minipage}{0.45\textwidth}%
      \epsfig{figure=Segment_generator_prog1.ps,width=\textwidth}
      \caption{Output of example program for the generic segment generator.}
      \label{figureSegmentGenerator}
    \end{minipage}%
    \hspace*{0.05\textwidth}%
    \begin{minipage}{0.45\textwidth}%
      \epsfig{figure=Segment_generator_prog2.ps,width=\textwidth}
      \caption{Output of example program for the generic segment
        generator using precomputed point locations.}
      \label{figureSegmentGeneratorFan}
    \end{minipage}%
  \end{figure}
\end{ccTexOnly}

\cprogfile{Segment_generator_prog1.C}

\begin{ccHtmlOnly}
  <A NAME="SegmentGenerator">
  <TABLE><TR><TD ALIGN=LEFT VALIGN=TOP WIDTH=60%>
    <A HREF="./Segment_generator_prog1.gif">Figure:</A>
    Output of example program for the generic segment generator.
  </TD><TD ALIGN=LEFT VALIGN=TOP WIDTH=5% NOWRAP>
  </TD><TD ALIGN=LEFT VALIGN=TOP WIDTH=35% NOWRAP>
    <A HREF="./Segment_generator_prog1.gif">
        <img src="./Segment_generator_prog1_small.gif" 
             alt="Segment Generator Example Output"></A>
  </TD></TR></TABLE>
\end{ccHtmlOnly}

The second example uses precomputed point vectors to generate a
regular structure of 100 segments, see 
\ccTexHtml{Figure~\ref{figureSegmentGeneratorFan}}{Figure <A
  HREF="#SegmentGeneratorFan"> <IMG SRC="cc_ref_up_arrow.gif"
  ALT="reference arrow" WIDTH="10" HEIGHT="10"></A>} for the example
output.

\cprogfile{Segment_generator_prog2.C}

\begin{ccHtmlOnly}
  <A NAME="SegmentGeneratorFan">
  <TABLE><TR><TD ALIGN=LEFT VALIGN=TOP WIDTH=60%>
    <A HREF="./Segment_generator_prog2.gif">Figure:</A>
    Output of example program for the generic segment generator using
    precomputed point locations.
  </TD><TD ALIGN=LEFT VALIGN=TOP WIDTH=5% NOWRAP>
  </TD><TD ALIGN=LEFT VALIGN=TOP WIDTH=35% NOWRAP>
    <A HREF="./Segment_generator_prog2.gif">
        <img src="./Segment_generator_prog2_small.gif" 
             alt="Segment Generator Example Output 2"></A>
  </TD></TR></TABLE>
\end{ccHtmlOnly}

\end{ccClassTemplate}

% +--------------------------------------------------------+
\beforecprogskip\parskip
\aftercprogskip0pt

% EOF
