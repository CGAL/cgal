\chapter{Circular kernel}
\label{chapter-circular-kernel}

\ccChapterAuthor{Monique Teillaud, Sylvain Pion}

\section{Introduction}

The CGAL kernel provides the user mainly with \textit{linear} objects
(points, line segments, lines...) and predicates on them. It also
defines circles but does not offer a lot of functionalities on them. 

The goal of the circular kernel is to offer to the user a large set of
functionalities on circles and circular arcs in the plane. All the
choices (interface, robustness, representation, and so on) made here
are consistent with the choices made in the CGAL kernel, for which we
refer the user to the 2D kernel manual. 

The circular kernel uses the extensibility of the 2D kernel. 
Three new main geometric objects are introduced: circular arcs, points
of circular arcs (used in particular for endpoints of arcs and
intersection points between arcs) and line segments whose endpoints
are points of this new type.

In this first release, all functionalities necessary for computing an
arrangement of circular arcs and these line segments are
defined. Three traits classes are provided for the CGAL arrangement
package. 

\section{Software design}

The design is done in such a way that the algebraic concepts and the
geometric concepts are clearly separated. The \ccc{Circular_kernel}
has therefore two template parameters: 
\begin{itemize}
\item {} the \ccc{BasicGeometricKernel}, from which the circular kernel derives,
provides all basic geometric objects like points, lines, circles, and
elementary functionality on them. In fact it is meant to be
instantiated by a CGAL kernel,  but the user may plug his own kernel
instead, as long at it follows the CGAL kernel concept. 
\item {} the \ccc{AlgebraicKernel} is responsible for computations on
polynomials and algebraic numbers. It has to be a model of concept 
\ccc{AlgebraicKernel_2_2} described in a separate chapter. The
robustness of the package relies on the fact that the algebraic kernel
provides exact computations on algebraic objects.
\end{itemize}

The types of \ccc{BasicGeometricKernel} are inherited by the circular kernel.
Some types are taken from the \ccc{AlgebraicKernel} parameter, and
some types are defined by the \ccc{Circular_kernel} itself.

In fact, the circular kernel is documented as a concept, and several
models are provided: \ccc{Circular_kernel} is the basic kernel, and
several filtered kernels are built on top of this basic kernel to
increase effciency: \footnote{to be written}

\section{Examples}

	\subsection{Computing an arrangement of random circles} 

This example shows how to construct incrementally an arrangement of
circles, using the traits class for arrangement of circular arcs
provided with the package.

\ccIncludeExampleCode{Curved_kernel/example_arrangement_random_circles.C}

	\subsection{Constructing an arrangement of circles and segments} 

In this example, the traits class using the
\ccAnchor{http://www.boost.org/doc/html/variant.html}{boost::variant}
is used in order to provide arrangements with curves that can be
either circular arcs or line segments.

\ccIncludeExampleCode{Curved_kernel/example_arrangement_random_circles_segments.C}

	\subsection{Using the ``lazy'' kernel} 
\footnote{TBD}

\section{Design and Implementation History}
