\begin{ccRefClass}{Curved_kernel<BasicGeometricKernel,AlgebraicKernel>}

\ccDefinition

\ccInclude{CGAL/Curved_kernel.h}

\ccIsModel{CircularKernel, ConicKernel}


\ccParameters

The curved kernel is parameterized by a \ccc{BasicGeometricKernel} parameter
(and derives from it), in order to reuse all needed functionalities on
basic linear objects provided by one of the CGAL kernels. It also
allows other implementations of these basic functionalities.

The second parameter, \ccc{AlgebraicKernel}, is meant to provide the
curved kernel with all the algebraic functionalities required for the
manipulation of algebraic curves. 
\footnote{default arguments?}

\ccInheritsFrom

\ccc{BasicGeometricKernel}

\ccTypes

\ccThree{typedef Circular_arc_endpoint_2<Curved-Kernel>}{Root_of_4xxx}{}
\ccThreeToTwo

The curved kernel takes its basic number type from either the basic geometric kernel or the algebraic kernel. 
\ccTypedef{typedef AlgebraicKernel::RT RT;}{Ring number type RT.}
or\\
\ccTypedef{typedef BasicGeometricKernel::RT RT;}{Ring number type RT.}
Therefore, these two number types must coincide. In the same way,
\ccTypedef{typedef AlgebraicKernel::RT FT;}{Ring number type FT.}
and\\
\ccTypedef{typedef BasicGeometricKernel::FT FT;}{Ring number type FT.}
must coincide. 


The types of \ccc{BasicGeometricKernel} are inherited by the curved kernel.
Some types are taken from the \ccc{AlgebraicKernel} parameter, and
some types are defined by the \ccc{Curved_kernel} itself.

\ccc{Curved_kernel} will be a model of \ccc{CircularKernel} if the 
\ccc{AlgebraicKernel} parameter follows the requirements of the 
concept \ccc{AlgebraicKernel_2_2}. The following types will then be usable. 

\ccTypedef{typedef Circle_2<Curved_kernel> Circle_2;}{} 
\ccGlue
\ccTypedef{typedef Circular_arc_2<Curved_kernel> Circular_arc_2;}{}
\ccGlue
\ccTypedef{typedef Circular_arc_endpoint_2<Curved_kernel> Circular_arc_endpoint_2;}{}

\footnote{I assume here that Circle\_2 will be moved to Curved kernel}

\ccc{Curved_kernel} will be a model of \ccc{ConicKernel} if the 
\ccc{AlgebraicKernel} parameter follows the requirements of the 
concept \ccc{AlgebraicKernel_4_2}. The following types will then be usable.

\ccTypedef{typedef Conic_2<Curved_kernel> Conic_2;}{}
\ccGlue
\ccTypedef{typedef Conic_arc_2<Curved_kernel> Conic_arc_2;}{}
\ccGlue
\ccTypedef{typedef Conic_arc_endpoint_2<Curved_kernel> Conic_arc_endpoint_2;}{}

\ccNestedType{Compare_x_2;}{Model of \ccc{CurvedKernel::CompareX_2}.}
\ccGlue
\ccNestedType{Compare_y_2;}{Model of \ccc{CurvedKernel::CompareY_2}.}
\ccGlue
\ccNestedType{Compare_xy_2;}{Model of \ccc{CurvedKernel::CompareXY_2}.}

\ccNestedType{Construct_circle_2;}{Model of ...}

!!!!!!!!!!!!!!!!!! a faire !!!!!!!!!!!!!!!!

\ccCreation
\ccCreationVariable{CK}

\ccConstructor{Curved_kernel()}{}

\ccPredicates

\ccMethod{Compare_x_2 compare_x_2_object() const;}{}
\ccGlue
\ccMethod{Compare_y_2 compare_y_2_object() const;}{}
\ccGlue
\ccMethod{Compare_xy_2 compare_xy_2_object() const;}{}

\ccHeading{Constructions}

\ccMethod{Construct_circle_2 construct_circle_2_object() const;}{}
\ccGlue
\ccMethod{Construct_circular_arc_2 construct_circular_arc_2_object() const;}{}
\ccGlue
\ccMethod{Construct_circular_arc_endpoint_2 construct_circular_arc_endpoint_2_object() const;}{}

\ccMethod{Construct_conic_2 construct_conic_2_object() const;}{}
\ccGlue
\ccMethod{Construct_conic_arc_2 construct_conic_arc_2_object() const;}{}
\ccGlue
\ccMethod{Construct_conic_arc_endpoint_2 construct_conic_arc_endpoint_2_object() const;}{}

\ccMethod{Make_x_monotone_2 make_x_monotone_2_object() const;}{}
\ccGlue
\ccMethod{Construct_intersections_2 construct_intersections_2_object() const;}{}

\ccSeeAlso

\ccRefIdfierPage{BasicGeometricKernel}\\
\ccRefIdfierPage{AlgebraicKernel_2_2}\\
\ccRefIdfierPage{AlgebraicKernel_4_2}

\end{ccRefClass}
