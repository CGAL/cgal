>\begin{ccRefFunction}{compare_x}

\ccDefinition

\ccFunction{template < class CircularKernel > 
	Comparison_result compare_x
	(const Circular_arc_endpoint_2<CircularKernel> &p,
	const Circular_arc_endpoint_2<CircularKernel> &q);}
        {Calls the operator() of \ccc{CircularKernel::CompareX_2}.}

\ccFunction{template < class ConicKernel > 
	Comparison_result compare_x
	(const Conic_arc_endpoint_2<ConicKernel> &p,
	const Conic_arc_endpoint_2<ConicKernel> &q);}
        {Calls the operator() of \ccc{ConicKernel::CompareX_2}.}

\ccSeeAlso

\ccRefConceptPage{CircularKernel::CompareX_2}\\
\ccRefConceptPage{ConicKernel::CompareX_2}

\end{ccRefFunction}
\begin{ccRefFunction}{compare_y}

\ccDefinition

\ccFunction{template < class CircularKernel > 
	Comparison_result compare_y
	(const Circular_arc_endpoint_2<CircularKernel> &p,
	const Circular_arc_endpoint_2<CircularKernel> &q);}
        {Calls the operator() of \ccc{CircularKernel::CompareY_2}.}

\ccFunction{template < class ConicKernel > 
	Comparison_result compare_y
	(const Conic_arc_endpoint_2<ConicKernel> &p,
	const Conic_arc_endpoint_2<ConicKernel> &q);}
        {Calls the operator() of \ccc{ConicKernel::CompareY_2}.}

\ccSeeAlso

\ccRefConceptPage{CircularKernel::CompareY_2}\\
\ccRefConceptPage{ConicKernel::CompareY_2}

\end{ccRefFunction}
\begin{ccRefFunction}{compare_xy}

\ccDefinition

\ccFunction{template < class CircularKernel > 
	Comparison_result compare_xy
	(const Circular_arc_endpoint_2<CircularKernel> &p,
	const Circular_arc_endpoint_2<CircularKernel> &q);}
        {Calls the operator() of \ccc{CircularKernel::CompareXY_2}.}

\ccFunction{template < class ConicKernel > 
	Comparison_result compare_xy
	(const Conic_arc_endpoint_2<ConicKernel> &p,
	const Conic_arc_endpoint_2<ConicKernel> &q);}
        {Calls the operator() of \ccc{ConicKernel::CompareXY_2}.}

\ccSeeAlso

\ccRefConceptPage{CircularKernel::CompareXY_2}\\
\ccRefConceptPage{ConicKernel::CompareXY_2}

\end{ccRefFunction}
\begin{ccRefFunction}{make_x_monotone}

\ccDefinition

\ccFunction{template < class CircularKernel, class OutputIterator >
    OutputIterator make_x_monotone
	(const Circular_arc_2<CircularKernel> &ca,
	OutputIterator res);}
        {Calls the operator() of \ccc{CircularKernel::MakeXMonotone_2}.}

\ccFunction{template < class ConicKernel, class OutputIterator >
    OutputIterator make_x_monotone
	(const Conic_arc_2<ConicKernel> &ca,
	OutputIterator res);}
        {Calls the operator() of \ccc{ConicKernel::MakeXMonotone_2}.}

\ccSeeAlso

\ccRefConceptPage{CircularKernel::MakeXMonotone_2}\\
\ccRefConceptPage{ConicKernel::MakeXMonotone_2}

\end{ccRefFunction}
\begin{ccRefFunction}{construct_intersections}

\ccDefinition

\ccFunction{template < class CircularKernel, class OutputIterator >
    OutputIterator construct_intersections
	(const Circle_2<CircularKernel> &ca1,
	const Circle_2<CircularKernel> &ca2, 
	OutputIterator res);}
        {Calls the operator() of \ccc{CircularKernel::ConstructIntersections_2}.}

\ccFunction{template < class CircularKernel, class OutputIterator >
    OutputIterator construct_intersections
	(const Circle_2<CircularKernel> &ca1,
	const Circle_2<CircularKernel> &ca2, 
	OutputIterator res, int foo);}
        {Calls the operator(int foo) of \ccc{CircularKernel::ConstructIntersections_2}.}
\footnote{difference between multiplicity and parity...?}

\ccFunction{template < class CircularKernel, class OutputIterator >
    OutputIterator construct_intersections
	(const Circular_arc_2<CircularKernel> &ca1,
	const Circular_arc_2<CircularKernel> &ca2, 
	OutputIterator res);}
        {Calls the operator() of \ccc{CircularKernel::ConstructIntersections_2}.}

\ccFunction{template < class CircularKernel, class OutputIterator >
    OutputIterator construct_intersections
	(const Circular_arc_2<CircularKernel> &ca1,
	const Circular_arc_2<CircularKernel> &ca2, 
	OutputIterator res, int foo);}
        {Calls the operator(int foo) of \ccc{CircularKernel::ConstructIntersections_2}.}

Idem for conics...

\ccSeeAlso

\ccRefConceptPage{CircularKernel::ConstructIntersections_2}\\
\ccRefConceptPage{ConicKernel::ConstructIntersections_2}

\end{ccRefFunction}
\begin{ccRefFunction}{solve}

\ccDefinition

\ccFunction{template < class AlgebraicKernel_2_2, class OutputIterator >
	OutputIterator solve
	(const AlgebraicKernel_2_2::Polynomial_circle_2_2 &p1,
	const AlgebraicKernel_2_2::Polynomial_circle_2_2 &p2,
	OutputIterator res);}
	{Calls the operator() of \ccc{AlgebraicKernel_2_2::Solve}.}

\ccFunction{template < class AlgebraicKernel_4_2, class OutputIterator >
	OutputIterator solve
	(const AlgebraicKernel_4_2::Polynomial_2_2 &p1,
	const AlgebraicKernel_4_2::Polynomial_2_2 &p2,
	OutputIterator res);}
	{Calls the operator() of \ccc{AlgebraicKernel_4_2::Solve}.}

\ccSeeAlso

\ccRefConceptPage{AlgebraicKernel_2_2::Solve}\\
\ccRefConceptPage{AlgebraicKernel_4_2::Solve}

\end{ccRefFunction}
\begin{ccRefFunction}{get_equation}

\ccFunction{template < class CircularKernel >
	CircularKernel::AlgebraicKernel_2_2::PolynomialCircle_2_2
	get_equation(const CircularKernel::Circle_2 & c);}
	{Calls the operator() of \ccc{CircularKernel::GetEquation}.}

\ccFunction{template < class ConicKernel >
	ConicKernel::AlgebraicKernel_4_2::Polynomial_2_2
	get_equation(const ConicKernel::Conic_2 & c);}
	{Calls the operator() of \ccc{ConicKernel::GetEquation}.}

\ccSeeAlso

\ccRefConceptPage{CircularKernel::GetEquation}\\
\ccRefConceptPage{ConicKernel::GetEquation}

\end{ccRefFunction}
\begin{ccRefFunction}{make_root_of_2}

\ccFunction{template < class RT >
	Root_of_2<RT>
	make_root_of_2(RT a, RT b, RT c, int i);}
	{Returns the \ccc{i}th root of equation $aX^2+bX+c=0$. \ccc{RT} is supposed to be a \ccc{RingNumberType}, and \ccc{Root_of_2<RT>} is the type given by \ccc{Root_of_traits_2<RT>}.}

\ccFunction{template < class RT >
	Root_of_2<RT>
	make_root_of_2(FT a, FT b, FT c, int i);}
	{.}

\footnote{groumpf. \ccc{Root_of_taits} is templated by RT but we also need (do we?) make-root-of with FT. Relation between RT and FT...?}

\ccSeeAlso

\ccRefIdfierPage{CGAL::Root_of_2<RT>}\\
\ccRefIdfierPage{CGAL::Root_of_traits_2<RT>}

\end{ccRefFunction}
