\begin{ccRefClass}{Circular_arc_2<CircularKernel>}

\ccDefinition

\ccInclude{CGAL/Circular_arc_2.h}

\ccParameters

\ccc{CircularKernel} 

\ccCreation
\ccCreationVariable{ca}

Default constructible, copy constructible.

\ccThree{Circular_arc_endpoint_2}{ca.is_x_monotone()}{}
\ccThreeToTwo

\ccConstructor{Circular_arc_2(const CircularKernel::Circle_2 &c)}
{Constructs an arc from a full circle.}

\ccConstructor{Circular_arc_2(const CircularKernel::Circle_2 &c, 
		   const CircularKernel::Circle_2 &c1, int i1,
		   const CircularKernel::Circle_2 &c2, int i2)}
{Constructs the unique circular arc that is oriented counterclockwise,
whose supporting circle is \ccc{c}, and whose source endpoint is the
intersection of \ccc{c} and \ccc{c1} with index $i1$, and whose target
is the intersection of \ccc{c} and \ccc{c2} of index $i2$, where
intersections are ordered lexicographically.
\ccPrecond{\ccc{c} intersects both \ccc{c1} and \ccc{c2}, and the arc
defined by the intersections has non-zero length.}} 
\footnote{there will be variants in the class...}

\ccAccessFunctions

\ccThree{CircularKernel::Circular_arc_endpoint_2}{ca.is_x_monotone()}{}
\ccThreeToTwo

\ccMethod{CircularKernel::Circle_2 supporting_circle();}{}

A circular arc is supposed to be oriented counterclockwise, from 
\ccc{begin} to \ccc{end}. 

\ccMethod{CircularKernel::Circular_arc_endpoint_2 begin();}{}
\ccGlue
\ccMethod{CircularKernel::Circular_arc_endpoint_2 end();}{}

When the methods \ccc{begin} and \ccc{end} return the same point, then 
the arc is in fact a full circle. \footnote{arcs of zero length are
points, not arcs}

When an arc is x-monotone, its left and right endpoints can be accessed
directly:

\ccMethod{CircularKernel::Circular_arc_endpoint_2 left();}{\ccPrecond{\ccVar.\ccc{is_x_monotone()}}.}
\ccGlue
\ccMethod{CircularKernel::Circular_arc_endpoint_2 right();}{\ccPrecond{\ccVar.\ccc{is_x_monotone()}}.}
\footnote{names for associated functors in CK...? similar to
\textit{ConstructVertex\_2} in kernel, whose name is not perfect...}

\ccQueryFunctions

\ccMethod{bool is_x_monotone();}{Tests whether the arc is x-monotone.}
\ccGlue
\ccMethod{bool is_y_monotone();}{Tests whether the arc is y-monotone.}

\ccHeading{I/O}

\ccFunction{istream& operator>> (std::istream& is, Circular_arc_2 & ca);}{}
\ccGlue
\ccFunction{ostream& operator<< (std::ostream& os, const Circular_arc_2 & ca);}{}

\end{ccRefClass}
