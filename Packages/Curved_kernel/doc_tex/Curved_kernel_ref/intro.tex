\chapter{Curved Kernel}

\textbf{DRAFT - Monique - with Sylvain's help...}

The main class is: 

\ccRefIdfierPage{CGAL::Curved_kernel<BasicGeometricKernel,AlgebraicKernel>}

%%%%%%%%%%%%%%%%%%%%%%%%%%%%%%%%%%%%%%%%%%%%%%%%%%%%%%%%%
\section*{Concepts}

\ccRefConceptPage{CircularKernel_2}\\
\ccRefConceptPage{ConicKernel_2}

\ccRefConceptPage{BasicGeometricKernel}

\ccRefConceptPage{AlgebraicKernel_2_2}\\
\ccRefConceptPage{AlgebraicKernel_4_2}\\
\ccRefConceptPage{AlgebraicKernel_2_2::RootOf_2}\\
\ccRefConceptPage{AlgebraicKernel_4_2::RootOf_4}\\
\ccRefConceptPage{Algebraic_kernel_2_2::PolynomialCircle_2_2}\\
\ccRefConceptPage{Algebraic_kernel_4_2::Polynomial_2_2}

\textit{Remark about the suffix \ccc{_x_y}: \ccc{_x} stands 
for the degree of the polynomials and the algebraic numbers, and
\ccc{_y} stands for the number of variables, which is analogous to the
dimension for geometric objects.}

	\subsubsection*{Functors} 
\ccRefConceptPage{CircularKernel::CompareX_2}\\
\ccRefConceptPage{ConicKernel::CompareX_2}\\
\ccRefConceptPage{CircularKernel::CompareY_2}\\
\ccRefConceptPage{ConicKernel::CompareY_2}\\
\ccRefConceptPage{CircularKernel::CompareXY_2}\\
\ccRefConceptPage{ConicKernel::CompareXY_2}

\ccRefConceptPage{CircularKernel::ConstructCircle_2}\\
\ccRefConceptPage{CircularKernel::ConstructCircularArc_2}\\
\ccRefConceptPage{CircularKernel::ConstructCircularArcEndpoint_2}\\
\ccRefConceptPage{ConicKernel::ConstructConic_2}\\
\ccRefConceptPage{ConicKernel::ConstructConicArc_2}\\
\ccRefConceptPage{ConicKernel::ConstructConicArcEndpoint_2}

\ccRefConceptPage{CircularKernel::IsXMonotone_2}\\
\ccRefConceptPage{ConicKernel::IsXMonotone_2}

\ccRefConceptPage{CircularKernel::MakeXMonotone_2}\\
\ccRefConceptPage{ConicKernel::MakeXMonotone_2}\\
\ccRefConceptPage{CircularKernel::ConstructIntersections_2}\\
\ccRefConceptPage{ConicKernel::ConstructIntersections_2}

\ccRefConceptPage{CircularKernel::GetEquation}\\
\ccRefConceptPage{ConicKernel::GetEquation}

\ccRefConceptPage{AlgebraicKernel_2_2::ConstructPolynomialCircle_2_2}\\
\ccRefConceptPage{AlgebraicKernel_4_2::ConstructPolynomial_2_2}

\ccRefConceptPage{AlgebraicKernel_2_2::Solve}\\
\ccRefConceptPage{AlgebraicKernel_4_2::Solve}
%%%%%%%%%%%%%%%%%%%%%%%%%%%%%%%%%%%%%%%%%%%%%%%%%%%%%%%%%
\section*{Classes}

\ccRefIdfierPage{CGAL::Curved_kernel<BasicGeometricKernel,AlgebraicKernel>}

	\subsubsection*{Curves and Surfaces}
\ccRefIdfierPage{CGAL::Circle_2<CircularKernel>}\\
\ccRefIdfierPage{CGAL::Conic_2<ConicKernel>}\\
\ccRefIdfierPage{CGAL::Ellipsis_2<ConicKernel>}\footnote{for ETH?}\\
\ccRefIdfierPage{CGAL::Sphere_3<SphericalKernel>}\\
\ccRefIdfierPage{CGAL::Quadric_3<Quadr...Kernel>}

	\subsubsection*{Arcs}
\ccRefIdfierPage{CGAL::Circular_arc_2<CircularKernel>}\\
\ccRefIdfierPage{CGAL::Conic_arc_2<ConicKernel>}\\
\ccRefIdfierPage{CGAL::Sphere_patch_3<SphericalKernel>}\\
\ccRefIdfierPage{CGAL::Quadric_patch_3<SphericalKernel>}

	\subsubsection*{Points}
\ccRefIdfierPage{CGAL::Circular_arc_endpoint_2<CircularKernel>}\\
\ccRefIdfierPage{CGAL::Conic_arc_endpoint_2<ConicKernel>}

	\subsubsection*{Polynomials and algebraic numbers} 
\footnote{how does the linking work when concepts and classes have the same name...? example \ccc{Polynomial_2_2}}

\ccRefIdfierPage{CGAL::Root_of_2<RT>}\\
\ccRefIdfierPage{CGAL::Root_of_4<RT>}

\ccRefIdfierPage{CGAL::Root_of_traits_2<RT>}\\
\ccRefIdfierPage{CGAL::Root_of_traits_4<RT>}

%%%%%%%%%%%%%%%%%%%%%%%%%%%%%%%%%%%%%%%%%%%%%%%%%%%%%%%%%
\section*{Functions}

	\subsubsection*{Predicates}
\ccRefIdfierPage{CGAL::compare_x}\\
\ccRefIdfierPage{CGAL::compare_y}\\
\ccRefIdfierPage{CGAL::compare_xy}

	\subsubsection*{Constructions}

\ccRefIdfierPage{CGAL::make_root_of_2}\\
\ccRefIdfierPage{CGAL::make_root_of_4}

\ccRefIdfierPage{CGAL::make_x_monotone}\\
\ccRefIdfierPage{CGAL::construct_intersections}

\ccRefIdfierPage{CGAL::solve}

	\subsubsection*{others?} 

\ccRefIdfierPage{CGAL::get_equation}

%%%%%%%%%%%%%%%%%%%%%%%%%%%%%%%%%%%%%%%%%%%%%%%%%%%%%%%%%
\section*{Traits class}

\ccRefIdfierPage{CGAL::Circular_arc_traits}\\
\ccRefIdfierPage{CGAL::Conic_arc_traits}
%%%%%%%%%%%%%%%%%%%%%%%%%%%%%%%%%%%%%%%%%%%%%%%%%%%%%%%%%
% \section*{Enums}
