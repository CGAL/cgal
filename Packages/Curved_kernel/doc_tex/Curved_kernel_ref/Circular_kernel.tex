\begin{ccRefClass}{Circular_kernel<LinearKernel,AlgebraicKernel>}

\ccDefinition

\ccInclude{CGAL/Circular_kernel.h}

\ccIsModel

\ccc{CircularKernel}\footnote{for Lutz: macro ``ccIsmodel'' is
apparently assuming that there is only one possible concept...}

\ccParameters

The circular kernel is parameterized by a \ccc{LinearKernel} parameter
(and derives from it), in order to reuse all needed functionalities on
basic linear objects provided by one of the CGAL kernels. It also
allows other implementations of these basic functionalities.

The second parameter, \ccc{AlgebraicKernel}, is meant to provide the
circular kernel with all the algebraic functionalities required for the
manipulation of algebraic curves. 
It must be a model of concept \ccc{AlgebraicKernel_2_2}. 

\ccInheritsFrom

\ccc{LinearKernel}

\ccTypes

\ccThree{typedef Circular_arc_point_2<Circular-Kernel>}{Root_of_4xxx}{}
\ccThreeToTwo

The circular kernel inherits its basic number types from the algebraic kernel:
\ccTypedef{typedef AlgebraicKernel::RT RT;}{Ring number type.}
\ccTypedef{typedef AlgebraicKernel::FT FT;}{Field number type.}
In fact, the two number types \ccc{AlgebraicKernel::RT} and
\ccc{LinearKernel::RT} must coincide, as well as
\ccc{AlgebraicKernel::FT} and \ccc{LinearKernel::FT}.
\footnote{how to avoid inconsistencies? a user might take an AK and a
BGK with different RTs... this is not a problem specific to CK, it already occurs 
in many places in CGAL...}. 

The types of \ccc{LinearKernel} are inherited by the circular kernel.
Some types are taken from the \ccc{AlgebraicKernel} parameter, and
some types are defined by the \ccc{Circular_kernel} itself.

\ccc{Circular_kernel} will be a model of \ccc{CircularKernel} if the 
\ccc{AlgebraicKernel} parameter follows the requirements of the 
concept \ccc{AlgebraicKernel_2_2}. The following types will then be
usable, as well as all the functioanlity on them described in the
\ccc{CircularKernel} concept. 

\ccTypedef{typedef Line_arc_2<Circular_kernel> Line_arc_2;}{}
\ccGlue
\ccTypedef{typedef Circular_arc_2<Circular_kernel> Circular_arc_2;}{}
\ccGlue
\ccTypedef{typedef Circular_arc_point_2<Circular_kernel> Circular_arc_point_2;}{}

\ccSeeAlso

\ccRefIdfierPage{LinearKernel}\\
\ccRefIdfierPage{AlgebraicKernel_2_2}

\end{ccRefClass}
