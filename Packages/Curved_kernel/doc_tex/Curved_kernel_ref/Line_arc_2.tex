\begin{ccRefClass}{Line_arc_2<CircularKernel>}

\ccDefinition

\ccInclude{CGAL/Line_arc_2.h}

\ccParameters

\ccc{CircularKernel} 

\ccCreation
\ccCreationVariable{la}

Default constructible, copy constructible.

\ccThree{Circular_arc_point_2}{ca.is_x_monotone()}{}
\ccThreeToTwo

\ccConstructor{Line_arc_2(const CircularKernel::Line_2 &l,
		const CircularKernel::Circle_2 &c1, int i1,
		 const CircularKernel::Circle_2 &c2, int i2)}
{Constructs the line segment whose supporting line is \ccc{l}, whose 
source endpoint is the \ccc{i1}th intersection of \ccc{l} with \ccc{c1}, 
and whose target endpoint is the \ccc{i2}th intersection of \ccc{l} 
and \ccc{c2}, where intersections are ordered lexicographically.}
\ccPrecond{\ccc{l} intersects both \ccc{c1} and \ccc{c2}, and the arc
defined by the intersections has non-zero length.}


\ccAccessFunctions

\ccThree{CircularKernel::Circular_arc_point_2}{ca.is_x_monotone()}{}
\ccThreeToTwo

\ccMethod{CircularKernel::Line_2 supporting_line();}{}

\ccMethod{CircularKernel::Circular_arc_point_2 source();}{}
\ccGlue
\ccMethod{CircularKernel::Circular_arc_point_2 target();}{}

\ccMethod{CircularKernel::Circular_arc_point_2 left();}{}
\ccGlue
\ccMethod{CircularKernel::Circular_arc_point_2 right();}{}

\ccQueryFunctions

\ccMethod{bool is_vertical();}{}

\ccHeading{I/O}

\ccFunction{istream& operator>> (std::istream& is, Line_arc_2 & ca);}{}
\ccGlue
\ccFunction{ostream& operator<< (std::ostream& os, const Line_arc_2 & ca);}{}

\end{ccRefClass}

