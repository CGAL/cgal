% +------------------------------------------------------------------------+
% | Reference manual page: Polyhedron_operator_rightshift.tex
% +------------------------------------------------------------------------+
% | 05.04.2004   Peter Hachenberger
% | Package: Nef_3
% | 
\RCSdef{\RCSNefpolyhedron3operatorostreamRev}{$Revision$}
\RCSdefDate{\RCSNefpolyhedron3operatorostreamDate}{$Date$}
% |
%%RefPage: end of header, begin of main body
% +------------------------------------------------------------------------+

\ccHtmlNoClassLinks
\begin{ccRefFunction}{operator>>}
\label{refNef_polyhedron_operator_rightshift_3}

\ccDefinition

This operator reads a Nef polyhedron, which is given in the proprietary file
format written by the input operator \emph{in} and assigns it to \emph{N}. It includes the 
complete incidence structure, the geometric data, and the marks of each item.

It is recommended to use the \cgal kernels \ccc{Homogeneous}, 
\ccc{Simple_homogeneous}, 
or \ccc{Extended_homogeneous} parametrized with any exact number type that models 
mathbb{Z}. The input and output iterators of Nef polyhedra parametrized with
either of these kernels are compatible as long as the Nef polyhedron is bounded.
An unbounded Nef polyhedron can only be read by a Nef polyhedron parametrized with
an extended kernel. It is also recommended to use the \cgal stream modifier 
\ccc{set_ascii_mode}.

The input operator and output operators of N

\ccInclude{CGAL/IO/Nef_polyhedron_iostream.h}

\ccGlobalFunction{template <class Nef_polyhedronTraits_3>
    istream& operator>>( istream& in,
                         CGAL::Nef_polyhedron_3<Nef_polyhedronTraits_3>& N);}
  
\ccSeeAlso

\ccRefIdfierPage{CGAL::Nef_polyhedron_3<Traits>}\\ 
\lcTex{\ccc{operator<<} \dotfill\ 
    page~\pageref{refNef_polyhedron_operator_leftshift_3}}%
\lcRawHtml{
    <I><A HREF="Function_operator.html">operator&lt;&lt;</A></I>.
}

\end{ccRefFunction}

% +------------------------------------------------------------------------+
%%RefPage: end of main body, begin of footer
% EOF
% +------------------------------------------------------------------------+
