%% Copyright (c) 2004  SciSoft.  All rights reserved.
%%
%% This file is part of CGAL (www.cgal.org); you may redistribute it under
%% the terms of the Q Public License version 1.0.
%% See the file LICENSE.QPL distributed with CGAL.
%%
%% Licensees holding a valid commercial license may use this file in
%% accordance with the commercial license agreement provided with the software.
%%
%% This file is provided AS IS with NO WARRANTY OF ANY KIND, INCLUDING THE
%% WARRANTY OF DESIGN, MERCHANTABILITY AND FITNESS FOR A PARTICULAR PURPOSE.
%%
%% $Name$
%%
%% Author(s)     : Fernando Cacciola <fernando_cacciola@hotmail.com>

\ccRefPageBegin

%%RefPage: end of header, begin of main body
% +------------------------------------------------------------------------+

\begin{ccRefConcept}{StraightSkeleton_2}

   A model for the \ccRefName\ concept defines the underlying
   combinatorial data structure (halfedge data structure) used to
   represent the 2D straight skeleton of a simple polygon.

The \ccRefName\ concept is a refinement of the \ccc{HalfedgeDS} concept. Hence, the requirements for the latter are requirements for the former.
The difference in requirements is that the
\ccHtmlNoLinksFrom{\ccStyle{Vertex} and \ccStyle{Halfedge}}
types should be models of the more refined concepts
\ccc{StraightSkeletonVertex_2} and \ccc{StraightSkeletonHalfedge_2}.

\ccTypes
    \ccNestedType{Vertex}{
       The vertex type. Should be a model of the
       \ccc{StraightSkeletonVertex_2} concept.}

    \ccNestedType{Halfedge}{
       The halfedge type. Should be a model of the
       \ccc{StraightSkeletonHalfedge_2} concept.}


\end{ccRefConcept} % StraightSkeleton_2
% +------------------------------------------------------------------------+
%%RefPage: end of main body, begin of footer
\ccRefPageEnd
% EOF
% +------------------------------------------------------------------------+
