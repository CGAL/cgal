% +------------------------------------------------------------------------+
% | Reference manual page: ImplicitSurfaceTraits_3.tex
% +------------------------------------------------------------------------+
% | 18.01.2006   Author
% | Package: Package
% | 
\RCSdef{\RCSImplicitSurfaceTraitsRev}{$Revision$}
\RCSdefDate{\RCSImplicitSurfaceTraitsDate}{$Date$}
% |
%%RefPage: end of header, begin of main body
% +------------------------------------------------------------------------+


\begin{ccRefConcept}{ImplicitSurfaceOracleTraits_3}

%% \ccHtmlCrossLink{}     %% add further rules for cross referencing links
%% \ccHtmlIndexC[concept]{} %% add further index entries

\ccDefinition
  
The concept \ccRefName\ provides the numerical type 
and the point type used in the surface oracle
\ccc{Implicit_surface_oracle< Traits, Func>}.

%\ccGeneralizes

%ThisConcept \\
%ThatConcept

\ccTypes

\ccNestedType{FT}{The numerical type}
\ccGlue
\ccNestedType{Point_3}{The point type}

%\ccCreation
%\ccCreationVariable{a}  %% choose variable name

%\ccConstructor{ImplicitSurfaceTraits_3();}{default constructor.}

%\ccOperations

%\ccMethod{void foo();}{some member functions}

\ccHasModels

Any CGAL Kernel.

%\ccc{Some_class},
% \ccc{Some_other_class}.

\ccSeeAlso

Some\_other\_concept,
\ccc{some_other_function}.

\ccExample

A short example program.
Instead of a short program fragment, a full running program can be
included using the 
\verb|\ccIncludeExampleCode{Package/ImplicitSurfaceTraits_3.C}| 
macro. The program example would be part of the source code distribution and
also part of the automatic test suite.

\begin{ccExampleCode}
void your_example_code() {
}
\end{ccExampleCode}

%% \ccIncludeExampleCode{Package/ImplicitSurfaceTraits_3.C}

\end{ccRefConcept}

% +------------------------------------------------------------------------+
%%RefPage: end of main body, begin of footer
% EOF
% +------------------------------------------------------------------------+

