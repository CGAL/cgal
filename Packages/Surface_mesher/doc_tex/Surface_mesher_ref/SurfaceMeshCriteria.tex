% +------------------------------------------------------------------------+
% | Reference manual page: SurfaceMeshCriteria.tex
% +------------------------------------------------------------------------+
% | 02.12.2005   Author
% | Package: Package
% | 
\RCSdef{\RCSSurfaceMeshCriteriaRev}{$Revision$}
\RCSdefDate{\RCSSurfaceMeshCriteriaDate}{$Date$}
% |
%%RefPage: end of header, begin of main body
% +------------------------------------------------------------------------+


\begin{ccRefConcept}{SurfaceMeshCriteria}

%% \ccHtmlCrossLink{}     %% add further rules for cross referencing links
%% \ccHtmlIndexC[concept]{} %% add further index entries

\ccDefinition
  
The Delaunay refinement process involved in  the surface mesher 
is guided by a set of refinement criteria.
The concept \ccRefName\ described the interface which
allow the mesher to take these criteria into account.

Typically the meshing criteria is a set
of elementary criterion, each of which
 the facet is required to meet.
The meshing algorithm will eliminate bad facets
which do not meet the criteria.

Furthemore final mesh depends upon the order according to which bad facets
are handled and the meshing algorithm 
needs to be able to quantify the facet qualities and to compare
the qualities of different faces.
Therefore the concept \ccRefName\ 
define a type \ccc{Quality} designed to measure
the quality of a mesh facet. 
Typically this quality
is a multicomponent variable, each component being
a numerical value associated with a given criterion
and the comparison operator is just a lexicographic 
comparison.


%\ccGeneralizes

%ThisConcept \\
%ThatConcept

\ccTypes

\ccNestedType{Quality}{A type to value the quality of mesh facets.
    This type is required have a comprison operator}

\ccCreation
\ccCreationVariable{criteria}  %% choose variable name

\ccConstructor{SurfaceMeshCriteria();}{default constructor.}

\ccOperations

\ccMethod{bool  is_bad(Facet f);}{return true if facet \ccc{f}
do not satisfy the required  criteria.}

\ccMethod{Quality quality(Facet f);}{Compute the quality
of facet \ccc{f}.}

\ccHasModels

\ccc{Some_class},
\ccc{Some_other_class}.

\ccSeeAlso

Some\_other\_concept,
\ccc{some_other_function}.

%\ccExample

%A short example program.
%Instead of a short program fragment, a full running program can be
%included using the 
%\verb|\ccIncludeExampleCode{Package/SurfaceMeshCriteria.C}| 
%macro. The program example would be part of the source code distribution and
%also part of the automatic test suite.

%\begin{ccExampleCode}
%void your_example_code() {
%}
%\end{ccExampleCode}

%% \ccIncludeExampleCode{Package/SurfaceMeshCriteria.C}

\end{ccRefConcept}

% +------------------------------------------------------------------------+
%%RefPage: end of main body, begin of footer
% EOF
% +------------------------------------------------------------------------+

