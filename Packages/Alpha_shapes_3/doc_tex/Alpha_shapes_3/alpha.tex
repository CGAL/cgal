% ======================================================================
%
% Copyright (c) 1999 The GALIA Consortium
%
% This software and related documentation is part of the
% Computational Geometry Algorithms Library (CGAL).
%
% Every use of CGAL requires a license. Licenses come in three kinds:
%
% - For academic research and teaching purposes, permission to use and
%   copy the software and its documentation is hereby granted free of  
%   charge, provided that
%   (1) it is not a component of a commercial product, and
%   (2) this notice appears in all copies of the software and
%       related documentation.
% - Development licenses grant access to the source code of the library 
%   to develop programs. These programs may be sold to other parties as 
%   executable code. To obtain a development license, please contact
%   the GALIA Consortium (at cgal@cs.uu.nl).
% - Commercialization licenses grant access to the source code and the
%   right to sell development licenses. To obtain a commercialization 
%   license, please contact the GALIA Consortium (at cgal@cs.uu.nl).
%
% This software and documentation is provided "as-is" and without
% warranty of any kind. In no event shall the CGAL Consortium be
% liable for any damage of any kind.
%
% The GALIA Consortium consists of Utrecht University (The Netherlands),
% ETH Zurich (Switzerland), Free University of Berlin (Germany),
% INRIA Sophia-Antipolis (France), Martin-Luther-University Halle-Wittenberg
% (Germany), Max-Planck-Institute Saarbrucken (Germany),
% and Tel-Aviv University (Israel).
%
% ----------------------------------------------------------------------
%
% package       : Alpha_shapes_2
% author(s)     : Tran Kai Frank DA <Frank.Da@sophia.inria.fr>
%
% coordinator   : INRIA Sophia-Antipolis (<Mariette.Yvinec@sophia.inria.fr>)
%
% ======================================================================

\RCSdef{\alphashapeRevision}{$Revision$}
\RCSdefDate{\alphashapeDate}{$Date$}

%----------------------------------------------------------------------

\chapter{Alpha-Shapes} \label{I1_ChapterAlphashapes}

%\ccChapterSubTitle{\alphashapeRevision, \alphashapeDate}

\section{Introduction}

This chapter presents a framework for alpha shapes. The description is
based on Edelsbrunner and M\"ucke.  Alpha shapes can be used for shape
reconstruction from a dense unorganized set of data points.  Alpha
shapes are the generalization of the convex hull of a point set. Let
$S$ be a finite set of points in $\R^d$, $d = 2,3$ and $\alpha$ a
parameter with $0 \leq \alpha \leq \infty$. For $\alpha =
\infty$, the $\alpha$-shape is the convex hull of $S$.  As $\alpha$
decreases, the $\alpha$-shape shrinks and develops cavities. Finally,
for $\alpha = 0$, the $\alpha$-shape is the set $S$ itself.

There is a close connection between alpha shapes and Delaunay
triangulations. More precisely, the $\alpha$-complex of $S$ is a
subcomplex of the Delaunay triangulation of $S$.  The corresponding
$\alpha$-shape is its underlying space.  Alternatively, the
$\alpha$-shape of $S$ is a polygon (resp.\ polytope) whose boundary
consists of $\alpha$-exposed $k$-simplices, $0 \leq k \leq d-1$. A
simplex is $\alpha$-exposed, if there is an open disk (resp.\ ball) of
radius $\sqrt{\alpha}$ through the vertices of the simplex that does
not contain any other point of $S$. In general, an alpha shape is a
non-connected, mixed-dimension polygon (rep.\ polytope).

The $\alpha$-shapes of $S$ form a discrete family, even though they
are defined for all real numbers $\alpha$ with $0 \leq \alpha
\leq \infty$. Thus, we can represent the entire family of $\alpha$ shapes
of $S$ by the Delaunay triangulation of $S$. In this representation
each $k$-simplex of the Delaunay triangulation is associated with an
interval that specifies for which values of $\alpha$ the $k$-simplex
belongs to the $\alpha$-shape. Relying on this result, the family of
$\alpha$-shapes can be computed efficiently and relatively
easily. Furthermore, we can select an appropriate $\alpha$-shape from a
finite number of different $\alpha$-shapes and corresponding
$\alpha$-values.

%----------------------------------------------------------------------

\section{Alpha Shape of Points in a Plane \label{I1_SectAlpha Shape_2}}

\begin{ccClassTemplate} {CGAL_Alpha_shape_2<Traits>}

\ccDefinition
The class \ccClassTemplateName\ represents the family of
$\alpha$-shapes of points in a plane for {\em all} positive
$\alpha$. It maintains the underlying Delaunay triangulation which
represents connectivity and order among its faces. Each
$k$-dimensional face of the Delaunay triangulation is associated with
an interval that specifies for which values of $\alpha$ the face
belongs to the $\alpha$-shape. There are links between the intervals
and the $k$-dimensional faces of the Delaunay triangulation.


\ccInclude{CGAL/Alpha_shape_2.h}

\ccInheritsFrom

\ccStyle{CGAL_Delaunay_Triangulation_2<Traits>}

The modifying functions \ccStyle{insert} and \ccStyle{remove} will overwrite
the inherited functions. At the moment, only the static version is implemented.

\ccTypes
\ccSetThreeColumns{CGAL_Oriented_side}{}{\hspace*{10cm}}
\ccThreeToTwo

\ccNestedType{Traits}{the alpha shape traits type.}
 
It contains the Delaunay triangulation traits class.  For example
\ccStyle{Traits::Point} is a mapping on a point class. Additionally,
it defines a mapping on a coordinate type class.

\ccTypedef{typedef Traits::Coord_type Coord_type;}{}

\ccNestedType{Alpha_iterator}{An iterator that allow to traverse 
the sorted sequence of different $\alpha$-values. The iterator is
bidirectional and non-mutable. Its \ccStyle{value_type}
is \ccStyle{Coord_type}}

\ccEnum{enum Classification_type {EXTERIOR, SINGULAR, REGULAR, INTERIOR};}
{Distinguishes the different cases for classifying a $k$-dimensional face
 of the underlying Delaunay triangulation of the $\alpha$-shape. \\
\ccStyle{EXTERIOR} if the face does not belong to the $\alpha$-complex.\\
\ccStyle{SINGULAR} if the face belongs to the boundary of the $\alpha$-shape,
 but is not incident to any 2-dimensional face of the $\alpha$-complex\\
\ccStyle{REGULAR} if the face belongs to the boundary of the $\alpha$-shape
 and is incident to a 2-dimensional face of the $\alpha$-complex\\
\ccStyle{INTERIOR} if the face belongs to the $\alpha$-complex, but does
not belong to the boundary of the $\alpha$-shape\\}

\ccEnum{enum Mode {GENERAL, REGULARIZED};}
{ In general, an alpha shape is a non-connected, mixed-dimension
polygon. Its regularized version is formed by the set of regular edges
and their vertices}

\ccCreation
\ccCreationVariable{A}

\ccConstructor{CGAL_Alpha_shape_2(Coord_type alpha = 0,
				  Mode m = GENERAL);}
{Introduces an empty $\alpha$-shape \ccVar\ for a positive $\alpha$-value
 \ccStyle{alpha}.
\ccPrecond \ccStyle{alpha}~$\geq~0$.}


\ccConstructor{template < class InputIterator >
		CGAL_Alpha_shape_2(
			InputIterator first,
			InputIterator last,
                const Coord_type& alpha = 0,
	        Mode m = GENERAL);}
{Initializes the family of alpha-shapes with the points in the range
$\left[\right.$\ccStyle{first}, \ccStyle{last}$\left.\right)$ and 
introduces an $\alpha$-shape \ccVar\ for a positive $\alpha$-value
\ccStyle{alpha}.  
\ccPrecond The \ccStyle{value_type} of \ccStyle{first} and
\ccStyle{last} is \ccStyle{Point}.\\
\ccStyle{alpha} $\geq 0$.}

\ccOperations

\ccMethod{template < class InputIterator >
		int make_Alpha_shape(
			InputIterator first,
			InputIterator last,
                        const Coord_type& alpha = 0,
	                Mode m = GENERAL);}
{Initialize the family of alpha-shapes with the points in the range
$\left[\right.$\ccStyle{first}, \ccStyle{last}$\left.\right)$ and 
introduces an $\alpha$-shape \ccVar\ for a positive $\alpha$-value
\ccStyle{alpha}. Returns the number of inserted points. \\
If the function is applied to an non-empty family of alpha-shape, it is cleared
before initialization.
\ccPrecond The \ccStyle{value_type} of \ccStyle{first} and
\ccStyle{last} is \ccStyle{Point}.\\
\ccStyle{alpha} $\geq 0$.}

\ccMethod{void
	clear();}
{Clears the structure.}

\ccMethod{Coord_type
	set_alpha(const Coord_type& alpha);}
{Sets the $\alpha$-value to \ccStyle{alpha}.
 Returns the previous $\alpha$-value.
\ccPrecond \ccStyle{alpha} $\geq 0$.}

\ccMethod{const Coord_type&
	get_alpha(void) const;}
{Returns the current $\alpha$-value.}

\ccMethod{const Coord_type&  get_nth_alpha(int n) const;}
{Returns the $n$-th alpha-value.
 \ccPrecond \ccStyle{n} < number of alphas.}

\ccMethod{int number_of_alphas() const;}
{Returns the number of different alpha-values.}

% dynamic version	
% 
% \ccMethod{Vertex* insert(const Point& p);}
% {Inserts point \ccStyle{p} in the alpha shape and returns the
% corresponding vertex of the underlying Delaunay triangulation.\\ If
% point \ccStyle{p} coincides with an already existing vertex, this
% vertex is returned and the alpha shape remains unchanged.\\ Otherwise,
% the vertex is inserted in the underlying Delaunay triangulation and
% the associated intervals are updated. }
% 
% \ccMethod{void remove(Vertex *v);}
% {Removes the vertex from the underlying Delaunay triangulation. The
% created hole is retriangulated and the associated intervals are
% updated.}
% 

\ccMethod{Mode
	set_mode(Mode m = GENERAL );}
{Sets \ccVar\ to its general or regularized version. 
Returns the previous mode.}

\ccMethod{Mode
	get_mode(void) const;}
{Returns whether \ccVar\ is general or regularized.}

\ccMethod{template  < class OutputIterator >
	  OutputIterator get_Alpha_shape_vertices(
					 OutputIterator result);}
{Writes the vertices of the alpha shape \ccVar\ for the current $\alpha$-value
to the container where \ccStyle{result} refers to. 
The \ccStyle{value_type} of \ccStyle{result} is \ccStyle{Vertex*}.
Returns an output iterator which is the end of the constructed range.}

\ccMethod{template  < class OutputIterator >
	  OutputIterator get_Alpha_shape_edges(
					 OutputIterator result);}
{Writes the edges 
of the alpha shape \ccVar\ for the current $\alpha$-value
to the container where \ccStyle{result} refers to. 
The \ccStyle{value_type} of \ccStyle{result} is \ccStyle{pair<Face*, int>}.
Returns an output iterator which is the end of the constructed range.}

\ccHeading{Predicates}


\ccMethod{Classification_type
           classify(const Point& p, 
	const Coord_type& alpha = get_alpha()) const;}
{Classifies a point \ccStyle{p} with respect to \ccVar.}

\ccMethod{Classification_type
           classify(Face* f, const Coord_type& alpha = get_alpha()) const;}
{Classifies the face \ccStyle{f} of the underlying Delaunay triangulation with respect to \ccVar.}

\ccMethod{Classification_type
           classify(pair<Face*, int> e, const Coord_type& alpha = get_alpha()) const;}
{Classifies the edge \ccStyle{e} of the underlying Delaunay triangulation with respect to \ccVar.}

\ccMethod{Classification_type
           classify(Face* f, int i, const Coord_type& alpha = get_alpha()) const;}
{Classifies the edge of the face \ccStyle{f} opposite to the vertex with index \ccStyle{i} of the underlying Delaunay triangulation with respect to \ccVar.}	

\ccMethod{Classification_type
           classify(Vertex* v, const Coord_type& alpha = get_alpha()) const;}
{Classifies the vertex \ccStyle{v} of the underlying Delaunay triangulation with respect to \ccVar.}


\ccHeading{Traversal of the $\alpha$-Values}

\smallskip
The alpha shape class defines an iterator that allows to visit the
sorted sequence of $\alpha$-values. This iterator is
non-mutable and bidirectional. Its value type is
\ccStyle{Coord_type}.

\ccMethod{Alpha_iterator alpha_begin() const;}
{Returns an iterator that allows to traverse the
sorted sequence of $\alpha$-values of the family of alpha shapes.}

\ccMethod{Alpha_iterator alpha_end() const;}
{Returns the corresponding past-the-end iterator.}

\ccMethod{Alpha_iterator alpha_find(const Coord_type& alpha) const;}
{Returns an iterator pointing to an element with $\alpha$-value
\ccStyle{alpha}, or the corresponding past-the-end iterator if such 
an element is not found.}

\ccMethod{Alpha_iterator alpha_lower_bound(const Coord_type& alpha) const;}
{Returns an iterator pointing to the first element with
$\alpha$-value not less than \ccStyle{alpha}.}

\ccMethod{Alpha_iterator alpha_upper_bound(const Coord_type& alpha) const;}
{Returns an iterator pointing to the first element with $\alpha$-value
greater than \ccStyle{alpha}.}

\ccHeading{Operations}

\ccMethod{int number_solid_components(const Coord_type& alpha = get_alpha()) const;}
{Returns the number of solid components of \ccVar, that is, the number of components of its regularized version.}

\ccMethod{Alpha_iterator find_optimal_alpha(int nb_components) const;}
{Returns an iterator pointing to the first element with $\alpha$-value
such that \ccVar\ satisfies the following two properties:\\
\ccStyle{nb_components} equals the number of solid components and \\
all data points are either on the boundary or in the interior of the regularized version of \ccVar.\\
If no such value is found, the iterator points to the first element with 
$\alpha$-value such that \ccVar\ satisfies the second property.}
\ccHeading{I/O}

The I/O operators are defined for \ccStyle{iostream}, and for
the window stream provided by \cgal. The format for the iostream
is an internal format. 

\ccInclude{CGAL/IO/ostream_2.h}

\ccFunction{ostream& operator<<(ostream& os,
                  const CGAL_Alpha_shape_2<Traits>& A);}
{Inserts the alpha shape \ccVar\ for the current $\alpha$-value into the stream \ccStyle{os}.
\ccPrecond The insert operator must be defined for \ccStyle{Point}.}

\ccInclude{CGAL/IO/Window_stream.h}

\ccFunction{CGAL_Window_stream& operator<<(CGAL_Window_stream& W,
                         const CGAL_Alpha_shape_2<Traits>& T);}
{Inserts the alpha shape \ccVar\ for the current $\alpha$-value into the window stream \ccStyle{W}.
\ccPrecond The insert operator must be defined for \ccStyle{Point} and \ccStyle{Segment}.}
\end{ccClassTemplate}

\ccImplementation
In the first static version, the set of intervals associated with the
$k$-dimensional faces of the underlying Delaunay triangulation are
stored only as sorted \ccStyle{vectors}. By using an interval tree the
alpha-shape could be constructed more efficiently. For the dynamic
version, we need \ccStyle{multimaps} or dynamic interval trees. The
cross links between the intervals and the $k$-dimensional faces of the
Delaunay triangulation are actually realized using
\ccStyle{multimaps} resp.\ \ccStyle{hash multimaps}.

\ccStyle{A.alpha find} uses linear search, while 
\ccStyle{A.alpha lower bound} and \ccStyle{A.alpha upper bound} 
use binary search.
\ccStyle{A.number solid components} performs a graph traversal and takes time linear in the number of faces of the underlying Delaunay triangulation.
\ccStyle{A.find optimal alpha} uses binary search and takes time
$O(\mbox{\em number of faces } \log{\mbox{\em number of faces}})$.

%----------------------------------------------------------------------

\section{Requirements\label{I1_SectRequirements}}


\subsection{Requirements for the Alpha-Shape Traits Class} 

The class \ccStyle{CGAL_Alpha_shape_2<Traits>} parameterized with a
traits class, that has the same requirements as the Delaunay
triangulation traits class.  The following requirement catalog lists
the primitives that must be defined additionnally.

\begin{ccClass} {Traits}
\subsection*{Traits (\mbox{\it Traits})}

\ccCreationVariable{t}

\ccDefinition
A class \ccClassName\ that satisfies the requirements of a alpha shape
traits class must provide the following predicate and operations in
addition to the requirements for the Delaunay triangulation traits
class.

\ccTypes

\ccNestedType{Coord_type}{A type to hold a coordinate type class. 
The type must provide a copy constructor, assignment, comparison
operators, negation, multiplication, division and allow the
declaration and initialization with a small integer constant
(cf. requirements for number types). An obvious choice would be
coordinate type of the point class.}

\ccCreation

Only a default constructor is required. Note that further constructors
can be provided. 

\ccConstructor{Traits();}
{A default constructor.}


\ccOperations

\ccMethod{Coord_type squared_radius_smallest_circumcircle(const Point& p0,
                                       		const Point& p1, 
						const Point& p2 );} 
{Returns the squared radius of smallest circumscribing circle of the
points \ccStyle{p0, p1, p2}. }

\ccMethod{Coord_type squared_radius_smallest_circumcircle(const Point& p0,
                                       const Point& p1);} 
{Returns the squared radius of smallest circumscribing circle of the
points \ccStyle{p0, p1}. }

\ccMethod{CGAL_Bounded_side side_of_bounded_circle(const Point& p0,
                                    const Point& p1,
                                    const Point& test);}
{Returns the relative position of point \ccStyle{test} to the circle
defined as the smallest circumscribing circle of the points \ccStyle{p0,
p1}.}

\end{ccClass}

\newpage
%----------------------------------------------------------------------

\section{Alpha Shape of Points in 3-Space \label{I1_SectAlpha Shape_3}}

\begin{ccClassTemplate} {CGAL_Alpha_shape_3<Traits>}

\ccDefinition
The class \ccClassTemplateName\ represents the family of
$\alpha$-shapes of points in a plane for {\em all} positive
$\alpha$. It maintains the underlying Delaunay tetrahedralization which
represents connectivity and order among its faces. Each
$k$-dimensional face of the Delaunay tetrahedralization is associated with
an interval that specifies for which values of $\alpha$ the face
belongs to the $\alpha$-shape. There are links between the intervals
and the $k$-dimensional faces of the Delaunay tetrahedralization.


\ccInclude{CGAL/Alpha_shape_3.h}

\ccInheritsFrom

\ccStyle{CGAL_Delaunay_Tetrahedralization_3<Traits>}

The modifying functions \ccStyle{insert} and \ccStyle{remove} overwrite
the inherited functions. At the moment, only the static version is implemented.

\ccTypes
\ccSetThreeColumns{CGAL_Oriented_side}{}{\hspace*{10cm}}
\ccThreeToTwo

\ccNestedType{Traits}{the alpha shape traits type.}
 
It contains the Delaunay tetrahedralization traits class.  For example
\ccStyle{Traits::Point} is a mapping on a point class. Additionally,
it defines a mapping on a coordinate type class.

\ccTypedef{typedef Traits::Coord_type Coord_type;}{}

\ccNestedType{Alpha_iterator}{An iterator that allow to traverse 
the sorted sequence of different $\alpha$-values. The iterator is
bidirectional and non-mutable. Its \ccStyle{value_type}
is \ccStyle{pair<Coord_type, Face*>}}

\ccEnum{enum Classification_type {EXTERIOR, SINGULAR, REGULAR, INTERIOR};}
{Distinguishes the different cases for classifying a $k$-dimensional face
 of the underlying Delaunay tetrahedralization of the $\alpha$-shape. \\
\ccStyle{EXTERIOR} if the face does not belong to the $\alpha$-complex.\\
\ccStyle{SINGULAR} if the face belongs to the boundary of the $\alpha$-shape,
 but is not incident to any 3-dimensional face of the $\alpha$-complex\\
\ccStyle{REGULAR} if the face belongs to the boundary of the $\alpha$-shape
 and is incident to a 3-dimensional face of the $\alpha$-complex\\
\ccStyle{INTERIOR} if the face belongs to the $\alpha$-complex, but does
not belong to the boundary of the $\alpha$-shape\\}

\ccEnum{enum Mode {GENERAL, REGULARIZED};}
{ In general, an alpha shape is a non-connected, mixed-dimension
polygon. Its regularized version is formed by the set of regular faces
and their vertices}

\ccCreation
\ccCreationVariable{A}

\ccConstructor{CGAL_Alpha_shape_3(Coord_type alpha = 0,
				  Mode m = GENERAL);}
{Introduces an empty $\alpha$-shape \ccVar\ for a positive $\alpha$-value
 \ccStyle{alpha}.
\ccPrecond \ccStyle{alpha}~$\geq~0$.}


\ccConstructor{template < class InputIterator >
		CGAL_Alpha_shape_3(
			InputIterator first,
			InputIterator last,
                const Coord_type& alpha = 0,
	        Mode m = GENERAL);}
{Initializes the family of alpha-shapes with the points in the range
$\left[\right.$\ccStyle{first}, \ccStyle{last}$\left.\right)$ and 
introduces an $\alpha$-shape \ccVar\ for a positive $\alpha$-value
\ccStyle{alpha}. 
\ccPrecond The \ccStyle{value_type} of \ccStyle{first} and
\ccStyle{last} is \ccStyle{Point}.\\
\ccStyle{alpha} $\geq 0$.}

\ccOperations

\ccMethod{template < class InputIterator >
		int make_Alpha_shape(
			InputIterator first,
			InputIterator last,
                const Coord_type& alpha = 0,
	        Mode m = GENERAL);}
{Initialize the family of alpha-shapes with the points in the range
$\left[\right.$\ccStyle{first}, \ccStyle{last}$\left.\right)$ and 
introduces an $\alpha$-shape \ccVar\ for a positive $\alpha$-value
\ccStyle{alpha}. Returns the number of inserted points. \\
If the function is applied to an non-empty family of alpha-shape, it is cleared
before initialization.
\ccPrecond The \ccStyle{value_type} of \ccStyle{first} and
\ccStyle{last} is \ccStyle{Point}.\\
\ccStyle{alpha} $\geq 0$.}

\ccMethod{void
	clear();}
{Clears the structure.}

\ccMethod{Coord_type
	set_alpha(const Coord_type& alpha);}
{Sets the $\alpha$-value to \ccStyle{alpha}.
 Returns the previous $\alpha$-value.
\ccPrecond \ccStyle{alpha} $\geq 0$.}

\ccMethod{const Coord_type&
	get_alpha(void) const;}
{Returns the current $\alpha$-value.}

\ccMethod{const Coord_type&  get_nth_alpha(int n) const;}
{Returns the $n$-th alpha-value.
 \ccPrecond \ccStyle{n} < number of alphas.}

\ccMethod{int number_of_alphas() const;}
{Returns the number of different alpha-values.}

% dynamic version	
% 
% \ccMethod{Vertex* insert(const Point& p);}
% {Inserts point \ccStyle{p} in the alpha shape and returns the
% corresponding vertex of the underlying Delaunay tetrahedralization.\\ If
% point \ccStyle{p} coincides with an already existing vertex, this
% vertex is returned and the alpha shape remains unchanged.\\ Otherwise,
% the vertex is inserted in the underlying Delaunay tetrahedralization and
% the associated intervals are updated. }
% 
% \ccMethod{void remove(Vertex *v);}
% {Removes the vertex from the underlying Delaunay tetrahedralization. The
% created hole is retriangulated and the associated intervals are
% updated.}
% 

\ccMethod{Mode
	set_mode(Mode m = GENERAL );}
{Sets \ccVar\ to its general or regularized version. 
Returns the previous mode.}

\ccMethod{Mode
	get_mode(void) const;}
{Returns whether \ccVar\ is general or regularized.}

\ccMethod{template  < class OutputIterator >
	  OutputIterator get_Alpha_shape_vertices(
					 OutputIterator result);}
{Writes the vertices of the alpha shape \ccVar\ for the current $\alpha$-value
to the container where \ccStyle{result} refers to. 
The \ccStyle{value_type} of \ccStyle{result} is \ccStyle{Vertex*}.
Returns an output iterator which is the end of the constructed range.}

\ccMethod{template  < class OutputIterator >
	  OutputIterator get_Alpha_shape_edges(
					 OutputIterator result);}
{Writes the edges  of the alpha shape \ccVar\ for the current $\alpha$-value
to the container where \ccStyle{result} refers to. 
The \ccStyle{value_type} of \ccStyle{result} is 
\ccStyle{CGAL_Triple<Simplex*, int, int>}.
Returns an output iterator which is the end of the constructed range.}

\ccMethod{template  < class OutputIterator >
	  OutputIterator get_Alpha_shape_faces(
					 OutputIterator result);}
{Writes the faces 
of the alpha shape \ccVar\ for the current $\alpha$-value
to the container where \ccStyle{result} refers to. 
The \ccStyle{value_type} of \ccStyle{result} is \ccStyle{pair<Simplex*, int>}.
Returns an output iterator which is the end of the constructed range.}

\ccHeading{Predicates}

\ccMethod{Classification_type
           classify(const Point& p, 
	const Coord_type& alpha = get_alpha()) const;}
{Classifies a point \ccStyle{p} with respect to \ccVar.}

\ccMethod{Classification_type
           classify(Simplex* s, const Coord_type& alpha = get_alpha()) const;}
{Classifies the simplex \ccStyle{s} of the underlying Delaunay tetrahedralization with respect to \ccVar.}

\ccMethod{Classification_type
           classify(pair<Simplex*, int> f, const Coord_type& alpha = get_alpha()) const;}
{Classifies the face \ccStyle{f} of the underlying Delaunay tetrahedralization with respect to \ccVar.}

\ccMethod{Classification_type
           classify(Simplex* s, int i, const Coord_type& alpha = get_alpha()) const;}
{Classifies the face of the simplex \ccStyle{s} opposite to the vertex with index \ccStyle{i} of the underlying Delaunay tetrahedralization with respect to \ccVar.}

\ccMethod{Classification_type
           classify(CGAL_Triple<Simplex*, int, int> e, const Coord_type& alpha = get_alpha()) const;}
{Classifies the edge \ccStyle{e} of the underlying Delaunay tetrahedralization with respect to \ccVar.}

\ccMethod{Classification_type
           classify(Simplex* s, int i, int j, const Coord_type& alpha = get_alpha()) const;}
{Classifies the edge of the simplex \ccStyle{s} from the vertex with index \ccStyle{i} to the vertex with index \ccStyle{j} of the underlying Delaunay tetrahedralization with respect to \ccVar.}

\ccMethod{Classification_type
           classify(Vertex* v, const Coord_type& alpha = get_alpha()) const;}
{Classifies the vertex \ccStyle{v} of the underlying Delaunay tetrahedralization with respect to \ccVar.}


\ccHeading{Traversal of the $\alpha$-Values}

\smallskip
The alpha shape class defines an iterator that allows to visit the
sorted sequence of $\alpha$-values. This iterator is
non-mutable and bidirectional. Its value type is
\ccStyle{Coord_type}.

\ccMethod{Alpha_iterator alpha_begin() const;}
{Returns an iterator that allows to traverse the
sorted sequence of $\alpha$-values of the family of alpha shapes.}

\ccMethod{Alpha_iterator alpha_end() const;}
{Returns the corresponding past-the-end iterator.}

\ccMethod{Alpha_iterator alpha_find(const Coord_type& alpha) const;}
{Returns an iterator pointing to an element with $\alpha$-value
\ccStyle{alpha}, or the corresponding past-the-end iterator if such 
an element is not found.}

\ccMethod{Alpha_iterator alpha_lower_bound(const Coord_type& alpha) const;}
{Returns an iterator pointing to the first element with
$\alpha$-value not less than \ccStyle{alpha}.}

\ccMethod{Alpha_iterator alpha_upper_bound(const Coord_type& alpha) const;}
{Returns an iterator pointing to the first element with $\alpha$-value
greater than \ccStyle{alpha}.}

\ccHeading{Operations}

\ccMethod{int number_solid_components(const Coord_type& alpha = get_alpha()) const;}
{Returns the number of solid components of \ccVar, that is, the number of components of its regularized version.}

\ccMethod{Alpha_iterator find_optimal_alpha(int nb_components) const;}
{Returns an iterator pointing to the first element with $\alpha$-value
such that \ccVar\ satisfies the following two properties:\\
\ccStyle{nb_components} equals the number of solid components and \\
all data points are either on the boundary or in the interior of the regularized version of \ccVar.\\
If no such value is found, the iterator points to the first element with 
$\alpha$-value such that \ccVar\ satisfies the second property.}
\ccHeading{I/O}

The I/O operators are defined for \ccStyle{iostream}, and for
the window stream provided by \cgal. The format for the iostream
is an internal format. 

\ccInclude{CGAL/IO/ostream_3.h}

\ccFunction{ostream& operator<<(ostream& os,
                  const CGAL_Alpha_shape_3<Traits>& A);}
{Inserts the alpha shape \ccVar\ for the current $\alpha$-value into the stream \ccStyle{os}.
\ccPrecond The insert operator must be defined for \ccStyle{Point}.}

\ccInclude{CGAL/IO/Geomview.h}

\ccFunction{CGAL_Geomview_stream& operator<<(CGAL_Geomview_stream& G,
                         const CGAL_Alpha_shape_3<Traits>& T);}
{Inserts the alpha shape \ccVar\ for the current $\alpha$-value into the window stream \ccStyle{G}.
\ccPrecond The insert operator must be defined for  \ccStyle{Segment} and \ccStyle{Triangle}.}
\end{ccClassTemplate}

\ccImplementation
In the first static version, the set of intervals associated with the
$k$-dimensional faces of the underlying Delaunay tetrahedralization are
stored only as sorted \ccStyle{vectors}. By using an interval tree the
alpha-shape could be constructed more efficiently. For the dynamic
version, we need \ccStyle{multimaps} or dynamic interval trees. The
cross links between the intervals and the $k$-dimensional faces of the
Delaunay tetrahedralization are actually realized using
\ccStyle{multimaps} resp.\ \ccStyle{hash multimaps}.

\ccStyle{A.alpha find} uses linear search, while 
\ccStyle{A.alpha lower bound} and \ccStyle{A.alpha upper bound} 
use binary search.
\ccStyle{A.number solid components} performs a graph traversal and takes time linear in the number of faces of the underlying Delaunay triangulation.
\ccStyle{A.find optimal alpha} uses binary search and takes time
$O(\mbox{\em number of faces } \log{\mbox{\em number of faces}})$.
%----------------------------------------------------------------------

\section{Requirements\label{I1_SectRequirements}}

\subsection{Requirements for the Alpha-Shape Traits Class} 

The class \ccStyle{CGAL_Alpha_shape_3<Traits>} parameterized with a
traits class, that has the same requirements as the Delaunay
tetrahedralization traits class.  The following requirement catalog lists
the primitives that must be defined additionnally.

\begin{ccClass} {Traits}
\subsection*{Traits (\mbox{\it Traits})}

\ccCreationVariable{t}

\ccDefinition
A class \ccClassName\ that satisfies the requirements of a alpha shape
traits class must provide the following predicates and geometric
primitives in addition to the requirements for the Delaunay
tetrahedralization traits class.

\ccTypes

\ccNestedType{Coord_type}{A type to hold a coordinate type class. 
The type must provide a copy constructor, assignment, comparison
operators, negation, multiplication, division and allow the
declaration and initialization with a small integer constant
(cf. requirements for number types). An obvious choice would be
coordinate type of the point class.}

\ccCreation

Only a default constructor is required. Note that further constructors
can be provided. 

\ccConstructor{Traits();}
{A default constructor.}


\ccOperations

\ccMethod{Coord_type squared_radius_smallest_circumsphere(const Point& p0,
                                       		const Point& p1, 
						const Point& p2,
	                                        const Point& p3);} 
{Returns the squared radius of smallest circumscribing sphere of the
points \ccStyle{p0, p1, p2, p3}. }

\ccMethod{Coord_type squared_radius_smallest_circumsphere(const Point& p0,
                                       		const Point& p1, 
						const Point& p2 );} 
{Returns the squared radius of smallest circumscribing sphere of the
points \ccStyle{p0, p1, p2}. }

\ccMethod{Coord_type squared_radius_smallest_circumsphere(const Point& p0,
                                       const Point& p1);} 
{Returns the squared radius of smallest circumscribing sphere of the
points \ccStyle{p0, p1}. }

\ccMethod{CGAL_Bounded_side side_of_bounded_sphere(const Point& p0,  const Point& p1, const Point& p2, const Point& test);}
{Returns the relative position of point \ccStyle{test} to the sphere
defined as the smallest circumscribing sphere of the points \ccStyle{p0,
p1, p2}.}

\ccMethod{CGAL_Bounded_side side_of_bounded_sphere(const Point& p0,
                                    const Point& p1,
                                    const Point& test);}
{Returns the relative position of point \ccStyle{test} to the sphere
defined as the smallest circumscribing sphere of the points \ccStyle{p0,
p1}.}

\end{ccClass}

\end{document} 
