% =============================================================================
% The CGAL Reference Manual
% Chapter: Geometric Optimisation
% Section: Smallest Enclosing Sphere of Spheres
% -----------------------------------------------------------------------------
% file  : doc_tex/basic/Optimisation/Optimisation_ref/Traits.tex
% author: Kaspar Fischer (fischerk@inf.ethz.ch)
% -----------------------------------------------------------------------------
% $Revision$
% $Date$
% $CGAL_Package$
% =============================================================================

\begin{ccRefConcept}{MinSphereOfSpheresTraits}

% -----------------------------------------------------------------------------
\ccDefinition

A model of concept \ccc{MinSphereOfSpheresTraits} must provide the
following constants, types, predicates and operations.

\ccHasModels
%\ccc{CGAL::Min_sphere_of_spheres_d_traits_d<K,FT,UseSqrt,Algorithm>}

\ccConstants 

\ccNestedType{D}{specifies the dimension of the spheres you want to
  compute the minsphere of.}

\ccTypes

\ccNestedType{Sphere}{is a typedef to to some class representing a sphere.
(The package will compute the minsphere of spheres of type
\ccc{Sphere}.)}

\ccNestedType{FT}{is a (exact or inexact) field number type.
  \ccRequire Currently, \ccc{FT} must either be \ccc{double} or
  \ccc{float}, or an exact field number type.}

\ccNestedType{Coordinate_iterator}{non-mutable model of the STL
concept \ccc{BidirectionalIterator} with value type \ccc{FT}. Used
to access the center coordinates of a sphere.}

\ccNestedType{Use_square_roots}{must typedef to either
\ccc{CGAL::Tag_true} or \ccc{CGAL::Tag_false}: In the former case, the
algorithm computing the minsphere will perform square-root operation on
instances of type \ccc{FT} where appropriate; \ccc{FT} must then be a
number type supporting square-roots (see \ccc{Number_type_traits}).
On the other hand, if the compile-time flag is \ccc{CGAL::Tag_false},
the algorithm will work without doing square-roots (so \ccc{FT} need
not provide a square-root operation).\\
%
\emph{Note:} On some platforms it is much faster to disable
square-roots (due to lack of hardware support for computing
square-roots).  So even in cases where
\ccc{Number_type_traits<FT>::Has_sqrt} is \ccc{Tag_true}, it might be
worth setting \ccc{Use_square_roots} to \ccc{Tag_false}.}

\ccNestedType{Algorithm}{selects the method to compute the minsphere.
  It must typedef to either \ccc{CGAL::Default_algorithm},
  \ccc{CGAL::LP_algorithm} or \ccc{CGAL::Farthest_first_heuristic}.
  The recommended choice is the first, which is a synonym to the one
  of the other two methods which we consider ``the best in practice.''
  In case of \ccc{CGAL::LP_algorithm}, the minsphere will be computed
  using the LP-algorithm~\cite{msw-sblp-92}, which in our
  implementation has maximal expected running time $O(2^d n)$ (in the
  number of operations on the number type \ccc{FT}).  In case of
  \ccc{CGAL::Farthest_first_heuristic}, a simple heuristic will be
  used instead which seems to work fine in practice, but comes without
  a guarantee on the running time.  For an inexact number type
  \ccc{FT} we strongly recommend \ccc{CGAL::Farthest_first_heuristic}
  since it handles most degeneracies in a satisfying manner.  Notice
  that this compile-time flag is taken as a hint only.  Should one of
  the methods not be available anymore in a future release, then the
  default algorithm will be chosen.}

\ccAccessFunctions

\ccCreationVariable{traits}

\ccMemberFunction{ FT radius(const
Sphere& s);}{ returns the radius of sphere \ccc{s}.
\ccPostcond The returned number is
 greater or equal to~$0$.}

\ccMemberFunction{ Coordinate_iterator begin(const Sphere& s);}{
  returns an iterator referring to the first of the \ccc{D} Cartesian
  coordinates of the center of \ccc{s}.}

\end{ccRefConcept}

