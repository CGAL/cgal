% +------------------------------------------------------------------------+
% | Reference manual page: insert_non_intersecting.tex
% +------------------------------------------------------------------------+
% | 
% | Package: Arrangement_2
% | 
% +------------------------------------------------------------------------+

\ccRefPageBegin

\begin{ccRefFunction}{insert_non_intersecting_curve}

\ccInclude{CGAL/Arrangement_2.h}

\ccFunction{template<class Traits, class Dcel, class PointLocation>
            typename Arrangement_2<Traits,Dcel>::Halfedge_handle 
            insert_non_intersecting_curve (Arrangement_2<Traits,Dcel>& arr,
                         const typename Traits::X_monotone_curve_2& c,
                         const PointLocation& pl);}
   {Inserts the given $x$-monotone curve \ccc{c} into the arrangement
    \ccc{arr}, where the interior of \ccc{c} is disjoint from all existing
    arrangement vertices and edges. Under this assumption, it is possible to
    locate the endpoints of \ccc{c} in the arrangement and use one of the
    specialized insertion member-functions of \ccc{arr} according to the
    results. As no intersection are computed, the \ccc{Traits} parameter
    must be a model of the restricted \ccc{ArrangementBasicTraits_2} concept.
    The point-location object \ccc{pl}, which must be a model of the
    \ccc{ArrangementPointLocation_2} concept, is used for answering
    the two point-location queries on \ccc{c}'s endpoints. As the insertion
    operations creates just a single new edge, the function returns a handle
    for this one of twin halfedges that form this edge.
    \ccPrecond{\ccc{pl} is attached to the given arrangement \ccc{arr}.}}

\ccFunction{template<class Traits, class Dcel>
            typename Arrangement_2<Traits,Dcel>::Halfedge_handle
            insert_non_intersecting_curve (Arrangement_2<Traits,Dcel>& arr,
                         const typename Traits::X_monotone_curve_2& c);}
   {Inserts the given $x$-monotone curve \ccc{c} into the arrangement
    \ccc{arr}, where the interior of \ccc{c} is disjoint from all existing
    arrangement vertices and edges. The function returns a handle for the
    one of the twin halfedges created by the insertion.
    The function operates as the function listed above, but it uses the default
    ``walk along line'' point-location strategy, so users need not provide a
    point-location object.}

\end{ccRefFunction}

\ccRefPageEnd
