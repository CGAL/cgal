% +------------------------------------------------------------------------+
% | Reference manual page: arr_insert_x_monotone.tex
% +------------------------------------------------------------------------+
% | 
% | Package: Arrangement_2
% | 
% +------------------------------------------------------------------------+

\ccRefPageBegin

\begin{ccRefFunction}{insert_x_monotone}

\ccInclude{CGAL/Arrangement_2.h}

\ccFunction{template<class Traits, class Dcel, class PointLocation>
            void insert_x_monotone (Arrangement_2<Traits,Dcel>& arr,
                         const typename Traits::X_monotone_curve_2& c,
                         const PointLocation& pl);}
   {Inserts the given $x$-monotone curve \ccc{c} into the arrangement
    \ccc{arr}. The \ccc{Traits} parameter should be a model of the
    \ccc{ArrangementXMonotoneTraits_2} concept. \ccc{c} is in inserted
    into the arrangement by locating its left endpoint and computing its
    zone until reaching the right endpoint.
    The point-location object \ccc{pl}, which should be a model of the
    \ccc{ArrangementPointLocation_2} concept, is used for locating the
    left endpoint of \ccc{c} in the exising arrangemnt.
    \ccPrecond{\ccc{pl} is attached to the given arrangement \ccc{arr}.}}

\ccFunction{template<class Traits, class Dcel>
            void insert_x_monotone (Arrangement_2<Traits,Dcel>& arr,
                         const typename Traits::X_monotone_curve_2& c);}
   {Inserts the given $x$-monotone curve \ccc{c} into the arrangement
    \ccc{arr}. The function operates as the function listed above, but it
    uses the default ``walk along line'' point-location strategy, so users
    need not provide a point-location object.}

\ccFunction{template<class Traits, class Dcel, InputIterator>
            void insert_x_monotone (Arrangement_2<Traits,Dcel>& arr,
                         InputIterator first, InputIterator last);}
   {Inserts the given range of $x$-monotone curves \ccc{[first,last)}
    into the arrangement \ccc{arr}. The insertion is performed in an aggregated
    manner, using the sweep-line algorithm. The \ccc{Traits} parameter should
    be a model of the \ccc{ArrangementXMonotoneTraits_2} concept.
    \ccPrecond{The value-type of \ccc{InputIterator} is 
               \ccc{Traits::X_monotone_curve_2}.}}

\end{ccRefFunction}

\ccRefPageEnd
