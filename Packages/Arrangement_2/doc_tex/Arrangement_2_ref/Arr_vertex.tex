% +------------------------------------------------------------------------+
% | Reference manual page: Arr_vertex.tex
% +------------------------------------------------------------------------+
% | 
% | Package: Arrangement_2
% | 
% +------------------------------------------------------------------------+

\ccRefPageBegin

\begin{ccRefClass}[Arrangement_2<Traits,Dcel>::]{Vertex}
\ccRefLabel{arr_ref:vertex}

\ccDefinition
An object $v$ of the class \ccRefName\ represents an arrangement vertex,
that is --- a $0$-dimensional cell, associated with a point on the plane. 

\ccInheritsFrom
    \ccHtmlNoLinksFrom{\ccc{typename Dcel::Vertex}}

\ccCreation
\ccCreationVariable{v}
%=====================

\ccConstructor{Vertex();}
    {default constructor.}    

\ccAccessFunctions
%=================

All non-const methods listed below also have \ccc{const} counterparts
that return constant handles, iterators or circulators:

\ccMethod{Vertex_handle handle();}
    {returns a handle for the vertex \ccVar{}.}

\ccMethod{bool is_isolated() const;}
    {checks if the vertex is isolated (has no incident edges).}

\ccMethod{typename Dcel::Size degree() const;}
    {returns the number of edges incident to \ccVar{}.}

\ccMethod{Halfedge_around_vertex_circulator incident_halfedges();}
    {returns a circulator circulator that allows going over the halfedges
     incident to \ccVar{} (that have \ccVar{} as their target).
     The edges are traversed in a clockwise direction around \ccVar{}.
     \ccPrecond{\ccVar{} is not an isolated vertex.}}

\ccMethod{const typename Traits::Point_2& point() const;}
    {returns the point associated with the vertex.}

\end{ccRefClass}

\ccRefPageEnd
