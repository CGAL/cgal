\ccRefPageBegin
\begin{ccRefClass}{Arr_segment_traits_2<Kernel>}
    
\ccDefinition 
  The traits class \ccStyle{Arr_segment_traits_2<Kernel>} is a model of the
  \ccc{ArrangementTraits_2} concept. It should be templated with a CGAL-Kernel
  model that is templated in turn with a number type. The number type should be
  exact to avoid robustness problems, although other number types could be used
  at the user's own risk. For example, \ccStyle{Cartesian<Quotient<MP_Float> >},
  \ccStyle{Homogeneous<leda_integer>}, \ccStyle{Cartesian<double>}.

  It achieves faster running times than the
  \ccStyle{Arr_non_caching_segment_traits_2} traits-class, when arrangements
  with relatively many intersection points are constructed. It also allows for
  working with less accurate, yet computationally efficient number types, such
  as \ccStyle{Quotient<MP_Float>}, which represents floating-point numbers with
  an unbounded mantissa, but with a bounded exponent. On the other hand, it
  uses more space and stores extra data with each segment, so constructing
  sparse arrangements could be more efficient with the
  \ccStyle{Arr_non_caching_segment_traits_2} traits-class.

\ccInclude{CGAL/Arr_segment_traits_2.h}
 
\ccIsModel
    \ccc{ArrangementTraits_2}

\end{ccRefClass}
\ccRefPageEnd
