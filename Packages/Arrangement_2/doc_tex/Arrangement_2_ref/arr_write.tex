% +------------------------------------------------------------------------+
% | Reference manual page: arr_write.tex
% +------------------------------------------------------------------------+
% | 
% | Package: Arrangement_2
% | 
% +------------------------------------------------------------------------+

\ccRefPageBegin

\begin{ccRefFunction}{write}

\ccInclude{CGAL/IO/Arr_iostream.h}

\ccFunction{template<class Traits, class Dcel, class Formatter>
            std::ostream& write (const Arrangement_2<Traits,Dcel>& arr,
                                 std::ostream& os,
                                 Formatter& formatter);}
   {Writes the arrangement \ccc{arr} into the given output stream using a specific
    output format. \ccc{formatter}, which must be a model of the
    \ccc{ArrangementOutputFormatter}, defines the output format.}

\ccFunction{template<class Traits, class Dcel>
            std::ostream& operator<< (std::ostream& os,
                                      const Arrangement_2<Traits,Dcel>& arr);}
   {Writes the arrangement \ccc{arr} into the given output stream using the
    output format defined by the \ccc{Arr_text_formatter} class --- that is,
    only the basic geometric and topological features of the arrangement are
    written, without any auxiliary data that may be attached to the \dcel\ features.}

\end{ccRefFunction}

\ccRefPageEnd
