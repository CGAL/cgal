% +------------------------------------------------------------------------+
% | Reference manual page: Arr_face_graph_adaptor.tex
% +------------------------------------------------------------------------+
% | 
% | Package: Arrangement_2
% | 
% +------------------------------------------------------------------------+

\ccRefPageBegin

\begin{ccRefClass}{Arr_bgl_dual_adator<Arrangement>}
    
\ccDefinition
%============

This class adapts an \ccc{Arrangement_2} instance to a BGL graph. It defines 
the necessary types and methods required by some BGL graph-concepts listed 
below. Thus, enabling the use of BGL algorithms that operate on models of
these graph concepts. The \ccc{Arrangement_2} faces are considered as the 
verticies of the graph. More precisely, \ccc{Arrangement_2::Face_handle} 
are the graph verticies, and \ccc{Arrangement_2::Halfedge_handle} are the 
edges of the graph.

The \ccClassTemplateName\ class template should be instantiated with a 
class that models the \ccc{Arranagement_2} concept, an instance of which is 
the data structure to be adapted.

The adaptor models three BGL concepts:
\begin{itemize}
\item \ccc{Graph}
\item \ccc{IncidenceGraph}
\item \ccc{VertexListGraph}
\end{itemize}

\ccInclude{CGAL/Arr_bgl_dual_adator.h}

\ccCreation
\ccCreationVariable{arr}
%=======================
    
\ccConstructor{Arr_bgl_dual_adator(Planar_map_2& in_arrangement);}
    {constructs an adaptor to BOOST graph for the given \ccc{in_arrangement}.}
    
\ccSeeAlso
    \ccc{ArrangementDcel}\lcTex{ 
      (\ccRefPage{ArrangementDcel})}\\
    \ccc{make_arr_bgl_dual_adaptor()}\lcTex{ 
      (\ccRefPage{make_arr_bgl_dual_adaptor})}

\end{ccRefClass}

\ccRefPageEnd
