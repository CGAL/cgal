% +------------------------------------------------------------------------+
% | Reference manual page: arr_insert.tex
% +------------------------------------------------------------------------+
% | 
% | Package: Arrangement_2
% | 
% +------------------------------------------------------------------------+

\begin{ccRefFunction}{insert}

\ccInclude{CGAL/Arrangement_2.h}

\ccFunction{template<class Traits, class Dcel, class PointLocation>
            void insert (Arrangement_2<Traits,Dcel>& arr,
                         const typename Traits::Curve_2& c,
                         const PointLocation& pl);}
   {Inserts the given curve \ccc{c} into the arrangement \ccc{arr}, where
    no restrictions are made on the nature of the inserted curve.
    The \ccc{Traits} parameter should be a model of the
    \ccc{ArrangementTraits_2}
    concept --- that is, it should define the \ccc{Curve_2} type and support
    its subdivision into $x$-monotone subcurves (and perhaps isolated points).
    Each subcurve is in turn inserted into the arrangement by locating its
    left endpoint and computing its zone until reaching the right endpoint.
    The point-location object \ccc{pl}, which should be a model of the
    \ccc{ArrangementPointLocation_2} concept, is used for answering
    point-location queries during the insertion process.
    \ccPrecond{\ccc{pl} is attached to the given arrangement \ccc{arr}.}}

\ccFunction{template<class Traits, class Dcel>
            void insert (Arrangement_2<Traits,Dcel>& arr,
                         const typename Traits::Curve_2& c);}
   {Inserts the given curve \ccc{c} into the arrangement \ccc{arr}, where
    no restrictions are made on the nature of the inserted curve.
    The function operates as the function listed above, but it uses the default
    ``walk along line'' point-location strategy, so users need not provide a
    point-location object.}

\ccFunction{template<class Traits, class Dcel, InputIterator>
            void insert (Arrangement_2<Traits,Dcel>& arr,
                         InputIterator first, InputIterator last);}
   {Aggregately inserts the given range of curves \ccc{[first,last)} into
    the arrangement \ccc{arr}, where no restrictions are made on the nature of
    the inserted curves. The \ccc{Traits} parameter should be a model of the
    \ccc{ArrangementTraits_2} concept. The input curves are subdivided into
    $x$-monotone subcurves (and perhaps isolated points), which are inserted
    into the arrangement using the sweep-line algorithm.
    \ccPrecond{The value-type of \ccc{InputIterator} is 
               \ccc{Traits::Curve_2}.}}

\end{ccRefFunction}
