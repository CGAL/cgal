\renewcommand{\Re}{{\rm I\!\hspace{-0.025em} R}}

\def\C{{\cal C}}
\def\G{{\cal G}}
\def\F{{\cal F}}
\def\I{{\cal I}}
\def\U{{\cal U}}
\def\M{{\cal M}}
\def\eps{{\varepsilon}}
\def\bd{{\partial}}
\def\dm{{\cal D}}

% ===============================================================
\section{Introduction}
\label{arr_sec:intro}
% ===================
Given a set $\mathcal{C}$ of planar curves, the {\em arrangement}
${\mathcal A}({\mathcal C})$ is the subdivision of the plane induced
by the curves in $\mathcal{C}$ into maximally connected cells. The cells
can be $0$-dimensional ({\em vertices}), $1$-dimensional ({\em edges})
or $2$-dimensional ({\em faces}). The {\em planar map} of ${\mathcal
A}({\mathcal C})$ is the embedding of the arrangement as a planar
graph, such that each arrangement vertex corresponds to a planar
point, and each edge corresponds to a planar subcurve of one of the
curves in ${\mathcal C}$.
The class \ccc{Arrangement_2<Traits,Dcel>} is a data structure that maintains
arrangements in the plane. In particular, the data structure maintains a
planar map perhaps with its features exsented with additional information.
