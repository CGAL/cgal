
\newcommand{\reals}{{\rm I\!\hspace{-0.025em} R}}
\newcommand{\C}{{\cal C}}
\newcommand{\A}{{\cal A}}
\newcommand{\eps}{{\varepsilon}}
\newcommand{\dcel}{{\sc Dcel}}


% ===============================================================
\section{Introduction}
\label{arr_sec:intro}
% ===================

Given a set $\C$ of planar curves, their {\em arrangement}
$\A(\C)$ is the subdivision they induce on the plane into
zero-dimensional, one-dimensional and two-dimensional cells,
called {\em vertices}, {\em edges} and {\em faces}, respectively.

As the curves in $\C$ can intersect and are not necessarily
$x$-monotone, we construct a collection $\C''$ in two steps. We
first decompose each curve in $\C$ into maximal $x$-monotone
curves (and possible isolated points), thus obtaining the
collection $\C'$. Then we decompose each curve in $\C'$ into
maximal connected pieces not intersecting any other curve (or
point) in $\C'$. This way we obtain the collection $\C''$ of
$x$-monotone subcurves that pairwise disjoint in their interiors.
This collection may also contain isolated points, if the curves of
$\C$ contain such points. The collection $\C''$ can be
conveniently embedded as a planar graph, whose vertices are
associated with curve endpoints or with isolated points and whose
edges are associated with subcurves. This graph can represented
using a {\em doubly-connected edge list} data-structure (\dcel\
for short), which consists of containers of vertices, edges and
faces and maintains the incidence relations among these objects.

The main idea behind the \dcel\ data-structure is to represent
each edge using a pair of directed {\em halfedges}, one going from
the (lexicographically) left endpoint of the curve to its right
endpoint, and the other --- known as its {\em twin} halfedge ---
going in the opposite direction. Halfedges are used to separate
faces and to connect vertices (with the exception of {\em isolated
vertices}, which are not connected).

If a vertex $v$ is the target of a halfedge $e$, we say that $v$
and $e$ are {\em incident} to each other. The halfedges incident
to a vertex $v$ form a circular list oriented in a clockwise order
around this vertex (we assume $v$ is not an isolated vertex, as
isolated vertices have no incident halfedges).

Each halfedge $e$ stores a pointer to its {\it incident face},
which is the face lying to its left. Moreover, every halfedge is
followed by another halfedge sharing the same incident face, such
that the target vertex of the halfedge is the same as the source
vertex of the next halfedge. The halfedges are therefore connected
in circular lists and form chains, such that all edges of a chain
are incident to the same face and wind along its boundary. We call
such chains a {\em connected component of the boundary} (or {\em
CCB} for short).

A bounded face has a unique CCB of halfedges winding is a
counterclockwise orientation along its boundary, that is defined
to be its {\em outer CCB}, while the unbounded face does not have
an outer boundary (as the arrangement module supports only bounded
curves, we always have exactly one unbounded face in the
arrangement). Any other connected component of the boundary of the
face is called a {\em hole} (or {\em inner CCB}), and can be
represented as a circular chain of halfedges winding in a
clockwise orientation around its outer boundary. Note that a hole
is not necessarily a single face --- it may have no area, or
alternatively it may consist of several connected faces.  Every
face can have several holes contained in its interior (or there
may be no holes). In addition, every face may contain isolated
vertices in its interior. See Figure~\ref{arr_sec:seg_dcel} for an
illustration of the various \dcel\ features.

\begin{figure}[t]
\begin{ccTexOnly}
\begin{center}
\documentclass[11pt]{report}
% ======== Font ===============================================================
\usepackage[T1]{fontenc}
\usepackage{lmodern}
\usepackage{slantsc}
\usepackage{amssymb}
\usepackage{amsfonts}
\usepackage[english]{babel}
\usepackage[latin1]{inputenc}
\usepackage{pifont}

% ======== Math ===============================================================
\usepackage{amsmath,amscd,amsthm}
\usepackage{empheq}

% ======== pdf, url and hyperlink =============================================
% \usepackage{url}
\usepackage{hyperref}
\usepackage[hyphenbreaks]{breakurl}

% ======== Tables =============================================================
\usepackage{multirow}
\usepackage{multicol}
\usepackage{tabularx}

% ======== Listings ===========================================================
\usepackage{alltt}
\usepackage{listings}
\usepackage{mdwlist}

% ======== Graphics ===========================================================
\usepackage{graphicx}
\usepackage{etex}
% \usepackage{pst-all}
\usepackage{pstricks,pst-node,pst-tree,pst-text,pst-3d,pst-plot,pst-eucl}
\usepackage{pst-xkey,pst-jtree}
\usepackage{pst-lens}
\usepackage{colordvi}
% The tikz package loads the graphicx package, so no need to load it again.
% Trying to load it with different options will cause the "Option
% clash" error. Use \PassOptionsToPackage{<options>}{graphicx} before
% loading tikz to pass any additional options to the internally loaded
% graphicx.
\usepackage{tikz}
\usepackage{pgfplots}
\usepackage{mathptmx}
\usetikzlibrary{calc,patterns,decorations.pathmorphing,decorations.markings,matrix,fit,decorations.pathreplacing,arrows,automata,positioning,shapes,chains,spy}
%\usetikzlibrary{quotes,angles,graph}
\usetikzlibrary{intersections,through,backgrounds}
\usepackage{tikz-3dplot}
\tikzset{%
  invisible/.style={opacity=0},
  only/.code args={<#1>#2}{\only<#1>{\pgfkeysalso{#2}}},
  alt/.code args={<#1>#2#3}{\alt<#1>{\pgfkeysalso{#2}}{\pgfkeysalso{#3}}},
  temporal/.code args={<#1>#2#3#4}{%
    \temporal<#1>{\pgfkeysalso{#2}}{\pgfkeysalso{#3}}{\pgfkeysalso{#4}}},
  point/.style={circle,inner sep=1.5pt,minimum size=1.5pt,draw,fill=#1},
  point/.default=red
}
\makeatletter
\def\nodesDef{\@ifnextchar[{\@nodesDefWith}{\@nodesDefWithout}}
\def\@nodesDefWith[#1]#2{\foreach \c in #2 { \node[point=#1] (n\c) at (\c) {}; }}
\def\@nodesDefWithout#1{\foreach \c in #1 { \node[point] (n\c) at (\c) {}; }}
\makeatother
% #1 --- list of points
% #2 --- list of segments
% #3 --- first point
% #4 --- draw color
% #5 --- fill color
\newcommand{\directedPolygon}[5]{
  \fill[#4] (#3) \foreach \pa/\pb in #2 {-- (\pb)};
  \nodesDef{#1}
  \foreach \pa/\pb in #2 {
    \draw[thick,#5,->,>=stealth] (n\pa) -- (n\pb);
  };
}
%
% ======== Miscellaneous ======================================================
\usepackage{fancybox}
\usepackage{calc}
\usepackage{xspace}
% \usepackage{picins}
\usepackage[absolute,overlay]{textpos}
\usepackage{ifthen}
\usepackage{etoolbox}
%

%\usepackage{pgfpages}
%\pgfpagesuselayout{4 on 1}[letterpaper,border shrink=5mm,landscape]
%\pgfpagesuselayout{4 on 1}[a4paper,landscape]

% ======== DO NOT MOVE - Reset awkward values changed by pst-jtree ============
\psset{treevshift=0,unit=1cm,xunit=1cm,yunit=1cm,everytree={},
       etcratio=.75,triratio=.5}
%
% ======== Figures ============================================================
\DeclareGraphicsExtensions{.png}
\DeclareGraphicsExtensions{.jpg}
% \DeclareGraphicsRule{.png}{eps}{.bb}{`convert -compress JPEG #1 eps2:-}
\DeclareGraphicsRule{.png}{eps}{.bb}{`convert #1 eps2:-}
\DeclareGraphicsRule{.jpg}{eps}{.bb}{`convert #1 eps2:-}
%
% ======== Bibliography =======================================================
\newcommand{\etalchar}[1]{$^{#1}$}
%
% ======== colors =============================================================
\definecolor{lightred}{rgb}{1,0.8,0.8}
\definecolor{lightblue}{rgb}{0.8,0.8,1}
\definecolor{mediumred}{rgb}{1,0.5,0.5}
\definecolor{mediumblue}{rgb}{0.5,0.5,1}
\definecolor{darkred}{rgb}{0.5,0,0}
\definecolor{darkblue}{rgb}{0,0,0.5}
\definecolor{darkcyan}{rgb}{0,0.5,0.5}
\definecolor{lightcyan}{rgb}{0.5,1,1}
\definecolor{lightyellow}{rgb}{0.9,0.9,0.7}
\definecolor{olivegreen}{rgb}{.42,.55,.14}
\definecolor{seagreen}{rgb}{.18,.54,.34}

\newrgbcolor{lessthan}{1 0.2 0.2}
\newrgbcolor{equal}{1 1 0.2}
\newrgbcolor{greaterthan}{0.2 0.2 1}

\newrgbcolor{pcolor}{0.7 0.3 0.1}
\newrgbcolor{qcolor}{0 0 1}
\newrgbcolor{mscolor}{0.5 0.5 0}
%
\definecolor{polyALightColor}{rgb}{0.75,0.75,1}
\definecolor{polyBLightColor}{rgb}{1,0.8,0.3}
\definecolor{polyCLightColor}{rgb}{1,1,0.25}
\definecolor{polyDLightColor}{rgb}{1,1,0.75}

\definecolor{polyAColor}{rgb}{0.5,0.7,1}
\definecolor{polyBColor}{rgb}{1,0.72,0.225}
\definecolor{polyCColor}{rgb}{0.8,0.9,0.2}
\definecolor{polyDColor}{rgb}{0.8,0.8,0.6}

\definecolor{polyADarkColor}{rgb}{0.2,0.4,0.67}
\definecolor{polyBDarkColor}{rgb}{0.67,0.4,0.12}
\definecolor{polyCDarkColor}{rgb}{0.53,0.6,0.13}
\definecolor{polyDDarkColor}{rgb}{0.53,0.53,0.4}

\definecolor{polyABDarkColor}{rgb}{0.5,0.35,0.4}

\definecolor{crossColor}{rgb}{1,0,0.5}

\newrgbcolor{hot_color}{1 0 0}
\newrgbcolor{nonactive_color}{0.7 0.7 0.7}
\newrgbcolor{discovered}{0 0 0}
\newrgbcolor{active_color}{0 0 1}

\definecolor{obstacle}{rgb}{0.9,0.2,0.1}
\definecolor{robot1}{rgb}{0.1,0.4,0.9}
\definecolor{robot2}{rgb}{0.4,0.1,0.9}
\definecolor{minksum0}{rgb}{0.8,0.9,0.2}

\newrgbcolor{annulus-color}{0.000 0.690 0.000}
\newrgbcolor{in-color}{0.9 0 0}  %% {0.000 0.490 0.000}
\newrgbcolor{out-color}{0 0 0.9} % {0.690 0.000 0.690}
\newrgbcolor{inout-color}{0.000 0.490 0.000}
\newrgbcolor{circles-color}{0.690 0.000 0.69}
%
% ======== Theorems ===========================================================
%% \newtheorem{observation}[theorem]{Observation}
%% \newtheorem{proposition}[theorem]{Proposition}
%% \newtheorem{application}[theorem]{Application}
%
% ======== general macros =====================================================
\def\gf{\textsc{Geometry Factory}}
\def\inria{\textsc{INRIA}}
\def\mpi{\textsc{MPII}}
\newcommand{\bez}{B{\'e}zier}
\newcommand{\unitsphere}{\ensuremath{{\mathbb S}^2}}
\newcommand{\Rtwo}{\ensuremath{\reals \rule{0.3mm}{0mm}^2}\xspace}    % R^2
\newcommand{\Rthree}{\ensuremath{\reals \rule{0.3mm}{0mm}^3}\xspace}  % R^3
\newcommand{\Rd}{\ensuremath{\reals \rule{0.3mm}{0mm}^d}\xspace}      % R^d
\newcommand{\Rx}[1]{\ensuremath{\reals \rule{0.3mm}{0mm} ^ {#1}}\xspace} % R^x
\newcommand{\SOd}[1]{\ensuremath{{\mathbb S} \rule{0.3mm}{0mm} ^ {#1}}}
\newcommand{\SOtwo}{\SOd{2}\xspace}
\newcommand{\R}{\ensuremath{\mathbb{R}}}                             %
\newcommand{\Z}{\ensuremath{\mathbb{Z}}}                             %
\newcommand{\Expect}{{\rm I\kern-.3em E}}

\def\eps{{\epsilon}}
\newcommand{\mindia}[1]{\ensuremath{{\mathcal M}(#1)}}
\newcommand{\distance}[2]{\ensuremath{\rho(#1, #2)}}
\newcommand{\distancesqr}[2]{\ensuremath{\rho^2(#1, #2)}}
\newcommand{\bisector}[2]{\ensuremath{B(#1, #2)}}
\newcommand{\cpp}{{C}{\tt ++}}
\newcommand{\windows}{\textsc{Windows}}
\newcommand{\unix}{\textsc{Unix}}
\newcommand{\linux}{\textsc{Linux}}
\newcommand{\cygwin}{\textsc{Cygwin}}
\newcommand{\stl}{\textsc{Stl}}
\newcommand{\cgal}{\textsc{Cgal}}
\newcommand{\cmake}{\textsc{CMake}}
\newcommand{\gmp}{\textsc{Gmp}}
\newcommand{\gnu}{\textsc{Gnu}}
\newcommand{\mpfr}{\textsc{Mpfr}}
\newcommand{\core}{\textsc{Core}}
\newcommand{\boost}{\textsc{Boost}}
\newcommand{\qt}{\textsc{Qt}}
\newcommand{\leda}{\textsc{Leda}}
\newcommand{\vlsi}{\textsc{Vlsi}}
\newcommand{\dcel}{\textsc{Dcel}}
\newcommand{\ccb}{\textsc{Ccb}}
\newcommand{\parms}{{\rm I\!\hspace{-0.025em} P}}
\newcommand{\reals}{{\rm I\!\hspace{-0.025em} R}}
\newcommand{\redpart}{{\sc \textcolor{red}{$R$}}}
\newcommand{\greenpart}{{\sc \textcolor{green}{$G$}}}
\newcommand{\bluepart}{{\sc \textcolor{blue}{$B$}}}
\newcommand{\purplepart}{{\sc \textcolor{magenta}{$P$}}}
\newcommand{\yellowpart}{{\sc \textcolor{orange}{$Y$}}}
\newcommand{\turquoisepart}{{\sc \textcolor{cyan}{$T$}}}
\newcommand{\kdtree}{\textsc{Kd}-tree}
\newcommand{\kdtrees}{\textsc{Kd}-trees}
\newcommand{\bgl}{\textsc{Bgl}}

\newcommand{\sgmo}{{\bf SGM}}
\newcommand{\cgmo}{{\bf CGM}}
\newcommand{\ngmo}{{\bf NGM}}
\newcommand{\ch}{{\bf CH}}
\newcommand{\Fuku}{{\bf Fuk}}

\def\concept#1{\textsf{\it #1}}

\def\calA{{\cal A}}
\def\calC{{\cal C}}
\def\calG{{\cal G}}
\def\calI{{\cal I}}
\def\calJ{{\cal J}}
\def\calK{{\cal K}}
\def\calL{{\cal L}}
\def\calM{{\cal M}}
\def\calP{{\cal P}}
\def\calS{{\cal S}}
\def\calT{{\cal T}}
\def\calU{{\cal U}}

% ======== Macros for Voronoi =================================================
\newcommand{\distsym}{\ensuremath{\rho}\xspace}
\newcommand{\mobius}{M\"o\-bi\-us\xspace}
\newcommand{\region}{\ensuremath{{Reg}}}
\newcommand{\vor}[1]{\ensuremath{Vor_\distsym(#1)}}
\newcommand{\FPFS}{Far\-thest-Point Far\-thest-Site}
%
% ======== Macros for English =================================================
\newcommand{\ie}{i.\,e.,\xspace}
\newcommand{\eg}{e.\,g.,\xspace}
\newcommand{\etal}{{et~al}.\xspace}
\newcommand{\Wlog}{w.\,l.\,o.\,g.\xspace}
\newcommand{\apriori}{{\it a~priori}}
%
% ======== Listings ===========================================================
\newlength{\ccBaseWidth}\setlength{\ccBaseWidth}{\textwidth/72}
\lstset{%
  language=C++,%
%  basewidth=\ccBaseWidth,
  keywordstyle=\color{blue},commentstyle=\color{red}%
}
\def\myLstinline#1{\lstinline[columns=fixed]{#1}}
%
% ======== html ===============================================================
\newrgbcolor{hrefcolor}{0.2 0.2 0.6}
\def\hrefc#1#2{\href{#1}{\textcolor{hrefcolor}{#2}}}
%
% ======== PS Tricks ==========================================================
\SpecialCoor
%
\def\myxunit{1}
\def\myyunit{1}
\def\arrset#1{\psset{unit=#1cm}\def\myxunit{#1}\def\myyunit{#1}}
\def\vertexradius{2pt}
\def\crossradius{0.15}
%
\def\arrCross(#1,#2){
  \psline[linewidth=0.75pt,linecolor=crossColor]
  (!#1 \crossradius\space \myxunit\space div sub #2)
  (!#1 \crossradius\space \myxunit\space div add #2)
  \psline[linewidth=0.75pt,linecolor=crossColor]
  (!#1 #2 \crossradius\space \myyunit\space div sub)
  (!#1 #2 \crossradius\space \myyunit\space div add)
}
\def\arrCrossVertex(#1)#2{
  \arrCross(#1)
  \pnode(#1){#2}
}
\def\arrEmptyVertex(#1)#2{
  \pscircle*[linecolor=white](#1){\vertexradius}
  \cnode[linewidth=0.5pt](#1){\vertexradius}{#2}
}
\def\arrColorVertex(#1)#2#3{
  \pscircle*[linecolor=#3](#1){\vertexradius}
  \cnode[linewidth=0.5pt](#1){\vertexradius}{#2}
}
\def\arrRedVertex(#1)#2{\arrColorVertex(#1){#2}{red}}
\def\arrCyanVertex(#1)#2{\arrColorVertex(#1){#2}{cyan}}
\def\arrGreenVertex(#1)#2{\arrColorVertex(#1){#2}{green}}
\def\arrOliveGreenVertex(#1)#2{\arrColorVertex(#1){#2}{olivegreen}}
\def\arrBlueVertex(#1)#2{\arrColorVertex(#1){#2}{blue}}
\def\arrGrayVertex(#1)#2{\arrColorVertex(#1){#2}{gray}}
\def\arrPurpleVertex(#1)#2{\arrColorVertex(#1){#2}{purple}}

\def\arrMainVertex(#1)#2{\arrRedVertex(#1){#2}}
\def\arrMinorVertex(#1)#2{\arrCyanVertex(#1){#2}}

\def\arrlvertex[#1](#2)#3#4{
  \arrMainVertex(#2){#3}
  \uput[#1]{0}(#2){#4}
}
\def\arrLabeledColorVertex[#1](#2)#3#4#5{
  \arrColorVertex(#2){#3}{#4}
  \uput[#1]{0}(#2){#5}
}
\def\arrp[#1](#2)#3{\arrlvertex[#1](#2){#3}{$p_{#3}$}}
\def\arrv[#1](#2)#3{\arrlvertex[#1](#2){#3}{$v_{#3}$}}
\def\arru[#1](#2)#3{\arrlvertex[#1](#2){#3}{$u_{#3}$}}
\def\nsarraxes(#1)(#2)(#3){\psaxes*[linecolor=black,linewidth=0.25pt,labels=none]{->}(#1)(#2)(#3)}
\def\arraxes(#1)(#2)(#3){\hspace{-3pt}\nsarraxes(#1)(#2)(#3)}
\def\arrIntersect{
  /arrDict 10 dict def
  arrDict begin
  /arrIntersect {
    /arrIntersectDict 14 dict def
    arrIntersectDict begin
    /x1 exch def
    /y1 exch def
    /x2 exch def
    /y2 exch def
    /x3 exch def
    /y3 exch def
    /x4 exch def
    /y4 exch def
    /dx43 x4 x3 sub def
    /dy43 y4 y3 sub def
    /dx13 x1 x3 sub def
    /dy13 y1 y3 sub def
    /dx21 x2 x1 sub def
    /dy21 y2 y1 sub def
    dx43 dy13 mul dy43 dx13 mul sub
    dy43 dx21 mul dx43 dy21 mul sub
    div dup dx21 mul x1 add exch dy21 mul y1 add
    end
  } bind def
  end
}
\def\arrColorIntersectionVertex(#1,#2)(#3,#4)(#5,#6)(#7,#8){
  \def\ArgI{{#1}}%
  \def\ArgII{{#2}}%
  \def\ArgIII{{#3}}%
  \def\ArgIV{{#4}}%
  \def\ArgV{{#5}}%
  \def\ArgVI{{#6}}%
  \def\ArgVII{{#7}}%
  \def\ArgVIII{{#8}}%
  \arrColorIntersectionVertexRelay
}
\def\arrColorIntersectionVertexRelay#1#2{
  \arrColorVertex(!\arrIntersect\space
    arrDict begin \ArgVIII \ArgVII \ArgVI \ArgV \ArgIV \ArgIII \ArgII \ArgI arrIntersect end
  ){#1}{#2}
}
\def\arrEmptyIntersectionVertex(#1,#2)(#3,#4)(#5,#6)(#7,#8)#9{
  \arrEmptyVertex(!\arrIntersect\space
    arrDict begin #8 #7 #6 #5 #4 #3 #2 #1 arrIntersect end
  ){#9}
}
\def\arrMainIntersectionVertex(#1,#2)(#3,#4)(#5,#6)(#7,#8)#9{
  \arrMainVertex(!\arrIntersect\space
    arrDict begin #8 #7 #6 #5 #4 #3 #2 #1 arrIntersect end
  ){#9}
}
\def\arrCrossIntersectionVertex(#1,#2)(#3,#4)(#5,#6)(#7,#8)#9{
  \psline[linewidth=0.75pt,linecolor=crossColor](!\arrIntersect\space
    arrDict begin #8 #7 #6 #5 #4 #3 #2 #1 arrIntersect end
    \crossradius\space \myyunit\space div sub
  )(!\arrIntersect\space
    arrDict begin #8 #7 #6 #5 #4 #3 #2 #1 arrIntersect end
    \crossradius\space \myyunit\space div add
  )
  \psline[linewidth=0.75pt,linecolor=crossColor](!\arrIntersect\space
    arrDict begin #8 #7 #6 #5 #4 #3 #2 #1 arrIntersect end
    exch \crossradius\space \myxunit\space div sub exch
  )(!\arrIntersect\space
    arrDict begin #8 #7 #6 #5 #4 #3 #2 #1 arrIntersect end
    exch \crossradius\space \myxunit\space div add exch
  )
  \pnode(!\arrIntersect\space
    arrDict begin #8 #7 #6 #5 #4 #3 #2 #1 arrIntersect end
  ){#2}
}
\def\arrTriangle(#1,#2)(#3,#4)(#5,#6)(#7,#8){
  \def\ArgI{{#1}}%
  \def\ArgII{{#2}}%
  \def\ArgIII{{#3}}%
  \def\ArgIV{{#4}}%
  \def\ArgV{{#5}}%
  \def\ArgVI{{#6}}%
  \def\ArgVII{{#7}}%
  \def\ArgVIII{{#8}}%
  \arrTriangleRelayA
}
\def\arrTriangleRelayA(#1,#2)(#3,#4)#5{
  \def\ArgIX{{#1}}%
  \def\ArgX{{#2}}%
  \def\ArgXI{{#3}}%
  \def\ArgXII{{#4}}%
  \pspolygon*[linecolor=#5]
    (!\arrIntersect\space
      arrDict begin
      \ArgVIII \ArgVII \ArgVI \ArgV \ArgIV \ArgIII \ArgII \ArgI arrIntersect
      end
    )
    (!\arrIntersect\space
      arrDict begin
      \ArgXII \ArgXI \ArgX \ArgIX \ArgVIII \ArgVII \ArgVI \ArgV arrIntersect
      end
    )
    (!\arrIntersect\space
      arrDict begin
      \ArgIV \ArgIII \ArgII \ArgI \ArgXII \ArgXI \ArgX \ArgIX arrIntersect
      end
    )
}
\def\arrPolygon(#1,#2)(#3,#4)(#5,#6)(#7,#8){
  \def\ArgI{{#1}}%
  \def\ArgII{{#2}}%
  \def\ArgIII{{#3}}%
  \def\ArgIV{{#4}}%
  \def\ArgV{{#5}}%
  \def\ArgVI{{#6}}%
  \def\ArgVII{{#7}}%
  \def\ArgVIII{{#8}}%
  \arrpolygonRelayA
}
\def\arrpolygonRelayA(#1,#2)(#3,#4)(#5,#6)(#7,#8){
  \def\ArgIX{{#1}}%
  \def\ArgX{{#2}}%
  \def\ArgXI{{#3}}%
  \def\ArgXII{{#4}}%
  \def\ArgXIII{{#5}}%
  \def\ArgXIV{{#6}}%
  \def\ArgXV{{#7}}%
  \def\ArgXVI{{#8}}%
  \arrpolygonRelayB
}
\def\arrpolygonRelayB#1{
  \pspolygon*[linecolor=#1]
    (!\arrIntersect\space
      arrDict begin
      \ArgVIII \ArgVII \ArgVI \ArgV \ArgIV \ArgIII \ArgII \ArgI arrIntersect
      end
    )
    (!\arrIntersect\space
      arrDict begin
      \ArgXII \ArgXI \ArgX \ArgIX \ArgVIII \ArgVII \ArgVI \ArgV arrIntersect
      end
    )
    (!\arrIntersect\space
      arrDict begin
      \ArgXVI \ArgXV \ArgXIV \ArgXIII \ArgXII \ArgXI \ArgX \ArgIX arrIntersect
      end
    )
    (!\arrIntersect\space
      arrDict begin
      \ArgIV \ArgIII \ArgII \ArgI \ArgXVI \ArgXV \ArgXIV \ArgXIII arrIntersect
      end
    )
}
\def\arrPentagon(#1,#2)(#3,#4)(#5,#6)(#7,#8){
  \def\ArgI{{#1}}%
  \def\ArgII{{#2}}%
  \def\ArgIII{{#3}}%
  \def\ArgIV{{#4}}%
  \def\ArgV{{#5}}%
  \def\ArgVI{{#6}}%
  \def\ArgVII{{#7}}%
  \def\ArgVIII{{#8}}%
  \arrPentagonRelayA
}
\def\arrPentagonRelayA(#1,#2)(#3,#4)(#5,#6)(#7,#8){
  \def\ArgIX{{#1}}%
  \def\ArgX{{#2}}%
  \def\ArgXI{{#3}}%
  \def\ArgXII{{#4}}%
  \def\ArgXIII{{#5}}%
  \def\ArgXIV{{#6}}%
  \def\ArgXV{{#7}}%
  \def\ArgXVI{{#8}}%
  \arrPentagonRelayB
}
\def\arrPentagonRelayB(#1,#2)(#3,#4)#5{
  \def\ArgXVII{{#1}}%
  \def\ArgXVIII{{#2}}%
  \def\ArgXIX{{#3}}%
  \def\ArgXX{{#4}}%
  \pspolygon*[linecolor=#5]
    (!\arrIntersect\space
      arrDict begin
      \ArgVIII \ArgVII \ArgVI \ArgV \ArgIV \ArgIII \ArgII \ArgI arrIntersect
      end
    )
    (!\arrIntersect\space
      arrDict begin
      \ArgXII \ArgXI \ArgX \ArgIX \ArgVIII \ArgVII \ArgVI \ArgV arrIntersect
      end
    )
    (!\arrIntersect\space
      arrDict begin
      \ArgXVI \ArgXV \ArgXIV \ArgXIII \ArgXII \ArgXI \ArgX \ArgIX arrIntersect
      end
    )
    (!\arrIntersect\space
      arrDict begin
      \ArgXX \ArgXIX \ArgXVIII \ArgXVII \ArgXVI \ArgXV \ArgXIV \ArgXIII arrIntersect
      end
    )
    (!\arrIntersect\space
      arrDict begin
      \ArgIV \ArgIII \ArgII \ArgI \ArgXX \ArgXIX \ArgXVIII \ArgXVII arrIntersect
      end
    )
}
\def\arredge#1#2{
    \ncline[linewidth=0.5pt,linecolor=blue]{#1}{#2}
    \ncline[linewidth=0.25pt,offset=3pt]{->}{#1}{#2}
    \ncline[linewidth=0.25pt,offset=3pt]{->}{#2}{#1}
}
%
\def\smallfilledsquare{\pspolygon*[linecolor=white](0,0)(0.1,0)(0.1,0.1)(0,0.1)}
\def\smallsquare{\pspolygon(0,0)(0.1,0)(0.1,0.1)(0,0.1)}
%
% ======== macros for package names ===========================================
\newcommand{\cgalPackage}[1]{{\emph{#1}}}
\newcommand{\iiDArrangementsPackage}{\cgalPackage{2D Arrangements}}
\newcommand{\iiDIntersectionofCurvesPackage}{\cgalPackage{2D Intersection of Curves}}
\newcommand{\iiDSnapRoundingPackage}{\cgalPackage{2D Snap Rounding}}
\newcommand{\EnvelopesofCurvesiniiDPackage}{\cgalPackage{Envelopes of Curves in 2D}}
\newcommand{\EnvelopesofSurfacesiniiiDPackage}{\cgalPackage{Envelopes of Surfaces in 3D}}
\newcommand{\iiDMinkowskiSumsPackage}{\cgalPackage{2D Minkowski Sums}}
\newcommand{\iiDRegularizedBooleanSetOperationsPackage}{\cgalPackage{2D Regularized Boolean Set-Operations}}
\newcommand{\iiDPolygonsPackage}{\cgalPackage{2D Polygons}}
\newcommand{\iiDBooleanOperationsonNefPolygonsPackage}{\cgalPackage{2D Boolean Operations on Nef Polygons}}
\newcommand{\iiDBooleanOperationsonNefPolygonsEmbeddedontheSpherePackage}{\cgalPackage{2D Boolean Operations on Nef Polygons Embedded on the Sphere}}
\newcommand{\iiiDBooleanOperationsonNefPolyhedraPackage}{\cgalPackage{3D Boolean Operations on Nef Polyhedra}}
%
% =============================================================================
%
\newenvironment{FramedVerb}%
	       {\VerbatimEnvironment
		 \begin{Sbox}\begin{minipage}{\textwidth}\begin{Verbatim}}%
	       {\end{Verbatim}\end{minipage}\end{Sbox}
		 \setlength{\fboxsep}{8pt}\fbox{\TheSbox}}

\newenvironment{fminipage}%
	       {\begin{Sbox}\begin{minipage}}%
	       {\end{minipage}\end{Sbox}\fbox{\TheSbox}}

\newrgbcolor{c1}{0.1 0.1 0.5}
\newrgbcolor{c2}{0.2 0.0 0.5}
\newrgbcolor{c3}{0.0 0.2 0.5}
\newrgbcolor{xpcolor}{1 0.5 0} % intersection point color
\newrgbcolor{epcolor}{0 0 0.8} % endpoint color
\newrgbcolor{fpcolor}{0 0.8 0} % ficticious point color
\newrgbcolor{fccolor}{0 0.5 0} % ficticious curve color

\newenvironment{packed-enum}{
\begin{enumerate}
  %\setlength{\topsep}{0pt}
  %\setlength{\partopsep}{0pt}
  \setlength{\itemsep}{0pt}
  %\setlength{\parsep}{0pt}
  \setlength{\rightmargin}{0pt}
  %\setlength{\listparindent}{0pt}
  %\setlength{\itemindent}{0pt}
  %\setlength{\labelsep}{0pt}
  %\setlength{\labelwidth}{0pt}
  %%\makelabel{label}
  \setlength{\leftmargin}{\labelwidth+\labelsep}
}{\end{enumerate}}
%
% ======= acolumns ============================================================
\usepackage{environ}% Required for \NewEnviron, i.e. to read the whole body of the environment
\makeatletter
\newcounter{acolumn}%  Number of current column
\newlength{\acolumnmaxheight}%   Maximum column height

% `column` replacement to measure height
\newenvironment{@acolumn}[1]{%
  \stepcounter{acolumn}%
  \begin{lrbox}{\@tempboxa}%
    \begin{minipage}{#1}%
}{%
    \end{minipage}
  \end{lrbox}
  \@tempdimc=\dimexpr\ht\@tempboxa+\dp\@tempboxa\relax
  % Save height of this column:
  \expandafter\xdef\csname acolumn@height@\roman{acolumn}\endcsname{\the\@tempdimc}%
  % Save maximum height
  \ifdim\@tempdimc>\acolumnmaxheight
  \global\acolumnmaxheight=\@tempdimc
  \fi
}

% `column` wrapper which sets the height beforehand
\newenvironment{@@acolumn}[1]{%
  \stepcounter{acolumn}%
  % The \autoheight macro contains a \vspace macro with the maximum height minus the natural column height
  \edef\autoheight{\noexpand\vspace*{\dimexpr\acolumnmaxheight-\csname acolumn@height@\roman{acolumn}\endcsname\relax}}%
  % Call original `column`:
  \orig@column{#1}%
}{%
  \endorig@column
}

% Save orignal `column` environment away
\let\orig@column\column
\let\endorig@column\endcolumn

% `columns` variant with automatic height adjustment
\NewEnviron{acolumns}[1][]{%
  % Init vars:
  \setcounter{acolumn}{0}%
  \setlength{\acolumnmaxheight}{0pt}%
  \def\autoheight{\vspace*{0pt}}%
  % Set `column` environment to special measuring environment
  \let\column\@acolumn
  \let\endcolumn\end@acolumn
  \BODY% measure heights
  % Reset counter for second processing round
  \setcounter{acolumn}{0}%
  % Set `column` environment to wrapper
  \let\column\@@acolumn
  \let\endcolumn\end@@acolumn
  % Finally process columns now for real
  \begin{columns}[#1]%
    \BODY
  \end{columns}%
}
\makeatother
%
% ======== Tikz museful macros==================================================
% Usage \arcThroughThoPoints[options]{center}{p}{q};
\newcommand{\arcThroughThoPoints}[4][]{
\coordinate (center) at (#2);
\draw[#1]
 let \p1=($(#3)-(center)$),
     \p2=($(#4)-(center)$),
     \n0={veclen(\p1)},       % Radius
     \n1={atan2(\x1,\y1)}, % angles
     \n2={atan2(\x2,\y2)},
     \n3={\n2>\n1?\n2:\n2+360}
   in (#3) arc(\n1:\n3:\n0);
}

\pagestyle{empty}
\begin{document}
\psscalebox{1.8}{\arrset{0.8}
  \pspicture[](-0.2,-0.2)(12.2,12.2)
  %\psgrid(0,0)(12,12)
  \arrMainVertex( 1, 3){a}
  \arrMainVertex( 2, 4){b}
  \arrMainVertex( 0, 6){c}
  \arrv[90](5,11){1}
  \arrMainVertex(9,11){f}
  \arrMainVertex(12, 8){g}
  \arrv[-60](8,8){2}
  \arrMainVertex( 8, 4){i}
  \arrMainVertex( 6, 4){j}
  \arrMainVertex( 6, 2){k}
  \arrMainVertex( 4, 0){m}
  \arrMainVertex( 9, 0){n}
  \arrMainVertex(11, 2){o}
  \arrMainVertex(12, 4){p}
  \arrMainVertex(12, 0){q}
  \arrMainVertex( 4, 2){r}
  \arrMainVertex( 3, 3){s}
  \arrMainVertex( 5, 3){t}
  %
  \ncline[linewidth=0.5pt,linecolor=blue]{a}{b}
  \ncline[linewidth=0.5pt,linecolor=blue]{b}{c}
  \ncline[linewidth=0.5pt,linecolor=blue]{c}{1}
  \ncline[linewidth=0.5pt,linecolor=blue]{1}{f}
  \ncline[linewidth=0.5pt,offset=3pt,nodesepB=6pt]{->}{f}{1}\Aput[1pt]{$e_{\mathrm{prev}}$}
  \ncline[linewidth=0.5pt,linecolor=blue]{f}{g}
  \ncline[linewidth=0.5pt,linecolor=blue]{g}{2}
  \ncline[linewidth=0.5pt,linecolor=blue]{g}{2}
  \ncline[linewidth=0.5pt,offset=3pt,nodesepB=4pt]{->}{2}{g}\Aput[1pt]{$e_{\mathrm{next}}$}
  \ncline[linewidth=0.5pt,linecolor=blue]{2}{i}
  \ncline[linewidth=0.5pt,linecolor=blue]{i}{j}
  \ncline[linewidth=0.5pt,linecolor=blue]{j}{k}
  \ncline[linewidth=0.5pt,linecolor=blue]{k}{m}
  \ncline[linewidth=0.5pt,linecolor=blue]{m}{a}
  \ncline[linewidth=0.5pt,linecolor=blue]{1}{2}
  \ncline[linewidth=0.5pt,linecolor=blue]{g}{i}
  \ncline[linewidth=0.5pt,linecolor=blue]{m}{n}
  \ncline[linewidth=0.5pt,linecolor=blue]{n}{i}
  \ncline[linewidth=0.5pt,linecolor=blue]{g}{n}
  \ncline[linewidth=0.5pt,linecolor=blue]{n}{o}
  \ncline[linewidth=0.5pt,linecolor=blue]{o}{p}
  \ncline[linewidth=0.5pt,linecolor=blue]{p}{q}
  \ncline[linewidth=0.5pt,linecolor=blue]{q}{o}
  \ncline[linewidth=0.5pt,linecolor=blue]{r}{s}
  \ncline[linewidth=0.5pt,offset=3pt]{->}{r}{s}
  \ncline[linewidth=0.5pt,offset=3pt]{->}{s}{r}
  \ncline[linewidth=0.5pt,linecolor=blue]{r}{t}
  \ncline[linewidth=0.5pt,offset=3pt]{->}{r}{t}
  \ncline[linewidth=0.5pt,offset=3pt]{->}{t}{r}
  %
  \ncline[linewidth=0.5pt,offset=3pt]{->}{b}{a}
  \ncline[linewidth=0.5pt,offset=3pt]{->}{c}{b}
  \ncline[linewidth=0.5pt,offset=3pt]{->}{1}{c}
  \ncline[linewidth=0.5pt,offset=3pt,nodesepA=4pt]{->}{g}{f}
  \ncline[linewidth=0.5pt,offset=3pt]{->}{i}{2}
  \ncline[linewidth=0.5pt,offset=3pt]{->}{j}{i}
  \ncline[linewidth=0.5pt,offset=3pt]{->}{k}{j}
  \ncline[linewidth=0.5pt,offset=3pt]{->}{m}{k}
  \ncline[linewidth=0.5pt,offset=3pt]{->}{a}{m}
  \ncline[linewidth=0.5pt,offset=3pt,nodesepA=6pt]{->}{1}{2}\Aput[1pt]{$e$}
  \ncline[linewidth=0.5pt,offset=3pt]{->}{2}{1}\Aput[1pt]{$e'$}
  %
  \rput(2,10){$f_0$}
  \rput(3,4){$f_1$}
  \rput(8.5,9.5){$f_2$}
  \rput(5,7){$f_3$}
  \rput(4.5,5.6){$f_4$}
  %
  \arru[90](2,6){1}
  \arru[90](5,9){2}
  %
  \arrMainVertex(4,5){a}
  \arrMainVertex(4,7){b}
  \arrMainVertex(5,8){c}
  \arrMainVertex(5,4){f}
  \arrMainVertex(5,6){e}
  \arrMainVertex(6,7){d}
  \ncline[linewidth=0.5pt,linecolor=blue]{a}{b}
  \ncline[linewidth=0.5pt,linecolor=blue]{b}{c}
  \ncline[linewidth=0.5pt,linecolor=blue]{c}{d}
  \ncline[linewidth=0.5pt,linecolor=blue]{d}{e}
  \ncline[linewidth=0.5pt,linecolor=blue]{e}{f}
  \ncline[linewidth=0.5pt,linecolor=blue]{f}{a}
  \ncline[linewidth=0.5pt,linecolor=blue]{b}{e}
  \ncline[linewidth=0.5pt,offset=3pt]{->}{a}{b}
  \ncline[linewidth=0.5pt,offset=3pt]{->}{b}{c}
  \ncline[linewidth=0.5pt,offset=3pt]{->}{c}{d}
  \ncline[linewidth=0.5pt,offset=3pt]{->}{d}{e}
  \ncline[linewidth=0.5pt,offset=3pt]{->}{e}{f}
  \ncline[linewidth=0.5pt,offset=3pt]{->}{f}{a}
  \endpspicture
}
\end{document}

\end{center}
\end{ccTexOnly}

\caption{An arrangement of interior-disjoint line segment, which
consists of a single connected component inside the unbounded face
$\tilde{f}$, with some of the \dcel\ records that represent it.
The half-edge $e$ (and its twin $e'$) correspond to a line segment
that connects the vertices $v_1$ and $v_2$ and separates the face
$f_1$ from $f_2$. The predecessor $e_{\rm prev}$ and successor
$e_{\rm next}$ of $e$ are part of the chain that form the outer
boundary of the face $f_2$. The face $f_1$ has a more complicated
structure as it contains two holes in its interior: One hole
consists of two adjacent faces $f_3$ and $f_4$, while the other
hole is comprised of two edges. $f_1$ also contains two isolated
vertices $u_1$ and $u_2$ in its interior.}
\label{arr_sec:seg_dcel}

%\begin{ccHtmlOnly}
%<P>
%<center>
%  <img src="arr_segs.gif"  border=0 alt="An arrangement of line segments">
%</center>
%\end{ccHtmlOnly}
\end{figure}

\section{The Main Arrangement Class}
\label{arr_sec:arr_class}
%===================================

The main guideline behind the design of the arrangement package is
the separation of the representation of the planar arrangement
from the various algorithms that operate on it. Another important
guideline is the separation between geometry an topology in the
arrangement representation.

The class \ccc{Arrangement_2<Traits,Dcel>} is a the main class in
the arrangement package. It is used to represent planar
arrangements and provides the interface needed to traverse them
and to maintain them. The arrangement is represented using the
\dcel\ class, which captures the {\em topological structure} of
the planar subdivision, and a geometric {\em traits} class that
defines the family of planar curves that form the arrangement and
supplies a range of geometric predicates (and constructions)
needed to construct and maintain the arrangement.

The separation between topology and geometry is defined by the two
template parameters of the \ccc{Arrangement_2} template:
\begin{itemize}
\item The \ccc{Traits} template-parameter should be instantiated with a
model of the \ccc{ArrangementBasicTraits} concept. The traits
class defines the types of $x$-monotone curves and two-dimensional
points --- \ccc{X_monotone_curve_2} and \ccc{Point_2},
respectively --- and supports basic geometric predicates on them.

In the first sections of this chapter we will always use
\ccc{Arr_segment_traits_2}, which enables the construction of
arrangements of line segments, as our traits class. However, the
arrangement package is shipped with several other traits classes
that can handle also poly-lines (continuous piecewise-linear
curves), conic arcs and arcs of rational functions. We will
exemplify the usage of these traits classes in
Section~\ref{arr_sec:traits}.

\item The \ccc{Dcel} template-parameter should be instantiated with a
class that is a model of the \ccc{ArranagementDcel} concept. The
value of this parameter is by default
\ccc{Arr_default_dcel<Traits>}, and it is recommended for most
users to stick to this default implementation of a \dcel\ class.
However, in some applications it is necessary to extend the \dcel\
features --- see Section~\ref{arr_sec:ex_dcel} for further
explanations and examples.
\end{itemize}

\subsection{Traversing the Arrangement}
\label{arr_ssec:traverse}
%--------------------------------------

The simplest and most fundamental arrangement operations are the
various traversal methods, that allow users to systematically go
over the relevant features of the arrangement at hand.

As mentioned before, the arrangement is represented as a \dcel,
and therefore stores three containers of vertices, halfedges and
faces. The \ccc{Arrangement_2} class therefore supplies iterators
for these containers. For example, the methods
\ccc{vertices_begin()} and \ccc{vertices_end()} return
\ccc{Arrangement_2::Vertex_iterator} objects that define the valid
range of arrangement vertices. The value type of this iterator is
\ccc{Arrangement_2::Vertex}. Moreover, the vertex-iterator type is
equivalent to \ccc{Arrangement_2::Vertex_handle}, which serves as
a pointer to a vertex. As we see next, all function related to
arrangement features accept handle types as input parameters and
return handle types as their output. Each of the three nested
classes support a \ccc{handle()} method that returns a handle to
the object at hand.

In addition to the iterators for arrangement vertices, halfedges
and faces, the arrangement class also support \ccc{edges_begin()}
and \ccc{edges_end()} that return
\ccc{Arrangement_2::Edge_iterator} objects for traversing the
arrangement edges. Note that the value type of this iterator is
\ccc{Arrangement_2::Halfedge}, representing one of the twin
halfedge that form the edge. In fact, as twin halfedges always are
always contiguous in the halfedge container, using this iterator
is equivalent to going over all halfedges and taking every second
halfedge.

All iterator, circulator and handle types also have non-mutable
(a.k.a. {\em const}) counterparts. For example, the arrangement
has a non-constant member function called \ccc{vertices_begin()}
that returns a \ccc{Vertex_iterator} object and another const
member function that returns a a \ccc{Vertex_const_iterator}
object. In fact, all methods listed in this section that return an
iterator, a circulator or a handle have non-mutable counterparts.
It should be noted that, for example, \ccc{Vertex_handle} can be
readily converted to a \ccc{Vertex_const_handle}, but not
vice-versa.

\subsubsection{Traversal Methods for an Arrangement Vertex}
\label{arr_sssec:tr_vertex}
%~~~~~~~~~~~~~~~~~~~~~~~~~~~~~~~~~~~~~~~~~~~~~~~~~~~~~~~~~~

A vertex is always associated with a geometric entity, namely with
a \ccc{Point_2} object, which can be retrieved by calling the
\ccc{point()} method of the nested \ccc{Vertex} class.

The \ccc{is_isolated()} method is used to determine whether a
vertex is isolated or not. Recall that the incident halfedges
incident to a regular (non-isolated) vertex form a circular list
around this vertex. The \ccc{incident_halfedges()} method returns
a circulator of type
\ccc{Arrangement_2::Halfedge_around_vertex_circulator} that
enables the traversal of this circular list in a clockwise
direction. The value type of this circulator is \ccc{Halfedge}.

The following function prints all the neighbors of a given
arrangement vertex (assuming the \ccc{Point_2} type can be
exported to the standard output using the \ccc{<<} operator). For
simplicity, we assume that the arrangement type and its nested
type are properly defined (see next for more complete examples):
\begin{alltt}
void print_neighboring_vertices (Vertex_const_handle v)
\{
  if (v->is_isolated())
  \{
    std::cout << "The vertex (" << v->point() << ") is isloated" << std::endl;
    return;
  \}

  Halfedge_around_vertex_const_circulator  first, curr;
  Vertex_const_handle                      u;

  std::cout << "The neighbors of the vertex (" << v->point() << ") are:";
  first = curr = v->incident_halfedges();
  do
  \{
    // Note that the current halfedge is (u -> v):
    u = curr->source();
    std::cout << " (" << u->point() << ")";

  \} while (curr != first);
  std::cout << std::endl;

  return;
\}
\end{alltt}

In case of an isolated vertex, it is possible to obtain the face
that contains this vertex. However, this is done via a method of
the \ccc{Arrangement_2} class, called \ccc{incident_face
(Vertex_handle v)}, and not using one of the methods of the nested
\ccc{Vertex} class.

\subsubsection{Traversal Methods for an Arrangement Halfedge}
\label{arr_sssec:tr_halfedge}
%~~~~~~~~~~~~~~~~~~~~~~~~~~~~~~~~~~~~~~~~~~~~~~~~~~~~~~~~~~~~

Each pair of twin halfedge is associated with an
\ccc{X_monotone_curve_2} object, which can be retrieved by calling
the \ccc{curve()} method of the nested \ccc{Halfedge} class.

The \ccc{source()} and \ccc{target()} methods return handles to
the halfedge source and target vertices, respectively. One can
obtain a handle to the twin halfedge using the \ccc{twin()}
method. From the definition of halfedges, it follows that if
\ccc{he} is a halfedge handle, then:
\begin{itemize}
\item \ccc{he->curve()} is equivalent to \ccc{he->twin()->curve()}.
\item \ccc{he->source()} is equivalent to \ccc{he->twin()->target()}.
\item \ccc{he->target()} is equivalent to \ccc{he->twin()->source()}.
\end{itemize}

Every halfedge has an incident face that lies to its left, which
can be obtained by using the \ccc{face()} method. Recall that a
halfedge is always one link in a connected chains of halfedges
that share the same incident face, known as a {\em CCB}. The
\ccc{prev()} and \ccc{next()} methods return handles to the
previous and next halfedges in the CCB (note that \ccc{he->prev()}
is not necessarily equivalent to \ccc{he->twin()->next()}).

As the CCB is actually a circular list, it is natural to traverse
it using a circulator. The \ccc{ccb()} method returns a
\ccc{Arrangement_2::Ccb_halfedge_circulator} object for the
halfedges along the CCB.

The following function prints all $x$-monotone curves and points
along a CCB (assuming that the \ccc{Point_2} and the
\ccc{X_monotone_curve_2} types can be exported to the standard
output using the \ccc{<<} operator):
\begin{alltt}
void print_ccb (Ccb_halfedge_const_circulator circ)
\{
  Ccb_halfedge_const_circulator  curr = circ;
  Halfedge_const_handle          he;

  std::cout << "(" << curr->source()->point() << ")"
  do
  \{
    he = curr->handle();
    std::cout << "   [" << he->curve() << "]   "
              << "(" << he->target()->point() << ")";

  \} while (curr != circ);
  std::cout << std::endl;

  return;
\}
\end{alltt}


\subsubsection{Traversal Methods for an Arrangement Face}
\label{arr_sssec:tr_face}
%~~~~~~~~~~~~~~~~~~~~~~~~~~~~~~~~~~~~~~~~~~~~~~~~~~~~~~~~

As the \ccc{Arrangement_2} class supports only finite curves, the
arrangement always has a single unbounded face, where all other
arrangement features form holes (or isolated vertices) inside this
face. (Note that an empty arrangement contains no other feature
{\em but} the unbounded face.) The function \ccc{unbounded_face()}
returns a handle to this face.

Given a \ccc{Face} object, we can use the \ccc{is_unbounded()}
method to determine whether it is unbounded. Bounded face have an
outer CCB, and the \ccc{outer_ccb()} method returns a circulator
for the halfedges along this CCB. Note that the halfedges along
this CCB wind in a counterclockwise orientation around the outer
boundary of the face.

A face can also contain disconnected components in its interior,
namely holes and isolated vertices:
\begin{itemize}
\item The \ccc{holes_begin()} and \ccc{holes_end()} methods return
\ccc{Arrangement_2::Holes_iterator} objects that define the range
of holes inside the face. The value type of this iterator is
\ccc{Ccb_halfedge_circulator}, defining the CCB that winds in a
clockwise orientation around a hole.

\item The \ccc{isolated_vertices_begin()} and
\ccc{isolated_vertices_end()} methods return
\ccc{Arrangement_2::Isolated_vertices_iterator} objects that
define the range of isolated vertices inside the face. The value
type of this iterator is \ccc{Vertex}.
\end{itemize}

The following function prints the outer and inner boundaries of a
given face, using the function \ccc{print_ccb()} that was
introduced in the previous subsection:
\begin{alltt}
void print_face (Face_const_handle f)
\{
  // Print the outer boundary.
  if (f.is_unbounded())
  \{
    std::cout << "Unbounded face. " << std::endl;
  \}
  else
  \{
    std::cout << "Outer boundary: ";
    print_ccb (f.outer_ccb());
  \}

  // Print the boundary of each of the holes.
  Holes_const_iterator              hole;
  int                               index = 1;

  for (hole = f->holes_begin(); hole != f->holes_end(); ++hole, ++index)
  \{
    std::cout << "    Hole #" << index << ": ";
    print_ccb (*hole);
  \}

  // Print the isolated vertices.
  Isolated_vertices_const_iterator  iv;

  for (iv = f->isolated_vertices_begin(), index = 1;
       iv != f->isolated_vertices_end(); ++iv, ++index)
  \{
    std::cout << "    Isolated vertex #" << index << ": "
              << "(" << iv->point() << ")" << std::endl;
  \}

  return;
\}
\end{alltt}

\subsubsection{Additional Example}
\label{arr_sssec:tr_ex}
%~~~~~~~~~~~~~~~~~~~~~~~~~~~~~~~~~~~~~~~~~~~~~~~~~~~~~~~~

To conclude the review of the various traversal methods, we bring
the following function that prints the current setting of the
arrangement:\footnote{To simplify the code, we assume that the
\ccc{Arrangement_2} type is defined as a proper instantiation of
the \ccc{Arrangement_2<Traits,Dcel>} template and all its nested
types are also defined. In the file \ccc{arr_print.h} that can be
found under the examples folder this function (and the rest of the
functions listed in this section) are brought in a more general
setting, where the arrangement type serves as a template parameter
for these functions, so different instantiations of the
\ccc{Arrangement_2<Traits,Dcel>} template can be provided to the
same function templates.}
\begin{alltt}
void print_arrangement (const Arrangement_2& arr)
\{
  // Print the arrangement vertices.
  Vertex_const_iterator      vit;

  std::cout << arr.number_of_vertices() << " vertices:" << std::endl;
  for (vit = arr.vertices_begin(); vit != arr.vertices_end(); ++vit)
  \{
    std::cout << "(" << vit->point() << ")";
    if (vit->is_isolated())
      std::cout << " - Isolated." << std::endl;
    else
      std::cout << " - degree " << vit->degree() << std::endl;
  \}

  // Print the arrangement edges.
  Edge_const_iterator        eit;

  std::cout << arr.number_of_edges() << " edges:" << std::endl;
  for (eit = arr.edges_begin(); eit != arr.edges_end(); ++eit)
    std::cout << "[" << eit->curve() << "]" << std::endl;

  // Print the arrangement faces.
  Face_const_iterator        fit;

  std::cout << arr.number_of_faces() << " faces:" << std::endl;
  for (fit = arr.faces_begin(); fit != arr.faces_end(); ++fit)
    print_face (fit);

  return;
\}
\end{alltt}

\section{Traits Classes}
\label{arr_sec:traits}
%=======================

\section{Extending the \dcel}
\label{arr_sec:ex_dcel}
%============================

\section*{Design and Implementation History}
%===========================================

The code of this package is the result of a long development process.
Initially (and until version~3.1), the code was spread among several
packages, namely \ccc{Topological_map}, \ccc{Planar_map_2},
\ccc{Planar_map_with_intersections_2} and \ccc{Arrangement_2}, that were
developed by~: \newline
Ester Ezra, Eyal Flato, Efi Fogel, Dan Halperin, Iddo Hanniel, Idit Haran,
Sariel Har-Peled, Shai Hirsch, Eugene Lipovetsky, Oren Nechushtan, Ron Wein,
Baruch Zukerman and Tali Zvi.

In version~3.2, as part of the ACS project, the packages have done through
a major re-design, resulting in an improved \ccc{Arrangement_2} package.
The code of the new package was restructured and developed by~: \newline
Efi Fogel, Idit Haran, Ron Wein and Baruch Zukerman.
