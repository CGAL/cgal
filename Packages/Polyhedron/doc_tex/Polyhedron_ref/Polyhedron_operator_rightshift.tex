% +------------------------------------------------------------------------+
% | Reference manual page: Polyhedron_operator_rightshift.tex
% +------------------------------------------------------------------------+
% | 09.09.2000   Lutz Kettner
% | Package: Polyhedron
% | 
\RCSdef{\RCSPolyhedronoperatorostreamRev}{$Revision$}
\RCSdefDate{\RCSPolyhedronoperatorostreamDate}{$Date$}
% |
%%RefPage: end of header, begin of main body
% +------------------------------------------------------------------------+


\ccHtmlNoClassLinks
\begin{ccRefFunction}{operator>>}
\label{refPolyhedron_operator_rightshift}

\ccDefinition

This operator reads a polyhedral surface in Object File Format, OFF,
with file extension {\tt .off}, which is also understood by
GeomView~\cite{p-gmgv15-94}, from the input stream \ccc{in} and
appends it to the polyhedral surface $P$.  Only the point coordinates
and facets from the input stream are used to build the polyhedral
surface. Neither normal vectors nor color attributes are evaluated. If
the stream \ccc{in} does not contain a permissible polyhedral surface
the \ccc{ios::badbit} of the input stream \ccc{in} is set and $P$ remains 
unchanged.

For OFF an ASCII and a binary format exist. The stream detects the
format automatically and can read both.

\ccInclude{CGAL/IO/Polyhedron_iostream.h}

\ccGlobalFunction{template <class PolyhedronTraits_3>
    istream& operator>>( istream& in,
                         CGAL::Polyhedron_3<PolyhedronTraits_3>& P);}
  
\ccSeeAlso

\ccRefIdfierPage{CGAL::Polyhedron_3<Traits>}\\ 
\ccRefIdfierPage{CGAL::Polyhedron_incremental_builder_3<HDS>}\\
\lcTex{\ccc{operator<<} \dotfill\ 
    page~\pageref{refPolyhedron_operator_leftshift}}%
\lcRawHtml{
    <I><A HREF="Function_operator.html">operator&lt;&lt;</A></I>.
}

\ccImplementation

This operator is implemented using the modifier mechanism for
polyhedral surfaces and the \ccc{CGAL::Polyhedron_incremental_builder_3}
class, which allows the construction in a single, efficient scan pass
of the input and handles also all the possible flexibility of the
polyhedral surface.


\end{ccRefFunction}

% +------------------------------------------------------------------------+
%%RefPage: end of main body, begin of footer
% EOF
% +------------------------------------------------------------------------+

