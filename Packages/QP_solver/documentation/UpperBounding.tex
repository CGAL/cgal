\documentclass[a4paper]{article}
%\usepackage{html}
\usepackage[dvips]{graphics,color,epsfig}
\usepackage{path}
\usepackage{amssymb}
\usepackage{amsfonts}
\usepackage{amsmath}
\usepackage{amsthm}
\usepackage{psfrag}
\usepackage{algorithm}
\usepackage{algpseudocode}
\newcommand{\N}{\ensuremath{\mathbb{N}}}
\newcommand{\F}{\ensuremath{\mathbb{F}}}
\newcommand{\Z}{\ensuremath{\mathbb{Z}}}
\newcommand{\R}{\ensuremath{\mathbb{R}}}
\newcommand{\Q}{\ensuremath{\mathbb{Q}}}
\newcommand{\C}{\ensuremath{\mathbb{C}}}

\newtheorem{lemma}{Lemma}
\newtheorem{assumption}{Assumption}
\newtheorem{definition}{Definition}


\title{Upper Bounding}
\author{Frans Wessendorp}
\begin{document}
\maketitle
\section{General Remarks}
\begin{eqnarray}
\label{def:QP_UB}
(QP)\quad minimize&  c^{T}x + x^{T} D x      	&  \nonumber \\
s.t.	 & \sum_{j=0}^{n-1}a_{ij}x_{j} = b_{i}	& i \in E  \nonumber \\
	 & \sum_{j=0}^{n-1}a_{ij}x_{j} \leq b_{i} & i \in I^{\leq} \\
	 & \sum_{j=0}^{n-1}a_{ij}x_{j} \geq b_{i} & i \in I^{\geq}  \nonumber \\
 	 & \mathit{l}_{j} \leq x_{j} \leq u_{j}	  & j \in \{0 \ldots n-1 \}
	 \nonumber
\end{eqnarray}
%\end{equation}
with $D$ positive semi definite, where $I:= I^{\leq} \cup I^{\geq}$ and
$\left| E \right| + \left| I \right| = m$,

The lower and upper bounds, in the finite case,  will be stored in the vectors
$\mathbf{l}$ and $\mathbf{u}$, $l \leq u$, both of size $\left|O\right|$, where $O$ denotes the set of original variables. The absence or presence of finite bounds will be indicated by means of two 0,1 valued vectors $f^{l}$ and $f^{u}$. Thus if 
$f_{j}^{l}=0$, for some $j \in O$, then the lower bound for the variable $x_{j}$ is $-\infty$, on the other hand, if $f_{j}^{l}=1$ for some $j \in O$ then the lower bound for variable $x_{j}$ is $l_{j}$. The upper bounds are described 
likewise. Thus a variable $x_{k}$ is \emph{free}, if $f_{k}^{u}=0$ and
$f_{k}^{l}=0$, and $x_{k}$ is \emph{fixed}, if $u_{k}=l_{k}$ as well as
$f_{k}^{l}=1$ and $f_{k}^{u}=1$ hold. The set of unrestricted original variables will be denoted by $U$ and the set of fixed original variables will be denoted by $F$.

Extending the vectors $f^{l}, f^{u}$ and $l, u$ for the set of slack variables $S$ and artificial variables $art$ comes naturally, such that we can introduce  
a function $b: O \cup S \cup art \rightarrow \{-1, 0, 1\}$ indicating which of the three possible values a nonbasic variable may take with respect to its bounds,
\begin{equation}
b(k):= \left\{
\begin{array}{ll}
-1
&
\mbox{if $x_{k}=l_{k} \wedge f_{k}^{l}=1$ for $k \in N \setminus F$} \\
0
&
\mbox{if $\left(f_{k}^{l}=1 \Rightarrow x_{k} > l_{k}\right) \wedge
          \left(f_{k}^{u}=1 \Rightarrow x_{k} < u_{k}\right)$
          for $k \in N \setminus F$} \\
1
&
\mbox{if $x_{k}=u_{k} \wedge f_{k}^{u}=1$ for $k \in N \setminus F$} \\
\end{array}
\right.
\end{equation}
where $N$ denotes the set of nonbasic variables. Note that by the very definition of slack and artificial variables $b(k)=-1$ for
$k \in (S \cup art) \cap N$.   
\section{Initialization}
Since the original variables for the initial feasible solution to the auxiliary problem are all nonbasic we have to assign initial values to them. Since for the free or unrestricted variables there is no finite bound to initialize the variables with we will initialize them with the value zero. Due to the amount of computation involved for the initialization of the auxiliary problem we extend the set of original variables that are initialized with zero, $Z$, in the following sense, 
\begin{equation}
\label{def:Z}
Z:=\{ k \in O \left|\right. 0 \in \left[l_{k}, u_{k}\right]\} 
\end{equation}
that is all the original variables that contain zero in their feasible intervall are initialized with zero. For the remaining variables $x_{k}$,
$k \in O \setminus Z$ we initialize the variable with the finite lower bound whenever there is choice,
\begin{equation}
x_{k}^{(0)}:=\left\{
\begin{array}{ll}
0
&
\mbox{for $k \in Z \cap O$} \\
l_{k}
&
\mbox{if $f_{k}^{l}=1, f_{k}^{u}=1$ for $k \in O \setminus Z$} \\
l_{k}
&
\mbox{if $f_{k}^{l}=1, f_{k}^{u}=0$ for $k \in O \setminus Z$} \\
u_{k}
&
\mbox{if $f_{k}^{l}=0, f_{k}^{u}=1$ for $k \in O \setminus Z$}
\end{array}
\right.
\end{equation}


\subsection{The auxiliary problem}
For each of the equality constraints $\sum_{j=0}^{n-1}a_{ij}x_{j}$, $i \in E$,
the original constraint matrix $A$ is augmented by an artificial column 
\begin{equation}
a= \left\{
\begin{array}{ll}
-e_{i}
&
\mbox{if $b_{i}-\sum_{k \in O \setminus Z}x_{k}^{(0)}$} \\
e_{i}
&
\mbox{otherwise}
\end{array}
\right.
\end{equation}
where $e_{i}$ denotes the $i$-th column of the $m \times m$ identity matrix. If the set of inequality constraints with infeasible origin, 
\begin{equation}
I_{inf}:=\left\{
  i \in I^{\leq}\left|\right. b_{i}-\sum_{k \in O \setminus Z}x_{k}^{(0)}
  \right\}
  \cup
  \left\{
    i \in I^{\geq} \left|\right. b_{i}-\sum_{k \in O \setminus Z}x_{k}^{(0)}
  \right\}
\end{equation}
is nonempty, the original constraint matrix $A$ is augmented by a special artificial column $a_{i}^{s}$
\begin{equation}
a_{i}^{s}= \left\{
\begin{array}{ll}
-1
&
\mbox{if $i \in I^{\leq}, b_{i}-\sum_{k \in O \setminus Z}x_{k}^{(0)}<0$} \\
1
&
\mbox{if $i \in I^{\geq}, b_{i}-\sum_{k \in O \setminus Z}x_{k}^{(0)}>0$} \\
0
&
\mbox{otherwise}
\end{array}
\right.
\end{equation}


\subsection{The initialization of the auxiliary problem}
Let $i_{0} \in I_{inf}$ be the index of a constraint that has a most infeasible
origin, that is 
\begin{equation}
  \left| b_{i_{0}}-\sum_{k \in O \setminus Z}x_{k}^{(0)} \right|
  \geq 
  \left|b_{i}-\sum_{k \in O \setminus Z}x_{k}^{(0)}\right|, \quad i \in I_{inf}
\end{equation}
then $B_{O}$, $B_{S}$ and the initial set of basic and nonbasic constraints
$S_{B}$ and $S_{N}$are initialized as
\begin{equation}
\begin{array}{ccccccc}
  \label{def:headings_init_io}
B_{O}^{(0)} &:=& art && B_{S}^{(0)} &:=& S \setminus
  \{\sigma^{-1}\left(i_{0}\right)\} \\
S_{B}^{(0)} &:=& I \setminus \{i_{0}\} && S_{N}^{(0)} & := & \{ i_{0} \} 
\end{array}
\end{equation}
If on the other hand, $I_{inf}=\emptyset$ then $B_{O}$, $B_{S}$, $S_{B}$ and
$S_{N}$ are initialized as
\begin{equation}
\begin{array}{ccccccc}
\label{def:headings_init_fo}
B_{O}^{(0)} &:=& art && B_{S}^{(0)} &:=& S \\
S_{B}^{(0)} &:=&I && S_{N}^{(0)}&:=& \emptyset
\end{array}
\end{equation}
where $art$ does not contain a special artificial variable.

\section{Optimality conditions}
By the KKT conditions for optimality a feasible solution $x^{*}$ to the
QP~(\ref{def:QP_UB}) is optimal, if there exists an $\left|E\right|$-vector
$\lambda$ and an $\left|I\right|+2n$-vector $\mu \geq 0$,
$\mu=\left({\mu^{\leq}}^{T}\left|\right. {\mu^{\geq}}^{T} \left|\right.
{\mu^{l}}^{T} \left|\right. {\mu^{u}}^{T}\right)^{T}$, such that
\begin{eqnarray*}
c + Dx^{*} +A_{E, \bullet}^{T}\lambda
&=&
-A_{I^{\leq}, \bullet}^{T}\mu^{\leq} + A_{I^{\geq}, \bullet}^{T}\mu^{\geq}
+I\mu^{l} -I\mu^{u}  \\
\mu_{i}^{\leq}\left(A_{i, \bullet}x^{*}-b_{i}\right)
&=&
0 \quad\quad\quad i \in I^{\leq}\\
\mu_{i}^{\geq}\left(b_{i}-A_{i, \bullet}x^{*}\right)
&=&
0 \quad\quad\quad i \in I^{\geq}\\
\mu_{i}^{l}\left(l_{i}-x_{i}^{*}\right)
&=& 
0 \quad\quad\quad i= 1 \dots n\\
\mu_{i}^{u}\left(x_{i}^{*}-u_{i}\right)
&=&
0 \quad\quad\quad i=1 \dots n
\end{eqnarray*}
Extending $\lambda$ appropriately and disposing of $\mu^{\leq}$ and $\mu^{\geq}$ in the above conditions we obtain
the following equivalent conditions:
A feasible solution $x^{*}$ to the QP~(\ref{def:QP_UB}) is optimal if there exists an $m$-vector $\lambda$, $\lambda_{i} \geq 0$ for $i \in I^{\leq}$,
$\lambda_{i} \leq 0$ for $i \in I^{\geq}$, and two $n$-vectors $\mu^{l}$ and
$\mu^{u}$, $\mu^{l} \geq 0$, $\mu^{u} \geq 0$ such that
\begin{eqnarray*}
c + Dx^{*} +A^{T}\lambda
&=&
I\left(\mu^{l} -\mu^{u}\right)  \\
\lambda^{T}\left(Ax^{*}-b\right)
&=&
0 \\
\mu_{i}^{l}\left(l_{i}-x_{i}^{*}\right)
&=& 
0 \quad\quad\quad i= 1 \dots n\\
\mu_{i}^{u}\left(x_{i}^{*}-u_{i}\right)
&=&
0 \quad\quad\quad i=1 \dots n
\end{eqnarray*}
Replacing $\mu^{l}-\mu^{u}$ by $\mu$ we finally obtain
\begin{lemma}
A feasible solution $x^{*}$ to QP~(\ref{def:QP_UB}) is optimal, if there exists an $m$-vector $\lambda$ and an $n$-vector $\mu$ such that 
\begin{eqnarray}
c + Dx^{*} +A^{T}\lambda
&=&
\mu  \\
\lambda^{T}\left(Ax^{*}-b\right)
&=&
0
\end{eqnarray}
and
\begin{eqnarray}
\label{def:mu_opt_cond}
\lambda_{i} \left\{
\begin{array}{ll}
\geq 0, & \text{for $i \in I^{\leq}$} \\
\leq 0, & \text{for $i \in I^{\geq}$}
\end{array}
\right.
&&
\mu_{j} \left\{
\begin{array}{ll}
\geq 0, & \text{if $x_{j}^{*}=l_{j}$} \\
= 0,    & \text{if $l_{j} < x_{j}^{*} < u_{j}$} \\
\leq 0, & \text{if $x_{j}^{*}=u_{j}$}
\end{array}
\right.
\end{eqnarray}
\end{lemma}

\section{Pricing}
Due to the fact that original nonbasic variables may assume several different values and according to Equations~(\ref{def:mu_opt_cond}) the pricing step not only has to deliver an entering variable $x_{j}$, $j \in N$, if the current solution is suboptimal, but a direction $d_{t} \in \{-1,1\}$ indicating whether the entering variable $x_{j}$ is to be increased or decreased. So if $d_{t}=1$ the entering variable is to be increased, otherwise it is to be decreased.

\subsection{Full exact pricing}
Generalizing the full exact pricing algorithm we introduce the following subsets of the set of nonbasic variables $N$
\begin{eqnarray}
N_{l}&:=&\{k \in N\left|\right. b\left(k\right)=-1\} \\
N_{0}&:=&\{k \in N\left|\right. b\left(k\right)=0\} \\
N_{u}&:=&\{k \in N\left|\right. b\left(k\right)=1\}
\end{eqnarray}
\paragraph{Full exact pricing with explicit bounds}
\begin{tabbing}
\texttt{FULL\_EXACT\_PRICING$(N)$} \\
\texttt{IF} \= \kill
\> \texttt{$j_{l}:=\arg\min_{k \in N_{l} \cup
                               N_{0}}\mu_{k}$}  \\
\> \texttt{$j_{u}:=\arg\max_{k \in N_{u} \cup
                               N_{0}}\mu_{k}$}  \\
\> \texttt{$j:=\arg\max_{k \in \{j_{l}, j_{u}\}}\left|\mu_{k}\right|$}  \\
\> \texttt{IF $\mu_{j} \neq 0$ THEN} \\
\texttt{IFIF} \= \kill
\> \texttt{IF $j = j_{l}$ THEN $dir:=1$ ELSE $dir:=-1$ END} \\
%\texttt{IFIFIF} \= \kill
%\> \texttt{dir:=1} \\
%\texttt{IFIF} \= \kill
%\> \texttt{ELSE} \\
%\texttt{IFIFIF} \= \kill
%\> \texttt{dir=-1} \\
%\texttt{IFIF} \= \kill
%\> \texttt{END} \\
\> \texttt{RETURN $(j, dir)$} \\
\texttt{IF} \= \kill
\> \texttt{ELSE} \\
\texttt{IFIF} \= \kill
\> \texttt{RETURN \emph{optimal}} \\
\texttt{IF} \= \kill
\> \texttt{END} \\
\texttt{END}
\end{tabbing}


\subsection{Partial exact pricing}
Generalizing the partial pricing algorithm given in \cite{Sven}, section 6.4.1 for explicit bounds we introduce the following subsets of the active set $\mathcal{A}$ 
\begin{eqnarray}
\mathcal{A}_{l}&:=&\{k \in \mathcal{A}\left|\right. b\left(k\right)=-1\} \\
\mathcal{A}_{0}&:=&\{k \in \mathcal{A}\left|\right. b\left(k\right)=0\} \\
\mathcal{A}_{u}&:=&\{k \in \mathcal{A}\left|\right. b\left(k\right)=1\}
\end{eqnarray}
\paragraph{Partial exact pricing with explicit bounds}
\begin{tabbing}
\texttt{PARTIAL\_EXACT\_PRICING$(\mathcal{A})$} \\
\texttt{IF} \= \kill
\> \texttt{$j_{l}:=\arg\min_{k \in \mathcal{A}_{l} \cup
                               \mathcal{A}_{0}}\mu_{k}$}  \\
\> \texttt{$j_{u}:=\arg\max_{k \in \mathcal{A}_{u} \cup
                               \mathcal{A}_{0}}\mu_{k}$}  \\
\> \texttt{$j:=\arg\max_{k \in \{j_{l}, j_{u}\}}\left|\mu_{k}\right|$}  \\
\> \texttt{IF $\left(\mu_{j} < 0 \wedge b\left(j\right) \leq 0 \right)
   \vee \left(\mu_{j} > 0 \wedge b\left(j\right) \geq 0 \right)$} \\
\texttt{IFIF} \= \kill
\> \texttt{IF $j=j_{l}$ THEN $dir:=1$ ELSE $dir:=-1$ END} \\
\> \texttt{RETURN $(j, dir)$} \\
\texttt{IF} \= \kill
\> \texttt{ELSE} \\
\texttt{IFIF} \= \kill
\> \texttt{$V:=\{k \in N \setminus \mathcal{A} \left|\right.
    \mu_{k} < 0 \wedge b\left(k\right) \leq 0 \vee
    \mu_{k} > 0 \wedge b\left(k\right) \geq 0\}$} \\
\> \texttt{IF $V=0$ THEN} \\
\texttt{IFIFIF} \= \kill  
\> \texttt{RETURN \emph{optimal}} \\
\texttt{IFIF} \= \kill
\> \texttt{ELSE} \\
\texttt{IFIFIF} \= \kill
\> \texttt{$\mathcal{A}:=\mathcal{A} \cup V$} \\
\> \texttt{RETURN $\arg\max_{j \in V}\left|\mu_{j}\right|$} \\
\texttt{IFIF} \= \kill
\> \texttt{END} \\
\texttt{IF} \= \kill
\> \texttt{END}
\end{tabbing}
\subsection{Full filtered pricing}
\subsection{Partial filtered pricing}

\section{Ratio Test Step~1}
According to (\ref{def:mu_opt_cond}) the value of an entering nonbasic variable may be increased or decreased.
The Ratio Test Step~1 determines the minimum positive increment or decrement for the entering variable $x_{j}$ with respect to the finite upper and lower bounds of the basic variables, the finite upper and lower bounds of the entering variable as well as $\mu_{j}$ becoming zero. We distinguish three events:
\begin{enumerate}
\item The entering variable $x_{j}$ takes the value of its finite upper,
$u_{j}$, or lower bound, $l_{j}$.
\item A basic variable takes the value of its finite upper or lower bound.
\item $\mu_{j}(x_{j})$ becomes zero.
\end{enumerate}
\marginpar{include artificial vars as well}  
%\begin{algorithm}
%\caption{Ratio Test Step~1 with upper bounding}
%\label{alg:ratio_test_step_1_0}
\begin{algorithmic}
\Function{ratio\_test\_1\_\_t\_i}{$B_{O}, B_{S}, d_{t}$}
\State $i_{min} \gets -1, \quad     d_{min} \gets 1, \quad  q_{min} \gets 0$
\If{$d_{t}=1$}
    \ForAll{$i \gets 0,  \left|B_{O}\right| - 1$}
        \If{$q_{B_{O}}[i] > 0 \wedge f_{B_{O}}^{l}[i]=1$}
            \If{$d_{min}*q_{B_{O}}[i] > (x_{B_{O}}[i]-l_{B_{O}}[i])*q_{min}$}
                \State $i_{min} \gets B_{O}[i],
                    \quad d_{min} \gets x_{B_{O}}[i]-l_{B_{O}}[i],
                    \quad q_{min} \gets q_{B_{O}}[i]$
            \EndIf
        \ElsIf{$q_{B_{O}}[i] < 0 \wedge f_{B_{O}}^{u}[i]=1$}
            \If{$d_{min}*-q_{B_{O}}[i] > (u_{B_{O}}[i]-x_{B_{O}}[i])*q_{min}$}
                \State $i_{min} \gets B_{O}[i],
                        \quad d_{min} \gets u_{B_{O}}[i]-x_{B_{O}}[i],
                        \quad q_{min} \gets -q_{B_{O}}[i]$
            \EndIf
        \EndIf
    \EndFor
    \ForAll{$i \gets 0, \left|B_{S}\right| - 1$}
        \If{$q_{B_{S}}[i] > 0$}
            \If{$d_{min}*q_{B_{S}}[i] < x_{B_{S}}[i]*q_{min}$}
                \State $i_{min} \gets B_{S}[i],
                    \quad d_{min} \gets x_{B_{S}}[i],
                    \quad q_{min} \gets q_{B_{S}}[i]$
            \EndIf	
        \EndIf 
    \EndFor
\Else
    \ForAll{$i \gets 0,  \left|B_{O}\right| - 1$}
        \If{$q_{B_{O}}[i] > 0 \wedge f_{B_{O}}^{u}[i]=1$}
            \If{$d_{min}*q_{B_{O}}[i] > (u_{B_{O}}[i]-x_{B_{O}}[i])*q_{min}$}
                \State $i_{min} \gets B_{O}[i],
                    \quad d_{min} \gets u_{B_{O}}[i]-x_{B_{O}}[i],
                    \quad q_{min} \gets q_{B_{O}}[i]$
            \EndIf
        \ElsIf{$q_{B_{O}}[i] < 0 \wedge f_{B_{O}}^{l}[i]=1$}
            \If{$d_{min}*-q_{B_{O}}[i] > (x_{B_{O}}[i]-l_{B_{O}}[i])*q_{min}$}
                \State $i_{min} \gets B_{O}[i],
                    \quad d_{min} \gets x_{B_{O}}[i]-l_{B_{O}}[i],
                    \quad q_{min} \gets -q_{B_{O}}[i]$
            \EndIf
        \EndIf
    \EndFor
    \ForAll{$i \gets 0, \left|B_{S}\right| - 1$}
        \If{$q_{B_{S}}[i] < 0$}
            \If{$d_{min}*-q_{B_{S}}[i] < x_{B_{S}}[i]*q_{min}$}
                \State $i_{min} \gets B_{S}[i],
                    \quad d_{min} \gets x_{B_{S}}[i],
                    \quad q_{min} \gets -q_{B_{S}}[i]$
            \EndIf	
        \EndIf 
    \EndFor
\EndIf
\State \textbf{return} $(i_{min}, d_{min}, q_{min})$
\EndFunction
\end{algorithmic}
%\end{algorithm}

\section{Ratio Test Step~2}
Since according to (\ref{def:mu_opt_cond}) $d_{t}=1 \Rightarrow \mu_{j}<0$ and $d_{t}=-1 \Rightarrow \mu_{j}>0$ for an entering variable, we distinguish these two cases and for each basic variable $x_{i}$ we distinguish whether  
%\begin{algorithm}
%\caption{Ratio Test Step~2 with upper bounding}
%\label{alg:ratio_test_step_2_0}
\begin{algorithmic}
\Function{ratio\_test\_2\_0\_\_mu\_i}{$B_{O},B_{S}, d_{t}$}
\State $i_{min} \gets -1, \quad d_{min} \gets 0, \quad  p_{min} \gets 1$
\If{$d_{t}=1$}
\Comment $\mu_{j} \leq 0$
    \ForAll{$i \gets 0,  \left|B_{O}\right| - 1$}
        \If{$p_{B_{O}}[i] < 0 \wedge f_{B_{O}}^{l}[i]=1$}
            \If{$d_{min}*p_{B_{O}}[i] > (x_{B_{O}}[i]-l_{B_{O}}[i])*p_{min}$}
                \State $i_{min} \gets B_{O}[i],
                            \quad d_{min} \gets x_{B_{O}}[i]-l_{B_{O}}[i],
                            \quad p_{min} \gets p_{B_{O}}[i]$
            \EndIf
        \EndIf 
        \If{$p_{B_{O}}[i] > 0 \wedge f_{B_{O}}^{u}[i]=1$}
            \If{$d_{min}*p_{B_{O}}[i] < (u_{B_{O}}[i]-x_{B_{O}}[i])*p_{min}$}
                \State $i_{min} \gets B_{O}[i],
                            \quad d_{min} \gets u_{B_{O}}[i]-x_{B_{O}}[i],
                            \quad p_{min} \gets -p_{B_{O}}[i]$
            \EndIf
        \EndIf 
    \EndFor
    \ForAll{$i \gets 0, \left|B_{S}\right| - 1$}
        \If{$p_{B_{S}}[i] < 0$}
            \If{$d_{min}*p_{B_{S}}[i] > x_{B_{S}}[i]*p_{min}$}
                \State $i_{min} \gets B_{S}[i],
                            \quad d_{min} \gets x_{B_{S}}[i],
                            \quad p_{min} \gets p_{B_{S}}[i]$
            \EndIf	
        \EndIf 
    \EndFor
\Else
\Comment $\mu_{j} \geq 0$
    \ForAll{$i \gets 0,  \left|B_{O}\right| - 1$}
        \If{$p_{B_{O}}[i] > 0 \wedge f_{B_{O}}^{l}[i]=1$}
            \If{$d_{min}*p_{B_{O}}[i] < (x_{B_{O}}[i]-l_{B_{O}}[i])*p_{min}$}
                \State $i_{min} \gets B_{O}[i],
                            \quad d_{min} \gets x_{B_{O}}[i]-l_{B_{O}}[i],
                            \quad p_{min} \gets p_{B_{O}}[i]$
            \EndIf
        \EndIf
        \If{$p_{B_{O}}[i] < 0 \wedge f_{B_{O}}^{u}[i]=1$}
            \If{$d_{min}*p_{B_{O}}[i] > (u_{B_{O}}[i]-x_{B_{O}}[i])*p_{min}$}
                \State $i_{min} \gets B_{O}[i],
                            \quad d_{min} \gets u_{B_{O}}[i]-x_{B_{O}}[i],
                            \quad p_{min} \gets -p_{B_{O}}[i]$
            \EndIf
        \EndIf 
    \EndFor
    \ForAll{$i \gets 0, \left|B_{S}\right| - 1$}
        \If{$p_{B_{S}}[i] > 0$}
            \If{$d_{min}*p_{B_{S}}[i] < x_{B_{S}}[i]*p_{min}$}
                \State $i_{min} \gets B_{S}[i],
                            \quad d_{min} \gets u_{B_{O}}[i]-x_{B_{S}}[i],
                            \quad p_{min} \gets p_{B_{S}}[i]$
            \EndIf	
        \EndIf 
    \EndFor
\EndIf
\State \textbf{return} $(i_{min}, x_{min}, p_{min})$
\EndFunction
\end{algorithmic}
%\end{algorithm}

\begin{thebibliography}{99}
\bibitem{Sven} Sven Sch\"{o}nherr. Quadratic Programming in Geometric Optimization:
Theory, Implementation, and Applications, Dissertation, Diss. ETH No 14738, ETH
Z\"{u}rich, Institute of Theoretical Computer Science, 2002.
\bibitem{Chvatal} Va\v{s}ek Chv\'{a}tal. \textit{Linear Programming}. W. H. Freeman and Company,
New York, Chapter 8, 1983 
\bibitem{Zielke} Gerhard Zielke. Inversion of Modified Symmetric Matrices. 
\textit{Journal of the Association for Computing Machinery}, Vol. 15, No. 3,
July 1968, pp. 402-408
\bibitem{Frans_Deg} Degeneracy
\end{thebibliography}
\end{document}