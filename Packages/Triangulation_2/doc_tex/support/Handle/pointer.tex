%% $RCSfile$
% $Revision$
% $Date$


%\chapter{Handles} \label{I1_ChapterPointer}
\section{Handles} 



%\section{Definition}
Most of the data structures and classes of the \cgal\ library
uses the concept of \ccc{Handle} in their user interface.
The concept \ccc{Handle} describes what is    often called 
a trivial iterator.  
A \ccc{Handle} is
mainly   a pointer to an object providing 
dereference operator \ccc{operator*()} and 
member access \ccc{operator->()} but no increment or decrement 
operators.
A  \ccc{Handle} should be used whenever the referenced
object
is not part of a logical sequence.
 
\paragraph{Model for a handle}
The class \ccc{CGAL::Pointer<T>}
provides a model for a handle
pointing to an object
of class \ccc{T}. 

Note that a simple pointer \ccc{T*},
an iterator or a circulator with value type \ccc{T},
can also be used as handles for object of type  \ccc{T}.

