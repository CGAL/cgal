% +------------------------------------------------------------------------+
% | Reference manual page: Handle_ref.tex
% +------------------------------------------------------------------------+
% | 10 04 2002   Mariette Yvinec
% | Package: Handle
% | 
\RCSdef{\RCSTriangulationRev}{$Revision$}
\RCSdefDate{\RCSTriangulationDate}{$Date$}
% |
%%RefPage: end of header, begin of main body
% +------------------------------------------------------------------------+

%\clearpage
%\section{Reference pages for Handles}
\chapter{Handles}

Most of the data structures and classes of the \cgal library
uses the concept of \ccc{Handle} in their user interface.
The concept \ccc{Handle} describes what is    often called 
a trivial iterator.  
A \ccc{Handle} is
mainly   a pointer to an object providing 
dereference operator \ccc{operator*()} and 
member access \ccc{operator->()} but no increment or decrement 
operators.
A  \ccc{Handle} should be used whenever the referenced
object
is not part of a logical sequence.
 
\subsection*{Concept}
\ccRefIdfierPage{Handle}

\subsection*{Class}
\ccRefIdfierPage{CGAL::Pointer<T>}
