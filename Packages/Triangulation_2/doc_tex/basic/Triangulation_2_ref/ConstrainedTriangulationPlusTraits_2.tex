% +------------------------------------------------------------------------+
% | Reference manual page: ConstrainedTriangulationPlusTraits_2.tex
% +------------------------------------------------------------------------+
% | 05.02.2002   Mariette Yvinec
% | Package: Triangulation_2
% | 
\RCSdef{\RCSConstrainedTriangulationPlusTraitsRev}{$Revision$}
\RCSdefDate{\RCSConstrainedTriangulationPlusTraitsDate}{$Date$}
% |
%%RefPage: end of header, begin of main body
% +------------------------------------------------------------------------+


\begin{ccRefConcept}{ConstrainedTriangulationPlusTraits_2}

%% \ccHtmlCrossLink{}     %% add further rules for cross referencing links
%% \ccHtmlIndexC[concept]{} %% add further index entries

\ccDefinition
  
The concept \ccRefName\ defines the requirements for the traits
class of the constrained triangulation \ccc{Triangulation}
plugged as first template argument of the class
\ccc{Constrained_Triangulation_Plus_2<Tiangulation,Itag>}.
The triangulation class \ccc{Triangulation} is either
a \ccc{Constrained_Triangulation_2<Traits,Tds>}
or a \ccc{Constrained_Delaunay_Triangulation_2<Traits,Tds>}.
However, if intersection of constraints are allowed
(i.e. the \ccc{Itag} template parameter
is either  \ccc{Tag_exact_intersections} or
\ccc{Tag_exact_predicates}),  the traits
 class  of \ccc{Triangulation}  has to provide
an object function to compute the intersection of two
constraint segment. Therefore, the concept
\ccRefName is a refinement of the
the traits  concept
(\ccc{TriangulationTraits_2} or \ccc{DelaunayTriangulationTraits_2})
of the base triangulation  class.

\ccGeneralizes

ThisConcept generalizes one of the following concept\\
\ccc{TriangulationTraits_2} \\
\ccc{DelaunayTriangulationTraits_2}


\ccTypes

\ccNestedType{Intersect_2}{A function object whose operator()
computes the intersection of two segments :
\ccc{Object_2 operator()(Segment s1, Segment s2);}
{Returns the intersection of \ccc{s1} and \ccc{s2}.}
}

\ccCreation
\ccCreationVariable{traits}  %% choose variable name

\ccConstructor{ConstrainedTriangulationPlusTraits_2();}{default constructor.}
\ccGlue
\ccConstructor{ConstrainedTriangulationPlusTraits_2(
                    const ConstrainedTriangulationPlusTraits_2& ctptraits);}
{Copy constructor}

\ccMethod{ConstrainedTriangulationPlusTraits_2 operator=
	(const ConstrainedTriangulationPlusTraits_2& ctptraits);}
{Assignment operator.}


\ccHeading{Access to constructor object}
\ccMethod{Intersect_2  intersect_2_object();}{}

\ccHasModels
The Cartesian and homogeneous kernels provided by \cgal.



\ccSeeAlso
\ccc{TriangulationTraits_2} \\
\ccc{DelaunayTriangulationTraits_2}


\ccExample

%A short example program.
%Instead of a short program fragment, a full running program can be
%included using the 
%\verb|\ccIncludeExampleCode{examples/Package/ConstrainedTriangulationPlusTraits_2_prog.C}| 
%macro. The program example would be part of the source code distribution and
%also part of the automatic test suite.

%\begin{ccExampleCode}
%void your_example_code() {
%}
%\end{ccExampleCode}

%% \ccIncludeExampleCode{examples/Package/ConstrainedTriangulationPlusTraits_2_prog.C}

\end{ccRefConcept}

% +------------------------------------------------------------------------+
%%RefPage: end of main body, begin of footer
% EOF
% +------------------------------------------------------------------------+

