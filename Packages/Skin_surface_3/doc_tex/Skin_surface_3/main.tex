% +------------------------------------------------------------------------+
% | Skin_surface_3/doc_tex/Skin_surface_3/main.tex
% +------------------------------------------------------------------------+
% | Meshing the 3d Skin surface defined for a set of spheres.
% | 
\RCSdef{\skinSurfaceRev}{$Revision$}
\RCSdefDate{\skinSurfaceDate}{$Date$}
% +------------------------------------------------------------------------+

\chapter{3D Skin Surface Meshing}
\label{chapterSkinSurface}
\ccChapterRelease{\skinSurfaceRev. \ \skinSurfaceDate}
\ccChapterAuthor{Nico Kruithof}

\minitoc

% +------------------------------------------------------------------------+
\section{Introduction}
\label{sectionSkinSurfaceIntro}

Skin surfaces, introduced by Edelsbrunner in \cite{cgal:e-dssd-99} (TODO: was
\cite{e-dssd-98}), have a rich and simple combinatorial and geometric
structure that makes them suitable for modeling large molecules in
biological computing.  Meshing such surfaces is often required for
further processing of their geometry, like in numerical simulation and
visualization.

A skin surface is parameterized by a set of weighted points (input
balls) and a shrink factor. If the shrink factor is equal to one, the
surface is just the boundary of the union of the input balls.  If the
shrink factor decreases, the skin surface becomes tangent continuous,
due to the appearance of patches of spheres and hyperboloids
connecting the balls.

This package constructs an isotopic mesh from a set of balls and a
shrink factor using the algorithm described in
\cite{cgal:kv-mssct-05}. An optimized algorithm is implemented for
meshing the union of a set of balls.

\section{Definition}
A skin surface is defined as the boundary of an infinite set of balls.
Let ${S}^{(w)}$ be a set of weighted points in $\R^3$. A weighted
point ${p}^{(w)}=(p,w_p)$ can also be seen as a ball with center $p$
and squared radius $w_p$. A vector space of weighted points is
inherited from the bijective map
\[\Pi({p}^{(w)}) = ({p, \|{p}\|^2-w_p})\]
where $\|{p}\|$ is the two-norm. Addition of two weighted points and
the multiplication of a weighted point by a scalar are defined in the
vector space structure inherited under $\Pi$.

A shrunk weighted point ${p}^{(w)}_{s}$ is obtained by multiplying the
weight of the weighted point ${p}^{(w)}$ with the scalar $s$:
${p}^{(w)}_{s}= (p,s w_p)$. A set of weighted points is shrunk by
shrinking each weighted point seperately.

The skin surface of ${S}^{(w)}$ with shrink factor $s$ is the boundary
of the weighted points in the convex hull of ${S}^{(w)}$ shrunk by a
factor $s$:
\[{\tt skn}_s({S}^{(w)}) =
\partial({\tt conv}({S}^{(w)}))_s\]

A polyhedral complex called the mixed complex subdivides the skin
surface into quadric patches (spheres and hyperboloids). A mixed cell
exists for every simplex in the Regular triangulation. Adding alpha to
the weight of each weighted point in $S^{(w)}$, does not change the
mixed complex. For a shrink factor of one, the mixed complex
degenerates into the Voronoi diagram.

\section{Skin surface mesh}
The mesh is constructed in several steps:
\begin{enumerate}
\item Triangulate the mixed complex
\item Extract the mesh from the constructed triangulation
\item Apply modifications to the mesh
\end{enumerate}
Each of these steps is of independent interest and we describe them in
more detail.

\subsection{Triangulation of the mixed complex.}

\subsubsection{Union of balls}

\subsection{Marching tetrahedra}
\subsection{Modifications to the mesh}

\section{Software Design}
Back references
\section{Example programs}
% EOF


