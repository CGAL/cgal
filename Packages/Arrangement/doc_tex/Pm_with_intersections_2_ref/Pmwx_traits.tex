% +------------------------------------------------------------------------+
% | Reference manual page: Pmwx_2_traits.tex (Arrangement)
% +------------------------------------------------------------------------+
% |
% | Package: arr (Planar_map_with_intersections_2)
% |
% +------------------------------------------------------------------------+

\ccRefPageBegin

%%RefPage: end of header, begin of main body
% +------------------------------------------------------------------------+
% +========================================================================+
%   Arrangement_2_traits
% +========================================================================+
\begin{ccRefConcept}{PlanarMapWithIntersectionsTraits_2}

\ccDefinition
A model of the \ccRefName\ concept aggregates the geometric types and
primitive operations used by the 
\ccc{Planar_map_with_intersections_2<Planar_map>} data structure.

Note that the concept \ccRefName\ refines the concept \ccc{PlanarMapTraits_2}
and inherits all its types and operations.

In addition to the requirements of the \ccc{PlanarMapTraits_2} concept, it must
provide the types and operations listed below.

\ccRefines
\ccc{PlanarMapTraits_2} \lcTex{(\ccRefPage{PlanarMapTraits_2})}.

\ccTypes

The geometric types defined below must have a default constructor,
copy constructor, and assignment operator.

\ccNestedType{Curve_2}{A type that holds a general curve in the plane.
  Its endpoints must be of type \ccc{Point_2}.
  Curves of type \ccc{Curve_2} can be inserted into a 
  \ccc{Planar_map_with_intersections_2<Dcel,Traits>} object and are 
  automatically split into \ccc{X_monotone_curve_2} objects.}

\ccCreationVariable{pmwx_traits}

\ccCreation
    
\ccConstructor{Traits();}
{A default constructor.}

\ccOperations

The following methods that have a curve parameter of type
\ccc{X_monotone_curve_2} have the implicit precondition that requires the
curve to be $x$-monotone.

\ccMethod{bool curves_overlap(const X_monotone_curve_2& cv1,
                              const X_monotone_curve_2& cv2);}
         {returns \ccc{true} if \ccc{cv1} and \ccc{cv2} overlap
          in a one-dimensional subcurve (i.e. at an infinite and uncountable 
          number of points), \ccc{false}. otherwise.}

%construction function
%---------------------   

\ccMethod{template<class OutputIterator>
          OutputIterator make_x_monotone(const Curve_2& cv,
                                         OutputIterator res);}
	 {cuts \ccc{cv} into $x$-monotone subcurves and stores them in a
	  sequence starting at \ccc{res}. The order in which they are stored
	  defines their order in the hierarchy tree. Returns past-the-end
	  iterator of the sequence. The value type of the output iterators
          must be \ccc{X_monotone_curve_2}.}

\ccMethod{void curve_split(const X_monotone_curve_2& cv,
                           X_monotone_curve_2& c1, X_monotone_curve_2& c2,
                           const Point_2& split_pt);}
         {splits $cv$ at \ccc{split_pt} into two curves, and assigns them to
           \ccc{c1} and \ccc{c2} respectively.
           \ccPrecond{\ccc{split_pt} is on \ccc{cv} but is not one of its
	     endpoints.}}

\ccMethod{bool nearest_intersection_to_right(const X_monotone_curve_2& c1,
                                             const X_monotone_curve_2& c2,
                                             const Point_2& pt);}
	 {finds the nearest intersection of \ccc{c1} and
	  \ccc{c2} lexicographically to the right of a reference point
	  \ccc{pt}, and returns an object that represents the intersection if
	  exists, or an empty object otherwise.
          Nearest is defined as the lexicographically nearest point, not
	  including the reference point itself. If the intersection of the
	  two curves is an \ccc{X_monotone_curve_2}, that is, there is an
	  overlapping subcurve, that contains the reference point in its
	  $x$-range, it returns an \ccc{X_monotone_curve_2} whose interior is
	  strictly to the right of the reference point (that is, whose
	  left endpoint is the projection of the reference point onto the 
	  overlapping subcurve).}

\ccMethod{bool nearest_intersection_to_left(const X_monotone_curve& c1,
                                            const X_monotone_curve& c2,
                                            const Point_2& pt);}
	 {finds the nearest intersection of \ccc{c1} and
	  \ccc{c2} lexicographically to the left of a reference point
	  \ccc{pt}, and returns an object that represents the intersection if
	  exists, or an empty object otherwise.
          Nearest is defined as the lexicographically nearest point, not
	  including the reference point itself. If the intersection of the
	  two curves is an \ccc{X_monotone_curve_2}, that is, there is an
	  overlapping subcurve, that contains the reference point in its
	  $x$-range, it returns an \ccc{X_monotone_curve_2} whose interior is
	  strictly to the left of the reference point (that is, whose
	  right endpoint is the projection of the reference point onto the 
	  overlapping subcurve).}

\ccHasModels
The following classes are actually models of the \ccc{ArrangementTraits_2} 
concept, that is a refinment of the \ccRefName\ concept. 

  \ccc{Arr_segment_traits_2<Kernel>}\\
  \ccc{Arr_segment_cached_traits_2<Kernel>}\\
  \ccc{Arr_polyline_traits<Kernel, Container>} \\
  \ccc{Arr_conic_traits_2<Kernel>}

\end{ccRefConcept}

\ccRefPageEnd
