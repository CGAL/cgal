% +------------------------------------------------------------------------+
% | Reference manual page: Arr_edge_node.tex (Arrangement)
% +------------------------------------------------------------------------+
% | 
% | Package: arr (Arrangement_2)
% | 
% +------------------------------------------------------------------------+

\ccRefPageBegin

%%RefPage: end of header, begin of main body
% +------------------------------------------------------------------------+
% +========================================================================+
%   Arrangement_2::Edge_node
% +========================================================================+
\begin{ccRefClass}[Arrangement_2<Dcel,Traits,Base_node>::]{Edge_node}

\ccDefinition An object $en$ of the class
    \ccStyle{Edge_node} is a node holding an edge curve in the
    hierarchy tree.

\ccInheritsFrom
    \ccc{Subcurve_node}

    \ccCreationVariable{en}

    In addition to the operations of \ccc{Subcurve_node} the following
    operation is implemented.
    
\ccOperations
    
    \ccMethod{bool is_edge_node();}{returns true;}
    
    \ccMethod{Halfedge_handle halfedge();}{returns the halfedge
       in the planar map that corresponds to the \ccc{X_curve} in \ccVar.
       The returned halfedge has the same direction as the $x$-curve $cv$
       in \ccVar{}
       (i.e., its source vertex holds the point that is the source of
       $cv$, and its target vertex holds the point that is $cv$'s target.)}


\end{ccRefClass}


% +------------------------------------------------------------------------+
%%RefPage: end of main body, begin of footer
\ccRefPageEnd
% EOF
% +------------------------------------------------------------------------+
