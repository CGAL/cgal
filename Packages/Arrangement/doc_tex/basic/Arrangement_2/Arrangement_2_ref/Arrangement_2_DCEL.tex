% +------------------------------------------------------------------------+
% | Reference manual page: Arrangement_2_DCEL.tex (Arrangement)
% +------------------------------------------------------------------------+
% | 
% | Package: arr (Arrangement_2)
% | 
% +------------------------------------------------------------------------+

\ccRefPageBegin

%%RefPage: end of header, begin of main body
% +------------------------------------------------------------------------+
% +========================================================================+
%   Arrangement_2_DCEL
% +========================================================================+

\begin{ccRefConcept}{Arrangement_2_Dcel}

\ccDefinition
    An \ccRefName\ is a refinement of the \ccc{Planar_map_Dcel} concept 
    \lcTex{(see \ccRefPage{Planar_map_Dcel})}. 
    The requirements of the latter are 
    the requirements of the former.
    The additional requirements of the \ccRefName\ are
    additional functions for the
    Dcel \ccc{Halfedge} class. Those requirements are listed below.

\ccHasModels
   \ccc{Arr_2_default_dcel<Traits>} 
   \lcTex{(see introduction, \ccRefPage{arr_ref_intro})}

   The \ccRefName\ concept consists of a set of requirements as any other 
   concept. The notions of a Dcel vertex, Dcel halfedge and Dcel face are
   part of the whole Dcel concept. One can regard these notions as 
   {\em subconcepts,} (not a formal term). 

   However, since the implementation is often separated into different 
   classes, the user does not have to write the whole Dcel anew. For example,
   the \ccc{Arr_2_default_dcel} inherits the \ccc{Pm_Dcel<V,H,F>}
   \lcTex{(\ccRefPage{Pm_Dcel<V,E,F>})} with the actual parameters
   \ccc{Arr_2_vertex_base}, \ccc{Arr_2_halfedge_base} and 
   \ccc{Arr_2_face_base}. The reason for this is that the refinement of 
   the \ccRefName\ is only with the additional halfedge requirements of it.

% WHAT ABOUT THIS
\ccHeading{Halfedge of Arrangement\_2\_Dcel}

%\begin{ccClass}[Arrangement_2_DCEL_]{Halfedge}
\begin{ccClass}{Arrangement_2_Dcel_Halfedge}

\ccCreationVariable{e}


\ccDefinition An object $e$ of the class
    \ccStyle{Halfedge} is a halfedge of the Dcel of the
    arrangement.
    It has an additional type \ccc{Base_node} which is the
    same as the template parameter \ccc{Base_node} of the
    arrangement (see section~\ref{sec:base_node}).
    It has the additional methods \ccc{edge_node()} 
    and \ccc{set_edge_node(Base_node*)}.

\ccOperations
    \ccMethod{Base_node* edge_node();}
    {returns a pointer to the edge node in the hierarchy tree which
       holds the curve from 
       which the halfedge was generated.}

    \ccMethod{void set_edge_node(Base_node* b);}
    {sets \ccc{b} to be the edge node in the hierarchy tree which
       holds the curve of the halfedge.}


\ccHasModels
  \ccc{CGAL::Arr_2_halfedge_base<Base_node>} (for example, used by
  \ccc{Arr_2_default_dcel<Traits>} )


\end{ccClass}
%\end{ccRefConcept}

\end{ccRefConcept}


%\end{ccRefConcept}

% +------------------------------------------------------------------------+
%%RefPage: end of main body, begin of footer
\ccRefPageEnd
% EOF
% +------------------------------------------------------------------------+
