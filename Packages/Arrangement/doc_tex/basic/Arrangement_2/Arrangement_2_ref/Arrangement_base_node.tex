% +------------------------------------------------------------------------+
% | Reference manual page: Arrangement_base_node.tex (Arrangement)
% +------------------------------------------------------------------------+
% | 
% | Package: arr (Arrangement_2)
% | 
% +------------------------------------------------------------------------+

\ccRefPageBegin

%%RefPage: end of header, begin of main body
% +------------------------------------------------------------------------+
% +========================================================================+
%  Arrangement_base_node Concept
% +========================================================================+
\begin{ccRefConcept}{Arrangement_base_node}
\label{sec:base_node}

\ccDefinition
    The arrangement class is parameterized with the
    \ccRefName\ class which defines the base class for the objects
    inside the hierarchy tree.
    In this section we present the formal requirements for an \ccRefName\
    class.

    An object $bn$ of the class
    \ccRefName\ must provide the following types and operations.

\ccTypes
    \ccNestedType{Curve}{curve type, corresponding to the traits
    \ccc{Curve} type.}

\ccCreationVariable{bn}
\ccOperations
    \ccMethod{const Curve& curve() const;}{returns the curve in \ccVar.}
    \ccMethod{void set_curve(const Curve& c);}{sets the curve in \ccVar{} 
    to $c$.}

\ccHasModels
    \ccc{Arrangement_base_node<Curve>}

\end{ccRefConcept}
% +------------------------------------------------------------------------+
%%RefPage: end of main body, begin of footer
\ccRefPageEnd
% EOF
% +------------------------------------------------------------------------+
