% +------------------------------------------------------------------------+
% | Reference manual page: Arr_conic_traits.tex (Arrangement)
% +------------------------------------------------------------------------+
% | 
% | Package: arr (Arrangement_2)
% | 
% +------------------------------------------------------------------------+

\ccRefPageBegin

%%RefPage: end of header, begin of main body
% +------------------------------------------------------------------------+
% +========================================================================+
%  Arr_conic_traits_2<Kernel>
% +========================================================================+
\begin{ccRefClass}{Arr_conic_traits_2<Kernel>}

\ccDefinition
    The class \ccc{Arr_conic_traits_2<Kernel>} is a model of the 
    \ccc{ArrangementTraits_2} concept and serves as a traits class for finite
    segments of algebraic curves of degree 2 at most (conic curves).
    The template parameter \ccc{Kernel} must be a model of the a CGAL Kernel 
    concept that is templated by an exact number type that support the 
    arithmetic operations plus the square root operation, to ensure the
    robustness of the predicates and constructions carried out by the traits
    class. For example, \ccStyle{Cartesian<leda_real>} would be a good choice
    for the kernel.

    A general conic curve $C$ is the locus of all points $(x,y)$ satisfying the
    equation $rx^2 + sy^2 + txy + ux + vy + w = 0$, where:
    \begin{itemize}
    \item If $4rs - t^2 > 0$, $C$ is an ellipse. A special case occurs when
          $r = s \neq 0$ and $t = 0$, where $C$ becomes a circle.
    \item If $4rs - t^2 < 0$, $C$ is a hyperbola.
    \item If $4rs - t^2 = 0$, $C$ is a parabola. A special case occurs when
          $r = s = t = 0$, where $C$ becomes a line. 
    \end{itemize}

    A bounded conic arc is defined as one of the following:
    \begin{itemize}
    \item A full ellipse (or a circle) $C$.
    \item The supporting conic curve $C$ and two endpoints $s$ and $t$ (the 
    source and target, respectively).
    \end{itemize}

    It is possible to construct conic arcs in several ways, specifying a full
    ellipse or a conic curve and two endpoints. It is also possible to specify
    implicit endpoint that are formed by an intersection of $C$ with two other
    conic curves $C_1$ and $C_2$ --- in this case the user is just required to
    supply some approximation to the endpoints.

    A very useful subset of the set of conic arcs are line segments and 
    circular arcs, since arrangements of circular arcs and line segments have 
    some interesting applications (e.g. offsetting polygons, motion planning 
    for a disc robot, etc.). Circular arcs and line segment are simpler 
    objects and can be dealt with more efficiently than arbitrary arcs. 
    For these reasons, it is possible to construct conic arcs from segments 
    and from circles. Using these constructors is highly recommended: It is 
    more straightforward and also speeds up the arrangement construction.

\ccInclude{CGAL/Arr_conic_traits_2.h}

\ccIsModel
    \ccc{ArrangementTraits_2} \lcTex{(\ccRefPage{ArrangementTraits_2}).}

% The Conic_arc_2 class:
%
\subsection*{Class \ccc{Arr_conic_traits_2<Kernel>::Curve_2}}

The \ccc{Curve_2} class nested within the conic arcs' traits can represent
arbitrary conic arcs and support their construction in various ways. The copy 
and default constructor as well as the assignment and equality operators are 
provided for conic arcs. In addition, an \ccc{operator<<} for the curves is 
defined for standard output streams and \ccc{Window_stream}.

\begin{ccClass}{Arr_conic_traits_2<Kernel>::Curve_2}
    \ccCreation
    \ccCreationVariable{cv}

\ccConstructor{Curve_2 (const Kernel::FT& r, 
                        const Kernel::FT& s, 
			const Kernel::FT& t, 
                        const Kernel::FT& u, 
                        const Kernel::FT& v, 
                        const Kernel::FT& w,
                        const Kernel::Point_2& ps,
                        const Kernel::Point_2& pt);}
  {constructs a conic arc supported by the curve $rx^2 + sy^2 + txy + ux + 
   vy + w = 0$ with \ccc{ps} as its source and \ccc{pt} as its target.
   \ccPrecond{\ccc{ps} and \ccc{pt} both satisfy the equation of the supporting
              curve and define a bounded segment from it (e.g. in case of a
              hyperbolic arc, both point should be located on the same branch
              of the hyperbola.}}

\ccConstructor{Curve_2(const Kernel::FT& r, 
                       const Kernel::FT& s, 
		       const Kernel::FT& t, 
                       const Kernel::FT& u, 
                       const Kernel::FT& v, 
                       const Kernel::FT& w);}
  {constructs a conic arc that corresponds to the full curve $rx^2 + sy^2 + 
   txy + ux + vy + w = 0$.
   \ccPrecond{The given curve is an ellipse, that is $4rs - t^2 > 0$.}}

\ccConstructor{Curve_2(const Kernel::Segment_2& seg);}
  {constructs an arc from the line segment \ccc{seg}.}

\ccConstructor{Curve_2(const Kernel::Circle_2& circ,
                       const Kernel::Point_2& ps,
                       const Kernel::Point_2& pt);}
  {constructs a circular arc supported by the circle \ccc{circ} and with 
   \ccc{ps} and \ccc{pt} as its endpoints. If \ccc{circ} has a positive
   orientation, then the arc is formed by going in a counterclockwise 
   direction from \ccc{ps} to \ccc{pt}, and going in a clockwise direction if
   the circle has a negative orientation.
   \ccPrecond{\ccc{ps} and \ccc{pt} both lie on the circle \ccc{circ}}.}

\ccConstructor{Curve_2(const Kernel::Circle_2& circ);}
  {constructs a circular arc that corresponds to the full circle \ccc{circ}.}

\ccConstructor{Curve_2 (const Kernel::FT& r, 
                        const Kernel::FT& s, 
			const Kernel::FT& t, 
                        const Kernel::FT& u, 
                        const Kernel::FT& v, 
                        const Kernel::FT& w,
                        const Kernel::Point_2& ps,
                        const Kernel::FT& r1, 
                        const Kernel::FT& s1, 
			const Kernel::FT& t1, 
                        const Kernel::FT& u1, 
                        const Kernel::FT& v1, 
                        const Kernel::FT& w1,
                        const Kernel::Point_2& pt,
                        const Kernel::FT& r2, 
                        const Kernel::FT& s2, 
			const Kernel::FT& t2, 
                        const Kernel::FT& u2, 
                        const Kernel::FT& v2, 
                        const Kernel::FT& w2);}
  {constructs a conic are supported by the curve $rx^2 + sy^2 + txy + ux + 
   vy + w = 0$ with \ccc{ps} as its source and \ccc{pt} as its target.
   In this case \ccc{ps} and  \ccc{pt} are just approximations of the 
   endpoints, and their exact values are given implicitly, as the 
   intersections of the supporting curve with $r_{1}x^2 + s_{1}y^2 + t_{1}xy +
    u_{1}x + v_{1}y + w_{1} = 0$ and $r_{2}x^2 + s_{2}y^2 + t_{2}xy +
    u_{2}x + v_{2}y + w_{2} = 0$, respectively.
   \ccPrecond{The two curves specifying the endpoints really intersect with
              the supporting conic curve.}}

\ccOperations

\ccMethod{bool is_full_conic() const;}
  {returns whether the arc represents a full conic curve (a full ellipse).}

\ccMethod{const Kernel::Point_2& source() const;}
  {returns the source point of the arc.}

\ccMethod{const Kernel::Point_2& target() const;}
  {returns the target point of the arc.}

\ccMethod{bool is_x_monotone() const;}{returns \ccc{true} if the arc is
   $x$-monotone, \ccc{false} otherwise.}

\ccMethod{bool is_segment() const;}
  {returns whether the arc is a line segment.}

\ccMethod{bool is_circular() const;}
  {returns whether the arc is supported by a circle.} 

\ccInclude{CGAL/IO/Conic_arc_2_Window_stream.h}

\end{ccClass}

\end{ccRefClass} % Arr_conic_traits

% +------------------------------------------------------------------------+
%%RefPage: end of main body, begin of footer
\ccRefPageEnd
% EOF
% +------------------------------------------------------------------------+
