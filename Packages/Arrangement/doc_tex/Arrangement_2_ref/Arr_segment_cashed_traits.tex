% +------------------------------------------------------------------------+
% | Reference manual page: Arr_segment_cached_traits.tex (Arrangement)
% +------------------------------------------------------------------------+
% | 
% | Package: arr (Arrangement_2)
% | 
% +------------------------------------------------------------------------+

\ccRefPageBegin

%%RefPage: end of header, begin of main body
% +------------------------------------------------------------------------+
% +========================================================================+
%  Arr_segment_cached_traits_2<Kernel>
% +========================================================================+

\begin{ccRefClass}{Arr_segment_cached_traits_2<Kernel>}

\ccDefinition
   The class \ccRefName\ is a traits class for handling arrangements of line
   segments that stores additional cached information with it curves in order
   to speed up the arrangement construction. It thus uses the geometric kernel
   method in a more efficient way than the straightforward implementation, at
   the cost of additional memory.

   The \ccc{Arr_segment_traits_2<Kernel>::Curve_2} type is derived from the
   \ccc{Kernel::Segment_2} class, thus it has the same functionality. An
   object of this type can be created directly from a \ccc{Kernel::Segment_2} 
   object or from two \ccc{Kernel::Point_2} objects that specify its 
   endpoints.

   The \ccc{Kernel} must use a number type that supports exact computations
   with rational numbers, such as \ccc{Quotient<Gmpz>} or \ccc{leda_rational}.
   In this case, since the caching methods help avoiding cascaded computations
   it is also safe to use \ccc{Quotient<MP_Float>} as the number type.
   Other inexact representations can be used at the user's own risk.

\ccInclude{CGAL/Arr_segment_cached_traits_2.h}

\ccIsModel
    \ccc{ArrangementTraits_2} \lcTex{(\ccRefPage{ArrangementTraits_2}).}


\end{ccRefClass} % Arr_segment_cached_traits_2<Kernel>

% +------------------------------------------------------------------------+
%%RefPage: end of main body, begin of footer
\ccRefPageEnd
% EOF
% +------------------------------------------------------------------------+
