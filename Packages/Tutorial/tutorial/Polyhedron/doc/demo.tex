The application demo runs on windows and features a standard MFC multi
document-view architecture. It features a trackball to interactively
manipulate the polyhedron. Accepted mesh file formats are ASCII OFF
and OBJ (1-based vertex indices for the latter). \\

\noindent {\bf Mouse interaction}\\
Left: rotation\\
Right: translation\\
Both: zoom.\\

\noindent {\bf Main keys}\\
Press 'r' to choose a rendering mode. Note that superimposing the
control edges during subdivision is only available for the
quad-triangle subdivision scheme.\\
Press 's' to subdivide the mesh using the quad-triangle scheme.\\
Press 'ctrl+c' to generate a 24-bits raster image output to the clipboard.\\

\noindent {\bf Compiling on Windows}\\
The application has been compiled on MS .NET 2003 using CGAL. Apply
the following steps to compile the demo:
\begin{itemize}
\item Install the last release of CGAL.
\item Define an environment variable CGALROOT with the path to the CGAL-3.0.1 folder. If you installed CGAL with the standard wizard installation, the CGALROOT variable was defined at that time.
\item Compile the CGAL library in multithread mode. Go in CGALROOT/src directory and open the cgallib project. Open from the main menu, Project Properties. Go to C/C++, Code Generation, Runtime Library and choose Multi-threaded Debug (/MTd). Go to Librarian, General, Output File and modify to ../lib/msvc7/CGAL\_MT\_DEBUG.LIB. If you want to build in release mode, choose at the Project Properties Configuration Release, and do the same steps as before but at the Runtime Library you choose Multi-threaded (/MT) and in the output of Librarian put ../lib/msvc7/CGAL\_MT\_RELEASE.LIB.

To build debug and release mode, you need to choose the mode in the main menu Build at the Configuration manager.

\item Open the Mesh project in the tutorial/Polyhedron/MFC/Subdivision/ directory. Choose from the Configuration manager the mode (release or debug) you want to build your application and then go to the next step.
\item Rebuild all.
\end{itemize}
