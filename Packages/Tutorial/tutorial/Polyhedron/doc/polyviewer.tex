The tutorial starts with an implementation of a basic polyhedron
viewer based on the \cgalpoly\ with the default configuration.  This
basic viewer is based on OpenGL and demonstrates basic functionalities
of a \cgalpoly . We describe how to import a polyhedron file in the
OBJ format based on the \italic{modifier callback mechanism} and the
\italic{incremental builder}. We also show the mesh traversal based on
the \italic{iterators} and the \italic{circulators} for rendering and
the OBJ file exporting.

An extended polyhedron viewer is then introduced by customizing the
\poly\ with extra attributes and functionalities. This enriched
polyhedron supports facet and vertex normals for rendering, supports
the axis-aligned bounding box of the polyhedron, and provides geometry
items specialized with algorithmic flags.  The superimposition of the
control mesh on the subdivision surfaces is implemented with the flags
of the halfedge items (\figurename\ \ref{fig:quad-triangle}).

The tutorial also features a trackball to interactively manipulate the
polyhedron, a snapshot function of the camera viewpoint and the
transformation states, a raster output to the clipboard, and the
vectorial output to a postscript file.
