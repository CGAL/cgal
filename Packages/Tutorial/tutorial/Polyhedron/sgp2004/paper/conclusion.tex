The CGAL polyhedron is a robust and efficient mesh data structure.
With the generic design, the CGAL polyhedron is highly flexible
to be specialized for specific algorithm implementations.
Having the additional set of CGAL geometry components, the 
developments of a set of geometry algorithms can be
persisted within one framework.

In this paper, we have demonstrated several 
mesh algorithms based on the
CGAL polyhedron and other CGAL geometry components. We 
presented a combinatorial subdivisions solution showing the 
possibility of the extensible algorithm model 
based on the flexibility of the CGAL polyhedron. We also 
implemented a $\sqrt{3}$ subdivision showing the 
efficiency of the CGAL polyhedron.
Two remeshing algorithms, based on the CGAL polyhedron
and Delaunay triangulation, were demonstrated to show 
the versatility of the CGAL framework. And a set of
geometry functionalities were presented to show [lutz]...

With the emerging of CGAL as the standard library of the 
computational geometry, more and more geometry algorithms 
are researched and designed within CGAL. Using the CGAL
components frees the researchers and developers from the
implementation details of the underlying data structures
and lets them focus on the core of the algorithms. It
furthermore benefits the geometry processing community.
