% +------------------------------------------------------------------------+
% | Reference manual page: SplitRule.tex
% +------------------------------------------------------------------------+
% | 1.07.2001   Johan W.H. Tangelder
% | Package: ASPAS
% | 
\RCSdef{\RCSSplitRuleRev}{$Revision$}
\RCSdefDate{\RCSSplitRuleDate}{$Date$}
% |
%%RefPage: end of header, begin of main body
% +------------------------------------------------------------------------+


\begin{ccRefFunctionObjectConcept}{SplitRule<Item>}

%% \ccHtmlCrossLink{}     %% add further rules for cross referencing links
%% \ccHtmlIndexC[concept]{} %% add further index entries


\ccDefinition
  
The split rule concept defines the requirements on a function implementing a
splitting rule. Some splitting rules, require that the ratio of the longest and shortest side
of a node do not exceed the value specified by the \ccc{Aspect_ratio} parameter.  Other
spitting rules do not require this parameter.

\ccParameters

Expects for the parameter \ccc{Item} an implementation of the
\ccc{Point} concept, for example \ccc{CGAL::Point_d}.

\ccTypes

\ccTypedef{Item::FT NT;}{Number type.} 

\ccCreation
\ccCreationVariable{s}  %% choose variable name

\ccConstructor{
SplitRule();}{Default constructor.}

\ccHeading{Member Functions}

\ccMethod{
Plane_separator<NT>* rule(Point_container<Item>& c, NT Aspect_ratio);} 
{Computes and returns the splitter, that can split \ccc{c}. Note that the parameter \ccc{Aspect_ratio}
is optional.}

\ccHasModels

\ccc{Median_of_max_spread},
\ccc{Fair},
\ccc{Sliding_fair},
\ccc{Sliding_midpoint},
\ccc{Median_of_box},
\ccc{MidPoint_of_max_spread},
\ccc{Midpoint_of_box}.

\ccSeeAlso

\ccc{Point_container},
\ccc{Plane_separator}.


\end{ccRefFunctionObjectConcept}

% +------------------------------------------------------------------------+
%%RefPage: end of main body, begin of footer
% EOF
% +------------------------------------------------------------------------+

