% +------------------------------------------------------------------------+
% | Reference manual page: FuzzyQueryItem.tex
% +------------------------------------------------------------------------+
% | 1.07.2001   Johan W.H. Tangelder
% | Package: ASPAS
% | 
\RCSdef{\RCSFuzzyQueryItemRev}{$Revision$}
\RCSdefDate{\RCSFuzzyQueryItemDate}{$Date$}
% |
%%RefPage: end of header, begin of main body
% +------------------------------------------------------------------------+


\begin{ccRefConcept}{FuzzyQueryItem}

%% \ccHtmlCrossLink{}     %% add further rules for cross referencing links
%% \ccHtmlIndexC[concept]{} %% add further index entries

\ccDefinition
  
The concept \ccRefName\ describes the requirements for fuzzy $d$-dimensional spatial objects.

\ccTypes

\ccNestedType{Point}{ represents a $d$-dimensional point.}
\ccNestedType{NT}{Number type.} 

\ccCreationVariable{q}  %% choose variable name



\ccOperations

\ccMethod{bool contains(Point p); const}{test whether $q$ contains $p$.}

\ccMethod{bool inner_range_intersects(const Kd_tree_rectangle<PointTraits>& rectangle); const}
{test whether the inner approximation of the spatial object intersects a rectangle
associated with a node of a tree. }

\ccMethod{bool outer_range_is_contained_by(const Kd_tree_rectangle<PointTraits>& rectangle); const}
{test whether the outer approximation of the spatial object encloses a rectangle
associated with a node of a tree.}


\ccHasModels

\ccc{CGAL::Fuzzy_sphere<PointTraits>},
\ccc{CGAL::Fuzzy_iso_box<PointTraits, IsoBox>}.

\ccSeeAlso

\ccc{CGAL::Fuzzy_sphere<PointTraits>},
\ccc{CGAL::Fuzzy_iso_box<PointTraits, IsoBox>}.


\end{ccRefConcept}

% +------------------------------------------------------------------------+
%%RefPage: end of main body, begin of footer
% EOF
% +------------------------------------------------------------------------+

