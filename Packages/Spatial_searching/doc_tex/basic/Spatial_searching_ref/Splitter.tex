% +------------------------------------------------------------------------+
% | Reference manual page: Splitter.tex
% +------------------------------------------------------------------------+
% | 1.07.2001   Johan W.H. Tangelder
% | Package: ASPAS
% | 
\RCSdef{\RCSSplitterRev}{$Revision$}
\RCSdefDate{\RCSSplitterDate}{$Date$}
% |
%%RefPage: end of header, begin of main body
% +------------------------------------------------------------------------+


\begin{ccRefConcept}{Splitter}

%% \ccHtmlCrossLink{}     %% add further rules for cross referencing links
%% \ccHtmlIndexC[concept]{} %% add further index entries

\ccDefinition
  
The concept \ccRefName\ defines the requirements for a function object class implementing a splitting rule.

\ccTypes

\ccNestedType{NT;}{Number type.}
\ccNestedType{Container;}{Point container.}
\ccNestedType{Separator;}{Separator.} 

\ccCreationVariable{s}  %% choose variable name


\ccOperations

\ccMethod{
void operator()(Separator& s, PointContainer& c0, PointContainer& c1, NT Aspect_ratio=NT(3));} 
{Modifies the separator $s$, 
and splits \ccc{c0} into \ccc{c0} and \ccc{c1},
using the splitting dimension and the splitting value of the modified separator.
\ccc{Aspect_ratio} is an optional parameter.
}
\ccHasModels

\ccc{CGAL::Fair<SpatialPoint,PointContainer,Separator>}, \\
\ccc{CGAL::Median_of_rectangle<SpatialPoint,PointContainer,Separator>}, \\
\ccc{CGAL::Median_of_max_spread<SpatialPoint,PointContainer,Separator>}, \\
\ccc{CGAL::Midpoint_of_rectangle<SpatialPoint,PointContainer,Separator>}, \\
\ccc{CGAL::Midpoint_of_max_spread<SpatialPoint,PointContainer,Separator>}, \\
\ccc{CGAL::Sliding_fair<SpatialPoint,PointContainer,Separator>}, \\
\ccc{CGAL::Sliding_midpoint<SpatialPoint,PointContainer,Separator>.}

\ccSeeAlso

PointContainer, Separator.


\end{ccRefConcept}

% +------------------------------------------------------------------------+
%%RefPage: end of main body, begin of footer
% EOF
% +------------------------------------------------------------------------+

