% +------------------------------------------------------------------------+
% | Reference manual page: Container.tex
% +------------------------------------------------------------------------+
% | 1.07.2001   Johan W.H. Tangelder
% | Package: ASPAS
% | 
\RCSdef{\RCSContainerRev}{$Revision$}
\RCSdefDate{\RCSContainerDate}{$Date$}
% |
%%RefPage: end of header, begin of main body
% +------------------------------------------------------------------------+


\begin{ccRefConcept}{Container}

%% \ccHtmlCrossLink{}     %% add further rules for cross referencing links
%% \ccHtmlIndexC[concept]{} %% add further index entries

\ccDefinition
  
The concept \ccRefName\ defines the requirements for
a point container. 

\ccParameters

\ccc{Point} represents a $d$-dimensional point.


\ccCreation
\ccCreationVariable{c}  %% choose variable name

\ccConstructor{Container(int d);}
{
Construct an empty container for storing \ccc{d}-dimensional points.
}

\ccConstructor{
template <class InputIterator>
Point_container(int d, InputIterator begin, InputIterator end);}
{
Construct the container of $d$-dimensional points of type \ccc{Point}
given by the iterator sequence \ccc{begin}, \ccc{end}.
}

\ccOperations

\ccMethod{template <class Separator>
void split_container(Container& C, Separator sep, bool sliding=false);}
{Given an empty container \ccc{C} with the same dimension as \ccc{c}, splits \ccc{c} into
\ccc{c} and \ccc{C} using the separator \ccc{sep}. If sliding is true after splitting 
each container contains at least one point. \ccc{c} should contain at least two points.}

\ccHasModels

\ccc{CGAL::Point_container<Point>}.

\ccSeeAlso

Point, Separator.
\ccc{CGAL::Point_container<Point>}.

\end{ccRefConcept}

% +------------------------------------------------------------------------+
%%RefPage: end of main body, begin of footer
% EOF
% +------------------------------------------------------------------------+

