% +------------------------------------------------------------------------+
% | Reference manual page: Kd_tree_traits_point.tex
% +------------------------------------------------------------------------+
% | 1.07.2001   Johan W.H. Tangelder
% | Package: ASPAS
% | 
\RCSdef{\RCSKdtreetraitspoint2Rev}{$Revision$}
\RCSdefDate{\RCSKdtreetraitspoint2Date}{$Date$}
% |
%%RefPage: end of header, begin of main body
% +------------------------------------------------------------------------+


\begin{ccRefClass}{Search_traits_2<Kernel>}

%% \ccHtmlCrossLink{}     %% add further rules for cross referencing links
%% \ccHtmlIndexC[class]{} %% add further index entries

\ccDefinition
  
The class \ccRefName\ can be used as a template parameter of the kd tree
and the search classes.


\ccInclude{CGAL/Search_traits_2.h}

\ccParameters
Expects for the template argument a model of the concept \ccc{Kernel},
for example \ccc{CGAL::Cartesian<double>} or \ccc{CGAL::Simple_cartesian<CGAL::Gmp_q>}.

\ccIsModel

\ccc{SearchTraits}.


\ccTypes
\ccTypedef{Kernel::FT FT;}{Number type.}

\ccTypedef {Kernel::Point_2 Point_d;}{Point type.}
\ccTypedef {Kernel::Iso_rectangle_2 Iso_box_d;}{Iso box type.}

\ccTypedef {Kernel::Cartesian_const_iterator_2 Cartesian_const_iterator_d;}{An iterator over the Cartesian coordinates.}

\ccTypedef {Kernel::Construct_cartesian_const_iterator_2 Construct_cartesian_const_iterator_d;}{A functor with
two function operators, which return the begin and past the end iterator for the Cartesian coordinates. 
The functor for begin has as argument a \ccc{Point_d}. The functor for the past the end iterator, 
has as argument a \ccc{Point_d} and an \ccc{int}.}

\ccTypedef {Kernel::Construct_iso_rectangle_2 Construct_iso_box_d;}{Functor with operator to construct 
the iso box from two points.}

\ccSeeAlso

\ccc{Search_traits_3<Kernel>}\\
\ccc{Search_traits<NT_,Point,CartesianConstIterator,ConstructCartesianConstIterator}

\end{ccRefClass}


% +------------------------------------------------------------------------+
%%RefPage: end of main body, begin of footer
% EOF
% +------------------------------------------------------------------------+

