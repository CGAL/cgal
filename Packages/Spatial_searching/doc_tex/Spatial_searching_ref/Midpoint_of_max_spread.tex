% +------------------------------------------------------------------------+
% | Reference manual page: Midpoint_of_max_spread.tex
% +------------------------------------------------------------------------+
% | 1.07.2001   Johan W.H. Tangelder
% | Package: ASPAS
% | 
\RCSdef{\RCSMidpointofmaxspreadRev}{$Revision$}
\RCSdefDate{\RCSMidpointofmaxspreadDate}{$Date$}
% |
%%RefPage: end of header, begin of main body
% +------------------------------------------------------------------------+


\begin{ccRefFunctionObjectClass}{Midpoint_of_max_spread<Item>}  %% add template arg's if necessary

%% \ccHtmlCrossLink{}     %% add further rules for cross referencing links
%% \ccHtmlIndexC[class]{} %% add further index entries

\ccDefinition
Implements the midpoint of max spread splitting rule.

\ccInclude{CGAL/Splitting_rules.h}

\ccIsModel

SplitRule

\ccTypes

\ccTypedef{Item::R::FT NT;}{Number type.} 

\ccCreation
\ccCreationVariable{s}  %% choose variable name

\ccConstructor{
Midpoint_of_max_spread();}{Default constructor.}

\ccHeading{Member Functions}

\ccMethod{
Plane_separator<NT>* rule(Point_container<Item>& c);} 
{Computes and returns the splitter, that can split \ccc{c}.}

\ccSeeAlso

SplitRule,
\ccc{Point_container},
\ccc{Plane_separator}.

\end{ccRefFunctionObjectClass}

% +------------------------------------------------------------------------+
%%RefPage: end of main body, begin of footer
% EOF
% +------------------------------------------------------------------------+

