% +------------------------------------------------------------------------+
% | Reference manual page: GeneralDistance.tex
% +------------------------------------------------------------------------+
% | 1.07.2001   Johan W.H. Tangelder
% | Package: ASPAS
% |
\RCSdef{\RCSGeneralDistanceRev}{$Revision$}
\RCSdefDate{\RCSGeneralDistanceDate}{$Date$}
% |
%%RefPage: end of header, begin of main body
% +------------------------------------------------------------------------+


\begin{ccRefConcept}{GeneralDistance<Query_item, Item>}

%% \ccHtmlCrossLink{}     %% add further rules for cross referencing links
%% \ccHtmlIndexC[concept]{} %% add further index entries

\ccDefinition

Requirements of a distance class defining a distance between a \ccc{Query_item}
denoting a spatial object and a \ccc{Item} denoting a point.
To optimize distance computations transformed distances are used.
E.g., for an Euclidian distance the transformed distance is the squared Euclidean distance.

\ccParameters

Expects for the parameter \ccc{Query_item} an implementation of a spatial object, for
example \ccc{Kd_tree_rectangle} implementing iso-rectangles.
Expects for the parameter \ccc{Item} an implementation of the
\ccc{Point} concept, for example \ccc{CGAL::Point_d}.

\ccTypes

\ccTypedef{Item::FT NT;}{Number type.}

\ccCreation
\ccCreationVariable{d}  %% choose variable name

\ccConstructor{GeneralDistance();}{Default constructor.}

\ccHeading{Member Functions}

\ccMethod{NT transformed_distance(NT d);} {Returns the transformed distance.}

\ccMethod{NT inverse_of_transformed_distance(NT d);} {Returns the inverse of the transformed distance.}

\ccMethod{NT distance(Query_item& q, Item& p);}{Returns the transformed distance between \ccc{q} and \ccc{p}.}

\ccMethod{NT min_distance_to_rectangle(Query_item& q, Kd_tree_rectangle<NT>& r;);}
{Returns the transformed distance between \ccc{q} and
the point on the boundary of \ccc{r} closest to \ccc{q}.}

\ccMethod{NT max_distance_to_rectangle(Query_item& q, Kd_tree_rectangle<NT>& r;);}
{Returns the transformed distance between \ccc{q} and
the point on the boundary of \ccc{r} farthest to \ccc{q}.}

\ccHasModels

\ccc{L1_distance_to_rectangles}.

\ccSeeAlso

\ccc{L1_distance_to_rectangles}.

\end{ccRefConcept}

% +------------------------------------------------------------------------+
%%RefPage: end of main body, begin of footer
% EOF
% +------------------------------------------------------------------------+

