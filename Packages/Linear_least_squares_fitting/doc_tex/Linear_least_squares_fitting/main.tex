\chapter{Principal Component Analysis}
\label{chap:pca}

\ccChapterAuthor{Pierre Alliez and Sylvain Pion}

\minitoc

\section{Introduction}

[TODO: replace next paragraphs by actual introduction]

We propose to elaborate upon a CGAL Package which implements some
of the state-of-the-art surface reconstruction methods. Priority will
be given to methods which take as input unorganized point sets and
which compute an implicit function, although we leave the room to
Delaunay-based techniques as all building blocks are available in
CGAL. A common implicit function is an approximate distance function
to the input points or to an estimate tangent plane, or an indicator
function.\\


The package will propose an interface to the CGAL Surface Mesh Generator~\cite{mariette06,oudot05},
and we leave the room to other surface contouring techniques such
as the Marching Cubes algorithm~\cite{37422} and its variants (a
user may prefer using his favorite contouring algorithm).\\


The input can be an unorganized point set, possibly with attributes
such as unoriented normals, oriented normals, confidence values. For
piecewise smooth reconstruction, it is also desirable to take not
only one attribute per point, but possibly several ones (for example,
two normals for a point on a sharp edge, or several normals for a
corner, or even a cone of normals). The input can also be a set of
arbitrary cross sections, be they defined by point sets localized
onto well-defined planes, or as polylines.\\


Since reconstruction methods often require to estimate and/or to orient
the normals of a point set, we plan to implement two components devoted
to this task. These components will make use of existing components
such as PCA or Jet fitting.\\


The output can be either an implicit function (ready for evaluation
by any contouring algorithm), or a surface mesh, compatible with the
BGL Graph concept and CGAL polyhedron when 2-manifold, or represented
as a polygon soup otherwise.


The intended audience of this package is researchers, developers or
students developing algorithms around surface reconstruction. We also
target end users and industrial applications.


% \section{Design and Implementation History}

\section{Examples}
\label{subsec:pca_examples}

\subsection{Centroid of a set of points}

In the following example we use \stl\ containers of 2D and 3D points, and
compute their centroid. The kernel from which the input points
come is automatically deduced by the function.

\ccIncludeExampleCode{Linear_least_squares_fitting/centroid.C}

\subsection{Barycenter of a set of points}

In the following example we use \stl\ containers of 2D and 3D weighted points,
and compute their barycenter. The kernel from which the input points come is
automatically deduced by the function.

\ccIncludeExampleCode{Linear_least_squares_fitting/barycenter.C}

\subsection{Best fitting line of a set of 2D points}

In the following example we use an \stl\ container of 2D points, and
compute the best fitting line. The kernel from which the input points
come is automatically deduced by the function.

\ccIncludeExampleCode{Linear_least_squares_fitting/linear_least_squares_fitting_2.C}

