%%%%%%%%%%%%%%%%%%%%%%%re-implement the base classes without curves and points
% hopefully this part will be shorter and include only reference to the HDS
%and base classes for tpm

\ccRefPageBegin

%%RefPage: end of header, begin of main body
% +------------------------------------------------------------------------+


%%%%%vertex
\begin{ccRefClass}{Tpm_vertex_base}
\label{DCEL_sec:vertex_base}
\ccDefinition The class \ccStyle{Tpm_vertex_base} is a
base class for the vertex of the DCEL.
%is parameterized 
%with a \ccStyle{Point} class ,which is stored in it. 
The methods it defines 
return \ccStyle{void*} since the \ccStyle{Halfedge} and \ccStyle{Face} classes 
are not known to it. A casting will be done inside the DCEL. 

\ccInclude{CGAL/Topological_map_bases.h}
%\ccTypes 

%\ccNestedType{Point}{a point stored in the vertices.}

\ccCreationVariable{v}
%\ccConstructor{Vertex(const Point& p);}{creates a vertex with point \ccStyle{p}.}
%\ccConstructor{Vertex(const Vertex&);}{copy constructor.}

%\ccOperations 
\ccAccessFunctions
\ccMethod{void* halfedge();}{an incident halfedge that has \ccVar{} as its target.}
\ccGlue
\ccMethod{const void* halfedge();}{}

\ccThree{void}{v.set_halfedge( void* h);}{}

\ccModifiers
\ccMethod{void set_halfedge(void* h); }{sets incident halfedge.}

\end{ccRefClass}


%%%%%halfedge
\begin{ccRefClass}{Tpm_halfedge_base}
\label{DCEL_sec:halfedge_base}
\ccDefinition The class \ccStyle{Tpm_halfedge_base} is 
a base class for the halfedge of the DCEL.
%parameterized 
%with a \ccStyle{Curve} class ,which is stored in it. 
The methods it defines
return \ccStyle{void*} since the \ccStyle{Vertex} and \ccStyle{Face} classes 
are not known to it. A casting will be done inside the DCEL. 

\ccInclude{CGAL/Topological_map_bases.h}

%\ccTypes

%\ccNestedType{Curve}{the curve of the halfedge.}

\ccCreationVariable{h}
%\ccConstructor{Halfedge(const Curve& cv);}{create a halfedge with curve \ccStyle{cv}.}
%\ccConstructor{Vertex(const Vertex&);}{copy constructor.}

%\ccOperations 
\ccAccessFunctions

\ccMethod{void* opposite();}{the twin halfedge.}
\ccGlue
\ccMethod{const void* opposite();}{}

\ccMethod{void*       next();}{the next halfedge around the face.}
\ccGlue
\ccMethod{const void* next() const; }{}

\ccMethod{void*       vertex();}{the target vertex.}
\ccGlue
\ccMethod{const void* vertex() const;}{}

\ccMethod{void*       face();}{the incident face.}
\ccGlue
\ccMethod{const void* face() const; }{}

\ccMethod{void set_vertex(void* v); }{sets target vertex.}
\ccGlue
\ccMethod{void set_face(void* f); }{sets incident face.}

\end{ccRefClass}

%%%%%face
\begin{ccRefClass}{Tpm_face_base}
\label{DCEL_sec:face_base}
\ccDefinition The class \ccStyle{Tpm_face_base} is a face in the DCEL. 
The methods it defines 
return \ccStyle{void*} since the \ccStyle{Vertex} and \ccStyle{Halfedge} classes 
are not known to it. A casting will be done inside the DCEL. 

\ccImplementation It uses an STL list as a holes container. The list stores
\ccStyle{void*} objects that point to a representative halfedge for every 
hole in the face. 

\ccInclude{CGAL/Topological_map_bases.h}

\ccTypes
%\ccNestedType{Holes_container}{a container of \ccStyle{void*}.}
\ccNestedType{Holes_iterator}{a bidirectional iterator for traversing the
holes container, its value type is \ccStyle{void*}.}
\ccGlue
\ccNestedType{Holes_const_iterator}{}

\ccCreationVariable{f}
\ccConstructor{Face();}{default constructor.}

%\ccOperations 
\ccAccessFunctions

\ccMethod{void* halfedge();}{an incident halfedge on the outer CCB of \ccVar{}.}
\ccGlue
\ccMethod{const void* halfedge() const;}{}

\ccMethod{Holes_iterator  holes_begin();}{a begin iterator of the holes container.}
\ccGlue
\ccMethod{Holes_iterator  holes_end();}{a past-the-end iterator of the holes container.}
\ccMethod{Holes_const_iterator  holes_begin();}{}
\ccGlue
\ccMethod{Holes_const_iterator  holes_end();}{}

\ccModifiers

\ccMethod{void set_halfedge(void* h); }{sets incident halfedge.}

\ccMethod{void add_hole(void* h);}{adds \ccStyle{h} to the holes container.}
\ccMethod{void erase_hole(Holes_iterator hit);}{removes the halfedge referenced by \ccStyle{hit} from the holes container (the halfedge itself is not erased
from the DCEL).}
\ccMethod{void erase_holes(Holes_iterator first, Holes_iterator last);}{removes
   the halfedges in the range [\ccStyle{first},\ccStyle{last}) from the holes 
   container.}

\end{ccRefClass}
