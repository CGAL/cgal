% +------------------------------------------------------------------------+
% | Reference manual page: Planar_map.tex (Planar_map)
% +------------------------------------------------------------------------+
% | 
% | Package: pm (Planar_map_2)
% | 
% +------------------------------------------------------------------------+

\ccRefPageBegin

%%RefPage: end of header, begin of main body
% +------------------------------------------------------------------------+


% +========================================================================+
%   Planar Map
% +========================================================================+
\begin{ccRefClass}{Planar_map_2<Dcel,Traits>}
\label{PM_sec:pm}
    
\ccDefinition
    An object $pm$ of the class
    \ccClassTemplateName\ is the planar subdivision
    induced by a set of $x$-monotone curves %that do not intersect in
    %their interiors.
    such that no curve intersects the interior of any other curve.
    The available traits and dcel classes will be described
    later.

\ccInclude{CGAL/Planar_map_2.h}

\ccInheritsFrom
    \ccc{CGAL::Topological_map<Dcel>} 

    The modifying functions \ccc{insert_in_face_interior},
    \ccc{insert_from_vertex}, \ccc{insert_at_vertices}, \ccc{split_edge},
    \ccc{merge_edge} and \ccc{remove_edge}
    overwrite the inherited functions and make use
    of the geometric information of the planar map.

\ccThree{Planar_map_2<Dcel,Traits> :: Traits}{}{\hspace*{6cm}}
\ccThreeToTwo

\ccTypes

%  \ccTypedef{typedef Planar_map_2<Dcel,Traits> Self;}{planar map class}
%
  \ccNestedType{Traits}{traits class.}
  \ccGlue
  \ccNestedType{Dcel}{DCEL class.}

  \ccHtmlNoLinksFrom{
  The \ccc{Vertex}, \ccc{Halfedge} and \ccc{Face}
  types of the planar map are defined as part of the planar map DCEL
  While the face of planar map is a face of the topological map, the Vertex and Halfedge have additional
  functionality. A Vertex has the additional \ccc{point} and \ccc{set_point} operations. A Halfedge has
  the additional \ccc{curve} and \ccc{set_curve} operations. 
  } % ccHtmlNoLinksFrom
  See concepts \ccc{PlanarMapDcel_2}\lcTex{ (\ccRefPage{PlanarMapDcel_2})},
  \ccc{PlanarMapDcelVertex_2} \lcTex{(\ccRefPage{PlanarMapDcelVertex_2})}
  and \ccc{PlanarMapDcelHalfedge_2} \lcTex{(\ccRefPage{PlanarMapDcelHalfedge_2})}.
%\ccc{Dcel::Vertex}\\
%\ccc{Dcel::Halfedge}\\

  \ccHtmlNoLinksFrom{
  \ccNestedType{Vertex}{represents a vertex of the planar map.}}
  \ccGlue
  \ccHtmlNoLinksFrom{
  \ccNestedType{Halfedge}{represents a halfedge of the topological map.}}
  \ccGlue
  \ccHtmlNoLinksFrom{
  \ccNestedType{Face}{represents a face of the topological map.}}


  \ccTypedef{typedef typename Traits::X_curve X_curve;}{a curve of the planar map.}
  \ccGlue
  \ccTypedef{typedef typename Traits::Point Point;}{a point of the planar map.}


%    \ccNestedType{Vertex}{Represents a vertex of the planar-map.}
%    \ccNestedType{Halfedge}{Represents an half-edge of the planar-map.}
%    \ccNestedType{Face}{Represents a face of the planar-map.}
    
%    \ccNestedType{Vertex_iterator}{A bidirectional iterator over the
%       vertices of the planar-map. Its value-type is
%       \ccStyle{Vertex}.}
%     
%    \ccNestedType{Halfedge_iterator}{A bidirectional iterator over the
%       halfedges of the planar-map.  Its value-type is \ccStyle{Halfedge}.}
%    
%    \ccNestedType{Face_iterator}{A bidirectional iterator over the
%       faces of the planar-map.  Its value-type is \ccStyle{Face}.}
%    
%    \ccNestedType{Ccb_halfedge_circulator}{A forward circulator over the
%       edges of a CCB (connected components of the boundary).  Its
%       value-type is \ccStyle{Halfedge}.}
%    
%    \ccNestedType{Halfedge_around_vertex_circulator}{A forward circulator over
%       the half-edges which have  the vertex as their source .  
%       The half-edges are traversed in
%       their clockwise order around the vertex. Its value-type is
%       \ccStyle{Halfedge}.}
%    
%    \ccNestedType{Holes_iterator}{A bidirectional iterator to
%       traverse all the holes ( i.e., inner CCBs ) of a face 
%       (\ccStyle{Holes_iterator++}{} is the next hole in the face).
%       Its value type is
%       \ccStyle{Ccb_halfedge_circulator}.}
    
%\def\ccLongParamLayout{\ccTrue}
\ccConstants
    \ccEnum{enum Locate_type {VERTEX = 1, EDGE, FACE, UNBOUNDED_VERTEX, UNBOUNDED_EDGE, UNBOUNDED_FACE};}    
    {an enumeration that specifies the result of \ccc{point location} 
       and \ccc{ray shooting} operations.}
%\def\ccLongParamLayout{\ccFalse}

\ccCreation
    \ccCreationVariable{pm}
    
    \ccConstructor{Planar_map<Dcel,Traits>();} {constructs an
       ``empty map'' containing one unbounded face, which corresponds to the
       whole plane.}
    
    \ccConstructor{Planar_map<Dcel,Traits>(const Self& pm);}{copy constructor;}
    
    \ccc{Self} here is an abbreviation for the \ccRefName\ type.

    %\ccConstructor{Planar_map<I>( list<X_curve> &l
%       );} {construct the planar-map of the curves  of \ccStyle{l}
%       \ccPrecond{The curves are $x$-monotone} 
%       \ccPrecond{No pair of curves of $l$ intersect in their interiors} 
%       }
        
%    \ccStyle{template <class InputIterator>} \\
%    \ccConstructor{ Planar_map<I>( InputIterator begin, 
%                           InputIterator end );} 
%                {construct the planar-map of the curves from a 
%                container iterated by
%                \ccStyle{InputIterator} from  \ccStyle{begin} to \ccStyle{end}
%                \ccPrecond{The value type of \ccStyle{InputIterator}
%                   is \ccStyle{X_curve_2} objects}
%                \ccPrecond{The curves are $x$-monotone} 
%                \ccPrecond{No pair of curves intersect in their interiors} 
%      }

%    \ccConstructor{Planar_map<I>( Planar_map<I>
%       &map, I::Info_face &i);} {construct the planar-map of all the faces of
%       \ccStyle{map} with attached information equal to \ccStyle{i}.}

%\newpage

\begin{ccAdvanced}

\ccHeading{Point Location and Bounding Box}

As described in the introduction, the planar map users can define
which algorithm to use in the point location queries.  This is done
by passing an instance of some point location class instance to the map in
this constructor. The point location class should be a model of the
\ccc{PlanarMapPointLocation_2} concept. The randomized trapezoidal
decomposition algorithm is the one used by default in our
implementation.  However, the users can choose to use the naive or
``walk'' algorithms (trading time for memory efficiency) or can
implement their own point location algorithm. This can be done with a
class derived from the \ccc{Point_location_base<Planar_map>}. The concept \ccc{PlanarMapPointLocation_2} lists the set of requirements 
(\ccRefPage{PlanarMapPointLocation_2}).

\ccConstructor{Planar_map<Dcel,Traits>(Pm_point_location_base<Self> *pl_ptr);}{}
  
%A more advanced strategies 
%\ccc{Pm_dynamic_closed_bounding_box},
%\ccc{Pm_dynamic_open_bounding_box},
%\ccc{Pm_static_closed_bounding_box} and 
%\ccc{Pm_static_open_bounding_box} 
%maintain the planar map along with a bounding box.
%This bounding box is supposed to be large enough (in the dynamic case) 
%to bound all the finite curves inserted into the map, and to intersect 
%all the unbounded curves.
%In the static case, the bounding box is as indicated by its name, static.
%In any case, each curve is represented as its intersection with the bounding box, 
%allowing the planar map to deal only with bounded curves. 


The planar map user can also determine the bounding box strategy
This is done with a {\em bounding box class} passed to the map in the following constructors. 
The \ccc{Pm_dynamic_open_bounding_box} strategy is the one used by default by the library. If only the default strategy is to be used, the user can ignore the following constructors.

\ccConstructor{Planar_map<Dcel,Traits>(Pm_bounding_box_base<Self> *bb_ptr);}{}
\ccConstructor{Planar_map<Dcel,Traits>(Pm_point_location_base<Self> * pl_ptr,
        Pm_bounding_box_base<Self> *bb_ptr);}{}
  

\ccc{Self} here is an abbreviation for the \ccRefName\ type.

\end{ccAdvanced}

\begin{ccAdvanced}
\ccHeading{Reading Planar map}

\ccMethod{bool read (istream &in);}
{reading ccRefName\ from a given input stream.
   The input stream should support the extractor operator (\ccc{>>}) for the \ccStyle{Point} 
   and \ccStyle{X_curve} types of \ccc{Planar_map}. } 

\ccStyle{template <class Scanner>}
\ccMethod{bool read (istream &in, Scanner& scanner);}
{reading \ccRefName\ from a given input stream when taking the scanner class as a parameter.
   The input stream should support the extractor operator (\ccc{>>}) for the \ccStyle{Point} 
   and \ccStyle{X_curve} types of \ccc{Planar_map} }

\end{ccAdvanced}

\ccThree{Halfedge_handle}{}{\hspace*{11cm}}

%\ccPredicates
%\ccHeading{Query Functions}
\ccQueryFunctions

The following two functions are query functions. Their time complexity
depends on the point location strategy used.

\ccMethod{Halfedge_handle locate(const Point& p , 
   Locate_type& lt );}{
   \ccHtmlNoLinksFrom{computes the location in \ccStyle{pm}
   where  \ccStyle{p} lies;  
   If \ccStyle{lt} returns VERTEX then \ccStyle{p} lies on the vertex which 
   is the target of the returned \ccStyle{Halfedge}. If \ccStyle{lt} returns 
   EDGE then \ccStyle{p} lies on the returned \ccStyle{Halfedge}. If 
   \ccStyle{lt} returns FACE then \ccStyle{p} lies on the face which is on 
   the left of the returned \ccStyle{Halfedge} . If \ccStyle{lt} returns 
   UNBOUNDED\_FACE then \ccStyle{p} lies on the unbounded face, and 
   the returned \ccStyle{Halfedge} is on the boundary of 
   a hole in the unbounded face.}
   }
    
%\ccMethod{ Halfedge_handle vertical_ray_shoot(const typename Traits::Point& p, 
\ccMethod{ Halfedge_handle vertical_ray_shoot(const Point& p, 
   Locate_type& lt, bool up_direction );} {if
   \ccStyle{up_direction} is \ccStyle{true} (respectively,
   \ccStyle{false}) returns the first edge of \ccStyle{pm} that
   intersects the upward (respectively, downward) vertical ray emanating
   from \ccStyle{p}. If several edges intersect the vertical ray
   in the same (end)point $q$, the function returns the
   first halfedge pointing at $q$, that is encountered when moving clockwise 
   from \ccTexHtml{$\vec{pq}$}{$pq$} around $q$ . 
   In that case the value of \ccStyle {lt} will
   be VERTEX . If the ray does not intersect any edge, the value 
   of \ccStyle {lt} 
   will be UNBOUNDED\_FACE and the 
  \ccHtmlNoLinksFrom{\ccStyle{Halfedge}} returned will be null
   valued. Otherwise the value of \ccStyle {lt} will be EDGE.
   \ccPrecond{\ccStyle{p} lies in the interior of a face of
      \ccStyle{pm}, i.e., not on an edge or on a vertex.}
   \ccPostcond{the returned edge belongs to one of the CCB's of the face
      in which \ccStyle{p} is found, null valued if none is 
      found. } } 

\begin{ccAdvanced}
    For some applications the users may want to have direct access to
    the point location strategy (e.g., query the default strategy about
    the state of its internal search structure). For this we have implemented 
    the following function, note that the returned pointer is \ccc{const} so
    the users cannot change the internal state.

    \ccMethod{const Pm_point_location_base<Self>* point_location();}{returns a
       const pointer to the point location strategy of the map.}

\ccc{Self} here is an abbreviation for the \ccRefName\ type.
\end{ccAdvanced}

\ccModifiers

\ccMethod{Halfedge_handle insert( const X_curve &cv );} {inserts the curve
   cv into the map.
   \ccPrecond{No curve \ccStyle{cv1} of \ccStyle{pm}
      intersects \ccStyle{cv} in the interiors of
      \ccStyle{cv} and \ccStyle{cv1}. }
   \ccPrecond{\ccc{cv} is not equivalent to a point. } 
   }                

\begin{ccAdvanced}

\ccHeading{Specialized Insertion Functions}

The following functions enable the usage of information about the map which 
was acquired beforehand, to save time in insertions. It is recommended to use
these functions with the naive point location strategy.
% (using
%them with the default point location strategy is not helpful since
%the time is spent in any case on the updating of the internal point location
%structure). 
           
    \ccMethod{Halfedge_handle insert_at_vertices( const X_curve & cv,
       Vertex_handle v1, Vertex_handle v2 );}{inserts \ccStyle{cv} as a new edge
       between \ccStyle{v1} and \ccStyle{v2}, where \ccStyle{v1} and
       \ccStyle{v2} are vertices of the map. \ccStyle{v1}
          and \ccStyle{v2} represent the source and target of
          the returned halfedge. 
       \ccPrecond{\ccc{v1->point()} and \ccc{v2->point()} are the two endpoints of
          \ccc{cv}.} 
       \ccPrecond{\ccc{cv} is not equivalent to a point. }}
    
    \ccMethod{Halfedge_handle insert_from_vertex( const X_curve & cv,
       Vertex_handle v1, bool source);}{inserts a new edge for
       which one endpoint, \ccStyle{v1}, is already in the
       map. The returned halfedge is the one that has \ccc{v1} as it's source.
       If \ccStyle{source} is \ccStyle{true} (respectively,
       \ccStyle{false}) 
       %then the source (respectively, target) of
       %the returned halfedge is \ccStyle{v1} and its target (respectively, source)
       %holds the point which is \ccc{cv's} target (respectively, source). 
       then the \ccc{target} of the returned halfedge holds the point which is
       \ccc{cv's} target (respectively, source).
       \ccPrecond{the second endpoint of \ccStyle{cv} is not 
          in the map.} 
       \ccPrecond{\ccc{cv} is not equivalent to a point. }}
    
    \ccMethod{Halfedge_handle insert_in_face_interior( const X_curve &
       cv, Face_handle f);}
       {inserts \ccStyle{cv} as a new inner component of \ccc{f}. The
       returned halfedge shares the same direction as the curve
       \ccc{cv}. \ccPrecond{ \ccStyle{cv} is contained completely in
       the interior of \ccStyle{f} (no vertex of \ccStyle{cv} meets
       any vertex on the map).}  \ccPrecond{\ccc{cv} is not equivalent
       to a point. }}

\end{ccAdvanced}              

\ccMethod{Halfedge_handle split_edge (
   Halfedge_handle e, const X_curve& c1, 
   const X_curve& c2);}
{splits the edge \ccStyle{e} into \ccStyle{e1} and  \ccStyle{e2} , and
   add a vertex in the splitting point. If the source point of \ccStyle{e}
   is identical to the source of the curve \ccStyle{c1} then \ccStyle{c1}
   will be assigned to \ccStyle{e1} and \ccStyle{c2} to \ccStyle{e2}. 
   Otherwise, the opposite will take place. The returned halfedge will
   be \ccStyle{e1} where \ccc{e2} is \ccc{e1->next_halfedge()}.
   \ccPrecond{the preconditions of 
      \ccStyle{Topological_map<Dcel>::split_edge}.}
   \ccPrecond{the target of the curve $c1$  is identical to the source of 
      the curve $c2$.}
   \ccPrecond{the source of the curve $c1$  is identical to the source point 
      of \ccStyle{e} and the target of the curve $c2$ is identical to the
      target point of \ccStyle{e}, or 
      the source of the curve $c1$  is identical to the target point 
      of \ccStyle{e} and the target of the curve $c2$ is identical to the
      source point of \ccStyle{e}.}
   }

\ccMethod{Halfedge_handle merge_edge(Halfedge_handle e1, Halfedge_handle e2, 
   const X_curve& cv);}
{merges the edges \ccStyle{e1} and \ccStyle{e2} into one edge which will hold 
   the curve \ccStyle{cv}. The return value is the halfedge with the same 
   source vertex that \ccc{e1} had and the same target \ccc{e2} had.
   \ccPrecond{the preconditions of 
      \ccStyle{Topological_map<Dcel>::merge_edge}.}
   \ccPrecond{the source of the curve $cv$  is identical to the source point 
      of \ccStyle{e1} and the target of the curve $cv$ is identical to the
      target point of \ccStyle{e2}, or 
      the target of the curve $cv$  is identical to the source point 
      of \ccStyle{e1} and the source of the curve $cv$ is identical to the
      target point of \ccStyle{e2}.}
}

\ccMethod{Face_handle remove_edge(Halfedge_handle e);}
{removes the edge \ccStyle{e} from \ccVar.
   If the operation causes two faces to merge, the merged face is returned.
   Otherwise, the face to which the edge was incident is returned.}

%already in the topological map
%    \ccMethod
%    {Face  unbounded_face() const;}
%    {return the unbounded face of the planar-map}



%        \ccMethod { bool is_valid(); } { check the validity of all the
%           features of \ccVar . Check for each vertex \ccStyle{v}
%           whether all incident half-edges (by performing
%           \ccStyle{get_twin().get_next()})have \ccStyle{v} as their
%           origin. Check that each half-edge \ccStyle{e} hold the right
%           curve -- \ccStyle{Traits.curve_source(e.curve())==e.source().point()}
%           and \ccStyle{Traits.curve_target(e.curve())== e.target().point()}. 
%           Check that all the edges on the boundary
%           of each face \ccStyle{f} point to \ccStyle{f} as their
%           face} 

% @@@ \end{ccClas3s}
\ccSeeAlso
     \ccc{Topological_map<Dcel>}\lcTex{ 
     (\ccRefPage{CGAL::Topological_map<Dcel>})}\\
     \ccc{PlanarMapDcel_2}\lcTex{ 
     (\ccRefPage{PlanarMapDcel_2})}

\end{ccRefClass}

% FILL IN THE REST OF THE FILES
% +------------------------------------------------------------------------+
%%RefPage: end of main body, begin of footer
\ccRefPageEnd
% EOF
% +------------------------------------------------------------------------+
