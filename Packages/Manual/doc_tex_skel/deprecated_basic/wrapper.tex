\documentclass{book}

\usepackage{cprog}
\usepackage{cc_manual}
\usepackage{cc_manual_index}
\usepackage{makeidx}
\usepackage{latex_converter}
\usepackage{amssymb}
\usepackage{graphicx}
\usepackage{path}
\usepackage{ipe}
\usepackage{alltt}
\usepackage{pslatex}
\usepackage{psfrag}
\usepackage{rotating}

% page dimensions
% ---------------
\textwidth 15.6cm 
\textheight 23 cm
\topmargin -14mm
\evensidemargin 3mm 
\oddsidemargin 3mm

% default column layout
% ---------------------
\newcommand{\cgalColumnLayout}{\ccTexHtml{%
    \ccSetThreeColumns{CGAL_Oriented_side}{}{\hspace*{8.5cm}}
    \ccPropagateThreeToTwoColumns}{}}

\newcommand{\cgalrelease}{2.2}

\ccDefGlobalScope{CGAL::}
%%%%%%%
%   for the new reference manual style, uncomment the following commands
%
%\marginparsep10mm
%\marginparwidth15mm
%\gdef\ccNewRefManualStyle{\ccTrue}
%
%% The tab marker are aligned with the top of the main text. To align
%% them with the page header, the following length definition can be used.
%\setlength{\ccRefTabLift}{12.5mm}
%
%%%%%%%
%%%%%%%
%
% you may also want to use these commands, but be careful to check that
% other redefinitions of ccRefPageBegin and End don't cause strange problems
% (e.g. if ccAdvanced is used to bracket a ref page)
%\def\ccRefPageBegin{\ccParDims\cgalColumnLayout}
%\def\ccRefPageEnd{\ccParDims\cgalColumnLayout}
%
%%%%%%%

\makeindex

\sloppy

\input{ionly}
\lcTex{\batchmode}

\begin{document}

% =============================================================================
% The CGAL Reference Manual
% Chapter: Geometric Optimisation
% -----------------------------------------------------------------------------
% file   : doc_tex/basic/Optimisation/Optimisation_ref/main.tex
% package: Optimisation_doc
% author : Sven Sch�nherr <sven@inf.ethz.ch>
% -----------------------------------------------------------------------------
% $Revision$
% $Date$
% =============================================================================

\section{Reference Pages}

% =============================================================================
% The CGAL Reference Manual
% Chapter: Geometric Optimisation
% -----------------------------------------------------------------------------
% file   : doc_tex/basic/Optimisation/Optimisation_ref/reference_part.tex
% package: Optimisation_doc
% author : Sven Sch�nherr <sven@inf.ethz.ch>
% -----------------------------------------------------------------------------
% $Revision$
% $Date$
% =============================================================================

\newcommand{\inputOpt}[1]{\input{Optimisation_ref/#1.tex}}

\newcommand{\linebreakByHand}{\ccTexHtml{\linebreak[4]}{}}
\newcommand{  \newlineByHand}{\ccTexHtml{\\}{}}

% cross references
\index{minimum enclosing|see{{smallest enclosing}}}
\index{minimum spanning|see{{smallest enclosing}}}
\index{concentric spheres|see{{annulus}}}

% -----------------------------------------------------------------------------
\section*{Introduction}

This chapter describes concepts, classes, and functions for solving
geometric optimisation problems. They are divided into four categories.

\paragraph{Bounding Areas and Volumes.}
Smallest enclosing circle and ellipse (2D), smallest enclosing rectangle,
parallelogram, and strip (2D), rectangular $p$-center (2D), smallest
enclosing sphere and annulus (dD).

\paragraph{Inscribed Areas.}
Maximum area and perimeter inscribed $k$-gon (2D), extremal inscribed
$k$-gon (2D).

\paragraph{Optimal Distances.}
All furthest neigbors (2D), width of point set (3D), polytope distance (dD).

\paragraph{Advanced Techniques.}
Monotone and sorted matrix search.

\section*{Assertions}
The optimisation code uses infix \ccc{OPTIMISATION} in the assertions,
e.g.\ defining the compiler flag
\ccc{CGAL_OPTIMISATION_NO_PRECONDITIONS} switches precondition
checking off, cf.~\cgalReferToAssertions


% -----------------------------------------------------------------------------
\subsection*{Bounding Areas and Volumes}

\ccRefIdfierPage{CGAL::Min_circle_2<Traits>}\\[1ex]
\ccRefIdfierPage{CGAL::Min_circle_2_traits_2<K>}\\[1ex]
\ccRefConceptPage{MinCircle2Traits}

\smallskip

\ccRefIdfierPage{CGAL::Min_ellipse_2<Traits>}\\[1ex]
\ccRefIdfierPage{CGAL::Min_ellipse_2_traits_2<K>}\\[1ex]
\ccRefConceptPage{MinEllipse2Traits}

\smallskip

\ccRefIdfierPage{CGAL::min_rectangle_2}\\
\ccRefIdfierPage{CGAL::min_parallelogram_2}\\
\ccRefIdfierPage{CGAL::min_strip_2}\\[1ex]
\ccRefIdfierPage{CGAL::Min_quadrilateral_default_traits_2<R>}\\[1ex]
\ccRefConceptPage{MinQuadrilateralTraits_2}

\smallskip

\ccRefIdfierPage{CGAL::rectangular_p_center_2}\\[1ex]
\ccRefIdfierPage{CGAL::Rectangular_p_center_default_traits_2<R>}\\[1ex]
\ccRefConceptPage{RectangularPCenterTraits_2}

\bigskip

\ccRefIdfierPage{CGAL::Min_sphere_d<Traits>}\\
\ccRefIdfierPage{CGAL::Min_annulus_d<Traits>}\\[1ex]
\ccRefIdfierPage{CGAL::Optimisation_d_traits_2<K,ET,NT>}\\
\ccRefIdfierPage{CGAL::Optimisation_d_traits_3<K,ET,NT>}\\
\ccRefIdfierPage{CGAL::Optimisation_d_traits_d<K,ET,NT>}\\[1ex]
\ccRefConceptPage{OptimisationDTraits}

% -----------------------------------------------------------------------------
\subsection*{Inscribed Areas}

\ccRefIdfierPage{CGAL::maximum_area_inscribed_k_gon_2}\\
\ccRefIdfierPage{CGAL::maximum_perimeter_inscribed_k_gon_2}\\
\ccRefIdfierPage{CGAL::extremal_polygon_2}\\[1ex]
\ccRefIdfierPage{CGAL::Extremal_polygon_area_traits_2<K>}\\
\ccRefIdfierPage{CGAL::Extremal_polygon_perimeter_traits_2<K>}\\[1ex]
\ccRefConceptPage{ExtremalPolygonTraits_2}

% -----------------------------------------------------------------------------
\subsection*{Optimal Distances}

%\ccRefIdfierPage{CGAL::width_2}%\\[1ex]
%\ccRefIdfierPage{CGAL::Min_quadrilateral_default_traits_2<K>}\\[1ex]
%\ccRefConceptPage{MinQuadrilateralTraits_2}

%\smallskip

\ccRefIdfierPage{CGAL::all_furthest_neighbors_2}\\[1ex]
%\ccRefIdfierPage{CGAL::All_furthest_neighbors_default_traits_2<R>}\\[1ex]
\ccRefConceptPage{AllFurthestNeighborsTraits_2}

\smallskip

\ccRefIdfierPage{CGAL::Width_3<Traits>}\\[1ex]
\ccRefIdfierPage{CGAL::Width_default_traits_3<K>}\\[1ex]
\ccRefConceptPage{WidthTraits_3}

\smallskip

\ccRefIdfierPage{CGAL::Polytope_distance_d<Traits>}\\[1ex]
\ccRefIdfierPage{CGAL::Optimisation_d_traits_2<K,ET,NT>}\\
\ccRefIdfierPage{CGAL::Optimisation_d_traits_3<K,ET,NT>}\\
\ccRefIdfierPage{CGAL::Optimisation_d_traits_d<K,ET,NT>}\\[1ex]
\ccRefConceptPage{OptimisationDTraits}

% -----------------------------------------------------------------------------
\subsection*{Advanced Techniques}

\ccRefIdfierPage{CGAL::monotone_matrix_search}\\[1ex]
\ccRefIdfierPage{CGAL::Dynamic_matrix<M>}\\[1ex]
\ccRefConceptPage{MonotoneMatrixSearchTraits}\\
\ccRefConceptPage{BasicMatrix}

\smallskip

\ccRefIdfierPage{CGAL::sorted_matrix_search}\\[1ex]
\ccRefIdfierPage{CGAL::Sorted_matrix_search_traits_adaptor<F,M>}\\[1ex]
\ccRefConceptPage{SortedMatrixSearchTraits}

\smallskip

% =============================================================================

% Bounding Areas and Volumes

\inputOpt{main_Min_circle_2}
\inputOpt{main_Min_ellipse_2}
\inputOpt{main_Min_quadrilateral_2}
\inputOpt{main_Rectangular_p_centers}

\inputOpt{main_Min_sphere_d}
\inputOpt{main_Min_annulus_d}
\inputOpt{main_Optimisation_d_traits}

% Inscribed Areas

\inputOpt{main_Extremal_polygons}

% Optimal Distances

\inputOpt{main_All_furthest_neighbors}

\inputOpt{main_Width_3}

\inputOpt{main_Polytope_distance_d}


% Advanced Techniques

\inputOpt{main_Matrix_search}


% ===== EOF ===================================================================


% ===== EOF ===================================================================

\pagestyle{plain}
\pagenumbering{roman}
\setcounter{page}{0}
%% =============================================================================
% The CGAL Reference Manual
% Chapter: Geometric Optimisation
% -----------------------------------------------------------------------------
% file   : doc_tex/basic/Optimisation/Optimisation_ref/main.tex
% package: Optimisation_doc
% author : Sven Sch�nherr <sven@inf.ethz.ch>
% -----------------------------------------------------------------------------
% $Revision$
% $Date$
% =============================================================================

\section{Reference Pages}

% =============================================================================
% The CGAL Reference Manual
% Chapter: Geometric Optimisation
% -----------------------------------------------------------------------------
% file   : doc_tex/basic/Optimisation/Optimisation_ref/reference_part.tex
% package: Optimisation_doc
% author : Sven Sch�nherr <sven@inf.ethz.ch>
% -----------------------------------------------------------------------------
% $Revision$
% $Date$
% =============================================================================

\newcommand{\inputOpt}[1]{\input{Optimisation_ref/#1.tex}}

\newcommand{\linebreakByHand}{\ccTexHtml{\linebreak[4]}{}}
\newcommand{  \newlineByHand}{\ccTexHtml{\\}{}}

% cross references
\index{minimum enclosing|see{{smallest enclosing}}}
\index{minimum spanning|see{{smallest enclosing}}}
\index{concentric spheres|see{{annulus}}}

% -----------------------------------------------------------------------------
\section*{Introduction}

This chapter describes concepts, classes, and functions for solving
geometric optimisation problems. They are divided into four categories.

\paragraph{Bounding Areas and Volumes.}
Smallest enclosing circle and ellipse (2D), smallest enclosing rectangle,
parallelogram, and strip (2D), rectangular $p$-center (2D), smallest
enclosing sphere and annulus (dD).

\paragraph{Inscribed Areas.}
Maximum area and perimeter inscribed $k$-gon (2D), extremal inscribed
$k$-gon (2D).

\paragraph{Optimal Distances.}
All furthest neigbors (2D), width of point set (3D), polytope distance (dD).

\paragraph{Advanced Techniques.}
Monotone and sorted matrix search.

\section*{Assertions}
The optimisation code uses infix \ccc{OPTIMISATION} in the assertions,
e.g.\ defining the compiler flag
\ccc{CGAL_OPTIMISATION_NO_PRECONDITIONS} switches precondition
checking off, cf.~\cgalReferToAssertions


% -----------------------------------------------------------------------------
\subsection*{Bounding Areas and Volumes}

\ccRefIdfierPage{CGAL::Min_circle_2<Traits>}\\[1ex]
\ccRefIdfierPage{CGAL::Min_circle_2_traits_2<K>}\\[1ex]
\ccRefConceptPage{MinCircle2Traits}

\smallskip

\ccRefIdfierPage{CGAL::Min_ellipse_2<Traits>}\\[1ex]
\ccRefIdfierPage{CGAL::Min_ellipse_2_traits_2<K>}\\[1ex]
\ccRefConceptPage{MinEllipse2Traits}

\smallskip

\ccRefIdfierPage{CGAL::min_rectangle_2}\\
\ccRefIdfierPage{CGAL::min_parallelogram_2}\\
\ccRefIdfierPage{CGAL::min_strip_2}\\[1ex]
\ccRefIdfierPage{CGAL::Min_quadrilateral_default_traits_2<R>}\\[1ex]
\ccRefConceptPage{MinQuadrilateralTraits_2}

\smallskip

\ccRefIdfierPage{CGAL::rectangular_p_center_2}\\[1ex]
\ccRefIdfierPage{CGAL::Rectangular_p_center_default_traits_2<R>}\\[1ex]
\ccRefConceptPage{RectangularPCenterTraits_2}

\bigskip

\ccRefIdfierPage{CGAL::Min_sphere_d<Traits>}\\
\ccRefIdfierPage{CGAL::Min_annulus_d<Traits>}\\[1ex]
\ccRefIdfierPage{CGAL::Optimisation_d_traits_2<K,ET,NT>}\\
\ccRefIdfierPage{CGAL::Optimisation_d_traits_3<K,ET,NT>}\\
\ccRefIdfierPage{CGAL::Optimisation_d_traits_d<K,ET,NT>}\\[1ex]
\ccRefConceptPage{OptimisationDTraits}

% -----------------------------------------------------------------------------
\subsection*{Inscribed Areas}

\ccRefIdfierPage{CGAL::maximum_area_inscribed_k_gon_2}\\
\ccRefIdfierPage{CGAL::maximum_perimeter_inscribed_k_gon_2}\\
\ccRefIdfierPage{CGAL::extremal_polygon_2}\\[1ex]
\ccRefIdfierPage{CGAL::Extremal_polygon_area_traits_2<K>}\\
\ccRefIdfierPage{CGAL::Extremal_polygon_perimeter_traits_2<K>}\\[1ex]
\ccRefConceptPage{ExtremalPolygonTraits_2}

% -----------------------------------------------------------------------------
\subsection*{Optimal Distances}

%\ccRefIdfierPage{CGAL::width_2}%\\[1ex]
%\ccRefIdfierPage{CGAL::Min_quadrilateral_default_traits_2<K>}\\[1ex]
%\ccRefConceptPage{MinQuadrilateralTraits_2}

%\smallskip

\ccRefIdfierPage{CGAL::all_furthest_neighbors_2}\\[1ex]
%\ccRefIdfierPage{CGAL::All_furthest_neighbors_default_traits_2<R>}\\[1ex]
\ccRefConceptPage{AllFurthestNeighborsTraits_2}

\smallskip

\ccRefIdfierPage{CGAL::Width_3<Traits>}\\[1ex]
\ccRefIdfierPage{CGAL::Width_default_traits_3<K>}\\[1ex]
\ccRefConceptPage{WidthTraits_3}

\smallskip

\ccRefIdfierPage{CGAL::Polytope_distance_d<Traits>}\\[1ex]
\ccRefIdfierPage{CGAL::Optimisation_d_traits_2<K,ET,NT>}\\
\ccRefIdfierPage{CGAL::Optimisation_d_traits_3<K,ET,NT>}\\
\ccRefIdfierPage{CGAL::Optimisation_d_traits_d<K,ET,NT>}\\[1ex]
\ccRefConceptPage{OptimisationDTraits}

% -----------------------------------------------------------------------------
\subsection*{Advanced Techniques}

\ccRefIdfierPage{CGAL::monotone_matrix_search}\\[1ex]
\ccRefIdfierPage{CGAL::Dynamic_matrix<M>}\\[1ex]
\ccRefConceptPage{MonotoneMatrixSearchTraits}\\
\ccRefConceptPage{BasicMatrix}

\smallskip

\ccRefIdfierPage{CGAL::sorted_matrix_search}\\[1ex]
\ccRefIdfierPage{CGAL::Sorted_matrix_search_traits_adaptor<F,M>}\\[1ex]
\ccRefConceptPage{SortedMatrixSearchTraits}

\smallskip

% =============================================================================

% Bounding Areas and Volumes

\inputOpt{main_Min_circle_2}
\inputOpt{main_Min_ellipse_2}
\inputOpt{main_Min_quadrilateral_2}
\inputOpt{main_Rectangular_p_centers}

\inputOpt{main_Min_sphere_d}
\inputOpt{main_Min_annulus_d}
\inputOpt{main_Optimisation_d_traits}

% Inscribed Areas

\inputOpt{main_Extremal_polygons}

% Optimal Distances

\inputOpt{main_All_furthest_neighbors}

\inputOpt{main_Width_3}

\inputOpt{main_Polytope_distance_d}


% Advanced Techniques

\inputOpt{main_Matrix_search}


% ===== EOF ===================================================================


% ===== EOF ===================================================================

%\cleardoublepage
\tableofcontents
\cleardoublepage
\pagenumbering{arabic}
 
\input{AllMains}

\bibliographystyle{alpha}
\bibliography{../cgal-manual,geom}


\ccTexHtml{\printindex}{}

\end{document}
