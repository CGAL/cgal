\newcommand{\reals}{{\rm I\!\hspace{-0.025em} R}}
\newcommand{\calC}{{\cal C}}
\newcommand{\calA}{{\cal A}}
\newcommand{\eps}{{\varepsilon}}
\newcommand{\dcel}{{\sc Dcel}}
\newcommand{\naive}{na\"{\i}ve}
\newcommand{\kdtree}{{\sc Kd}-tree}

% ===============================================================
\section{Introduction}
\label{bobs_sec:intro}
% ===============================================================
%
This package consists of the implementation of Boolean set-operations
on point sets bounded by $x$-monotone curves in 2-dimensional
Euclidean space. In particular, it contains the implementation of
{\em regularized} Boolean set-operations, intersection predicates, and
point containment predicates.
% and Minkowski sum computations.

A regularized Boolean set-operation $\mbox{op}^*$ can be obtained by
first taking the interior of the resultant point set of an {\em ordinary}
Boolean set-operation $(P\ \mbox{op}\ Q)$ and then by taking the
closure~\cite{cgal:h-sm-04}. That is
$P\ \mbox{op}^*\ Q = \mbox{closure}(\mbox{interior} (P\ \mbox{op}\ Q))$.
Regularized Boolean set-operations appear in Constructive Solid
Geometry (CSG), because regular sets are closed under regularized
Boolean set-operations, and because regularization eliminates lower
dimensional features, thus simplifying and restricting the
representation to physically meaningful solids.
% A simplified representation does not have to store selection marks;
% they are implicitly always set for vertices and edges, and for faces
% they are deduced from a suitably chosen orientation condition on their
% boundaries.
Ordinary Boolean set-operations, which distinguish between the
interior and the boundary of a polygon, are not implemented yet. These
operations operate on, and result with, general polygons, which
include isolated curves and points, and open shapes. The \ccc{Nef_2}
package supports these operation for (linear) polygons; see
Chapter~\ref{chap:nef_2}. Bounding box constructions, extreme vertex
provision, and point containment predicates on a single (linear)
polygon are provided by the \ccc{Polygon_2} class; see
Chapter~\ref{Polygon}.

\begin{wrapfigure}{r}{4cm}
\vspace{-3ex}
\pspicture[](0,0)(2,2)
\psset{unit=1cm,linewidth=1pt}
  \pspolygon*[linecolor=gray](0,2)(0,0)(2,0)
  \pspolygon*[linecolor=gray](2,0)(4,0)(4,2)
  \pspolygon(0,2)(0,0)(2,0)
  \pspolygon(2,0)(4,0)(4,2)
\endpspicture
\caption{A non simple polygon.}
\label{fig:non_simple_polygon}
\end{wrapfigure}
A polygon $P$ is said to be {\em simple} (or Jordan) if the
only points of the plane belonging to two polygon edges of $P$ are the
polygon vertices of $P$. Such a polygon has a well-defined interior
and exterior. Simple polygons are topologically equivalent to
a disk. A polygon in our context must be simple. The cyclic
sequence of alternating polygon edges and polygon vertices is referred
to as the polygon {\em boundary}. A polygon whose boundary contains the
same vertex twice or more is connected and simple but not
necessary strictly simple; see figure~\ref{fig:non_simple_polygon}. We extend
the notion of a polygon to a point set in $\mathrm{E}^2$ that has a
topology of a polygon and its boundary edges must map to $x$-monotone
curves, and refer to it as a {\em general polygon}. We sometimes use
the term {\em polygon} instead of general polygon for simplicity here
after.

In many cases a simple operation that operates on two strictly simple
polygons that have no holes are required. The most simple functions in
the package operate on two polygons and accept them as two separate
parameters. However, even a simple operation, such as the union of two
strictly simple polygons with no holes, may result with a polygon with
holes, see Figure~\ref{fig:simple}~(a). Moreover, a simple operation,
such as the intersection of two strictly simple polygons with no
holes, may result with a set of disjoint polygons; see
Figure~\ref{fig:simple}~(b). Finally, only a polygon with holes can
represent the complement of a polygon without holes. Notice that this
polygon has no outer boundary; see Figure~\ref{fig:simple}~(c).

\begin{figure}[!hbp]
\begin{center}
\begin{tabular}{ccc}
\pspicture[](0,0)(4,2)
\psset{unit=1cm,linewidth=1pt}
  \pspolygon*[linecolor=gray](0,1)(2,0)(1,1)(2,2)
  \pspolygon*[linecolor=gray](3,1)(2,0)(4,1)(2,2)
  \pspolygon(0,1)(2,0)(1,1)(2,2)
  \pspolygon(3,1)(2,0)(4,1)(2,2)
\endpspicture &
\pspicture[](0,0)(5,2)
\psset{unit=1cm,linewidth=1pt}
  \pspolygon*[linecolor=lightgray](0,0)(1.5,1.5)(2.5,0.5)(3.5,1.5)(5,0)
  \pspolygon*[linecolor=lightgray](0,2)(5,2)(3.5,0.5)(2.5,1.5)(1.5,0.5)
  \pspolygon*[linecolor=gray](1,1)(1.5,1.5)(2,1)(1.5,0.5)
  \pspolygon*[linecolor=gray](3,1)(3.5,1.5)(4,1)(3.5,0.5)
  \pspolygon(0,0)(1.5,1.5)(2.5,0.5)(3.5,1.5)(5,0)
  \pspolygon(0,2)(5,2)(3.5,0.5)(2.5,1.5)(1.5,0.5)
\endpspicture &
\pspicture[](0,0)(3,2)
\psset{unit=1cm,linewidth=1pt}
  \pspolygon*[linecolor=gray](0,2)(0,0)(3,0)(3,2)
  \pspolygon*[linecolor=lightgray](1,0.67)(2,0.67)(2,1.34)(1,1.34)
  \pspolygon[linecolor=black](1,0.67)(2,0.67)(2,1.34)(1,1.34)
\endpspicture
\\
(a) & (b) & (c)
\end{tabular}
\caption{Operations on strictly simple polygons. (a) The union of two
strictly simple polygons. (b) The intersection of two strictly simple
polygons. (c) The complement of a strictly simple polygon.} 
\label{fig:simple}
\end{center}
\end{figure}

% ===============================================================
\section{Boolean Set-Operations on General Polygons}
\label{bobs_sec:bops}
% ===============================================================
For two point sets $P$ and $Q$, the (ordinary) Boolean set-operations are
defined as follows:
\begin{description}
\item [Intersection predicate] tests whether the two polygons $P$ and $Q$
  overlap without explicitly computing the overlapping cell (or cells).     
\item[Intersection] computes the intersection $R = P \cap Q$.
\item[Join] computes the union $R = P \cup Q$.
\item [Difference] computes the difference $R = P \setminus Q$ 
\item [Symmetric Difference] computes the symmetric difference
  $P = (P \setminus Q) \cup (Q \setminus P)$
\item[Complement] computes the complement $R = \overline{P}$.
\end{description}

% ===============================================================
\subsection{Regularized Boolean-Set Operations on General Polygons}
\label{bobs_ssec:regularized_bops}
% ===============================================================
A general polygon, which is an operand or a result of a regularized
Boolean set-operation is considered as being a closed point-set (that
consist of its interior and its boundary). The distinction between the
interior and  the exterior of a polygon is determined by the order of
its vertices. The definition of the regularized Boolean set-operations 
is slightly different than the definition of the ordinary as mentioned
above. For example, the intersection predicate tests whether the
interior of two polygons $P$ and $Q$ overlap. Thus, if two polygons
have a finite set of points, or even a boundary curve, in common, they
are concidered non-intersecting.

All functions that implement the regularized Boolean set-operations
are parameterized by a \ccc{GeneralPolygonSetTraits_2} type. The
result is provided through a single general polygon or an output
iterator of a range of general polygons. If  an output iterator is
used, the iterator type is an additional template parameter.

% ---------------------------------------------------------------
\subsection{A Simple Example}
\label{bobs_ssec:simple_example}
% ---------------------------------------------------------------
\begin{wrapfigure}{r}{2cm}
\pspicture[](-1,-1)(1,1)
\psset{unit=1cm,linewidth=1pt}
  \pspolygon*[linecolor=lightgray](-1,1)(0,-1)(1,1)
  \pspolygon*[linecolor=lightgray](-1,-1)(1,-1)(0,1)
  \pspolygon*[linecolor=gray](-0.5,0)(0,-1)(0.5,0)(0,1)
  \pspolygon(-1,1)(0,-1)(1,1)
  \pspolygon(-1,-1)(1,-1)(0,1)
\endpspicture
% \caption{.} 
\label{fig:example}
\vspace{-2cm}
\end{wrapfigure}
The example listed below tests whether two triangles depicted on the
right intersect.
%
% \begin{alltt}
% #include <CGAL/Cartesian.h>
% #include <CGAL/CORE_algebraic_number_traits.h>
% #include <CGAL/Arr_conic_traits_2.h>
% #include <CGAL/Boolean_set_operations_2.h>
% 
% typedef CGAL::CORE_algebraic_number_traits            Nt_traits;
% typedef Nt_traits::Rational                           Rational;
% typedef Nt_traits::Algebraic                          Algebraic;
% typedef CGAL::Cartesian<Rational>                     Rat_kernel;
% typedef CGAL::Arr_conic_traits_2<Rat_kernel, Alg_kernel, Nt_traits>
%                                                       Arr_traits_2;
% typedef CGAL::Circular_polygon_traits<Arr_traits_2>   Traits;
% typedef Traits_2::Point_2                             Point_2;
% 
% int main ()
% \{
%   Point_2 center1(0,0), center2(0,1);
%   CGAL::Conic_polygon_2<Traits> circ1(center1, 1), circ2(center1, 1), res;
%   CGAL::reg_intersection(circ1, circ2, res);
%   return 0;
% \}
% \end{alltt}
%
%\begin{fminipage}{\textwidth}
\begin{alltt}
#include <CGAL/Cartesian.h>
#include <CGAL/Polygon_2.h>
#include <CGAL/Gmpq.h>
#include <CGAL/Boolean_set_operations_2.h>

typedef CGAL::Gmpq                                      Number_type;
typedef CGAL::Cartesian<Number_type>                    Kernel;
typedef Kernel::Point_2                                 Point;
typedef CGAL::Polygon_2<Kernel>                         Polygon;

int main ()
\{
  Polygon p1, p2;
  p1.push_back(Point(-1,1));
  p1.push_back(Point(0,-1));
  p1.push_back(Point(1,1));
  p2.push_back(Point(-1,-1));
  p2.push_back(Point(1,-1));
  p2.push_back(Point(0,1));
  std::cout << CGAL::do_intersect(p1, p2) ? "TRUE" : "FALSE";
  return 0;
\}
\end{alltt}
% \end{fminipage}

% ---------------------------------------------------------------
\subsection{Interface}
\label{bobs_ssec:interface}
% ---------------------------------------------------------------
The methods that implement the Boolean set-operations provided by this
package come in two flavors as follows. Some of the methods accept two
parameters that represent two operands. These methods store the result
in a third instance, the current object. Their counterpart methods
accept only a single parameter that represents one of the operands.
The other one is provided by the current object, which is replaced by
the result.

If you want to compute the union of a general polygon $P$ with a
general-polygon set $R$, and store the result in $R$, you can construct
a general-polygon set $S(P)$, and apply the {\em union} operation to
$R$ providing $S(P)$:

\begin{alltt}
General_polygon_2 S(P);
R.join(S);
\end{alltt}

As a matter of fact, you can apply the union operation directly:

\begin{alltt}
R.join(P);
\end{alltt}

However, if the polygon does not intersect any one of the general
polygons represented by $R$, you can use the more efficient method
\ccc{insert()}:

\begin{alltt}
R.insert(P);
\end{alltt}

Both methods, like many others, are overloaded, and accept a range of
general polygons. In case of the \ccc{insert} method, all the general
polygons in the input range and the general polygons represented by
$R$ must be pairwise disjoint in their interiors.

% ===============================================================
\section{The Main Components}
\label{bobs_sec:main_components}
% ===============================================================
The central component in the Boolean Set-Operations package is the
\ccc{General_polygon_set_2} class-template. It employs the
\ccc{Arrangement_2} class to represent a point set in the plane as a
planar arrangement; see Chapter~\ref{chapterArrangement_2}. 
An instance of the \ccc{General_polygon_set_2} class-template can be
constructed from general polygons or general polygons with holes. The
precise definitions of these polygons are provided in the following
subsections. It provides methods to apply regularized Boolean 
set-operations on pairs of \ccc{General_polygon_set_2} objects or general
polygon (or general polygon with holes) directly, and a few other utility
methods. The input and output of these methods consist of one or more
general polygons, some of which may have holes.

The package also contains global functions that perform common operations,
and are provided for convenience. A typical such function constructs a pair
of \ccc{General_polygon_set_2} instances, invokes the appropriate method to
apply the desired boolean operation, and results with one or more general
polygons. When several operations are performed in a sequence, it is
much more efficient to use the member functions of
\ccc{General_polygon_set_2} directly, as the extraction of the general
polygons from the internal representation for some operation, and the
reconstruction of the internal representation for the succeeding operation
could be time consuming.


% ===============================================================
\subsection{The Traits}
\label{bobs_ssec:traits}
% ===============================================================
\begin{wrapfigure}{r}{2.5cm}
\begin{center}
\vspace{-3ex}
\pspicture[](-1,-1)(1,1)
\psset{unit=1cm,linewidth=1pt}
\pscircle[fillstyle=solid,fillcolor=lightgray](0,0){1}
\qdisk(-1,0){2pt}
\qdisk(1,0){2pt}
\endpspicture
\caption{A general polygon.}
\label{fig:general_polygon}
\end{center}
\end{wrapfigure}
A polygon is a closed point set bounded by a piecewise lienar curve, such
as the entities represented by \cgal 's \ccc{Polygon_2} type. However,
the geometric mapping of the polygon edges in our context is not
necessarily linear. The operations provided by this package operate on
point sets bounded by $x$-monotone segments of general curves (e.g.,
conic arcs and rational arcs). The central class \ccc{General_polygon_set_2}
that offers methods that implement these operations (and the global
functions) are parameterized with a {\em traits} class that defines
the abstract interface between the operation and the geometric
primitives used. The traits class models the traits concept
\ccc{GeneralPolygonSetTraits_2}, and is tailored to handle a specific
family of curves. The concept \ccc{GeneralPolygonSetTraits_2} refines
the \ccc{ArrangementXMonotoneTraits_2} concept. Thus, a model of this
concepts must define the type \ccc{X_monotone_curve_2}, which
represents an $x$-monotone curve, and the type of the curve endpoint 
\ccc{Point_2}, which represents a planar point. It also must provide
various operations on these types listed by the refined concept.
Just as with the case of computations using models of the 
\ccc{ArrangementXMonotoneTraits_2} concept, operations are robust only
when exact arithmetic is used. When inexact arithmetic is used,
(nearly) degenerate configurations may result in abnormal termination
of the program or even incorrect results.

Every model of the traits-class concept must also define the types
\ccc{General_polygon_2} and \ccc{General_polygon_with_holes_2}, and it
must provide operations on objects of these types sufficient to enable
the boolean-set operations.  The type \ccc{General_polygon_2} represents a
simple point-set in the plane, whose boundary curves are $x$-monotone.
This type could be plain, as it does not necessarily provide methods
to access the curves boundary, nor does it necessarily provide a
constructor from a range of curves. Instead, the traits concept
requires these operations. Conceptually, the vertices of such a
general polygon can be ordered clockwise or counterclockwise, but we
guaranty counterclockwise order for the results of all
operations. Operations that operate on general polygons may result
with one or more general polygons with holes, which is described
below.

The traits parameter is optional when \ccc{General_polygon}
is instantiated with the type \ccc{Polygon_2}. In this case the
traits is obtained through the utility class
\ccc{Default_general_polygon_set_traits_2<General_polygon>}. For others
instantiated values of \ccc{General_polygon}, it must be specified.
We plan to increase the set of general-polygon type for which a default
traits is obtained in the future.

% ===============================================================
\subsection{General Polygons with Holes}
\label{bobs_ssec:general_polygons_with_holes}
% ===============================================================
\begin{wrapfigure}{r}{2.5cm}
\begin{center}
\vspace{-3ex}
\pspicture[](-1,-1)(1,1)
\psset{unit=1cm,linewidth=1pt}
\pscircle[fillstyle=solid,fillcolor=lightgray](0,0){1}
\pscircle[fillstyle=solid,fillcolor=white](0,0){0.5}
\qdisk(-1,0){2pt}\qdisk(1,0){2pt}
\qdisk(-0.5,0){2pt}\qdisk(0.5,0){2pt}
\endpspicture
\caption{A general polygon with holes.}
\label{fig:general_polygon_with_holes}
\end{center}
\end{wrapfigure}
% The \ccc{Triangle_2} and \ccc{Iso_rectangle_2} classes for example,
% which represent a triangle and a parallel-axis rectangle in
% $\mathrm{E}^2$ are special cases of convex general polygon. The apriori
% knowledge of whether the input polygons are convex may sometimes expedite
% the operation or simplify the representation of the result. For example,
% the intersection of two convex polygons is either an empty polygon or a
% convex polygon.
Regular sets are closed under regularized Boolean set-operations.
These operations accept as input, and may output, general
polygons with holes. The concept \ccc{GeneralPolygonSetTraits_2}
requires a type of polygon with holes that corresponds to the concept
\ccc{GeneralPolygonWithHoles_2} to be defined. This concept requires 
access to the outer boundary, which is of \ccc{General_polygon_2} type,
and to the holes, where each hole is also of type \ccc{General_polygon_2}.
The outer boundary could be empty, and the number of holes could be
zero. Vertices of holes and vertices of the boundary may coincide.
Conceptually, the order of vertices is irrelevant, but for the results
of all operations, we guarantee counterclockwise order for the general
polygon that represents the outer boundary, and clockwise order for
the general polygons that represent the holes.

The exact definition of the obtained general polygon with holes as a
result of a Boolean set-operation or a sequence of such operations is
closely related to the definition of regularized Boolean 
set-operations, being the colsure of the interior of the corresponding
ordinary operation as explained next.
\begin{wrapfigure}{l}{2.5cm}
\begin{center}
\vspace{-4ex}
\pspicture[](0,0)(2,1.732)
\psset{unit=1cm,linewidth=1pt}
\pspolygon*[linecolor=gray](0,0)(1,0)(0.5,0.866)
\pspolygon*[linecolor=gray](1,0)(2,0)(1.5,0.866)
\pspolygon*[linecolor=gray](0.5,0.866)(1.5,0.866)(1,1.732)
\pspolygon(0,0)(1,0)(0.5,0.866)
\pspolygon(1,0)(2,0)(1.5,0.866)
\pspolygon(0.5,0.866)(1.5,0.866)(1,1.732)
\endpspicture
\caption{Triangles.}
\label{fig:unique}
\end{center}
\end{wrapfigure}
There are many ways to arrive at a particual regular set. For
rexample, the regular set depicted in Figure~\ref{fig:unique} is the
result of the union of three small triangles translated
appropriately. It is also the result of the difference between a large
triangle and a small upside down triangle. Every point set that can
represented by an instance $R$ of the \ccc{Gnereal_polygon_set_2} type
has a unique internal representation as a (unique) planar arrangement
regardless of the particular sequence of operations that were applied
to arrive at $R$. Moreover, the set of
\ccc{General_polygon_with_holes_2} instances that represent $R$, and
can be obtained by the user through the
\ccc{general_polygon_with_holes()} method is also unique. The point
set deipcted in Figure~\ref{fig:unique} is represented as a single
general polygon with holes that has a single hole (and not as a triple
general polygons with holes that have no holes at all). The boundaries
of the holes in a general polygon with holes are parts of the polygon
(and not the holes), and as a general rule, if two point sets are
connected, then they belong to the same polygon with holes.
 
Instances of \ccc{General_polygon_2} used as input must be strictly simple
as a precondition. Instances of \ccc{General_polygon_2} obtained as output
are guaranteed to be stritcly simple. On the other hand, neither the outer
boundary, nor any one of the holes, of an instance of
\ccc{General_polygon_with_holes_2} used as input, or obtained as output,
must be strictly simple.

% ===============================================================
\subsection{Extended General Polygons}
\label{bobs_ssec:extended_general_polygons}
% ===============================================================
The \ccc{ExtendedGeneralPolygon_2} is a concept that facilitates the
production of general-polygon set traits classes. A model of this
concept represents a simple point-set in the plane bounded
by $x$-monotone curves. As opposed to the plain
\ccc{General_polygon_2} type defined by any traits class, it must
define the type \ccc{X_monotone_curve_2}, which represents an
$x$-monotone curve of the point-set boundary. It must provide a
constructor from a range of such curves, and a pair of methods, namely
\ccc{begin_curves()} and \ccc{end_curves()}, that can be used to
iterate over the point-set boundary curves.
 
The
\ccc{General_polygon_set_traits_2<Arr_X_monotone_traits,Extended_general_polygon,>}
class template is a model of the \ccc{GeneralPolygonSetTraits_2}
concept. Its implementation is rather simple, as it exploits the
methods provided by the instantiated parameter
\ccc{Extended_general_polygon} --- a model of the concept
\ccc{ExtendedGeneralPolygon_2}, and it is derived from the second
parameter \ccc{Arr_X_monotone_traits} inheriting its necessary types
and methods. 

A third class template
\ccc{Arr_general_polygon_2<Arr_x_monotone_traits>}
generates models of the \ccc{ExtendedGeneralPolygon_2} concept. It is
parameterized by a model of the \ccc{ArrangementXMonotoneTraits_2}
concepts from which it obtains the \ccc{X_monotone_curve_2} type, and
it uses the necessary operations on this type provided by such a model
to maintain a container of \ccc{X_monotone_curve_2} objects, which
represents the boundary of the general polygon.

The code excerpt listed below defines a general-polygon set type that
can be used to perform Boolean set-operations on point sets bounded by
linear segments used by the \ccc{Arrangement_2} class by default. A
model of the \ccc{ExtendedGeneralPolygon_2} concept that represents a
(linear) polygon bounded by curves of type \ccc{Arr_segment_2} is
generated. The later is obtained from the instantiated parameter
\ccc{Arr_segment_traits_2}, which defines \ccc{Arr_segment_2} to be
its exposed type \ccc{X_monotone_curve_2}.
\begin{alltt}
typedef CGAL::Gmpq                                      Number_type;
typedef CGAL::Cartesian<Number_type>                    Kernel;
typedef CGAL::Arr_segment_traits_2<Kernel>              Arr_traits;
typedef CGAL::Arr_general_polygon_2<Arr_traits>         Extended_polygon;
typedef CGAL::General_polygon_set_traits_2<Arr_traits,Extended_polygon>
                                                        Traits;
typedef CGAL::General_polygon_set_2<Traits>             General_polygon_set;
\end{alltt}

Swapping the linear arrangement traits \ccc{Arr_segment_traits_2}
above with a traits class that handle conic arcs, such as
\ccc{Arr_conic_traits_2}, results with the definition of a
general-polygon set type that can be used to perform Boolean 
set-operations on point sets bounded by conic arcs of type
\ccc{Arr_conic_2}.
