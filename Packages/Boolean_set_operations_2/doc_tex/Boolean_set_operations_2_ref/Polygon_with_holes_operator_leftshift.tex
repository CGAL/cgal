\RCSdef{\RCSPolygonwithholesoperatoristreamrev}{$Revision$}
\RCSdefDate{\RCSPolygonwithholesoperatoristreamDate}{$Date$}

\ccHtmlNoClassLinks
\begin{ccRefFunction}{operator<<}
\label{refPolygon_with_holes_operator_leftshift}

\ccDefinition

This operator exports a general polygon or a general polygon with
holes $P$ to the output stream \ccc{out}. The output is in ASCII
format. Only the point coordinates of the boundary are exported.

An ASCII and a binary format exist. The format can be selected with
the \cgal\ modifiers for streams, \ccc{set_ascii_mode} and
\ccc{set_binary_mode} respectively. The modifier \ccc{set_pretty_mode}
can be used to allow for (a few) structuring comments in the
output. Otherwise, the output would be free of comments.  The default
for writing is ASCII without comments.

\ccInclude{CGAL/IO/General_polygon_iostream.h}

\ccGlobalFunction{template <class ArrTraits>
    ostream& operator<<(ostream& out, 
                        const CGAL::General_polygon_2<ArrTraits>& P);}
  
\ccInclude{CGAL/IO/General_polygon_with_holes_iostream.h}

\ccGlobalFunction{template <class Polygon>
  ostream& operator<<(ostream& out, 
                      const CGAL::General_polygon_with_holes_2<Polygon>& P);}
  
\ccSeeAlso

\ccRefIdfierPage{CGAL::Polygon_2<Kernel>}\\ 
\ccRefIdfierPage{CGAL::General_polygon_2<ArrTraits>}\\ 
\ccRefIdfierPage{CGAL::General_polygon_with_holes_2<Polygon>}\\ 
  \lcTex{\ccc{operator>>} \dotfill\ page~\pageref{refPolygon_with_holes_operator_rightshift}}%
  \lcRawHtml{<I><A HREF="Function_operator--.html">operator&gt;&gt;</A></I>}

\end{ccRefFunction}
