\ccRefPageBegin
\label{ref_boolean_operations}

% ============================================================================
\begin{ccRefFunction}{do_intersect}

\ccThree{bool}{do_intersect}{}
\ccThreeToTwo

\ccDefinition

\ccInclude{CGAL/IO/Boolean_set_operations.h}

\ccGlobalFunction{bool do_intersect(const Type1 & p1, const Type2 & p2);}
{Each one of this functions return \ccc{true}, if the two given polygons
\ccc{p1} and \ccc{p2} intersect in their interior, and \ccc{false}
otherwise.}

\begin{tabular}{|l|l|}
\hline
\textbf{Arg 1 Type} & \textbf{Arg 2 Type}\\
\hline
\hline
\ccc{Polygon_2} & 
\ccc{Polygon_2}\\
\hline
\ccc{Polygon_2} & 
\ccc{General_polygon_with_holes_2}\\ 
\hline
\ccc{General_polygon_with_holes_2} &
\ccc{Polygon_2}\\ 
\hline
\ccc{General_polygon_2} & 
\ccc{General_polygon_2}\\
\hline
\ccc{General_polygon_2} & 
\ccc{General_polygon_with_holes_2}\\ 
\hline
\ccc{General_polygon_with_holes_2} &
\ccc{General_polygon_2}\\ 
\hline
\ccc{General_polygon_with_holes_2} &
\ccc{General_polygon_with_holes_2}\\
\hline
\end{tabular}

\ccGlobalFunction{template <class Kernel, class Container>
bool do_intersect(const Polygon_2<Kernel, Container> & p1,
                  const Polygon_2<Kernel, Container> & p2);}
\ccGlue
\ccGlobalFunction{template <class Kernel, class Container>
bool do_intersect(const Polygon_2<Kernel, Container> & p1,
                  const General_polygon_with_holes_2<Polygon_2<Kernel,Container> > & p2);}
\ccGlue
\ccGlobalFunction{template <class Kernel, class Container>
bool do_intersect(const General_polygon_with_holes_2<Polygon_2<Kernel,Container> > & p1,
                  const Polygon_2<Kernel, Container> & p2);}
\ccGlue
\ccGlobalFunction{template <class Traits>
bool do_intersect(const General_polygon_2<Traits> & p1,
                  const General_polygon_2<Traits> & p2);}
\ccGlue
\ccGlobalFunction{template <class Traits>
bool do_intersect(const General_polygon_2<Traits> & p1,
                  const General_polygon_with_holes_2<General_polygon_2<Traits> > & p2);}
\ccGlue
\ccGlobalFunction{template <class Traits>
bool do_intersect(const General_polygon_with_holes_2<General_polygon_2<Traits> > & p1,
                  const General_polygon_2<Traits> & p2);}
\ccGlue
\ccGlobalFunction{template <class Polygon>
bool do_intersect(const General_polygon_with_holes_2<Polygon> & p1,
                  const General_polygon_with_holes_2<Polygon> & p2);}

\ccGlobalFunction{template <class InputIterator>
bool intersection(InputIterator & begin, InputIterator & end);}
{Return \ccc{true}, if the set the general polygons (or general
polygons with holes) in the given range intersect in their interior,
and \ccc{false} otherwise. (The value type of the input iterator is
used to distinguish between the two.)}

\ccGlobalFunction{template <class InputIterator1, class InputIterator2>
bool intersection(InputIterator1 & pgn_begin1,
                  InputIterator1 & pgn_end1,
                  InputIterator2 & pgn_begin2,
	          InputIterator2 & pgn_end2);}
{Return \ccc{true}, if the set the general polygons and general
polygons with holes in the given two ranges respectively intersect in
their interior, and \ccc{false} otherwise.}

\ccSeeAlso
\ccRefIdfierPage{CGAL::intersection}\\
\ccRefIdfierPage{CGAL::join}\\
\ccRefIdfierPage{CGAL::difference}\\
\ccRefIdfierPage{CGAL::symmetric_difference}

\end{ccRefFunction}
% ============================================================================
\begin{ccRefFunction}{intersection}

\ccThree{OutputIterator}{intersection}{}
\ccThreeToTwo

\ccDefinition

\ccInclude{CGAL/IO/Boolean_set_operations.h}

\ccGlobalFunction{
OutputIterator intersection(const Type1 & p1, const Type2 & p2,
                            OutputIterator oi);}
{Each one of this functions computes the intersection of two given
polygons \ccc{p1} and \ccc{p2}, inserts the resulting polygons with
holes into an output container through a given output iterator
\ccc{oi}, and returns the output iterator. The value type of the
\ccc{OutputIterator} is \ccc{General_polygon_with_holes_2}.}

\begin{tabular}{|l|l|}
\hline
\textbf{Arg 1 Type} & \textbf{Arg 2 Type}\\
\hline
\hline
\ccc{Polygon_2} & 
\ccc{Polygon_2}\\
\hline
\ccc{Polygon_2} & 
\ccc{Polygon_with_holes_2}\\
\hline
\ccc{Polygon_with_holes_2} & 
\ccc{Polygon_2}\\
\hline
\ccc{General_polygon_2} & 
\ccc{General_polygon_2}\\
\hline
\ccc{General_polygon_2} & 
\ccc{General_polygon_with_holes_2}\\
\hline
\ccc{General_polygon_with_holes_2} & 
\ccc{General_polygon_2}\\
\hline
\ccc{General_polygon_with_holes_2} & 
\ccc{General_polygon_with_holes_2}\\
\hline
\end{tabular}

\ccGlobalFunction{template <class Kernel, class Container, class OutputIterator>
OutputIterator intersection(const Polygon_2<Kernel, Container> & p1,
                            const Polygon_2<Kernel, Container> & p2,
                            OutputIterator oi);}
\ccGlue
\ccGlobalFunction{template <class Kernel, class Container, class OutputIterator>
OutputIterator intersection(const General_polygon_with_holes_2<Polygon_2<Kernel,Container> > & p1,
                            const Polygon_2<Kernel, Container> & p2,
                            OutputIterator oi);}
\ccGlue
\ccGlobalFunction{template <class Kernel, class Container, class OutputIterator>
OutputIterator intersection(const Polygon_2<Kernel, Container> & p1,
                            const General_polygon_with_holes_2<Polygon_2<Kernel,Container> > & p2,
                            OutputIterator oi);}
\ccGlue
\ccGlobalFunction{template <class Traits, class OutputIterator>
OutputIterator intersection(const General_polygon_2<Traits> & p1,
                            const General_polygon_2<Traits> & p2,
                            OutputIterator oi);}
\ccGlue
\ccGlobalFunction{template <class Traits, class OutputIterator>
OutputIterator intersection(const General_polygon_with_holes_2<General_polygon_2<Traits> > & p1,
                            const General_polygon_2<Traits> & p2,
                            OutputIterator oi);}
\ccGlue
\ccGlobalFunction{template <class Traits, class OutputIterator>
OutputIterator intersection(const General_polygon_2<Traits> & p1,
                            const General_polygon_with_holes_2<General_polygon_2<Traits> > & p2,
                            OutputIterator oi);}
\ccGlue
\ccGlobalFunction{template <class Polygon, class OutputIterator>
OutputIterator intersection(const General_polygon_with_holes_2<Polygon> & p1,
                            const General_polygon_with_holes_2<Polygon> & p2,
                            OutputIterator oi);}

\ccGlobalFunction{template <class InputIterator, class OutputIterator>
OutputIterator intersection(InputIterator & begin, InputIterator & end,
                            OutputIterator oi,
                            Traits & traits);}
{Computes the intersection of the general polygons (or general polygons with
holes) in the given range. (The value type of the input iterator is
used to distinguish between the two.) The result, represented by a set
of general poygon with holes, is inserted into an output container
through a given output iterator \ccc{oi}. The output iterator is
returned. The value type of the \ccc{OutputIterator} is
\ccc{General_polygon_with_holes_2}.}

\ccGlobalFunction{template <class InputIterator1, class InputIterator2,
		  class OutputIterator>
OutputIterator intersection(InputIterator1 & pgn_begin1,
                            InputIterator1 & pgn_end1,
                            InputIterator2 & pgn_begin2,
	                    InputIterator2 & pgn_end2,
		            OutputIterator oi);}
{Computes the intersection of the general polygons and general polygons
with holes in the given two ranges. The result, represented by a set
of general poygon with holes, is inserted into an output container
through a given output iterator \ccc{oi}. The output iterator is
returned. The value type of the \ccc{OutputIterator} is
\ccc{General_polygon_with_holes_2}.}

\ccSeeAlso
\ccRefIdfierPage{CGAL::do_intersect}\\
\ccRefIdfierPage{CGAL::join}\\
\ccRefIdfierPage{CGAL::difference}\\
\ccRefIdfierPage{CGAL::symmetric_difference}

\end{ccRefFunction}
% ============================================================================
\begin{ccRefFunction}{join}

\ccThree{OutputIterator}{join}{}
\ccThreeToTwo

\ccDefinition

\ccInclude{CGAL/IO/Boolean_set_operations.h}

\ccGlobalFunction{
bool join(const Type1 & p1, const Type2 & p2,
	  General_polygon_with_holes_2 & p);}
{Each one of these functions computes the union of two given polygons
\ccc{p1} and \ccc{p2}. If the two given polygons overlap, it returns
\ccc{true}, and places the resulting polygon in \ccc{p}. Otherwise, it
returns \ccc{false}.}

\begin{tabular}{|l|l|}
\hline
\textbf{Arg 1 Type} & \textbf{Arg 2 Type}\\
\hline
\hline
\ccc{Polygon_2} & 
\ccc{Polygon_2}\\
\hline
\ccc{Polygon_2} & 
\ccc{Polygon_with_holes_2}\\
\hline
\ccc{Polygon_with_holes_2} & 
\ccc{Polygon_2}\\
\hline
\ccc{General_polygon_2} & 
\ccc{General_polygon_2}\\
\hline
\ccc{General_polygon_2} & 
\ccc{General_polygon_with_holes_2}\\
\hline
\ccc{General_polygon_with_holes_2} & 
\ccc{General_polygon_2}\\
\hline
\ccc{General_polygon_with_holes_2} & 
\ccc{General_polygon_with_holes_2}\\
\hline
\end{tabular}

\ccGlobalFunction{template <class Kernel, class Container>
bool join(const Polygon_2<Kernel, Container> & p1,
          const Polygon_2<Kernel, Container> & p2,
	  General_polygon_with_holes_2<Polygon_2<Kernel, Container> > & p);}
\ccGlue
\ccGlobalFunction{template <class Kernel, class Container>
bool join(const Polygon_2<Kernel, Container> & p1,
          const General_polygon_with_holes_2<Polygon_2<Kernel,Container> > & p2,
	  General_polygon_with_holes_2Polygon_2<Kernel, Container> > & p);}
\ccGlue
\ccGlobalFunction{template <class Kernel, class Container>
bool join(const General_polygon_with_holes_2<Polygon_2<Kernel,Container> > & p2,
          const Polygon_2<Kernel, Container> & p1,
	  General_polygon_with_holes_2Polygon_2<Kernel, Container> > & p);}
\ccGlue
\ccGlobalFunction{template <class Traits>
bool join(const General_polygon_2<Traits> & p1,
          const General_polygon_2<Traits> & p2,
	  General_polygon_with_holes_2<General_polygon_2<Traits> > & p);}
\ccGlue
\ccGlobalFunction{template <class Traits>
bool join(const General_polygon_2<Traits> & p1,
          const General_polygon_with_holes_2<General_polygon_2<Traits> > & p2,
	  General_polygon_with_holes_2<General_polygon_2<Traits> > & p);}
\ccGlue
\ccGlobalFunction{template <class Traits>
bool join(const General_polygon_with_holes_2<General_polygon_2<Traits> > & p2,
          const General_polygon_2<Traits> & p1,
	  General_polygon_with_holes_2<General_polygon_2<Traits> > & p);}
\ccGlue
\ccGlobalFunction{template <class Polygon>
bool join(const General_polygon_with_holes_2<Polygon> & p1,
          const General_polygon_with_holes_2<Polygon> & p2,
	  Traits::Polygon_with_holes_2 & p);}

\ccGlobalFunction{template <class InputIterator, class OutputIterator>
OutputIterator join(InputIterator & begin, InputIterator & end,
                    OutputIterator oi);}
{Computes the union of the general polygons (or general polygons with
holes) in the given range. (The value type of the input iterator is
used to distinguish between the two.) The result, represented by a set
of general poygon with holes, is inserted into an output container
through a given output iterator \ccc{oi}. The output iterator is
returned. The value type of the \ccc{OutputIterator} is
\ccc{General_polygon_with_holes_2}.}

\ccGlobalFunction{template <class InputIterator1, class InputIterator2,
		  class OutputIterator>
OutputIterator join(InputIterator1 & pgn_begin1, InputIterator1 & pgn_end1,
                    InputIterator2 & pgn_begin2, InputIterator2 & pgn_end2,
		    OutputIterator oi);}
{Computes the union of the general polygons and general polygons
with holes in the given two ranges. The result, represented by a set
of general poygon with holes, is inserted into an output container
through a given output iterator \ccc{oi}. The output iterator is
returned. The value type of the \ccc{OutputIterator} is
\ccc{General_polygon_with_holes_2}.}

\ccSeeAlso
\ccRefIdfierPage{CGAL::do_intersect}\\
\ccRefIdfierPage{CGAL::intersection}\\
\ccRefIdfierPage{CGAL::difference}\\
\ccRefIdfierPage{CGAL::symmetric_difference}

\end{ccRefFunction}
% ============================================================================
\begin{ccRefFunction}{difference}

\ccThree{OutputIterator}{difference}{}
\ccThreeToTwo

\ccDefinition

\ccInclude{CGAL/IO/Boolean_set_operations.h}

\ccGlobalFunction{
OutputIterator difference(const Type1 & p1, const Type2 & p2,
                          OutputIterator oi);}
{Each one of these functions computes the difference between two given
polygons \ccc{p1} and \ccc{p2}, and inserts the resulting polygons
with holes into an output container through the output iterator \ccc{oi}.
The value type of the \ccc{OutputIterator} is
\ccc{General_polygon_with_holes_2}.}

\begin{tabular}{|l|l|}
\hline
\textbf{Arg 1 Type} & \textbf{Arg 2 Type}\\
\hline
\hline
\ccc{Polygon_2} & 
\ccc{Polygon_2}\\
\hline
\ccc{Polygon_2} & 
\ccc{Polygon_with_holes_2}\\
\hline
\ccc{Polygon_with_holes_2} & 
\ccc{Polygon_2}\\
\hline
\ccc{General_polygon_2} & 
\ccc{General_polygon_2}\\
\hline
\ccc{General_polygon_2} & 
\ccc{General_polygon_with_holes_2}\\
\hline
\ccc{General_polygon_with_holes_2} & 
\ccc{General_polygon_2}\\
\hline
\ccc{General_polygon_with_holes_2} & 
\ccc{General_polygon_with_holes_2}\\
\hline
\end{tabular}

\ccGlobalFunction{template <class Kernel, class Container, class OutputIterator>
OutputIterator difference(const Polygon_2<Kernel, Container> & p1,
                          const Polygon_2<Kernel, Container> & p2,
                          OutputIterator oi);}
\ccGlue
\ccGlobalFunction{template <class Kernel, class Container, class OutputIterator>
OutputIterator difference(const General_polygon_with_holes_2<Polygon_2<Kernel,Container> > & p1,
                          const Polygon_2<Kernel, Container> & p2,
                          OutputIterator oi);}
\ccGlue
\ccGlobalFunction{template <class Kernel, class Container, class OutputIterator>
OutputIterator difference(const Polygon_2<Kernel, Container> & p1,
                          const General_polygon_with_holes_2<Polygon_2<Kernel,Container> > & p2,
                          OutputIterator oi);}
\ccGlue
\ccGlobalFunction{template <class Traits, class OutputIterator>
OutputIterator difference(const General_polygon_2<Traits> & p1,
                          const General_polygon_2<Traits> & p2,
                          OutputIterator oi);}
\ccGlue
\ccGlobalFunction{template <class Traits, class OutputIterator>
OutputIterator difference(const General_polygon_with_holes_2<General_polygon_2<Traits> > & p1,
                          const General_polygon_2<Traits> & p2,
                          OutputIterator oi);}
\ccGlue
\ccGlobalFunction{template <class Traits, class OutputIterator>
OutputIterator difference(const General_polygon_2<Traits> & p1,
                          const General_polygon_with_holes_2<General_polygon_2<Traits> > & p2,
                          OutputIterator oi);}
\ccGlue
\ccGlobalFunction{template <class Polygon, class OutputIterator>
OutputIterator difference(const General_polygon_with_holes_2<Polygon> & p1,
                          const General_polygon_with_holes_2<Polygon> & p2);}

\ccGlobalFunction{template <class InputIterator, class OutputIterator>
OutputIterator difference(InputIterator & begin1, InputIterator & end1,
                          InputIterator & begin2, InputIterator & end2,
                          OutputIterator oi);}
{Computes the difference between the union of one set of general
polygons (or general polygons with holes) given in the first range, and
another set given in the second range. (The value type of the input iterator
is used to distinguish between the two types.) The result, represented
by a set of general poygon with holes, is inserted into an output container
through a given output iterator \ccc{oi}. The output iterator is
returned. The value type of the \ccc{OutputIterator} is
\ccc{General_polygon_with_holes_2}.}

\ccGlobalFunction{template <class InputIterator1, class InputIterator2,
		  class OutputIterator>
OutputIterator difference(InputIterator1 & begin11, InputIterator1 & end11,
                          InputIterator1 & begin12, InputIterator2 & end12,
			  InputIterator2 & begin21, InputIterator1 & end21,
			  InputIterator2 & begin22, InputIterator1 & end22,
		          OutputIterator oi);}
{Computes the difference between the union of a set of general polygons
and general polygons with holes given in the first two ranges
<\ccc{begin11},\ccc{end11}> and <\ccc{begin12},\ccc{end12}> and another
set given in two succeeding ranges <\ccc{begin21},\ccc{end21}> and
<\ccc{begin22},\ccc{end22}>. The result, represented by a set
of general poygon with holes, is inserted into an output container
through a given output iterator \ccc{oi}. The output iterator is
returned. The value type of the \ccc{OutputIterator} is
\ccc{General_polygon_with_holes_2}.}

\ccSeeAlso
\ccRefIdfierPage{CGAL::do_intersect}\\
\ccRefIdfierPage{CGAL::intersection}\\
\ccRefIdfierPage{CGAL::join}\\
\ccRefIdfierPage{CGAL::symmetric_difference}

\end{ccRefFunction}
% ============================================================================
\begin{ccRefFunction}{symmetric_difference}

\ccThree{OutputIterator}{symmetric_difference}{}
\ccThreeToTwo

\ccDefinition

\ccInclude{CGAL/IO/Boolean_set_operations.h}

\ccGlobalFunction{
OutputIterator intersection(const Type1 & p1, const Type2 & p2,
                            OutputIterator oi);}
{Each one of these functions computes the symmetric difference between
two given polygons \ccc{p1} and \ccc{p2}, and inserts the resulting
polygons with holes into an output container through the output
iterator \ccc{oi}. The value type of the \ccc{OutputIterator} is
\ccc{General_polygon_with_holes_2}.}

\begin{tabular}{|l|l|}
\hline
\textbf{Arg 1 Type} & \textbf{Arg 2 Type}\\
\hline
\hline
\ccc{Polygon_2} & 
\ccc{Polygon_2}\\
\hline
\ccc{Polygon_2} & 
\ccc{Polygon_with_holes_2}\\
\hline
\ccc{Polygon_with_holes_2} & 
\ccc{Polygon_2}\\
\hline
\ccc{General_polygon_2} & 
\ccc{General_polygon_2}\\
\hline
\ccc{General_polygon_2} & 
\ccc{General_polygon_with_holes_2}\\
\hline
\ccc{General_polygon_with_holes_2} & 
\ccc{General_polygon_2}\\
\hline
\ccc{General_polygon_with_holes_2} & 
\ccc{General_polygon_with_holes_2}\\
\hline
\end{tabular}

\ccGlobalFunction{template <class Kernel, class Container, class OutputIterator>
OutputIterator symmetric_difference(const Polygon_2<Kernel, Container> & p1,
                                    const Polygon_2<Kernel, Container> & p2,
                                    OutputIterator oi);}
\ccGlue
\ccGlobalFunction{template <class Kernel, class Container, class OutputIterator>
OutputIterator symmetric_difference(const General_polygon_with_holes_2<Polygon_2<Kernel,Container> > & p1,
                                    const Polygon_2<Kernel, Container> & p2,
                                    OutputIterator oi);}
\ccGlue
\ccGlobalFunction{template <class Kernel, class Container, class OutputIterator>
OutputIterator symmetric_difference(const Polygon_2<Kernel, Container> & p1,
                                    const General_polygon_with_holes_2<Polygon_2<Kernel,Container> > & p2,
                                    OutputIterator oi);}
\ccGlue
\ccGlobalFunction{template <class Traits, class OutputIterator>
OutputIterator symmetric_difference(const General_polygon_2<Traits> & p1,
                                    const General_polygon_2<Traits> & p2,
                                    OutputIterator oi);}
\ccGlue
\ccGlobalFunction{template <class Traits, class OutputIterator>
OutputIterator symmetric_difference(const General_polygon_with_holes_2<General_polygon_2<Traits> > & p1,
                                    const General_polygon_2<Traits> & p2,
                                    OutputIterator oi);}
\ccGlue
\ccGlobalFunction{template <class Traits, class OutputIterator>
OutputIterator symmetric_difference(const General_polygon_2<Traits> & p1,
                                    const General_polygon_with_holes_2<General_polygon_2<Traits> > & p2,
                                    OutputIterator oi);}
\ccGlue
\ccGlobalFunction{template <class Polygon, class OutputIterator>
OutputIterator 
symmetric_difference(const General_polygon_with_holes_2<Polygon> & p1,
                     const General_polygon_with_holes_2<Polygon> & p2,
                     OutputIterator oi);}

\ccSeeAlso
\ccRefIdfierPage{CGAL::do_intersect}\\
\ccRefIdfierPage{CGAL::intersection}\\
\ccRefIdfierPage{CGAL::join}\\
\ccRefIdfierPage{CGAL::difference}

\end{ccRefFunction}
\ccRefPageEnd
