\ccRefPageBegin
\label{ref_boolean_operations}

\ccThree{OutputIterator}{symmetric_difference}{}
\ccThreeToTwo

\ccInclude{CGAL/Bops_regularized_bops.h}

% ============================================================================
\begin{ccRefFunction}{do_intersect}

\ccDefinition

Each one of the following functions returns \ccc{true} if the two given
general polygons \ccc{p1} and \ccc{p2} intersect in their interior,
and \ccc{false} otherwise.

\ccInclude{CGAL/IO/Boolean_set_operations.h}

\ccGlobalFunction{template <class Kernel>
bool do_intersect(const Polygon_2<Kernel> & p1,
                  const Polygon_2<Kernel> & p2);}
\ccGlue
\ccGlobalFunction{template <class Traits>
bool do_intersect(const General_polygon_2<Traits> & p1,
                  const General_polygon_2<Traits> & p2);}

\ccGlobalFunction{template <class Traits>
bool do_intersect(const Traits::Polygon_2 & p1,
                  const Traits::Polygon_2 & p2);}
\ccGlue
\ccGlobalFunction{template <class Traits>
bool do_intersect(const Traits::Polygon_2 & p1,
                  const Traits::Polygon_with_holes_2 & p2);}
\ccGlue
\ccGlobalFunction{template <class Traits>
bool do_intersect(const Traits::Polygon_with_holes_2 & p1,
                  const Traits::Polygon_2 & p2);}
\ccGlue
\ccGlobalFunction{template <class Traits>
bool do_intersect(const Traits::Polygon_with_holes_2 & p1,
                  const Traits::Polygon_with_holes_2 & p2);}

\ccGlobalFunction{template <class Traits>
bool do_intersect(const Traits::Polygon_2 & p1,
                  const Traits::Polygon_2 & p2,
                  Traits & traits);}
\ccGlue
\ccGlobalFunction{template <class Traits>
bool do_intersect(const Traits::Polygon_2 & p1,
                  const Traits::Polygon_with_holes_2 & p2,
                  Traits & traits);}
\ccGlue
\ccGlobalFunction{template <class Traits>
bool do_intersect(const Traits::Polygon_with_holes_2 & p2
                  const Traits::Polygon_2 & p1,
                  Traits & traits);}
\ccGlue
\ccGlobalFunction{template <class Traits>
bool do_intersect(const Traits::Polygon_with_holes_2 & p1,
                  const Traits::Polygon_with_holes_2 & p2,
                  Traits & traits);}

\ccSeeAlso
\ccRefIdfierPage{CGAL::intersection}\\
\ccRefIdfierPage{CGAL::join}\\
\ccRefIdfierPage{CGAL::difference}\\
\ccRefIdfierPage{CGAL::symmetric_difference}

\end{ccRefFunction}
% ============================================================================
\begin{ccRefFunction}{intersection}

\ccDefinition

Each one of the following four functions computes the intersection of
two given general polygons \ccc{p1} and \ccc{p2}, inserts the
resulting polygons with holes into an output container through a given
output iterator \ccc{oi}, and returns the output iterator. The value
type of the \ccc{OutputIterator} is \ccc{General_polygon_with_holes_2}.

\ccInclude{CGAL/IO/Boolean_set_operations.h}

\ccGlobalFunction{template <class Kernel, class OutputIterator>
OutputIterator intersection(const Polygon_2<Kernel> & p1,
                            const Polygon_2<Kernel> & p2,
                            OutputIterator oi);}
\ccGlue
\ccGlobalFunction{template <class Traits, class OutputIterator>
OutputIterator intersection(const General_polygon_2<Traits> & p1,
                            const General_polygon_2<Traits> & p2,
                            OutputIterator oi);}

\ccGlobalFunction{template <class Traits, class OutputIterator>
OutputIterator intersection(const Traits::Polygon_2 & p1,
                            const Traits::Polygon_2 & p2,
                            OutputIterator oi);}
\ccGlue
\ccGlobalFunction{template <class Traits, class OutputIterator>
OutputIterator intersection(const Traits::Polygon_2 & p1,
                            const Traits::Polygon_with_holes_2 & p2,
                            OutputIterator oi);}
\ccGlue
\ccGlobalFunction{template <class Traits, class OutputIterator>
OutputIterator intersection(const Traits::Polygon_with_holes_2 & p2,
                            const Traits::Polygon_2 & p1,
                            OutputIterator oi);}
\ccGlue
\ccGlobalFunction{template <class Traits, class OutputIterator>
OutputIterator intersection(const Traits::Polygon_with_holes_2 & p1,
                            const Traits::Polygon_with_holes_2 & p2,
                            OutputIterator oi);}

\ccGlobalFunction{template <class Traits, class OutputIterator>
OutputIterator intersection(const Traits::Polygon_2 & p1,
                            const Traits::Polygon_2 & p2,
                            OutputIterator oi,
                            Traits & traits);}
\ccGlue
\ccGlobalFunction{template <class Traits, class OutputIterator>
OutputIterator intersection(const Traits::Polygon_2 & p1,
                            const Traits::Polygon_with_holes_2 & p2,
                            OutputIterator oi,
                            Traits & traits);}
\ccGlue
\ccGlobalFunction{template <class Traits, class OutputIterator>
OutputIterator intersection(const Traits::Polygon_with_holes_2 & p2,
                            const Traits::Polygon_2 & p1,
                            OutputIterator oi,
                            Traits & traits);}
\ccGlue
\ccGlobalFunction{template <class Traits, class OutputIterator>
OutputIterator intersection(const Traits::Polygon_with_holes_2 & p1,
                            const Traits::Polygon_with_holes_2 & p2,
                            OutputIterator oi,
                            Traits & traits);}
  
\ccGlobalFunction{template <class InputIterator, class OutputIterator>
OutputIterator intersection(InputIterator & begin, InputIterator & end,
                            OutputIterator oi,
                            Traits & traits);}
{Computes the intersection of the general polygons (or general polygons with
holes) in the given range. (The value type of the input iterator is
used to distinguish between the two.) The result, represented by a set
of general poygon with holes, is inserted into and output container
through a given output iterator \ccc{oi}. The output iterator is
returned. The value type of the \ccc{OutputIterator} is
\ccc{General_polygon_with_holes_2}.}

\ccGlobalFunction{template <class InputIterator1, class InputIterator2,
		  class OutputIterator>
OutputIterator intersection(InputIterator1 & pgn_begin,
                            InputIterator1 & pgn_end,
                            InputIterator2 & pgn_with_holes_begin,
	                    InputIterator2 & pgn_with_holes_end,
		            OutputIterator oi);}
{Computes the intersection of the general polygons and general polygons
with holes in the given two ranges. The result, represented by a set
of general poygon with holes, is inserted into and output container
through a given output iterator \ccc{oi}. The output iterator is
returned. The value type of the \ccc{OutputIterator} is
\ccc{General_polygon_with_holes_2}.}

\ccSeeAlso
\ccRefIdfierPage{CGAL::do_intersect}\\
\ccRefIdfierPage{CGAL::join}\\
\ccRefIdfierPage{CGAL::difference}\\
\ccRefIdfierPage{CGAL::symmetric_difference}

\end{ccRefFunction}
% ============================================================================
\begin{ccRefFunction}{join}

\ccDefinition

\ccInclude{CGAL/IO/Boolean_set_operations.h}

Each one of the following four functions computes the union of two
given general polygons \ccc{p1} and \ccc{p2}. If the two given
polygons overlap, it returns \ccc{true} and places the resulting
polygon in \ccc{p}. Otherwise, it returns \ccc{false}.

\ccGlobalFunction{template <class Kernel>
bool join(const Polygon_2<Kernel> & p1,
          const Polygon_2<Kernel> & p2,
	  const Traits::Polygon_with_holes_2 & p);}
\ccGlue
\ccGlobalFunction{template <class Traits>
bool join(const General_polygon_2<Traits> & p1,
          const General_polygon_2<Traits> & p2,
	  const Traits::Polygon_with_holes_2 & p);}

\ccGlobalFunction{template <class Traits>
bool join(const Traits::Polygon_2 & p1,
          const Traits::Polygon_2 & p2,
	  const Traits::Polygon_with_holes_2 & p);}
\ccGlue
\ccGlobalFunction{template <class Traits>
bool join(const Traits::Polygon_2 & p1,
          const Traits::Polygon_with_holes_2 & p2,
	  const Traits::Polygon_with_holes_2 & p);}
\ccGlue
\ccGlobalFunction{template <class Traits>
bool join(const Traits::Polygon_with_holes_2 & p2,
          const Traits::Polygon_2 & p1,
	  const Traits::Polygon_with_holes_2 & p);}
\ccGlue
\ccGlobalFunction{template <class Traits>
bool join(const Traits::Polygon_with_holes_2 & p1,
          const Traits::Polygon_with_holes_2 & p2,
	  const Traits::Polygon_with_holes_2 & p);}

\ccGlobalFunction{template <class Traits>
bool join(const Traits::Polygon_2 & p1,
          const Traits::Polygon_2 & p2,
	  const Traits::Polygon_with_holes_2 & p,
          Traits & traits);}
\ccGlue
\ccGlobalFunction{template <class Traits>
bool join(const Traits::Polygon_2 & p1,
          const Traits::Polygon_with_holes_2 & p2,
	  const Traits::Polygon_with_holes_2 & p,
          Traits & traits);}
\ccGlue
\ccGlobalFunction{template <class Traits>
bool join(const Traits::Polygon_with_holes_2 & p2,
          const Traits::Polygon_2 & p1,
	  const Traits::Polygon_with_holes_2 & p,
          Traits & traits);}
\ccGlue
\ccGlobalFunction{template <class Traits>
bool join(const Traits::Polygon_with_holes_2 & p1,
          const Traits::Polygon_with_holes_2 & p2,
	  const Traits::Polygon_with_holes_2 & p,
          Traits & traits);}

\ccGlobalFunction{template <class InputIterator, class OutputIterator>
OutputIterator join(InputIterator & begin, InputIterator & end,
                    OutputIterator oi);}
{Computes the union of the general polygons (or general polygons with
holes) in the given range. (The value type of the input iterator is
used to distinguish between the two.) The result, represented by a set
of general poygon with holes, is inserted into and output container
through a given output iterator \ccc{oi}. The output iterator is
returned. The value type of the \ccc{OutputIterator} is
\ccc{General_polygon_with_holes_2}.}

\ccGlobalFunction{template <class InputIterator1, class InputIterator2,
		  class OutputIterator>
OutputIterator join(InputIterator1 & pgn_begin, InputIterator1 & pgn_end,
                    InputIterator2 & pgn_with_holes_begin,
	            InputIterator2 & pgn_with_holes_end,
		    OutputIterator oi);}
{Computes the union of the general polygons and general polygons
with holes in the given two ranges. The result, represented by a set
of general poygon with holes, is inserted into and output container
through a given output iterator \ccc{oi}. The output iterator is
returned. The value type of the \ccc{OutputIterator} is
\ccc{General_polygon_with_holes_2}.}

\ccSeeAlso
\ccRefIdfierPage{CGAL::do_intersect}\\
\ccRefIdfierPage{CGAL::intersection}\\
\ccRefIdfierPage{CGAL::difference}\\
\ccRefIdfierPage{CGAL::symmetric_difference}

\end{ccRefFunction}
% ============================================================================
\begin{ccRefFunction}{difference}

\ccDefinition

\ccInclude{CGAL/IO/Boolean_set_operations.h}

Each one of the following four functions computes the difference
between two given general polygons \ccc{p1} and \ccc{p2}, and inserts
the resulting polygons with holes into an output container through the
output iterator \ccc{oi}. The value type of the \ccc{OutputIterator}
is \ccc{General_polygon_with_holes_2}.

\ccGlobalFunction{template <class Kernel, class OutputIterator>
OutputIterator difference(const Polygon_2<Kernel> & p1,
                          const Polygon_2<Kernel> & p2);}
\ccGlue
\ccGlobalFunction{template <class Traits, class OutputIterator>
OutputIterator difference(const General_polygon_2<Traits> & p1,
                          const General_polygon_2<Traits> & p2);}

\ccGlobalFunction{template <class Traits, class OutputIterator>
OutputIterator difference(const Traits::Polygon_2 & p1,
                          const Traits::Polygon_2 & p2,
                          OutputIterator oi);}
\ccGlue
\ccGlobalFunction{template <class Traits, class OutputIterator>
OutputIterator difference(const Traits::Polygon_2 & p1,
                          const Traits::Polygon_with_holes_2 & p2,
                          OutputIterator oi);}
\ccGlue
\ccGlobalFunction{template <class Traits, class OutputIterator>
OutputIterator difference(const Traits::Polygon_with_holes_2 & p2,
                          const Traits::Polygon_2 & p1,
                          OutputIterator oi);}
\ccGlue
\ccGlobalFunction{template <class Traits, class OutputIterator>
OutputIterator difference(const Traits::Polygon_with_holes_2 & p1,
                          const Traits::Polygon_with_holes_2 & p2,
                          OutputIterator oi);}

\ccGlobalFunction{template <class Traits, class OutputIterator>
OutputIterator difference(const Traits::Polygon_2 & p1,
                          const Traits::Polygon_2 & p2,
                          OutputIterator oi,
                          Traits & traits);}
\ccGlue
\ccGlobalFunction{template <class Traits, class OutputIterator>
OutputIterator difference(const Traits::Polygon_2 & p1,
                          const Traits::Polygon_with_holes_2 & p2,
                          OutputIterator oi,
                          Traits & traits);}
\ccGlue
\ccGlobalFunction{template <class Traits, class OutputIterator>
OutputIterator difference(const Traits::Polygon_with_holes_2 & p2,
                          const Traits::Polygon_2 & p1,
                          OutputIterator oi,
                          Traits & traits);}
\ccGlue
\ccGlobalFunction{template <class Traits, class OutputIterator>
OutputIterator difference(const Traits::Polygon_with_holes_2 & p1,
                          const Traits::Polygon_with_holes_2 & p2,
                          OutputIterator oi,
                          Traits & traits);}

\ccSeeAlso
\ccRefIdfierPage{CGAL::do_intersect}\\
\ccRefIdfierPage{CGAL::intersection}\\
\ccRefIdfierPage{CGAL::join}\\
\ccRefIdfierPage{CGAL::symmetric_difference}

\end{ccRefFunction}
% ============================================================================
\begin{ccRefFunction}{symmetric_difference}

\ccDefinition

\ccInclude{CGAL/IO/Boolean_set_operations.h}

Each one of the following four functions computes the symmetric
difference between two given general polygons \ccc{p1} and \ccc{p2},
and inserts the resulting polygons with holes into an output container
through the output iterator \ccc{oi}. The value type of the
\ccc{OutputIterator} is \ccc{General_polygon_with_holes_2}.

\ccGlobalFunction{template <class Kernel, class OutputIterator>
OutputIterator symmetric_difference(const Polygon_2<Kernel> & p1,
                                    const Polygon_2<Kernel> & p2,
                                    OutputIterator oi);}
\ccGlue
\ccGlobalFunction{template <class Traits, class OutputIterator>
OutputIterator symmetric_difference(const General_polygon_2<Traits> & p1,
                                    const General_polygon_2<Traits> & p2,
                                    OutputIterator oi);}

\ccGlobalFunction{template <class Traits, class OutputIterator>
OutputIterator symmetric_difference(const Traits::Polygon_2 & p1,
                                    const Traits::Polygon_2 & p2,
                                    OutputIterator oi);}
\ccGlue
\ccGlobalFunction{template <class Traits, class OutputIterator>
OutputIterator symmetric_difference(const Traits::Polygon_2 & p1,
                                    const Traits::Polygon_with_holes_2 & p2,
                                    OutputIterator oi);}
\ccGlue
\ccGlobalFunction{template <class Traits, class OutputIterator>
OutputIterator symmetric_difference(const Traits::Polygon_with_holes_2 & p2,
                                    const Traits::Polygon_2 & p1,
                                    OutputIterator oi);}
\ccGlue
\ccGlobalFunction{template <class Traits, class OutputIterator>
OutputIterator symmetric_difference(const Traits::Polygon_with_holes_2 & p1,
                                    const Traits::Polygon_with_holes_2 & p2,
                                    OutputIterator oi);}

\ccGlobalFunction{template <class Traits, class OutputIterator>
OutputIterator symmetric_difference(const Traits::Polygon_2 & p1,
                                    const Traits::Polygon_2 & p2,
                                    OutputIterator oi,
                                    Traits & traits);}
\ccGlue
\ccGlobalFunction{template <class Traits, class OutputIterator>
OutputIterator symmetric_difference(const Traits::Polygon_2 & p1,
                                    const Traits::Polygon_with_holes_2 & p2,
                                    OutputIterator oi,
                                    Traits & traits);}
\ccGlue
\ccGlobalFunction{template <class Traits, class OutputIterator>
OutputIterator symmetric_difference(const Traits::Polygon_with_holes_2 & p2,
                                    const Traits::Polygon_2 & p1,
                                    OutputIterator oi,
                                    Traits & traits);}
\ccGlue
\ccGlobalFunction{template <class Traits, class OutputIterator>
OutputIterator symmetric_difference(const Traits::Polygon_with_holes_2 & p1,
                                    const Traits::Polygon_with_holes_2 & p2,
                                    OutputIterator oi,
                                    Traits & traits);}

\ccSeeAlso
\ccRefIdfierPage{CGAL::do_intersect}\\
\ccRefIdfierPage{CGAL::intersection}\\
\ccRefIdfierPage{CGAL::join}\\
\ccRefIdfierPage{CGAL::difference}

\end{ccRefFunction}
\ccRefPageEnd
