\ccRefPageBegin

\begin{ccRefConcept}{GeneralPolygon_2}

% \ccTypes
% %=======
% \ccNestedType{Size}{number of edges size type.}
% \ccGlue
% \ccNestedType{Vertex_iterator}{a bidirectional iterator over the
%        vertices of the polygonal region. Its value-type is \ccc{Vertex}.}
% \ccGlue
% \ccNestedType{X_monotone_curve_2}{represents a planar (weakly) $x$-monotone
%   curve. The type of the geometric mapping of the polygonal edges.}
% \ccGlue
% \ccNestedType{Curve_const_iterator}{a bidirectional iterator over the
%   geometric mapping of the polygon edges. Its value type is
%   \ccc{X_monotone_curve_2}.}

\ccDefinition
% ===========
A model of this concept represents a simple basic general-polygon. The
concept requires the ability to access the geometric mapping of the
edges of the polygon. These curves must be $x$-monotone curves. Curves
of non-adjacent edges do not intersect in their interiors. The
vertices of such a polygon can be ordered clockwise or counterclockwise. 

\ccCreation
\ccCreationVariable{polygon}

\ccThree{Edge_const_iterator~~~}{}{\hspace*{7cm}}
\ccThreeToTwo

\ccConstructor{GeneralPolygon_2();}{default constructor.}
\ccGlue
\ccConstructor{GeneralPolygon_2(GeneralPolygon_2 other);}
{copy constructor.}
\ccGlue
\ccMethod{GeneralPolygon_2  operator=(other);}{assignment operator.}
% \ccGlue
% \ccConstructor{template <class InputIterator>
% GeneralPolygon_2(InputIterator begin, InputIterator end);}
% {constructs a general polygon from a given range of curves.}

% \ccOperations
% %====================================
% 
% \ccMethod{Vertex_iterator vertices_begin();} {returns the begin
% iterator of the vertices.}
% \ccGlue
% \ccMethod{Vertex_iterator vertices_end();} {returns the past-the-end
% iterator of the vertices.}
% 
% \ccMethod{Curve_const_iterator curves_begin();} {returns the begin
% iterator of the curves.}
% \ccGlue
% \ccMethod{Curve_const_iterator curves_end();} {returns the past-the-end
% iterator of the curves.}
% 
% \ccMethod{Size size();} {returns the number of edges.}

\ccPredicates
% ===========
\ccMethod{bool is_simple();}{returns \ccc{true} if the polygonal
region is simple, and \ccc{false} otherwise. Used as precondition for
some of the operations.}
% \ccMethod{bool is_convex();}{returns \ccc{true} if the polygonal
% region is convex, and \ccc{false} otherwise.}
% \ccMethod{Orientation orientation();} {returns the orientation of
% the polygonal region. If the number of vertices $\ccc{size()} < 3$ then
% \ccc{COLLINEAR} is returned. 
% \ccPrecond{ \ccStyle{is_simple()}.}}

\ccHasModels
% ==========
\ccc{CGAL::Polygon_2<Kernel_2>}

\end{ccRefConcept}

\ccRefPageEnd
