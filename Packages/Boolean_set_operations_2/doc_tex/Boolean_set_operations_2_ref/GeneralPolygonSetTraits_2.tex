\ccRefPageBegin

\begin{ccRefConcept}{GeneralPolygonSetTraits_2}

\ccThree{GeneralPolygonSetTraits_2}{General_polygon_with_holes_2}{}
\ccThreeToTwo

\ccDefinition
% ===========
This concept defines the minimal set of geometric predicates needed to
perform the Boolean-set operations. It refines the $x$-monotone
arrangement-traits concept. In addition to the \ccc{Point_2} and
\ccc{X_monotone_curve_2} types defined in the generalized concept, it defines
a type that represents a general polygon and another one that represents
general polygon with holes. It also requires operations that operate on these
types.

\ccRefines
\ccc{ArrangementXMonotoneTraits_2}

\ccTypes
% ======

\ccNestedType{Polygon_2}{represents a (general) polygon.}
\ccGlue
\ccNestedType{Polygon_with_holes_2}{represents a (general) polygon
  with holes.}

\ccNestedType{Curve_const_iterator}
{A const iterator of curves. Its value type is const
\ccc{X_monotone_curve_2}.}

\ccHeading{Functor Types}
% =======================

\ccThree{Construct_polygon_2}{}{\hspace*{\ccwComment}}
\ccThreeToTwo

\ccNestedType{Construct_general_polygon_2}
{a functor that constructs a (general) polygon from a range of
$x$-monotone curves. It uses the operator\\
 \ccc{void operator() (InputIterator begin, Input iterator end,
                       Polygon_2 & pgn)}, \\
 parameterized by the \ccc{InputIterator} type.}

\ccNestedType{Construct_curves_2}
{a functor that returns a pair that consists of the begin and
 past-the-end iterators of the $x$ monotone curves of the boundary of
 a given (general) polygon. It uses the operator\\
 \ccc{std:pair<Curve_const_iterator,
               Curve_const_iterator>
  operator() (const Polygon_2 & pgn)}.}

\ccCreation
\ccCreationVariable{traits}
% =========================

\ccConstructor{GeneralPolygonSetTraits_2();}{default constructor.}
\ccGlue
\ccConstructor{GeneralPolygonSetTraits_2(GeneralPolygonSetTraits_2 other);}
{copy constructor}
\ccGlue
\ccMethod{GeneralPolygonSetTraits_2  operator=(other);}{assignment operator.}


\ccHeading{Accessing Functor Objects}
%====================================

\ccMethod{Construct_polygon_2 construct_polygon_2_object();}
{returns a functor that constructs a (general) polygon.}

\ccMethod{Construct_curves_2 construct_curves_2_object();}
{returns a functor that obtains the curves of a (general) polygon.}

\ccPredicates
% ===========
\ccMethod{bool is_simple(Polygon_2 & pgn);}
{returns \ccc{true} if the (general) polygon \ccc{pgn} is simple, and
\ccc{false} otherwise. Used as precondition for some of the operations.}

\ccHasModels
%===========
\ccc{CGAL::Polygon_set_traits_2<Kernel>}

\ccSeeAlso
% ========
\ccSeeAlso
  \ccc{Arrangement_2}\lcTex{(\ccRefPage{Arrangement_2})}\\
  \ccc{ArrangementXMonotoneTraits_2}\lcTex{(\ccRefPage{ArrangementXMonotoneTraits_2})}

\end{ccRefConcept}

\ccRefPageEnd
