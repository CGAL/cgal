% +------------------------------------------------------------------------+
% | CGAL Reference Manual: Reference manual for Qt_widget.tex
% +------------------------------------------------------------------------+
% |
% | 03.01.2001  Radu Ursu
% | 
% | \RCSdef{\qtwidgetRev}{$Revision$}
% | \RCSdefDate{\qtwidgetDate}{$Date$}
% +------------------------------------------------------------------------+

% +-----------------------------------------------------+
\begin{ccRefClass}{Qt_widget_view}

\ccDefinition

Views are classes that draw on the drawing area of a \ccc{Qt_widget}. The
class \ccRefName\ is a base class for views and provides empty event
handlers. 

\ccInclude{CGAL/IO/Qt_widget_view.h}
\ccGlue

\ccCreation
\ccCreationVariable{view}
\ccSetTwoColumns{Qt_widget_view}{}

\ccConstructor{Qt_widget_view();}{The default constructor.}
\ccSetThreeColumns{const_iterator}{container.begin() const;}{}

\ccMethod{virtual void mousePressEvent(QMouseEvent *, Qt_widget&);}{}
\ccGlue
\ccMethod{virtual void mouseReleaseEvent(QMouseEvent *, Qt_widget&);}{}
\ccGlue
\ccMethod{virtual void wheelEvent(QMouseEvent *, Qt_widget&);}{}
\ccGlue
\ccMethod{virtual void mouseDoubleClickEvent(QMouseEvent *, Qt_widget&);}{}
\ccGlue
\ccMethod{virtual void mouseMoveEvent(QMouseEvent *, Qt_widget&);}{}
\ccGlue
\ccMethod{virtual void keyPressEvent(QKeyEvent *, Qt_widget&);}{}
\ccGlue
\ccMethod{virtual void keyReleaseEvent(QMouseEvent *, Qt_widget&);}{}
\ccGlue
\ccMethod{virtual void enterEvent(QEvent *, Qt_widget&);}{}
\ccGlue
\ccMethod{virtual void leaveEvent(QEvent *, Qt_widget&);}{These virtual functions can
be overloaded in the derived class. They are called by the \ccc{Qt_widget} 
to which the view is attached.}
\ccGlue

\ccHeading{Public Slots}

\ccMethod{virtual void draw(Qt_widget& q)=0;}{This virtual member function
 must be overloaded. This is where the drawing code goes. The argument \ccc{q} 
refers to the widget the view can draw on.}

\ccExample
The following example of a view draws the points of a triangulation in green.

\begin{ccExampleCode}
#include <CGAL/IO/Qt_widget_view.h>
#include <qobject.h>


namespace CGAL {

template <class T>
class Qt_view_show_points : public Qt_widget_view {
public:
  typedef typename T::Point             Point;
  typedef typename T::Segment           Segment;
  typedef typename T::Vertex            Vertex;
  typedef typename T::Vertex_iterator   Vertex_iterator;

  Qt_view_show_points(T &t) : tr(t){};

  void draw_view(Qt_widget &widget)
  {
    Vertex v;
    Vertex_iterator it = tr.vertices_begin(), 
                beyond = tr.vertices_end();
    widget << CGAL::GREEN << CGAL::PointSize (3) << CGAL::PointStyle (CGAL::DISC);
    while(it != beyond)
    {
      v = *it;
      widget << v.point();
      ++it;
    }
  };
private:
  T     &tr;
};//end class 
} // namespace CGAL
\end{ccExampleCode}

\end{ccRefClass}
% +-----------------------------------------------------+
% EOF








