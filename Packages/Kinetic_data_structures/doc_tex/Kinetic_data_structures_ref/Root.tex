% +------------------------------------------------------------------------+
% | Reference manual page: RootEnumerator.tex
% +------------------------------------------------------------------------+
% | 20.03.2005   Author
% | Package: Kinetic_data_structures
% | 
\RCSdef{\RCSRootEnumeratorRev}{$Revision$}
\RCSdefDate{\RCSRootEnumeratorDate}{$Date$}
% |
%%RefPage: end of header, begin of main body
% +------------------------------------------------------------------------+


\begin{ccRefConcept}{Root}

%% \ccHtmlCrossLink{}     %% add further rules for cross referencing links
%% \ccHtmlIndexC[concept]{} %% add further index entries

\ccDefinition
  
The concept \ccRefName\ for the values used to represent roots of functions and times.


\ccCreation
\ccCreationVariable{re}  %% choose variable name

\ccConstructor{Root();}{default constructor. The value is undefined.}
\ccConstructor{Root(NT v);}{Construct a root from a some number. The number types supported must include \ccc{double} and the function coefficient type.}


\ccOperations 

Roots must support all binary comparisons with other roots as well as
(possibly through constructors) with the constant 0. 


\ccGlobalFunction{std::ostream &operator<<(std::ostream &, Root);}{Write the root in some human readable format.}

\ccGlobalFunction{double to_double(Root);}{Return a double approximation of the root value.}

\ccGlobalFunction{std::pair<double, double> to_interval(Root);}{Return an interval containing the root value.}


\ccGlobalFunction{Root infinity<Root>();}{This function template must
be specialized to return a valid value which will be used to represent
infinity. Alternatively, \ccc{std::numeric_limits<Root>} could be
specialized and the \ccc{std::numeric_limits<Root>::infinity()} value
\ccc{std::numeric_limits<Root>::max()} value will be used. }


\ccHasModels

\ccc{double}, \ccc{CGAL::POLYNOMIAL/Simple_interval_root.h}.

\ccSeeAlso

FunctionKernel, RootEnumerator.




\end{ccRefConcept}

% +------------------------------------------------------------------------+
%%RefPage: end of main body, begin of footer
% EOF
% +------------------------------------------------------------------------+

