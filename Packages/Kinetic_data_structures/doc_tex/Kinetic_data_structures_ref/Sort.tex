% +------------------------------------------------------------------------+
% | Reference manual page: Sort.tex
% +------------------------------------------------------------------------+
% | 27.03.2005   Author
% | Package: KDS
% | 
\RCSdef{\RCSSortRev}{$Revision$}
\RCSdefDate{\RCSSortDate}{$Date$}
% |
%%RefPage: end of header, begin of main body
% +------------------------------------------------------------------------+


\begin{ccRefClass}{KDS::Sort<Traits>}  %% add template arg's if necessary

%% \ccHtmlCrossLink{}     %% add further rules for cross referencing links
%% \ccHtmlIndexC[class]{} %% add further index entries

\ccDefinition
  
The class \ccRefName\ maintains a sorted list of objects. It is the
simplest kinetic data structure provided and is a good place to start
when looking at the basics of implementing a kinetic data
structure. 

Note that it does not expose any programming interface, as it is meant
mostly as an example.

\ccInclude{CGAL/KDS/Sort.h}

\ccCreation
\ccCreationVariable{s}  %% choose variable name

\ccConstructor{Sort(Traits tr);}{The basic constructor.}

\ccSeeAlso

\ccc{CGAL::KDS::Ref_counted},
\ccc{CGAL::KDS::Delaunay_triangulation_2}.
\ccc{CGAL::KDS::Simple_kds_base}.

\ccExample


\ccIncludeExampleCode{Kinetic_data_structures/sort.C}| 


\end{ccRefClass}

% +------------------------------------------------------------------------+
%%RefPage: end of main body, begin of footer
% EOF
% +------------------------------------------------------------------------+

