% +------------------------------------------------------------------------+
% | Reference manual page: natural_neighbor_coordinates_2.tex
% +------------------------------------------------------------------------+
% | 
% | August 2003  Julia Floetotto
% | Package:   Interpolation
% | 
% |
% +------------------------------------------------------------------------+

%\renewcommand{\ccRefPageBegin}{\begin{ccAdvanced}}
%\renewcommand{\ccRefPageEnd}{\end{ccAdvanced}}
\begin{ccRefFunction}{natural_neighbor_coordinates_2}  %% add template arg's if necessary
\ccIndexSubitemBegin{interpolation}{natural_neighbor_coordinates_2}

\ccDefinition
  
The function \ccRefName\ computes natural neighbor coordinates, also
called Sibson's coordinates, for $2D$ points provided a two-dimensional
triangulation and a query point inside the convex hull of the vertices
of the triangulation.

\ccInclude{CGAL/natural_neighbor_coordinates_2.h}


\ccFunction{template <class Dt, class OutputIterator, class Traits>
  std::pair< OutputIterator, typename Traits::FT >
  natural_neighbor_coordinates_2(const Dt& dt, const typename
  Traits::Point_2& p, OutputIterator out, const Traits& traits,
  typename Dt::Face_handle start = typename Dt::Face_handle(NULL));} {
  computes the natural neighbor coordinates for \ccc{p} with respect
  to the points in the two-dimensional Delaunay triangulation \ccc{dt}.
  The template class \ccc{Dt} should be of type
  \ccc{Delaunay_triangulation_2}.  The value type of the
  \ccc{OutputIterator} is a pair of \ccc{Dt::Point_2} and the
  coordinate value of type \ccc{Dt::Geom_traits::FT}. The sequence of
  point-coordinate pairs that is computed by the function is placed
  starting at \ccc{out}. The function returns a pair with an iterator
  that is placed past-the-end of the resulting sequence of
  point-coordinate pairs and the normalization factor of the
  coordinates.
  \ccPrecond %
  \ccc{p} lies in the convex hull of the
  points in \ccc{dt}.
  }  

\ccFunction{template < class Dt, class OutputIterator > std::pair<
  OutputIterator, typename Dt::Geom_traits::FT >
  natural_neighbor_coordinates_2(const Dt& dt, const typename
  Dt::Geom_traits::Point_2& p, OutputIterator out, typename
  Dt::Face_handle start = typename Dt::Face_handle(NULL));} 
%
{The same as above. This function instantiates the template parameter
  \ccc{Traits} to be \ccc{Dt::Geom_traits}.}


\ccFunction{template <class Dt, class OutputIterator, class Traits,
  class EdgeIterator > std::pair< OutputIterator, typename Traits::FT
  > natural_neighbor_coordinates_2(const Dt& dt, const typename
  Traits::Point_2& p, OutputIterator out, EdgeIterator hole_begin,
  EdgeIterator hole_end, const Traits& traits);} { The same as above.
  \ccc{hole_begin} and \ccc{hole_end} determines the iterator range
  over the boundary edges of the conflict zone of \ccc{p} in the
  triangulation. It is the result of the function
  \ccc{T.get_boundary_of_conflicts(p,std::back_inserter(hole),
    start);}, see \ccc{Delaunay_triangulation_2}.}

\ccFunction{template <class Dt, class OutputIterator> std::pair<
  OutputIterator, typename Dt::Geom_traits::FT >
  natural_neighbor_coordinates_2(const Dt& dt, typename
  Dt::Vertex_handle vh, OutputIterator out);}{This function computes
  the natural neighbor coordinates of the point contained in the
  vertex referenced by \ccc{vh}, \ccc{vh->point()}, with respect to
  \ccc{Dt} excluding \ccc{vh->point()}. The same as above for the
  remaining parameters.}

\ccHeading{Requirements}
\begin{enumerate}
\item \ccc{Dt} are equivalent to the class
  \ccc{Delaunay_triangulation_2}.
\item \ccc{OutputIterator::value_type} is equivalent to
  \ccc{std::pair<Dt::Point_2, Dt::Geom_traits::FT>}, i.e.\ a pair
  asscociating a point and its natural neighbor coordinate.
\item  \ccc{Traits} is a model of the concept \ccc{DelaunayTriangulationTraits_2} %\ccIndexMainItem[c]{DelaunayTriangulationTraits_2}.
  Only the following members of this traits class are used:
  \begin{itemize}
  \item \ccc{Construct_circumcenter_2}
  \item \ccc{FT}
  \item \ccc{Point_2}
  \item \ccc{construct_circumcenter_2_object}
  \item[] Additionally, \ccc{Traits} must meet the requirements for
    the traits class of the \ccc{polygon_area_2} function, notably,
  \item \ccc{Compute_area_2}
  \item \ccc{Construct_triangle_2}
  \item \ccc{compute_area_2_object}
  \item \ccc{construct_triangle_2_object}
  \end{itemize}
\end{enumerate}

\ccSeeAlso
\ccRefIdfierPage{CGAL::linear_interpolation.h} \\
\ccRefIdfierPage{CGAL::Sibson_c1_interpolation.h}\\ 
\ccRefIdfierPage{CGAL::surface_coordinates_3.h} \\
\ccRefIdfierPage{CGAL::regular_neighbor_coordinates_2.h} 

\ccImplementation This function computes the areas stolen from the
Voronoi cells of points in \ccc{dt} by the insertion of \ccc{p}. The
total area of the Voronoi cell of \ccc{p} is also computed and
returned by the function.

\ccIndexSubitemEnd{Interpolation}{natural_neighbor_coordinates_2}
\end{ccRefFunction}
%\renewcommand{\ccRefPageBegin}{}
%\renewcommand{\ccRefPageEnd}{}

% +------------------------------------------------------------------------+
% RefPage: end of main body, begin of footer
% EOF
% +------------------------------------------------------------------------+

