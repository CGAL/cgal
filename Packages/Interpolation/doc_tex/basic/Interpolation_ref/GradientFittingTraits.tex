% +------------------------------------------------------------------------+
% | Reference manual page: GradientFittingTraits.tex
% +------------------------------------------------------------------------+
% | 06.04.2000   Author
% | Package: Package
% | 
\RCSdef{\RCSGradientFittingTraitsRev}{$Revision$}
\RCSdefDate{\RCSGradientFittingTraitsDate}{$Date$}
% |
%%RefPage: end of header, begin of main body
% +------------------------------------------------------------------------+


\begin{ccRefConcept}{GradientFittingTraits}

%% \ccHtmlCrossLink{}     %% add further rules for cross referencing links
%% \ccHtmlIndexC[concept]{} %% add further index entries

\ccDefinition
  
The function \ccc{sibson_gradient_fitting} is parameterized by a
traits class that defines the primitives used by the algorithm.   The
concept \ccRefName\ defines this common set of requirements.


\ccTypes \ccNestedType{FT}{The number type.}  \ccNestedType{Point}{}
\ccNestedType{Vector}{} \ccNestedType{Aff_transformation}{defines a
  matrix type.
  Must provide the following member functions~: \\
  \ccc{Aff_transformation tr.inverse ()} which gives the inverse
  transformation, and \\
  \ccc{Aff_transformation tr.transform( Vector v)} which returns the
  multiplication of \ccc{tr} with \ccc{v}. %, and, \\
  %the constructor \ccc{Aff_transformation(SCALING, FT scale)} which introduces a 
  %scaling by a scale factor \ccc{scale}. 
  }



\ccNestedType{Construct_vector} {A constructor object for
  \ccc{Vector}.
  Provides~: \\
  \ccc{Vector operator() (Point a, Point b)} which produces the vector
  \ccc{b -
    a} and \\
  \ccc{Vector operator() (Null_vector NULL_VECTOR)} which introduces
  the null vector.}
\ccNestedType{Construct_scaled_vector}{Constructor object
  \ccc{Vector}.
  Provides~: \\
  \ccc{Vector operator() (Vector v,FT scale)} which produces the
  vector v scaled by a factor scale.}
\ccNestedType{Construct_null_matrix}{Constructor object for
  \ccc{Aff_transformation}. Provides~: \\
  \ccc{Aff_transformation operator()()} which introduces an affine
  transformation whose matrix has only zero entries.}
\ccNestedType{Construct_scaling_matrix}{Constructor object for
  \ccc{Aff_transformation}. Provides~: \\
  \ccc{Aff_transformation operator()(FT scale)()} which introduces a
  scaling by a scale factor \ccc{scale}.}
\ccNestedType{Construct_sum_matrix}{Constructor object for
  \ccc{Aff_transformation}. Provides~: \\
  \ccc{Aff_transformation operator()(Aff_transformation tr1,
    Aff_transformation tr2)} which returns the sum of the two matrices
  representing \ccc{tr1} and \ccc{tr2}.}
\ccNestedType{Construct_outer_product} {Constructor object for
  \ccc{Aff_transformation}. Provides~: \\
  \ccc{Aff_transformation operator()(Vector v)} which returns the
  outerproduct, i.\ e.\ the quadratic matrix \ccc{v}$^t$ \ccc{v}.}

\ccCreation
\ccCreationVariable{traits}  %% choose variable name
%A default constructor, a copy constructor
% and an assignement operator are required. 
%Note that further constructors
%can be provided. 
%%\ccThree{InterpolationTraits}{traits=gtrxx  }{}
%\ccConstructor{InterpolationTraits();}{default constructor.}
%\ccGlue
%\ccConstructor{InterpolationTraits(InterpolationTraits gtr);}
%{Copy constructor}
%\ccMethod{InterpolationTraits operator=(InterpolationTraits gtr);}
%{Assignment operator.}

\ccOperations
\ccHeading{Predicate functions}
The following functions that create instances of the above predicate object
types must exist.
\ccMethod{Construct_vector construct_vector_object();}{}
\ccGlue
\ccMethod{Construct_scaled_vector construct_scaled_vector_object();}{}
\ccGlue
\ccMethod{Construct_null_matrix construct_null_matrix_object();}{}
\ccGlue
\ccMethod{Construct_sum_matrix construct_sum_matrix_object();}{}
\ccGlue
\ccMethod{Construct_outer_product construct_outer_product_object();}{}

\ccHasModels
\ccc{CGAL::Interpolation_gradient_fitting_traits_2<K>} \\

\ccSeeAlso
%\ccRefIdfierPage{CGAL::linear_interpolation} \\
\ccRefIdfierPage{CGAL::sibson_c1_interpolation} \\
\ccRefIdfierPage{CGAL::sibson_gradient_fitting} \\
\ccRefIdfierPage{CGAL::farin_c1_interpolation} \\
\ccRefIdfierPage{CGAL::quadratic_interpolation} \\
%\ccRefIdfierPage{CGAL::natural_neighbor_coordinates_2}\\
%\ccRefIdfierPage{CGAL::regular_neighbor_coordinates_2} \\
%\ccRefIdfierPage{CGAL::surface_neighbor_coordinates_2_3}\\
\end{ccRefConcept}

% +------------------------------------------------------------------------+
%%RefPage: end of main body, begin of footer
% EOF
% +------------------------------------------------------------------------+

