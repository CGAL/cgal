\chapter{2D Conforming Triangulations\\ and Meshes}
\label{user_chapter_2D_Meshes}

\minitoc

This package implements Shewchuk's algorithm to construct conforming
triangulations and 2D meshes.
Conforming triangulations will be described in the
section~\ref{sec:Mesh_2_conforming_triangulation} and meshes in the
section~\ref{sec:Mesh_2_meshes}.


%\section{Definitions}
%\label{sec:Mesh_2_definitions}

%\begin{description}

%\item[Conforming triangulation] A conforming triangulation is a refinement
%  of a constrained triangulation, obtained by inserting points (named
%  \emph{Steiner points}) on constrained segments, so that the resulting
%  triangulation is a \emph{Delaunay} or a \emph{Gabriel}
%  triangulation. Delaunay triangulations are defined in the \cgal\ user
%  manual of \ccc{Triangulation_2} package. A Gabriel triangulation of a
%  sets of points and constrained segments is a constrained Delaunay
%  triangulation with the extra following \emph{Gabriel property}: the
%  \emph{diametral circle} of each constrained segments contains no point in
%  its interior. The Gabriel property is stronger that the Delaunay property:
%  a conforming Gabriel triangulation is also a conforming Delaunay
%  triangulation.
%\item[Quality mesh] In this package, a quality mesh is a contrained
%  Delaunay triangulation of a domain containing only triangles whose shapes
%  and sizes satisfy several criterias.
%\end{description}


\section{Conforming triangulations}
\label{sec:Mesh_2_conforming_triangulation}

\subsection{Definition}
\label{sec:Mesh_2_conforming_defition}

A constrained Delaunay triangulation is said to be a \emph{conforming
  Delaunay triangulation} if every constrained edge is a Delaunay
edge, where we call  {\em Delaunay edge}  an edge that would  appear 
in the (unconstrained) Delaunay triangulation of the set of
vertices. Because any edge in a constrained Delaunay triangulation
is either a Delaunay edge or a constrained edge, a 
conforming   Delaunay triangulation is in fact 
a Delaunay triangulation. The only difference
is that  some of the edges  are marked as constrained edges.

A constrained Delaunay triangulation is said to be a \emph{conforming
  Gabriel triangulation} if every constrained edge is a Gabriel edge,
meaning that its diametral circle includes no vertex of the
triangulation in its interior. The Gabriel property is stronger that
the Delaunay property and each Gabriel edge is a Delaunay edge. Thus
conforming Gabriel triangulations are conforming Delaunay
triangulations.

Any contrained Delaunay triangulation can be refined into a conforming
Delaunay or conforming Gabriel triangulation by adding vertices,
called \emph{Steiner vertices}, on constrained edges until they are
cut into subconstraints small enough to be Delaunay or Gabriel edges.

\subsection{Building conforming triangulations}
\label{sec:Mesh_2_building_conforming}

Constrained Delaunay triangulations can be refined into
conforming triangulations 
by two global functions: \\
\ccc{template<class CDT> void make_conforming_Delaunay_2 (CDT& t)} \\
\ccc{template<class CDT> void make_conforming_Gabriel_2 (CDT& t)}. 

 In both cases, the
template parameter \ccc{CDT} must be instanciated by a
constrained Delaunay triangulation class.
Such a class 
can be either  a plain constrained Delaunay triangulation
(\ccc{Constrained_Delaunay_triangulation_2<Gt, Tds>} ) or
a derived class such as 
\ccc{Constrained_triangulation_plus_2<CDT2>}  or 
\ccc{Triangulation_hierarchy_2<CDT2>} 
where \ccc{CDT2} is a
\ccc{Constrained_Delaunay_triangulation_2<Gt, Tds>}.

There are some requirements on the geometric traits od the 
constrained Delaunay triangulation used to instantiate
the parameter \ccc{CDT}.
In case of \ccc{make_conforming_Delaunay_2 (CDT& t)}
he geometric traits  has to be a model of the concept 
\ccc{ConformingDelaunayTriangulationTraits_2} which refines
the traits \ccc{ConstrainedDelaunayTriangulationTraits_2}.
In case of  \ccc{make_conforming_Gabriel_2 (CDT& t),
the geometric traits  has to be a model of the concept
\ccc{ConformingGabrielTriangulationTraits_2}
which further refines \ccc{ConformingDelaunayTriangulationTraits_2}.

The constrained Delaunay triangulation \ccc{t} is passed by reference
and is refined into a  conforming Delaunay triangulation or 
a conforming Gabriel triangulation  by adding
vertices  that is the triangulation is modified. If you want to keep
the original triangulation, please make a copy of it.

The advanced user can also used the class
\ccc{Conforming_Delaunay_triangulation_2<CDT>} to refine
a constrained Delaunay triangulation into
a conforming Delaunay or a conforming Gabriel triangulation.

\subsection{Example: making a triangulation conforming Delaunay and then
  conforming Gabriel}
\label{sec:Mesh_2_example_making_conforming}

This example inserts several segments in a constrained Delaunay
triangulation, makes it conforming Delaunay, and then conforming
Gabriel. At each step, the number of vertices of the triangulation is
printed.

\ccIncludeExampleCode{Mesh_2/conform.C}

\section{Meshes}
\label{sec:Mesh_2_meshes}


\subsection{Definition}
\label{sec:Mesh_2_meshes_definition}

A mesh is a partition of a given domain into simplices, whose shapes
and sizes satisfy several criterias.

The domain to be mesh is bounded and can include internal
constraints to be respected.
Such a domain is defined by a \emph{planar straight
line graphes}, PSLG for short,  and a set of seed points.

A PSLG is a one dimensional simplicial complex, that
is a set of vertices and segments such that : \\
- the endpoints of any segment in the set are vertices of the set, \\
- two segments in the set are either disjoint or share a vertex of the 
set. \\
The segments of the PSLG described the boundaries and the internal
constraints of the domain.

The PSLG divides the plan into several connected components.
The domain is usually defined as  the union of the 
bounded connected components including 
the seed points.  Conversely, the seed points can be used to mark
the connected components which are outside the domain.




%Domains are unions of connected component of \emph{planar straight
%  line graphes} (PSLG), which are sets of vertices and segments such
%that all endpoints of every segments are in the set and that segments
%intersect only at end-points. PSLG can represented by a \cgal\ 
%constrained triangulation, whose constrained segments are the segments
%of the domain.

%By default, the domain to be meshed is the whole plane but the
%connected component of the infinite vertex. If the domain is not
%bounded by a polyline of constrained segments, this domain can be
%empty and the meshing algorithm will not do anything.

%However, one can define more precisely which connected components of
%the PSLG are in the domain by setting \ccc{seeds}. Seeds are a set of
%points, that are not in the set of vertices of the triangulation. A
%boolean marker tells if the seeds are in the domain or not, and the
%connected component of each seed is marked the same way. Anyway, the
%(infinite) connected component of the infinite vertex is always marked
%as the exterior of the domain. By default, the set of seeds is empty,
%and the domain is the union of all finite connected components of the
%PSLG.

\subsection{Shape and size criteria}
\label{sec:Mesh_2_criteria}

The shape criteria on triangles is a lower bound $B$ on the ratio
between the circumradius and the shortest edge length.
Such a bound implies  a lower bound  of $\arcsin{\frac{1}{2B}}$
on the  minimum angle of the triangle  and an upper
bound of $\pi - 2* \arcsin{\frac{1}{2B}}$ on the maximun angle.
Unfortunalty, the
terminaison of the algorithm is guaranted only if $B \ge \sqrt{2}$
which corresponds to a lower bound of $20.7$~degrees
on the angles.


%should be that the circumradius is
%lower than a bound $B$ times the shortest edge of the triangle. This
%is equivalent to say that the minimum angle of the triangle is greater
%than $\arcsin{\frac{1}{2B}}$. If not angles are smaller than $\theta$,
%then no angles are greater than $\pi - \theta$. Unfortunalty, the
%terminaison of the algorithm is guaranted only if $B \ge \sqrt{2}$.
%This package cannot, for the moment, insure that angles of meshes are
%greater than $20.7$~degrees.

 The size criteria can be any criteria that
tends to prefer small triangles.
The size bound  can be varying over the domain. 

Both types of criterias are defined
in a nested type \ccc{Is_bad} of the geometric traits class.


\subsection{The meshing algorithm}

Thus, the input to a  meshing problem is 
a PSLG,
a set of seeds describing the domain to be meshed
and  a set of valid size and shape criteria. 
The algorithm implemented in this package
starts with a Constrained Delaunay triangulation 
of the input PSLG  and
produces a mesh
using the Delaunay refinement method.

If all angles between incident segments of the input PSLG
are greater than $60$~degrees, and if the bound on the
circumradius/edge ratio is greater than $\sqrt{2}$
the algorithm is guaranteed to end up with a mesh
satisfying the size and shape criteria.

If some input 
angles are smaller than $60$~degrees, the algorithm
will end up with a mesh in which some triangles 
near small input angles violate the  criteria.
This is unavoidable. Indeed  small angles formed by input segments
cannot be suppressed. Furthermore, 
it  has been proved (\cite{s-mgdsa-00}),
that some domain  with small input angles
cannot be meshed with angles even smaller that the small input
angles.
Note that if the domain is a polygonal region
the resulting mesh will satisfy size and shape criteria
except for the small input angles.
Note also that though it is not guaranteed, the algorithm may succeed 
in achieving meshes with a lower angle bound
greater than $20.7$~degrees.




%that one cannot compute a triangular
%mesh the domain without adding even smaller angles in the mesh.

%because  it has been proved 
%that PSLG with small angles cannot be meshed without
%creating small angles.

% the criterias on angle and size are
%garanted to be fulfilled. If some input incident segments forme an
%angle smaller than $60$~degrees, these segments formed a
%\textit{cluster}. Near clusters, the algorithm cannot garanty the
%shape criterias. Of course small angles formed by input segments
%cannot be suppressed. What is more, if the domain is not a polygonal
%region, and includes angles smaller than $60$~degrees,
%\cite{s-mgdsa-00} has prooved that one cannot compute a triangular
%mesh the domain without inserting even smaller angles in the mesh.

%See~\cite{s-mgdsa-00} for details.


\subsection{Building meshes}
\label{sec:Mesh_2_building_meshes}

In this package, meshes are  obtained from
constrained Delaunay triangulation by calling the global function \\
\ccc{template<class CDT> void refine_Delaunay_mesh_2 (CDT &t, typename
  CDT::Geom_traits gt)}. \\
They can also be obtained by using the class
\ccc{Delaunay_mesh_2<CDT>} that derives from \ccc{CDT}. 
In both cases,
the template parameter \ccc{CDT} must be instanciated by a
constrained Delaunay triangulation class.
Such a class 
can be either  a plain constrained Delaunay triangulation
(\ccc{Constrained_Delaunay_triangulation_2<Gt, Tds>} ) or
a derived class such as 
\ccc{Constrained_triangulation_plus_2<CDT2>}  or 
\ccc{Triangulation_hierarchy_2<CDT2>} 
where \ccc{CDT2} is a
\ccc{Constrained_Delaunay_triangulation_2<Gt, Tds>}.

The geometric traits class of the instance of \ccc{CDT} 
has to be a model of the concept \ccc{DelaunayMeshTraits_2}.
The concept \ccc{DelaunayMeshTraits_2} refines the concept
\ccc{ConformingGabrielTriangulationTraits_2}
adding 
the geometric criteria that the triangles have to satisfy. 
\cgal provides models for this  concept such as:
\begin{itemize}
\item \ccc{Delaunay_mesh_traits_2<K>} defines a shape criteria that
  bounds the minimum angle of triangles.
\item \ccc{Delaunay_mesh_size_traits<K>} adds to the previous one a
  bound on the maximum edge lenght.
\end{itemize}

The class \ccc{Delaunay_mesh_2<CDT>} derives from \ccc{CDT} and has
several member functions to define the domain and mesh it. See example 
and the reference manual for details. 
Note that the insertin of verices and constraints
is not overwritten,
thus any insertion will break the size and shape guarantee
 of the mesh until a meshing
function is  called again.

%If, after a call to one of the
%meshing function, one inserts vertices of constrained edges, the
%triangulation is no longer guaranted to be meshed and the meshing
%function should be called again.

\subsection{Example using the global function and shape and size default
  criterias}

The following example inserts several segments in a contrained
triangulation, then mesh it using the global function
\ccc{refine_Delaunay_mesh_2} and the default traits class.

\ccIncludeExampleCode{Mesh_2/mesh_global.C}

\subsection{Example using the class \ccc{Delaunay_mesh_2<CDT>} and several
  criterias}

This example uses the class \ccc{Delaunay_mesh_2<CDT>} in order to refine
the mesh with new criterias, after the first meshing. One can use
\ccc{refine_Delaunay_mesh_2} twice, with different geometric traits class,
but it is less efficient, because some internal structures needed by the
algorithm are calculated twice.

\ccIncludeExampleCode{Mesh_2/mesh_class.C}

\todo{Exemple avec seeds.}

\todo{Exemple avec la fonction globale.}

%%% For emacs/AucTeX:
%%% Local Variables: ***
%%% mode:latex ***
%%% TeX-master: "user_manual.tex"  ***
%%% End: ***
