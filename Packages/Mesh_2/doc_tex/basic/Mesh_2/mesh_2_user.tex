\chapter{2D conforming triangulations\\ and meshes}
\label{user_chapter_2D_Meshes}

\minitoc

This package implements Shewchuk's algorithm to construct conforming
triangulations and 2D meshes.

Conforming triangulations will be described in the
section~\ref{sec:Mesh_2_conforming_triangulation} and meshes in the
section~\ref{sec:Mesh_2_meshes}.


%\section{Definitions}
%\label{sec:Mesh_2_definitions}

%\begin{description}

%\item[Conforming triangulation] A conforming triangulation is a refinement
%  of a constrained triangulation, obtained by inserting points (named
%  \emph{Steiner points}) on constrained segments, so that the resulting
%  triangulation is a \emph{Delaunay} or a \emph{Gabriel}
%  triangulation. Delaunay triangulations are defined in the \cgal\ user
%  manual of \ccc{Triangulation_2} package. A Gabriel triangulation of a
%  sets of points and constrained segments is a constrained Delaunay
%  triangulation with the extra following \emph{Gabriel property}: the
%  \emph{diametral circle} of each constrained segments contains no point in
%  its interior. The Gabriel property is stronger that the Delaunay property:
%  a conforming Gabriel triangulation is also a conforming Delaunay
%  triangulation.
%\item[Quality mesh] In this package, a quality mesh is a contrained
%  Delaunay triangulation of a domain containing only triangles whose shapes
%  and sizes satisfy several criterias.
%\end{description}


\section{Conforming triangulations}
\label{sec:Mesh_2_conforming_triangulation}

\subsection{Definition}
\label{sec:Mesh_2_conforming_defition}

A constrained Delaunay triangulation is said to be a \emph{conforming
  Delaunay triangulation} if every constrained edge is a Delaunay
edge, that is appears in the Delaunay triangulation of the set of
vertices. Thus a conforming Delaunay triangulation is a Delaunay
triangulation, where some edges are marked as constrained edges.

A constrained Delaunay triangulation is said to be a \emph{conforming
  Gabriel triangulation} if every constrained edge is a Gabriel edge,
meaning that its diametral circle includes no vertex of the
triangulation in its interior. The Gabriel property is stronger that
the Delaunay property and each Gabriel edge is a Delaunay edge. Thus
conforming Gabriel triangulations are conforming Delaunay
triangulations.

Any contrained Delaunay triangulation can be refined into a conforming
Delaunay or conforming Gabriel triangulation by adding vertices,
called \emph{Steiner vertices}, on constrained edges until they are
cut into subconstraints small enough to be Delaunay or Gabriel edges.

\subsection{Building conforming triangulations}
\label{sec:Mesh_2_building_conforming}

Conforming triangulations can be obtained by two global functions:
\ccc{template<class CDT> void make_conforming_Delaunay_2 (CDT& t)} and
\ccc{template<class CDT> void make_conforming_Gabriel_2 (CDT& t)}. The
template parameter \ccc{CDT} must be instanciated by a
\ccc{Constrained_Delaunay_triangulation_2<Gt, Tds>}, or a
\ccc{Constrained_triangulation_plus_2<CDT2>} or a
\ccc{Triangulation_hierarchy_2<CDT2>} where \ccc{CDT2} is a
\ccc{Constrained_Delaunay_triangulation_2<Gt, Tds>}. In order to test
the Delaunay or the Gabriel property, and to construct Steiner points,
the geometric traits class of \ccc{CDT} has to be a model of
\ccc{ConformingDelaunayTriangulationTraits_2}. It can be, for example,
any \cgal\ kernel.

The constrained Delaunay triangulation \ccc{t} is passed by reference
and is made conforming Delaunay or conforming Gabriel by adding
vertices, that is the triangulation is modified. If you want to keep
the original triangulation, please make a copy of it.

\subsection{Example: making a triangulation conforming Delaunay and then
  conforming Gabriel}
\label{sec:Mesh_2_example_making_conforming}

This example inserts several segments in a constrained Delaunay
triangulation, makes it conforming Delaunay, and then conforming
Gabriel. At each step, the number of vertices of the triangulation is
printed.

\ccIncludeExampleCode{Mesh_2/conform.C}

\section{Meshes}
\label{sec:Mesh_2_meshes}


\subsection{Definition}
\label{sec:Mesh_2_meshes_definition}

A mesh is a partition of a given domain into simplices, whose shapes
and sizes satisfy several criterias.

Domains are unions of connected component of \emph{planar straight
  line graphes} (PSLG), which are sets of vertices and segments such
that all endpoints of every segments are in the set and that segments
intersect only at end-points. PSLG can represented by a \cgal\ 
constrained triangulation, whose constrained segments are the segments
of the domain.

By default, the domain to be meshed is the whole plane but the
connected component of the infinite vertex. If the domain is not
bounded by a polyline of constrained segments, this domain can be
empty and the meshing algorithm will not do anything.

However, one can define more precisely which connected components of
the PSLG are in the domain by setting \ccc{seeds}. Seeds are a set of
points, that are not in the set of vertices of the triangulation. A
boolean marker tells if the seeds are in the domain or not, and the
connected component of each seed is marked the same way. Anyway, the
(infinite) connected component of the infinite vertex is always marked
as the exterior of the domain. By default, the set of seeds is empty,
and the domain is the union of all finite connected components of the
PSLG.

\subsection{Shape and size criteria}
\label{sec:Mesh_2_criteria}

The shape criteria on triangles should be that the circumradius is
lower than a bound $B$ times the shortest edge of the triangle. This
is equivalent to say that the minimum angle of the triangle is greater
than $\arcsin{\frac{1}{2B}}$. If not angles are smaller than $\theta$,
then no angles are greater than $\pi - \theta$. Unfortunalty, the
terminaison of the algorithm is guaranted only if $B \ge \sqrt{2}$.
This package cannot, for the moment, insure that angles of meshes are
greater than $20.7$~degrees. The size criteria can be any criteria that
tends to prefere small triangles. Both types of criterias are defined
in a nested type \ccc{Is_bad} of the geometric traits class.

If all angles between contrained segments of the initial triangulation
are greater than $60$~degrees, the criterias on angles and size are
garanted to be fulfilled. If some input incident segments forme an
angle smaller than $60$~degrees, these segments formed a
\textit{cluster}. Near clusters, the algorithm cannot garanty the
shape criterias. Of course small angles formed by input segments
cannot be suppressed. What is more, if the domain is not a polygonal
region, and includes angles smaller than $60$~degrees,
\cite{s-mgdsa-00} has prooved that one cannot compute a triangular
mesh the domain without inserting even smaller angles in the mesh.

See~\cite{s-mgdsa-00} for details.


\subsection{Building meshes}
\label{sec:Mesh_2_building_meshes}

In this package, meshes are constrained Delaunay triangulation whose
triangles satify shape and size criteria. They can be obtained from
constrained Delaunay triangulation by called the global function
\ccc{template<class CDT> void refine_Delaunay_mesh_2 (CDT &t, typename
  CDT::Geom_traits gt)}. They can also be obtained by using the class
\ccc{Delaunay_mesh_2<CDT>} that derives from \ccc{CDT}. In both cases,
the template parameter \ccc{CDT} has to be instanciated by
\ccc{Constrained_Delaunay_triangulation_2<Gt, Tds>}, or a
\ccc{Constrained_triangulation_plus_2<CDT2>} or a
\ccc{Triangulation_hierarchy_2<CDT2>} where \ccc{CDT2} is a
\ccc{Constrained_Delaunay_triangulation_2<Gt, Tds>}.

The geometric traits class of the instance of \ccc{CDT} should define
the geometric criteria that the triangles have to satisfy. That is why
it has to be a model of \ccc{DelaunayMeshTraits_2}. The latter concept 
has two models provided by this package:
\begin{itemize}
\item \ccc{Delaunay_mesh_traits_2<K>} defines a shape criteria that
  bounds the minimum angle of triangles.
\item \ccc{Delaunay_mesh_size_traits<K>} adds to the previous one a
  bound on the maximum edge lenght.
\item \todo{J'ai volontairement oubli\'e des mod\`eles.}
\end{itemize}

The class \ccc{Delaunay_mesh_2<CDT>} derives from \ccc{CDT} and has
several member functions to define the domain and mesh it. See example 
and the reference manual for details. If, after a call to one of the
meshing function, one inserts vertices of constrained edges, the
triangulation is no longer guaranted to be meshed and the meshing
function should be called again.

\subsection{Example using the global function and shape and size default
  criterias}

The following example inserts several segments in a contrained
triangulation, then mesh it using the global function
\ccc{refine_Delaunay_mesh_2} and the default traits class.

\ccIncludeExampleCode{Mesh_2/mesh_global.C}

\subsection{Example using the class \ccc{Delaunay_mesh_2<CDT>} and several
  criterias}

This example uses the class \ccc{Delaunay_mesh_2<CDT>} in order to refine
the mesh with new criterias, after the first meshing. One can use
\ccc{refine_Delaunay_mesh_2} twice, with different geometric traits class,
but it is less efficient, because some internal structures needed by the
algorithm are calculated twice.

\ccIncludeExampleCode{Mesh_2/mesh_class.C}

\todo{Exemple avec seeds.}

\todo{Exemple avec la fonction globale.}

%%% For emacs/AucTeX:
%%% Local Variables: ***
%%% mode:latex ***
%%% TeX-master: "user_manual.tex"  ***
%%% End: ***
