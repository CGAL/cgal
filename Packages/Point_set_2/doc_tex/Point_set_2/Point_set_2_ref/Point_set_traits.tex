\begin{ccRefConcept} {Point_set_traits}

\subsection{Requirements for the Point\_set traits class}

A point set traits class has to provide some primitives that are used by the point set class.
The following catalog lists the involved primitives.
The types used in the traits class must have a copy constructor and an assignment operator.

\ccCreationVariable{ps_traits}
\ccTypes

\ccNestedType{FT}%
       {The field type of the representation class of the point, segment, circle and line types.}

\ccNestedType{Point}%
       {The point type on which the point set operates.}

\ccNestedType{Circle}%
       {The circle type on which the point set operates.}


\ccHtmlLinksOff
       
\ccNestedType{Line}%
       {The line type on which the point set operates.}
       
\ccHtmlLinksOn

\ccNestedType{Segment}%
       {The segment type on which the point set operates.}
       
\ccNestedType{Compare_xy_2}%
       {The primitive must provide 
	\ccc{int operator()(const Point& p1,const Point& p2) const}
	and has to be derived from $leda\_cmp\_base<Point>$.
	The operator has to compare the points $p1$ and $p2$ lexicographically in
	$xy$ order. If $p1$ is smaller, the operator has to return -1, if $p1$
        is larger, the operator has to return 1, 0 otherwise.
       }

\ccNestedType{Compare_dist_2}%
       {The primitive must provide 
	\ccc{Comparison_result operator()(const Point& p1,const Point& p2, const Point& p3) const}.
	The operator has to compare the distances of points $p2$ and $p3$ to $p1$.
        The operator returns \ccc{SMALLER}, if $p2$ is closer (to $p1$), \ccc{LARGER} if $p3$ is closer,
        \ccc{EQUAL} otherwise.
       }      

 \ccNestedType{Orientation}%
	{The primitive must provide 
	\ccc{Orientation operator()(const Point& p1,const Point& p2,const Point& p3) const}.
	This operator has to return \ccc{COLLINEAR} if the 3 points are collinear,
	\ccc{LEFTTURN} if $p3$ lies on the left of the directed line through $p1$ and $p2$,
	\ccc{RIGHTTURN} otherwise.
	}
	
 \ccNestedType{Side_of_oriented_circle_2}%
	{The primitive must provide 
	 \ccc{Oriented_side operator()(const Point& p1,const Point& p2,const Point& p3, const Point& p4) const}.
	 This operator has to return \ccc{ON_POSITIVE_SIDE} if $p4$ lies on the positive side of the circle through
	 the points $p1$,$p2$ and $p3$, \ccc{ON_NEGATIVE_SIDE} if $p4$ lies on the negative side of the circle,
	 \ccc{ON_ORIENTED_BOUNDARY} otherwise.
	}
	
 \ccNestedType{Side_of_halfspace_2}%
	{The primitive must provide
	 \ccc{Orientation operator()(const Point& a,const Point& b, const Point& p3) const }.
	 This operator has to return \ccc{LEFTTURN} if $p3$ lies in the open halfspace $h$ orthogonal to vector
	 $b-a$ containing $b$ and having $a$ on its boundary, \ccc{COLLINEAR} if $p3$ is on the boundary of $h$ and
	 \ccc{RIGHTTURN} otherwise.
	}
	
 \ccNestedType{Segment_has_on_2}%
	{The primitive must provide
	\ccc{bool operator()(const Segment& seg, const Point& p) const }.
	This operator has to return true if $seg$ contains $p$, false otherwise.}
	
 \ccNestedType{Squared_distance}%
	{The primitive must provide 
	\ccc{FT operator()(const Point& p1,const Point& p2) const}.
	The operator has to return the squared distance from $p1$ to $p2$.}
	
 \ccNestedType{Squared_distance_to_line}%
	{The primitive must provide 
	\ccc{FT operator()(const Line& l,const Point& p) const}.
	The operator has to return the squared distance from $l$ to $p$.}
	
 \ccNestedType{Circle_bounded_side_2}%
	{The primitive must provide 
	\ccc{Bounded_side operator()(const Circle& c, const Point& p) const}.
	The operator has to return \ccc{ON_BOUNDED_SIDE} if $p$ lies on the bounded side of $c$,
	\ccc{ON_UNBOUNDED_SIDE} if $p$ lies on the unbounded side of $c$, \ccc{ON_BOUNDARY} otherwise.}
	
 \ccNestedType{Circle_center_2}%
	{The primitive must provide 
	\ccc{Point operator()(const Circle& c) const}.
	The operator has to return the center of $c$.}
	
 \ccNestedType{Construct_circle_2}%
	{This primitive constructs a circle. It must provide
	\ccc{Circle operator()(const Point& p1,const Point& p2, const Point& p3) const }.
	The points $p1$, $p2$ and $p3$ are on the boundary.} 
	
 \ccNestedType{Construct_segment_2}%
	{This primitive constructs a segment. It must provide
	\ccc{Segment operator()(const Point& p1,const Point& p2) const}.
	Point $p1$ is the source of the segment, $p2$ the target.}
	
 \ccNestedType{Construct_line_2}%
	{This primitive constructs a line. It must provide
	\ccc{Line operator()(const Point& p1,const Point& p2) const}.
	The line is constructed through $p1$ and $p2$.}

\end{ccRefConcept}
