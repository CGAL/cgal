% +------------------------------------------------------------------------+
% | Reference manual page: Convex_hull_3_ref/intro.tex
% +------------------------------------------------------------------------+
% | 10.5.2001   Susan Hert 
% | Package: Convex_hull_3
% | 
% |
% +------------------------------------------------------------------------+

%\clearpage
%\section{Reference Pages for 3D Convex Hulls}
\chapter{3D Convex Hulls}
\label{chap:convex_hull_3_ref}

A subset $S \subseteq \R^3$ is convex if for any two points $p$ and $q$
in the set the line segment with endpoints $p$ and $q$ is contained
in $S$. The convex hull\ccIndexMainItemDef{convex hull} of a set $S$ is 
the smallest convex set containing
$S$. The convex hull of a set of points $P$ is a convex 
polytope with vertices in $P$.  A point in $P$ is an extreme point 
(with respect to $P$)\ccIndexMainItemDef{extreme point} if it is a vertex 
of the convex hull of $P$.

\cgal\ provides functions for computing convex hulls in two, three 
and arbitrary dimensions as well as functions for testing if a given set of 
points in is strongly convex or not.  This chapter describes the functions
available for three dimensions. 


\ccHeading{Assertions}
\ccIndexSubitem{convex hull, 3D}{assertion flags}
\ccIndexSubitem{assertion flags}{convex hull, 3D}
The assertion flags for the convex hull and extreme point algorithms
use \ccc{CH} in their names (\textit{e.g.}, \ccc{CGAL_CH_NO_POSTCONDITIONS}).
For the convex hull algorithms, the postcondition
check tests only convexity (if not disabled), but not containment of the
input points in the polygon or polyhedron defined by the output points.
The latter is considered an expensive checking and can be enabled by
defining \ccc{CGAL_CH_CHECK_EXPENSIVE}%
\ccIndexAssertionFlagName{CGAL_CH_CHECK_EXPENSIVE}.


\ccHeading{Concepts}

%\ccRefConceptPage{ConvexHull3fromD_Rep}\\
\ccRefConceptPage{ConvexHullPolyhedron_3}\\
\ccRefConceptPage{ConvexHullPolyhedronFacet_3}\\
\ccRefConceptPage{ConvexHullPolyhedronHalfedge_3}\\
\ccRefConceptPage{ConvexHullPolyhedronVertex_3}\\
\ccRefConceptPage{ConvexHullTraits_3} \\
\ccRefConceptPage{IsStronglyConvexTraits_3}

\ccHeading{Traits Classes}

\ccRefIdfierPage{CGAL::Convex_hull_traits_3<R>}

\ccHeading{Convex Hull Functions}

\ccRefIdfierPage{CGAL::convex_hull_3} \\
\ccRefIdfierPage{CGAL::convex_hull_incremental_3} 

\ccHeading{Convexity Checking Function}

\ccRefIdfierPage{CGAL::is_strongly_convex_3} 

\clearpage

\lcHtml{\ccHeading{Alphabetical Listing of Reference Pages}}
