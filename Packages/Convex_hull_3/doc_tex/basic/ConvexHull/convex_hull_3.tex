\section{3D Convex Hull}
\label{sec:convex_hull_3}
\begin{ccPackage}{convex hull, 3D}

One can compute the convex hull of a set of points in three dimensions
in one of three ways in \cgal: using a static algorithm,
using an incremental construction algorithm, or using a
triangulation to get a fully dynamic computation.

\subsection{Example of Static Construction}
\ccIndexSubitem{convex hull, 3D}{static}
\ccIndexSubitem{convex hull, 3D}{quickhull}

The function 
\ccc{convex_hull_3}\ccIndexMainItem[C]{convex_hull_3} provides an 
implementation of the quickhull algorithm \cite{bdh-qach-96} for three 
dimensions\ccIndexMainItem{quickhull, 3D}.  There are two versions of this
function available, one that can be used when it is known that the output
will be a polyhedron (\textit{i.e.}, there are more than three points and
they are not all collinear) and one that handles all degenerate cases
and returns a \ccc{CGAL::Object}, which may be a point, a segment, a
triangle, or a polyhedron.  Both versions accept a range of input
iterators defining the set of points whose convex hull is to be computed
and a traits class defining the geometric types and predicates used in
computing the hull.

The following program computes the convex hull of a set of 250 random
points chosen from a sphere of radius 100.  It then determines if the 
resulting hull is a segment or a polyhedron.  

\ccIncludeExampleCode{Convex_hull_3/quickhull_3_ex.C}


\subsection{Example of Incremental Construction}
\ccIndexSubitem{convex hull, 3D}{incremental}

The function \ccc{convex_hull_incremental_3} %
\ccIndexMainItem[C]{convex_hull_incremental_3} provides an
interface similar to \ccc{convex_hull_3} for the $d$-dimensional 
incremental construction algorithm \cite{cms:fourresults-93}.  
implemented by the class \ccc{CGAL::Convex_hull_d<R>} that is specialized 
to three dimensions. This function accepts an iterator range over a set of
input points and returns a polyhedron, but it does not have a traits class
in its interface.  It uses the kernel
class \ccc{Kernel} used in the polyhedron type to define an instance of the 
adapter traits class \ccc{CGAL::Convex_hull_d_traits_3<Kernel>}.

In most cases, the function \ccc{convex_hull_3} will be faster than
\ccc{convex_hull_incremental_3}.  The latter is provided mainly 
for comparison purposes. 

To use the full functionality available with the $d$-dimensional class 
\ccc{CGAL::Convex_hull_d<R>} in three dimensions (\textit{e.g.}, the ability
to insert new points and to query if a point lies in the convex hull or not), 
you can instantiate the class \ccc{CGAL::Convex_hull_d<K>} with the adapter
traits class \ccc{CGAL::Convex_hull_d_traits_3<K>}, as shown in the following
example.

\ccIncludeExampleCode{Convex_hull_3/incremental_hull_3_demo.C}

\subsection{Example of Dynamic Construction}
\ccIndexSubitem{convex hull, 3D}{dynamic}

Fully dynamic maintenance of a convex hull can be achieved by using the
class \ccc{CGAL::Delaunay_triangulation_3}.  This class supports insertion
and removal of points (\textit{i.e.}, vertices of the triangulation) and the 
convex hull edges are simply the finite edges of infinite faces.  
The following example illustrates the dynamic construction of a convex hull.
First, random points from a sphere of a certain radius are generated and are
inserted into a triangulation.  Then the number of points of the convex hull 
are obtained by counting the number of triangulation vertices incident to the 
infinite vertex.  Some of the points are removed and then the number of points 
remaining on the hull are determined.  Notice that the vertices incident to the
infinite vertex of the triangulation are on the convex hull but it may be that
not all of them are vertices of the hull.

\ccIncludeExampleCode{Convex_hull_3/dynamic_hull_3_ex.C}

\end{ccPackage}

