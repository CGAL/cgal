\section*{Design and Implementation History}
%===========================================

The code of this package is the result of a long development process.
Initially (and until version~3.1), the code was spread among several
components, namely, \ccc{Topological_map}, \ccc{Planar_map_2},
\ccc{Planar_map_with_intersections_2} and \ccc{Arrangement_2}, that were
developed by Ester Ezra, Eyal Flato, Efi Fogel, Dan Halperin, Iddo
Hanniel, Idit Haran, Shai Hirsch, Eugene Lipovetsky, Oren Nechushtan,
Sigal Raab, Ron Wein, Baruch Zukerman, and Tali Zvi.

In version~3.2, as part of the ACS project, the packages have gone
through a major re-design, resulting in an improved and unified
\emph{2D Arrangements} package.
The code of the new package was restructured and developed by
Efi Fogel, Idit Haran, Ron Wein, and Baruch Zukerman. This
version included for the first time a new geometry-traits
class that handles circular and linear curves, and is based
on the circular kernel. The circular kernel was developed
by Monique Teillaud, Sylvain Pion, and Julien Hazebrouck.

Version~3.3 features arrangements of unbounded curves for the first
time. The design and development of this feature required yet another
restructuring of the entire package. All this was done by Eric
Berberich, Efi Fogel, Dan Halperin, Ophir Setter, and Ron
Wein. Michael Hemmer helped tuning up parts of the geometry-traits 
concept related to unbounded curves.

Version~3.7 introduced a geometry-traits class
that handles planar algebraic curves of arbitrary degree.
It was developed by Eric Berberich and Michael Kerber.

Version~3.9 introduced a new geometry-traits class that handles
rational arcs. It was developed by Oren Salzman and Michael Hemmer.
It replaced an old traits, which handled the same family of
curves, developed by Ron Wein.

Version~4.1 introduces a new trapezoid RIC point location class.
It was developed by Michal Kleinbort and Michael Hemmer.
The new implementation is a revamp of the old one 
by Oren Nechushtan, and can now guarantee logarithmic query time
in all cases and handle unbounded curves.