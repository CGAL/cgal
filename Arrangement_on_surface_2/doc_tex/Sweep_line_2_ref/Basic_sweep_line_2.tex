% +------------------------------------------------------------------------+
% | Reference manual page: Basic_sweep_line_2.tex
% +------------------------------------------------------------------------+
% | 
% | Package: Arrangement_2
% | 
% +------------------------------------------------------------------------+

\ccRefPageBegin

\begin{ccRefClass}{Basic_sweep_line_2<Traits, BasicVisitor>}
%\label{arr_ref:arr_obs}

\ccDefinition
%============

\ccClassTemplateName\ 

\ccInclude{CGAL/Basic_sweep_line_2.h}

\ccTypes
%=======


\ccTypedef{typedef typename Basic_sweep_line_2<Traits, BasicVisitor>::Traits_2 Traits_2;}
    {the traits type.}
\ccGlue

\ccTypedef{typedef typename Basic_sweep_line_2<Traits, BasicVisitor>::X_monotone_curve_2 X_monotone_curve_2;}
    {the $x$-monotone curve type.}
\ccGlue
\ccTypedef{typedef typename Basic_sweep_line_2<Traits, BasicVisitor>::Point_2 Point_2;}
    {the point type.}
\ccGlue

\ccTypedef{typedef typename Basic_sweep_line_2<Traits, BasicVisitor>::Event Event;}
    {the event type.}
\ccGlue

\ccTypedef{typedef typename Basic_sweep_line_2<Traits, BasicVisitor>::Subcurve Subcurve;}
    {the subcurve type.}
\ccGlue

\ccTypedef{typedef typename Basic_sweep_line_2<Traits, BasicVisitor>::Status_line_iterator Status_line_iterator;}
{the iterator type of the status line, its value type is  \ccc{Subcurve*}.}
\ccGlue
\ccTypedef{typedef typename Basic_sweep_line_2<Traits, BasicVisitor>::Event_queue_iterator Event_queue_iterator;}
{the iterator type of the event queue, its value type is  \ccc{Event*}.}
\ccGlue

\ccTypedef{typedef typename Basic_sweep_line_2<Traits, BasicVisitor>::Visitor Visitor;}
    {the visitor type.}
\ccGlue

\ccCreation
\ccCreationVariable{obs}
%=======================

\ccConstructor{Basic_sweep_line_2 (Visitor *visitor);} 
    {constructs a sweep line object with a pointer to a sweep line visitor.}
 
\ccConstructor{Basic_sweep_line_2 (Traits_2 *traits, Visitor *visitor);} 
    {constructs a sweep line object with pointer to a triats object and a pointer to a sweep-line visitor.}

\ccMethods
%===========

\ccMethod{template<class CurveInputIterator>
  void sweep (CurveInputIterator curves_begin,
              CurveInputIterator curves_end);}
   {Run the sweep-line algorithm on a given range of x-monotone curves.
    \ccc{curves_begin} is an iterator for the first curve in the range.
    \ccc{curves_end} is a past-the-end iterator for the range.
    \ccPrecond {The value-type of CurveInputIterator is X_monotone_curve_2.}}
    
    
\ccMethod{template<class CurveInputIterator, class PointInputIterator>
  void sweep (CurveInputIterator curves_begin,
              CurveInputIterator curves_end,
              PointInputIterator action_points_begin,
              PointInputIterator action_points_end);}
   {Run the sweep-line algorithm on a range of x-monotone curves and a range 
    of action event points (if a curve passed through an action point, it will be split).
   \ccc{curves_begin} is an iterator for the first x-monotone curve in the range.
   \ccc{curves_end} is a past-the-end iterator for this range.
   \ccc{points_begin} An iterator for the first point in the range.
   \ccc{points_end} is a past-the-end iterator for this range.
   \ccPrecond{ The value-type of XCurveInputIterator is the traits-class
    X_monotone_curve_2, and the value-type of PointInputIterator is the traits-class Point_2.}}
    
    
    \ccMethod{template<class CurveInputIterator, class ActionPointItr,class QueryPointItr>
  void sweep (CurveInputIterator curves_begin,
              CurveInputIterator curves_end,
              ActionPointItr action_points_begin,
              ActionPointItr action_points_end,
              QueryPointItr query_points_begin,
              QueryPointItr query_points_end);}
   { Run the sweep-line alogrithm on a range of x-monotone curves, a range   
     of action event points (if a curve passed through an action point, it will
     be split) and a range of query points (if a curve passed through a
     query point,it will not be splitted).
   \ccc{curves_begin} An iterator for the first x-monotone curve in the range.
   \ccc{curves_end} A past-the-end iterator for this range.
   \ccc{points_begin} An iterator for the first point in the range.
   \ccc{points_end} A past-the-end iterator for this range.
   \ccPrecond{ The value-type of XCurveInputIterator is the traits-class 
        X_monotone_curve_2, and the value-type of PointInputIterator is the 
        traits-class Point_2.}}


\ccHeading{Utility methods that can be used by the visitor during the sweep process.}
%=========================================================

\ccMethod{Status_line_iterator status_line_begin();}
    {Get an iterator for the first subcurve in the status line.}
\ccGlue
\ccMethod{Status_line_iterator status_line_end();}
    {Get a past-the-end iterator for the subcurves in the status line.}

\ccMethod{unsigned int status_line_size() const;}
    {Get the status line size.}
\ccGlue
\ccMethod{bool is_status_line_empty() const;}
    {Check if the status line is empty.}

\ccMethod{Event_queue_iterator event_queue_begin();}
    {Get an iterator for the first event in event queue.}
\ccGlue
\ccMethod{Event_queue_iterator event_queue_end();}
    {Get a past-the-end iterator for the events in the in event queue.}
    
    \ccMethod{unsigned int event_queue_size() const;}
    {Get the event queue size.}
    
    \ccMethod{bool is_event_queue_empty() const;}
    {Check if the event queue is empty.}
    
    \ccMethod{void stop_sweep();}
    {Stop the sweep by erasing the event queue (except for the current event).
     This function may called by the visitor during 'arter_handle_event' in
     order to stop the sweep-line process.}
     
     \ccMethod{void deallocate_event(Event* event);}
    { Deallocate event object.
      This method is made public to allow the visitor to manage the events
      deallocation (as necessary). }
    
    \ccMethod{Event* current_event();}
    {Get the current event.}
    
    \ccMethod{Traits_2* traits ();}
    {Get the traits object.}
    

\end{ccRefClass}

\ccRefPageEnd
