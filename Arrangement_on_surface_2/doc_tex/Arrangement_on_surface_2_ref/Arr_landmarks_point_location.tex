% +------------------------------------------------------------------------+
% | Reference manual page: Arr_landmarks_point_location.tex
% +------------------------------------------------------------------------+
% |
% | Package: Arrangement_2
% |
% +------------------------------------------------------------------------+

\ccRefPageBegin

\begin{ccRefClass}{Arr_landmarks_point_location<Arrangement,Generator>}
\label{arr_ref:lm_pl}

The \ccRefName\ class implements a Jump \& Walk algorithm, where special
points, referred to as ``landmarks'', are chosen in a preprocessing stage,
their place in the arrangement is found, and they are inserted into a
data-structure that enables efficient nearest-neighbor search (a
{\sc Kd}-tree). Given a query point, the nearest landmark is located and a
``walk'' strategy is applied from the landmark to the query point.

There are various strategies to select the landmark set in the
arrangement, where the strategy is determined by the
\ccc{Generator} template parameter. The following landmark-generator
classes are available:
\begin{description}
\item[\ccc{Arr_landmarks_vertices_generator} ---]
The arrangement vertices are used as the landmarks set.

\item[\ccc{Arr_random_landmarks_generator} ---]
$n$ random points in the bounding box of the arrangement are chosen
as the landmarks set.

\item[\ccc{Arr_halton_landmarks_generator} ---]
$n$ Halton points in the bounding box of the arrangement are chosen
as the landmarks set.

\item[\ccc{Arr_middle_edges_landmarks_generator} ---]
The midpoint of each arrangement edge is computed, and the resulting
set of points is used as the landmarks set. This generator can be applied
only for arrangements of line segments.

\item[\ccc{Arr_grid_landmarks_generator} ---]
A set of $n$ landmarks are chosen on a
$\lceil \sqrt{n} \rceil \times \lceil \sqrt{n} \rceil$
grid that covers the bounding box of the arrangement.
\end{description}
The \ccc{Arr_landmarks_vertices_generator} class is the default generator
and used if no \ccc{Generator} parameter is specified.

It is recommended to use the landmarks point-location strategy
when the application frequently issues point-location queries on a
rather static arrangement that the changes applied to it are mainly
insertions of curves and not deletions of them.

\ccInclude{CGAL/Arr_landmarks_point_location.h}

\ccIsModel
  \ccc{ArrangementPointLocation_2} \\
  \ccc{ArrangementVerticalRayShoot_2}

\end{ccRefClass}

\ccRefPageEnd
