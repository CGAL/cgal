% +------------------------------------------------------------------------+
% | Reference manual page: Arr_non_caching_segment_traits.tex
% +------------------------------------------------------------------------+
% | 
% | Package: Arrangement_2
% | 
% +------------------------------------------------------------------------+

\ccRefPageBegin

\begin{ccRefClass}{Arr_non_caching_segment_traits_2<Kernel>}
    
\ccDefinition 

The traits class \ccRefName\ is a model of the \ccc{ArrangementTraits_2}
concept that allows the construction and maintenance of arrangements of
line segments. It is parameterized with a \cgal-Kernel type, and it
is derived from it. This traits class is a thin layer above the
parameterized kernel. It inherits the \ccc{Point_2} from the kernel and its
\ccc{X_monotone_curve_2} and \ccc{Curve_2} types are both defined as
\ccc{Kernel::Segment_2}. Most traits-class functor are inherited from the
kernel functor, and the traits class only supplies the necessary functors
that are not provided by the kernel. The kernel is parameterized with a
number type, which should support exact rational arithmetic in order to
avoid robustness problems, although other number types could be used at the
user's own risk.

The traits-class implementation is very simple, yet may lead to
a cascaded representation of intersection points with exponentially long
bit-lengths, especially if the kernel is parameterized with a number type
that does not perform normalization (e.g. \ccc{Quotient<MP_Float>}).
The \ccStyle{Arr_segment_traits_2} traits class avoids this cascading
problem, and should be the default choice for implementing arrangements of
line segments. It is recommended to use \ccRefName\ only for very sparse
arrangements of huge sets of input segments.

While \ccRefName{} models the concept
\ccc{ArrangementDirectionalXMonotoneTraits_2}, the implementation of
the \ccc{Arr_mergeable_2} operation does not enforce the input curves
to have the same direction as a precondition. Moreover, \ccRefName{}
supports the merging of curves of opposite directions.

\ccInclude{CGAL/Arr_non_caching_segment_traits_2.h}
 
\ccIsModel
    \ccc{ArrangementTraits_2}\\
    \ccc{ArrangementLandmarkTraits_2}\\
    \ccc{ArrangementDirectionalXMonotoneTraits_2}

\ccInheritsFrom
    \ccc{Arr_non_caching_segment_basic_traits_2<Kernel>}

\ccSeeAlso
    \ccc{Arr_segment_traits_2<Kernel>}

\end{ccRefClass}
\ccRefPageEnd
