% +------------------------------------------------------------------------+
% | Reference manual page: Arr_dcel_halfedge.tex
% +------------------------------------------------------------------------+
% | 
% | Package: Arrangement_2
% | 
% +------------------------------------------------------------------------+

\ccRefPageBegin

\begin{ccRefConcept}{ArrangementDcelHalfedge}

\ccDefinition

A halfedge record in a \dcel\ data structure. Two halfedges with opposite
directions always form an edge (a halfedge pair). The halfedges form together
chains, defining the boundaries of connected components, such that all
halfedges along a chain have the same incident face. Note that the chain the
halfedge belongs to may form the outer boundary of a bounded face (an outer
CCB) or the boundary of a hole inside a face (an inner CCB). 

An edge is always associated with a curve, but the halfedge records only
store a pointer to the associated curve, and the actual curve objects
are stored elsewhere. Two opposite halfedges are always associated with
the same curve.

\ccCreationVariable{e}

\ccTypes
%=======

\ccNestedType{Vertex}{the corresponding \dcel\ vertex type.}
\ccGlue
\ccNestedType{Face}{the corresponding \dcel\ face type.}
\ccGlue
\ccNestedType{Hole}{the corresponding \dcel\ hole type.}

\ccNestedType{X_monotone_curve}{the curve type associated with the edge.} 

\ccCreation
%==========

\ccConstructor{Arr_dcel_halfedge();}
   {default constructor.}

\ccMethod{void assign (const Self& other);}
   {assigns \ccVar{} with the contents of the \ccc{other} halfedge.}

\ccHtmlNoLinksFrom{   % to avoid linkage of Vertex, etc to HDS::Vertex

\ccAccessFunctions
%=================

\ccMethod{Arr_halfedge_direction direction() const;}
    {returns \ccc{ARR_LEFT_TO_RIGHT} if \ccVar{}'s source vertex is
      lexicographically smaller than it target, and
      \ccc{ARR_RIGHT_TO_LEFT} if it is lexicographically larger than
      the target.}

\ccMethod{bool is_on_hole() const;}
    {determines whether the \ccVar{} lies on an outer CCB of a bounded face,
     or on an inner CCB (a hole inside a face). The function returns \ccc{true}
     if \ccVar{} lies on a hole.}

All functions below also have \ccc{const} counterparts, returning non-mutable
pointers or references:

\ccMethod{Halfedge* opposite();}
    {returns the twin halfedge.}

\ccMethod{Halfedge* prev();}
    {returns the previous halfedge along the chain.}

\ccMethod{Halfedge* next();}
    {returns the next halfedge along the chain.}

\ccMethod{Vertex* vertex();}
    {returns the target vertex.}

\ccMethod{Face* face();}
    {returns the incident face.
     \ccPrecond{\ccVar{} lies on the outer boundary of this face.}}

\ccMethod{Hole* hole();}
    {returns the hole (inner CCB) \ccVar{} belongs to.
     \ccPrecond{\ccVar{} lies on a hole inside its incident face.}}

\ccMethod{bool has_null_curve() const;}
    {returns whether the vertex is not associated with a valid curve.}
 
\ccMethod{X_monotone_curve& curve();}
    {returns the associated curve.
     \ccPrecond{\ccVar{} is associated with a valid curve.}}

\ccModifiers
%===========

\ccMethod{void set_opposite (Halfedge* opp);}
    {sets the opposite halfedge.}

\ccMethod{void set_direction (Arr_halfedge_direction dir);}
    {sets the lexicographical order between \ccVar{}'s source and target
     vertices to be \ccc{dir}.
     The direction of the opposite halfedge is also set to the
     opposite direction.}

\ccMethod{void set_prev (Halfedge* prev);}
    {sets the previous halfedge of \ccVar{} along the chain,
     and updates the cross-pointer \ccc{prev->next()}.}

\ccMethod{void set_next (Halfedge* next);}
    {sets the next halfedge of \ccVar{} along the chain,
     and updates the cross-pointer \ccc{next->prev()}.}

\ccMethod{void set_vertex (Vertex* v);}
    {sets the target vertex.}

\ccMethod{void set_face (Face* f);}
    {sets the incident face, marking that \ccVar{} lies on the outer CCB
     of the face \ccc{f}.}

\ccMethod{void set_hole (Hole* ho);}
    {sets the incident hole, marking that \ccVar{} lies on an inner CCB.}

\ccMethod{void set_curve (X_monotone_curve* c);}
    {sets the associated curve of \ccVar{} and its opposite halfedge.}

} % ccHtmlNoLinksFrom

\ccSeeAlso

     \ccc{ArrangementDcel}\lcTex{ 
     (\ccRefPage{ArrangementDcel})}\\
     \ccc{ArrangementDcelVertex}\lcTex{
     (\ccRefPage{ArrangementDcelVertex})}\\
     \ccc{ArrangementDcelFace}\lcTex{
     (\ccRefPage{ArrangementDcelFace})}\\
     \ccc{ArrangementDcelHole}\lcTex{
     (\ccRefPage{ArrangementDcelHole})}\\

\end{ccRefConcept}  

\ccRefPageEnd
