% +------------------------------------------------------------------------+
% | Reference manual page: Linear_cell_complex_traits.tex
% +------------------------------------------------------------------------+
% | 04.02.2010   Guillaume Damiand
% | Package: Linear_cell_complex
% +------------------------------------------------------------------------+
\ccRefPageBegin
%%RefPage: end of header, begin of main body
% +------------------------------------------------------------------------+

\begin{ccRefClass}{Linear_cell_complex_traits<d,K>}

\ccInclude{CGAL/Linear_cell_complex_traits.h}

\ccDefinition

This geometric traits concept is used in the
\ccc{Linear_cell_complex} class.  It can take as parameter any model of the
concept \ccc{Kernel} (for example any \cgal\ kernel), and defines inner
types and functors corresponding to the given dimension.

\ccIsModel
\ccRefConceptPage{LinearCellComplexTraits}

\ccInheritsFrom
\ccc{K}.

\ccParameters
\ccc{d} the dimension of the kernel,\\
\ccc{K} a model of the concept \ccc{Kernel} if \ccc{d==2} or 
 \ccc{d==3}; a model of the concept \ccc{Kernel_d} otherwise. 

There is a default template arguments for \ccc{K} which is
\ccc{CGAL::Exact_predicates_inexact_constructions_kernel}
if \ccc{d} is 2 or 3, and is \ccc{CGAL::Cartesian_d<double>}
otherwise.

Note that the default argument used for \ccc{K} when
\emph{d}\mygt{}3 does not use exact predicates because operations that
use predicates are only defined in 2D and 3D.

\ccConstants
\ccVariable{static unsigned int ambient_dimension = d;}{}

\ccSeeAlso
\ccRefIdfierPage{CGAL::Linear_cell_complex<d,d2,LCCTraits,CMItems,Alloc>}

\end{ccRefClass}
% +------------------------------------------------------------------------+
%%RefPage: end of main body, begin of footer
\ccRefPageEnd
% EOF
% +------------------------------------------------------------------------+
