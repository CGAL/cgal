\lcTex{%
  \newlength{\widthExtra}\setlength{\widthExtra}{1.1cm}
  \newlength{\widthLineReal}\setlength{\widthLineReal}{\linewidth}
  \addtolength{\widthLineReal}{-\widthExtra}
  \newlength{\minipageSpace}\setlength{\minipageSpace}{0.2cm}

  \newlength{\widthLeft}
  \newlength{\widthRight}
}

\newcommand{\reals}{\mathbb{R}}
\newcommand{\calC}{{\cal C}}
\newcommand{\calA}{{\cal A}}
\newcommand{\eps}{{\varepsilon}}
\newcommand{\dcel}{{\sc Dcel}}
\newcommand{\naive}{na\"{\i}ve}
\newcommand{\kdtree}{{\sc Kd}-tree}
\newcommand{\Cpp}{{C}{\tt ++}}

% ====================
\section{Introduction}
\label{mink_sec:intro}
% ====================

Given two sets $A,B \in \reals^d$, their \emph{Minkowski sum},
denoted by $A \oplus B$, is the set $\left\{ a + b ~\vert~ a \in
A, b \in B \right\}$. Minkowski sum are used in many applications,
such as motion planning and computer-aided design and
manufacturing. This package contains functions for computing planar
Minkowski sums of two polygons (namely $A$ and $B$ are two closed
polygons in $\reals^2$), and for a polygon and a disc (an operation
also known as \emph{offsetting} or \emph{dilating} a polygon).

\section{Computing the Minkowski Sum of Two Polygons}
\label{mink_sec:sum_poly}
%====================================================

Computing the Minkowski sum of two convex polygons $P$ ans $Q$ with
$m$ and $n$ vertices respectively is very easy, as $P \oplus Q$ is a
convex polygon bounded by copies of the $m + n$ edges, and these edges
are sorted by the angle they form with the $x$-axis. As the two
input polygons are convex, their edges are already sorted by the
angle they form with the $x$-axis. The Minkowski sum can therefore be
computed in $O(m + n)$ time, by starting from two botommost vertices
in $P$ and in $Q$ and performing ``merge sort'' on the edges.

\begin{figure}[t]
\begin{ccTexOnly}
\begin{center}
    \begin{tabular}{c c}
        \psfig{figure=Minkowski_sum_2/fig/onecyc_in.eps,width=2.5in,silent=} ~&~
        \psfig{figure=Minkowski_sum_2/fig/onecyc_out.eps,width=2.5in,silent=}
    \end{tabular}
\end{center}
\end{ccTexOnly}
\begin{ccHtmlOnly}
  <p><center>
  <img src="./fig/conv_onecycle.gif" border=0 alt="Convolution cycle">
  </center>
\end{ccHtmlOnly}
\caption{Computing the convolution of a convex polygon and a
non-convex polygon (left). The convolution consists of a single
self-intersecting cycle, drawn as a sequence of arrows (right).
The winding number associated with each face of the arrangement
induced by the segments forming the cycle appears in dashed circles.
The Minkowski sum of the two polygons is shaded.}
\label{fig:onecyc}
\end{figure}

If the polygons are not convex, it is possible to use one of the following
approaches:
\begin{description}
\item[Decomposition:]
We decompose $P$ and $Q$ into convex sub-polygons, namely we obtain two
sets of convex polygons $P_1, \ldots, P_k$ and $Q_1, \ldots, Q_\ell$ such
that $\bugcup_{i = 1}{k}{P_i} = P$ and $\bugcup_{i = j}{\ell}{Q_j} = Q$.
We then calculate the pairwise sums $S_{ij} = P_i \oplus Q_j$ using the
simple procedure described above, and compute the union $P \oplus Q =
\bigcup_{ij}{S_{ij}}$.

This approach relies on a decoposition strategy that computes the convex
decomposition of the input polygons and its performance depends on the
quality of the decomposition.
%
\item[Convolution:]
Let us denote the vertices of the input polygons by
$P = \left( p_0, \ldots, p_{m-1} \right)$ and
$Q = \left( q_0, \ldots, q_{n-1} \right)$. We assume that both $P$ and $Q$
have positive orientations (i.e. their boundaries wind in a counterclockwise
order around their interiors) and compute the convolution of the two polygon
boundaries. The {\em convolution} of these two polygons~\cite{grs-kfcg-83},
denoted $P * Q$, is a collection of line segments of the form
$[p_i + q_j, p_{i+1} + q_j]$, where the vector $\overrightarrow{p_i p_{i+1}}$
lies between $\overrightarrow{q_{j-1} q_j}$ and $\overrightarrow{q_j
q_{j+1}}$,\footnote{We say that a vector $\vec{v}$ lies between
two vectors $\vec{u}$ and $\vec{w}$ if we reach $\vec{v}$ strictly
before reaching $\vec{w}$ if we move all three vectors to the origin
and rotate $\vec{u}$ counterclockwise. Note that this also covers
the case where $\vec{u}$ has the same direction as $\vec{v}$.} and
--- symmetrically --- of segments of the form $[p_i + q_j, p_i + q_{j+1}]$,
where the vector $\overrightarrow{q_j q_{j+1}}$ lies between
$\overrightarrow{p_{i-1} p_i}$ and $\overrightarrow{p_i p_{i+1}}$.

The segments of the convolution form a number of closed (not
necessarily simple) polygonal curves called \emph{convolution
cycles}. The Minkowski sum $P \oplus Q$ is the set of points
having a non-zero winding number with respect to the cycles
of $P * Q$.\footnote{Informally speaking, the winding number of a point
$p \in \reals^2$ with respect to some planar curve $\gamma$ is an
integer number counting how many times does $\gamma$ wind in a
counterclockwise direction around $p$.} See Figure~\ref{fig:onecyc} for
an illustration.
\end{description}

\begin{figure}[t]
\begin{ccTexOnly}
  \begin{center}
  \input{Minkowski_sum_2/fig/sum_triangles.pstex_t}
  \end{center}
\end{ccTexOnly}
\begin{ccHtmlOnly}
  <p><center>
  <img src="./fig/sum_triangles.gif" border=0 alt="Minkowski sum of two triangles">
  </center>
\end{ccHtmlOnly}
\caption{Computing the Minkowski sum of two triangles, as done
in the example program \ccc{ex_sum_triangles.C}.}
\label{mink_fig:sum_tri}
\end{figure}

\ccIncludeExampleCode{../examples/Minkowski_sum_2/ex_sum_triangles.C}



\section{Offsetting a Polygon}
\label{mink_sec:offset}
%=============================
