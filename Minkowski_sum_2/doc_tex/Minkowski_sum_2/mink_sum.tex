\lcTex{%
  \newlength{\widthExtra}\setlength{\widthExtra}{1.1cm}
  \newlength{\widthLineReal}\setlength{\widthLineReal}{\linewidth}
  \addtolength{\widthLineReal}{-\widthExtra}
  \newlength{\minipageSpace}\setlength{\minipageSpace}{0.2cm}

  \newlength{\widthLeft}
  \newlength{\widthRight}
}

\newcommand{\reals}{\mathbb{R}}
\newcommand{\calC}{{\cal C}}
\newcommand{\calA}{{\cal A}}
\newcommand{\eps}{{\varepsilon}}
\newcommand{\dcel}{{\sc Dcel}}
\newcommand{\naive}{na\"{\i}ve}
\newcommand{\kdtree}{{\sc Kd}-tree}
\newcommand{\Cpp}{{C}{\tt ++}}

% ====================
\section{Introduction}
\label{mink_sec:intro}
% ====================

Given two sets $A,B \in \reals^d$, their \emph{Minkowski sum},
denoted by $A \oplus B$, is the set $\left\{ a + b ~\vert~ a \in
A, b \in B \right\}$. Minkowski sum are used in many applications,
such as motion planning and computer-aided design and
manufacturing. This package contains functions for computing
Minkowski sums of two polygons (namely $A$ and $B$ are two closed
polygons in $\reals^2$), and for a polygon and a disc (an operation
also known as \emph{offsetting}).

\begin{figure}[t]
\begin{ccTexOnly}
  \begin{center}
  \input{Minkowski_sum_2/fig/sum_triangles.pstex_t}
  \end{center}
\end{ccTexOnly}
\begin{ccHtmlOnly}
  <p><center>
  <img src="./fig/sum_triangles.gif" border=0 alt="Minkowski sum of two triangles">
  </center>
\end{ccHtmlOnly}
\caption{Computing the Minkowski sum of two triangles, as done
in the example program \ccc{ex_sum_triangles.C}.}
\label{mink_fig:sum_tri}
\end{figure}


\ccIncludeExampleCode{../examples/Minkowski_sum_2/ex_sum_triangles.C}
