% +------------------------------------------------------------------------+
% | Reference manual page: inset_polygon.tex
% +------------------------------------------------------------------------+
% | 
% | Package: Minkowski_sum_2
% | 
% +------------------------------------------------------------------------+

\ccRefPageBegin

\begin{ccRefFunction}{inset_polygon_2}

\ccInclude{CGAL/offset_polygon_2.h}

\ccFunction{template<class ConicTraits, class Container, class OutputIterator>
            OutputIterator inset_polygon_2
                 (const Polygon_2<typename ConicTraits::Rat_kernel,
                                  Container>& P,
                  const typename ConicTraits::Rat_kernel::FT& r,
                  const ConicTraits& traits,
                  OutputIterator oi);}
   {Computes the inset, or inner offset, of the given polygon \ccc{P} by a
    given radius \ccc{r} --- namely, the function computes the set of points
    inside the polygon whose distance from $P$'s boundary is at least $r$:
    $\{ p \in P \;|\; {\rm dist}(p, \partial P) \geq r \}$.
    Note that as the input polygon may not be convex, its inset may comprise
    several disconnected components. The result is therefore represented as a
    sequence of generalized polygons, such that the edges of each polygon
    correspond to line segments and circular arcs, both are special types of
    conic arcs, as represented by the \ccc{traits} class.
    The output sequence is returned via the output iterator \ccc{oi}, whose
    value-type must be \ccc{Gps_traits_2<ConicTraits>::Polygon_2}.
    \ccPrecond{\ccc{P} is a simple polygon.}}

\end{ccRefFunction}

\ccRefPageEnd
