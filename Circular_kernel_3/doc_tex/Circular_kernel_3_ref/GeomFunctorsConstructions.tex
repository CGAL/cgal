\begin{ccRefFunctionObjectConcept}{SphericalKernel::Split_3}

\ccCreationVariable{fo}

A model \ccVar\ of this type must provide:

\ccMemberFunction{void operator()
	(const SphericalKernel::Circular_arc_3 &a, 
	const SphericalKernel::Circular_arc_point_3 &p,
	SphericalKernel::Circular_arc_3 &a1, 
	SphericalKernel::Circular_arc_3 &a2);}
{Splits arc $a$ at point $p$, which creates arcs $a1$ and $a2$.
\ccPrecond{The point \ccc{p} lies in the interior of the input arc \ccc{a}.}}

\ccMemberFunction{void operator()
	(const SphericalKernel::Line_arc_3 &l, 
	 const SphericalKernel::Circular_arc_point_3 &p,
	 SphericalKernel::Line_arc_3 &l1, SphericalKernel::Line_arc_3 &l2);}
{Same for a line arc.}

\end{ccRefFunctionObjectConcept}

\begin{ccRefFunctionObjectConcept}{SphericalKernel::MakeThetaMonotone_3}

\ccCreationVariable{fo}

A model \ccVar\ of this concept must provide:

% \ccConstructor{MakeThetaMonotone_3(const Sphere_3 &sphere)}
% {Constructs a functor \ccVar\ to build $\theta$-monotone circular arcs from a circle or a circular arc, 
% relatively to \ccc{sphere}.}

\ccMemberFunction{
  template<class OutputIterator>
  OutputIterator operator()
	(const SphericalKernel::Circular_arc_3 &a,OutputIterator res);}
{
Copies in the output iterator the results of the split of arc $a$ at the $\theta$-extremal
point(s) of its supporting circle relatively to the context sphere used by the function \ccc{SphericalKernel::make_theta_monotone_3_object}
(Refer to section~\ref{section-SK-objects} for the definition of these points.)
The output iterator may contain no arc (if the supporting circle is a bipolar circle),
one arc (if $a$ is already $\theta$-monotone), two arcs (if only one $\theta$-extremal point is on $a$), or
three arcs (if two $\theta$-extremal points are on $a$).
\ccPrecond{\ccc{a} lies on the context sphere used by the function \ccc{SphericalKernel::make_theta_monotone_3_object},
and the supporting circle of \ccc{a} is not bipolar. }
}


\ccMemberFunction{
  template<class OutputIterator>
  OutputIterator operator()
	(const SphericalKernel::Circle_3 &c,OutputIterator res);}
{Copies in the output iterator the results of the split of circle $c$ at its $\theta$-extremal
point(s) relatively to the context sphere used by the function \ccc{SphericalKernel::make_theta_monotone_3_object}.
(Refer to section~\ref{section-SK-objects} for the definition of these points.)
The output iterator may contain no arc (if the circle is bipolar),
one arc (if the circle is polar or threaded), or two arcs (if the circle is normal).

The source and target are such that
the circular arc is the set of points of the circle that lie between the source 
and the target when traversing the circle counterclockwise
seen from the positive side of the plane of the circle.

In this definition, we say that a normal vector (a,b,c) is \textit{positive} if 
$(a>0) || (a==0) \&\& (b>0) || (a==0)\&\&(b==0)\&\&(c>0)$.


For a threaded circle, the arc returned the one built using the full circle.

For a polar circle, the arc returned is the full circle, the source and target correspond to the pole the circle goes through.
\ccPrecond{\ccc{c} lies on the context sphere used by the function \ccc{SphericalKernel::make_theta_monotone_3_object},
and \ccc{c} is not bipolar.}
}

\ccSeeAlso

\ccRefIdfierPage{SphericalKernel::IsThetaMonotone_3}

\end{ccRefFunctionObjectConcept}





