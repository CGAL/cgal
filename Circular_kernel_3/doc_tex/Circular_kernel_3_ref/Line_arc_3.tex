\begin{ccRefClass}{Line_arc_3<SphericalKernel>}

\ccInclude{CGAL/Line_arc_3.h}

\ccIsModel

\ccc{SphericalKernel::LineArc_3}

\ccCreation
\ccCreationVariable{la}

\ccThree{Circular_arc_point_3}{ca.is_x_monotone()}{}
\ccThreeToTwo

\ccConstructor{Line_arc_3(const Line_3<SphericalKernel> &l,
		const Circular_arc_point_3<SphericalKernel> &p1,
		const Circular_arc_point_3<SphericalKernel> &p2)}
{Construct the line segment supported by \ccc{l}, whose source 
is \ccc{p1}, and whose target is \ccc{p2}.
\ccPrecond{\ccc{p1} and \ccc{p2} lie on \ccc{l}. 
\ccc{p1} and \ccc{p2} are different.}}

\ccConstructor{Line_arc_3(const Line_3<SphericalKernel> &l,
		const Point_3<SphericalKernel> &p1,
		const Point_3<SphericalKernel> &p2)}
{Same.}

\ccConstructor{Line_arc_3(const Segment_3<SphericalKernel> &s)}
{}

\ccAccessFunctions

\ccThree{Circular_arc_point_3<SphericalKernel>}{ca.is_x_monotone()}{}
\ccThreeToTwo

\ccMethod{Line_3<SphericalKernel> supporting_line();}{}

\ccMethod{Circular_arc_point_3<SphericalKernel> source();}{}
\ccGlue
\ccMethod{Circular_arc_point_3<SphericalKernel> target();}{}

\ccMethod{Circular_arc_point_3<SphericalKernel> min();}
{Constructs the minimum vertex according to the lexicographic ordering 
of coordinates.} 
\ccGlue
\ccMethod{Circular_arc_point_3<SphericalKernel> max();}
{Same for the maximum vertex.}

\ccQueryFunctions

\ccMethod{bool is_vertical();}{Returns true \ccc{iff} the segment is
  vertical.}

\ccOperations

\ccFunction{bool operator==(const Line_arc_3<SphericalKernel> &s1,
			const Line_arc_3<SphericalKernel> &s2);}
{Test for equality. Two segments are equal, iff their non-oriented
  supporting lines are equal (i.e. they define the same set of
  points), and their endpoints are the same.}

\ccFunction{bool operator!=(const Line_arc_3<SphericalKernel> &s1,
			const Line_arc_3<SphericalKernel> &s2);}
{Test for nonequality.} 

\ccHeading{I/O}

\ccFunction{istream& operator>> (std::istream& is, Line_arc_3 & ca);}{}
\ccGlue
\ccFunction{ostream& operator<< (std::ostream& os, const Line_arc_3 & ca);}{}

The format for input/output is, for each line arc: a \ccc{Line_3} 
(the supporting line) and two \ccc{Circular_arc_point_3} (the two endpoints), 
under the condition that the endpoints are actually lying on the line.

\ccSeeAlso

\ccRefIdfierPage{CGAL::Circular_arc_point_3<SphericalKernel>}\\
\ccRefIdfierPage{CGAL::Circular_arc_3<SphericalKernel>}

\end{ccRefClass}

