\section{Introduction}

The goal of the 3D spherical kernel is to offer to the user a large
set of functionalities on spheres, circles and circular arcs in the 3D
space. These functionalities require computing on algebraic numbers,
which motivates the creation of a new kernel concept extending the
\cgal\ \ccc{Kernel} concept, which is restricted to objects
and functionality in a \ccc{FieldNumberType}.

All the choices (interface, robustness, representation, and so on)
made here are consistent with the choices made in the \cgal\ kernel,
for which we refer the user to the kernel manual
(Chapter~\ref{chapter-kernel-23}).

\section{Spherical Kernel Objects\label{section-SK-objects}}

New main geometric objects are introduced by \ccc{Spherical_kernel_3}:
circular arcs ((model of \ccc{SphericalKernel::CircularArc_3}), points
of circular arcs (model of \ccc{SphericalKernel::CircularArcPoint_3}),
and line segments (model of \ccc{SphericalKernel::LineArc_3}) whose
endpoints are points of this new type.

\ccc{SphericalKernel::CircularArcPoint_3} is used in particular for
endpoints of arcs and intersection points between spheres, circles or
arcs. The coordinates of these points are algebraic numbers of degree
two. Therefore, general predicates offered by the \ccc{Kernel} on
\ccc{Point_3}, which have coordinates in a \ccc{FieldNumberType},
would require heavy algebraic computations in algebraic extensions
of higher degrees and thus are not provided on
them, which explains the need for a new point type. 

A consistent set of predicates and constructions is offered on this
new type of points. The spherical kernel currently implements a set of
fundamental functionalities like intersection, comparisons, inclusion,
etc. More might be provided in the future, as long as only algebraic
numbers of degree two are used.

\section{Software Design}

The design of \ccc{Spherical_kernel_3} is similar to the design of
\ccc{Circular_kernel_2} (see Chapter~\ref{chapter-circular-kernel}). 
It has two template parameters:
\begin{itemize}
\item {} the first parameter must model the \cgal\ 
three dimensional \ccc{Kernel} concept. The spherical kernel derives
from it, and it provides all elementary geometric objects like points,
lines, spheres, circles and elementary functionality on them.
%The \cgal\ 2D Kernel can be used as parameter,
%but the user may plug his own kernel 
%instead, as long as it follows the \cgal\ kernel concept. 
\item {} the second parameter is the algebraic kernel, which is 
responsible for computations on polynomials and algebraic numbers. It 
must model the concept \ccc{AlgebraicKernelForSpheres}. The
robustness of the package relies on the fact that the algebraic kernel
provides exact computations on algebraic objects.
\end{itemize}

The 3D spherical kernel uses the extensibility scheme presented in the
kernel manual (see Section~\ref{section-extensible-kernel}). The types
of \ccc{Kernel} are inherited by the 3D spherical kernel and some
types are taken from the \ccc{AlgebraicKernelForSpheres}
parameter. \ccc{Spherical_kernel_3} introduces new geometric objects
as mentioned in Section~\ref{section-SK-objects}.

In fact, the spherical kernel is documented as a concept,
\ccc{SphericalKernel} and two models are provided: 
\begin{itemize}
\item {} \ccc{Spherical_kernel_3<Kernel,AlgebraicKernelForSpheres>}, the basic kernel,
\item {} and a predefined kernel \ccc{Exact_spherical_kernel_3}.
\end{itemize}

\section{Examples}

The first example shows how to construct spheres and compute
intersections on them using the global function.

\ccIncludeExampleCode{Circular_kernel_3/intersecting_spheres.cpp}

The second example illustrates the use of a functor. 

\ccIncludeExampleCode{Circular_kernel_3/functor_has_on_3.cpp}

\section{Design and Implementation History}

This package follows the 2D circular kernel package (see
Chapter~\ref{chapter-circular-kernel}), which induced the basic
choices of design.

Julien Hazebrouck and Damien Leroy participated in a first
prototype. Sylvain Pion is acknowledged for helpful discussions.

This work was partially supported by the IST Programme of the 6th
Framework Programme of the EU as a STREP (FET Open Scheme) Project
under Contract No IST-006413 (\ccAnchor{http://acs.cs.rug.nl/}{ACS} -
Algorithms for Complex Shapes).
