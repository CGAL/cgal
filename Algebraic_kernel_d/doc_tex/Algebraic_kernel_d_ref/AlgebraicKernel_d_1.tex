\begin{ccRefConcept}{AlgebraicKernel_d_1} 

\ccDefinition

The \ccc{AlgebraicKernel_d_1} concept is meant to provide the
algebraic functionalities on univariate polynomials of general degree
$d$ necessary as a basis for the curved kernel.

\ccTypes

A model of \ccc{AlgebraicKernel_d_1} is supposed to provide

\ccNestedType{Coefficient}{A model of \ccc{IntegralDomain}. }

\ccNestedType{Polynomial_1}{A model of \ccc{Polynomial_1}, for
univariate polynomials on an \ccc{IntegralDomain} coefficient type.} 

\ccNestedType{Algebraic_real_1}{A model of \ccc{AlgebraicReal_1}, for
algebraic numbers.}

\ccNestedType{Construct_polynomial_1}{
Must provide\\
\ccc{template <class InputIterator>}\\
\ccc{Polynomial_1 operator()(InputIterator first, InputIterator last)}\\
that constructs a polynomial from a range of coefficients of type
\ccc{Coefficient}.
}
\footnote{how are coefficients ordered? 
from lower to higher power of the monomials?}

\ccNestedType{Solve_1}{A model of the concept
\ccc{AlgebraicKernel_d_1::Solve_1}.} 

\ccNestedType{Sign_at_1}{A model of the concept
\ccc{AlgebraicKernel_d_1::SignAt_1}.}

\ccNestedType{Derivative_1}{A model of the concept
\ccc{AlgebraicKernel_d_1::Derivative}.} 

\ccNestedType{Compare_1}{A model of the concept 
\ccc{AlgebraicKernel_d_1::Compare_1}.}

\ccHasModels

\ccSeeAlso


\end{ccRefConcept}
