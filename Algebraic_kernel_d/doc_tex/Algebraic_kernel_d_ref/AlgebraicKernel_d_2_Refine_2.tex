\begin{ccRefFunctionObjectConcept}{AlgebraicKernel_d_2::RefineX_2}

\ccDefinition
Refines the first coordinate of a given
\ccc{AlgebraicKernel_d_2::AlgebraicReal_2}.



\ccOperations
\ccCreationVariable{fo}
%\ccThree{xxxxxxxxxxx}{xxxxxxxxxxx}{}
\ccThree{result_type}{fo(first_argument_type,++}{}

A model \ccVar\ of this type must provide:

\ccMethod{
void operator()(AlgebraicKernel_d_2::AlgebraicReal_2 &ar2);}{ 
  This operator at least half's the current interval of the first
  coordinate of $ar2$. \\
  Note that an interval may also be degenerated to a single point.
} 

\ccMethod{
void operator()(AlgebraicKernel_d_2::AlgebraicReal_2& ar2, int rel_prec);}{ 
  This operator refines the current interval of the first coordinate $x$
  of $ar2$ with respect to the given relative precision. \\ 
  That is: $|lower - upper| / |x| \leq 2^{-rel\_prec}$, where $lower$
  and $upper$ are lower and upper bounds for $x$, respectively.\\
  In case $x$ equals zero the interval degenerates to the zero interval. 
} 

\ccSeeAlso
\ccRefIdfierPage{AlgebraicKernel_d_2::AlgebraicReal_2}\\
\ccRefIdfierPage{AlgebraicKernel_d_2::RefineY_2}\\
\end{ccRefFunctionObjectConcept}


\begin{ccRefFunctionObjectConcept}{AlgebraicKernel_d_2::RefineY_2}

\ccDefinition
Refines the second coordinate of a given
\ccc{AlgebraicKernel_d_2::AlgebraicReal_2}.


\ccOperations
\ccCreationVariable{fo}
%\ccThree{xxxxxxxxxxx}{xxxxxxxxxxx}{}
\ccThree{result_type}{fo(first_argument_type,++}{}

A model \ccVar\ of this type must provide:

\ccMethod{
void operator()(AlgebraicKernel_d_2::AlgebraicReal_2& ar2);}{ 
  This operator at least half's the current interval of the second
  coordinate of $ar2$. \\ 
  Note that an interval may also be degenerated to a single point.
} 

\ccMethod{
void operator()(AlgebraicKernel_d_2::AlgebraicReal_2& ar2, int rel_prec);}{ 
  This operator refines the current interval of the second coordinate $y$
  of $ar2$ with respect to the given relative precision.\\
  That is: $|lower - upper| / |y| \leq 2^{-rel\_prec}$, where $lower$
  and $upper$ are lower and upper bounds for $y$, respectively.\\\\
  In case $y$ equals zero the interval degenerates to the zero interval. 
} 
  
\end{ccRefFunctionObjectConcept}
