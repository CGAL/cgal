\begin{ccRefConcept}{AlgebraicKernel_d_2} 

\ccDefinition

The \ccc{AlgebraicKernel_d_2} concept is meant to provide the curved
kernel with all the algebraic functionalities on bivariate polynomials
required for the manipulation of arcs of algebraic curves of general degree
$d$ in $\R^2$.

\texttt{Question: what is the righ interface with AK for \ccc{compare_y_at_x_right}?}

\ccTypes

A model of \ccc{AlgebraicKernel_d_2} is supposed to provide

\ccNestedType{Coefficient}{A model of \ccc{IntegralDomain}. }

\ccNestedType{Polynomial_2}{A model of \ccc{Polynomial_2}, for
bivariate polynomials on an \ccc{IntegralDomain} coefficient type.} 

\ccNestedType{Algebraic_real_2}{A model of
\ccc{AlgebraicKernel_d_2::AlgebraicReal_2}, for solutions of systems
of two bivariate polynomials of type \ccc{Polynomial_2}.}

\ccNestedType{Construct_polynomial_2}{
Must provide 
\ccc{template <class InputIterator> 
Polynomial_2 operator()(InputIterator first, InputIterator last)}
that constructs a polynomial from a range of coefficients of type
\ccc{Coefficient}, given in the lexicographic order of monomial degrees.
}
\footnote{to be added: sparse polynomials}

\ccNestedType{Solve_2}{A model of the concept
\ccc{AlgebraicKernel_d_2::Solve_2}.} 

\ccNestedType{Sign_at_2}{A model of the concept
\ccc{AlgebraicKernel_d_2::SignAt_2}.}

\ccNestedType{Derivative_x_2}{A model of the concept
\ccc{AlgebraicKernel_d_2::DerivativeX_2}.} 

\ccNestedType{Derivative_y_2}{A model of the concept
\ccc{AlgebraicKernel_d_2::DerivativeY_2}.} 

\ccNestedType{X_critical_points_2}{A model of the concept 
\ccc{AlgebraicKernel_d_2::XCriticalPoints_2}.}
\ccGlue
\ccNestedType{Y_critical_points_2}{A model of the concept 
\ccc{AlgebraicKernel_d_2::YCriticalPoints_2}.}

\ccNestedType{Compare_x_2}{A model of the concept 
\ccc{AlgebraicKernel_d_2::CompareX_2}.}
\ccGlue
\ccNestedType{Compare_y_2}{A model of the concept 
\ccc{AlgebraicKernel_d_2::CompareY_2}.}
\ccGlue
\ccNestedType{Compare_xy_2}{A model of the concept 
\ccc{AlgebraicKernel_d_2::CompareXY_2}.}

\ccHasModels

\ccSeeAlso


\end{ccRefConcept}
