\begin{ccRefClass}{Algebraic_kernel_d_1<Coeff>}
%\label{Algebraic_kernel_d_1}

\ccInclude{CGAL/Algebraic_kernel_d_1.h}

\ccDefinition

The class represents an algebraic real root by a square free polynomial and an
isolating interval that uniquely defines the root.
The template argument \ccc{Coeff} determines the coefficient type of the 
kernel, which is also the coefficient type of the supported polynomials.  

Currently, the following coefficient types are supported:\\
-- \ccc{Gmpz}, \ccc{Gmpq}, (requires configuration with external libraries GMP, MPFR and MPFI)\\
-- \ccc{CORE::BigInt}, \ccc{CORE::BigRat}, (requires configuration with external library GMP) \\ 
-- \ccc{leda_integer}, \ccc{leda_rational}. (requires configuration with external library LEDA)\\

\begin{ccAdvanced}
The template argument type can also be set to \ccc{Sqrt_extension<NT,ROOT>}, where \ccc{NT} is 
one of the types listed above. \ccc{ROOT} should be one of the integer types. 
See also the documentation of \ccc{Sqrt_extension<NT,ROOT>}. 
\end{ccAdvanced}

 The current method 
to isolate roots is the bitstream Descartes method presented in~\cite{eigenwillig-phd-08}.
The used method to refine the approximation of an algebraic real root is a slightly modified 
(filtered) version of the one presented in~\cite{abbott-qir-06}. 
The method has quadratic convergence.

\ccIsModel
\ccc{AlgebraicKernel_d_1}.


\ccTypes \ccThree{}{+++++++++++++}{++++++++}

\ccNestedType{Coefficient}{Same type as the template argument \ccc{Coeff}.}

\ccNestedType{Polynomial_1}{A model of \ccc{AlgebraicKernel_d_1::Polynomial_1}.}

\ccNestedType{Algebraic_real_1}{A model of \ccc{AlgebraicKernel_d_1::AlgebraicReal_1}.}

\ccNestedType{Bound}{The choice of \ccc{Coeff} also determines the provided bound, type. 
In case of \ccc{Coeff} is:\\
-- \ccc{Gmpz} or \ccc{Gmpq} this is \ccc{Gmpq},\\
-- \ccc{CORE::BigInt} or \ccc{CORE::BigInt} this is \ccc{CORE::BigRat},\\
-- \ccc{leda_integer} or \ccc{leda_integer} this is \ccc{leda_rational}.}

\ccNestedType{Multiplicity_type}{The multiplicity type is \ccc{int}.}



\ccSeeAlso
\ccRefConceptPage{AlgebraicKernel_d_1}\\
\ccRefConceptPage{Polynomial_d}\\
\ccRefIdfierPage{CGAL::Algebraic_kernel_d_2<Coeff>}

\end{ccRefClass}
