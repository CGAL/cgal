\begin{ccRefConcept}{Solution_2}

\ccDefinition

The concept \ccc{Solution_2} is meant to store both coordinates of a point
on a curve.

\ccCreationVariable{sol2}

\ccAccessFunctions

\ccMethod{Solution_1 x();}{
  returns $x$-coordinate. 
  May throw some exception if algebraic degree becomes to large.
  Usual always computed and therefore easy to access.
}

\ccMethod{Solution_1 y();}{
  returns $y$-coordinate.
  May throw some exception if algebraic degree becomes to large.
  Note that it can be costly to compute this value, so accessing it
  should be handled with care. 
}

\ccMethod{std::pair<Boundary,Boundary> s1.approximate_x(int prec);}{
  Refines the representation of $x$ to the given precision 
  (binary digits after point).
  Internally the precision can already be higher. Note that it can just
  refer to x().approximate(), but it can be more efficient or just possible
  to compute it without constructing the exact value.
} 

\ccMethod{std::pair<Boundary,Boundary> s1.approximate_y(int prec);}{
  Refines the representation of $y$ to the given precision 
  (binary digits after point).
  Internally the precision can already be higher. Note that it can just
  refer to y().approximate(), but it can be more efficient or just possible
  to compute it without constructing the exact value.
} 


\end{ccRefConcept}

