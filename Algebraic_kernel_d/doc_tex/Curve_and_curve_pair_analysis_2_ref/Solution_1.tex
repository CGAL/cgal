\begin{ccRefConcept}{Solution_1}

\ccDefinition

The concept \ccc{Solution_1} is meant to store a coordinate ($x$, or $y$) of
a point on curve. 

\ccRefines
 \ccc{RealEmbeddable}

Remark: Michael Hemmer proposed to extend the concept \ccc{RealEmbeddable} 
with the notion of compactification (adding/dealing with $\pm\infty$). A
valid model has to implement handling of infinity.

\begin{ccAdvanced}

\ccTypes

\ccTypedef{typedef typename Solution_1::Boundary Boundary;}{A NT being able
to represent values between two Solution\_1}

\ccCreationVariable{sol1}

\ccAccessFunctions

Is this type meant to be really abstract (only RealEmbeddable) or 
do we want to have access
to certain entries? In general, it should be possible to use an appropriate
NT here, i.e., a NT that fits the needs of the CA/CPA.

TODO: Defining polynomial?

TODO: For some purposes (e.g., drawing, seperation) it is useful to have the 
following methods:

\ccMethod{Boundary between(Solution_1 s);}{
  returns a rational between \ccc{sol1} and \ccc{s}
  \ccPrecond{sol1 != s}
}

\ccMethod{std::pair<Boundary,Boundary> s1.approximate(int prec);}{
  Refines the representation to the given precision (binary digits 
  after point). Internally the precision can already be higher.
} 

The advanced methods can be seen as a refined concept for enabling drawing. 
Maybe, we can have it orthogonal to the normal concept.

\end{ccAdvanced}

\end{ccRefConcept}
