\begin{ccRefConcept}{AlgebraicKernelCCPA_d_2}

\ccDefinition

The \ccc{AlgebraicKernelCCPA_d_2} concept refines the \ccc{AlgebraicKernel_d_2}
concept with functionality on bivariate polynomials
required for the manipulation of arcs of algebraic curves of general degree
$\R^2$ using an $y$-per-$x$-view on the curves.

TODO: Solution\_d should be merged with AlgebraicReal\_d.

\ccRefines
\ccc{AlgebraicKernel_d_2}

\ccTypes

\ccNestedType{RealSolution_1 or AlgebraicReal_1}{
Open question!
}

\ccNestedType{RealSolution_2 or AlgebraicReal_2}{
Open question!
}

The following nested types should already be part of AlgebraicKernel\_d\_2!

\ccNestedType{MakeCoprime_1}{
}
\ccNestedType{MakeSquareFree_1}{
}
\ccNestedType{SquareFreeFactorization_1}{
}

\ccNestedType{MakeCoprime_2}{
}
\ccNestedType{MakeSquareFree_2}{
}
\ccNestedType{SquareFreeFactorization_2}{
}

What else? Monique?

These are new:

\ccNestedType{CurveAnalysis_2}{A model of
\ccc{CurvePairAnalyis_2::CurveAnalysis_2}, for analysing single
curves defined as bivariate polynomials of type \ccc{Polynomial_2}.}

\ccNestedType{CurvePairAnalysis_2}{A model of
\ccc{AlgebraicKernelCCPA_2::CurvePairAnalysis_2}, 
for analysing a pair of curves
defined as two analyses of type \ccc{CurveAnalysis_2}.}


%Note that polynomails need to have some functionality. Due to that reason
%we repeat these types here:

%\ccNestedType{Polynomial_1}{A model of \ccc{Polynomial_1}, for
%univariate polynomials when the \ccc{Coefficient} type is an 
%\ccc{IntegralDomain}. It must also be possible to perform \ccc{Canonicalize, 
%    GcdUpToConstantFactor, 
%    IntegralDivUpToConstantFactor, 
%    MakeSquareFreeUpToConstantFactor, and 
%    SquareFreeFactorizationUpToConstantFactor. 
%    Also needed some PolynomialConstruction concept.}
%} 

%\ccNestedType{Polynomial_2}{A model of \ccc{Polynomial_2}, for
%  bivariate polynomials on an \ccc{IntegralDomain} coefficient type. 
%  it must also be possible to perform \ccc{Canonicalize, 
%    GcdUpToConstantFactor, 
%    IntegralDivUpToConstantFactor, 
%    MakeSquareFreeUpToConstantFactor, and 
%    SquareFreeFactorizationUpToConstantFactor} on this type - maybe using a
%  traits class. Also needed some PolynomialConstruction concept.
%} 

{\small TODO: Boundary: RealEmbeddable that is 'dense R'??}

\end{ccRefConcept}
