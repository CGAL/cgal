

\section{Design and Implementation History}

This package is clearly split into a univariate and bivariate
kernel. However, with respect to its history the package splits into
a design part and an implementation part. 

The concepts, which make up the design part, 
were written by Eric Berberich, Michael Hemmer, and
Monique Teillaud. 
The design history of the package is fairly old and several
ideas that influenced this package can already be found
in~\cite{cgal:bhkt-risak-07}. Since then, the initial design underwent
considerable changes. For instance, it was decided that the algebraic
numbers should be under the control of the algebraic kernel. On the other
hand the initial support for polynomials was extended to a separate
and independent package that is not restricted to a certain number of
variables. Thus, the authors want to thank for all the useful feedback and
ideas that was brought to them throughout the last years. In particular,
they want to thank Menelaos Karavelas and Elias Tsigaridas for their
initial contributions.

The two generic models %, \ccc{CGAL::Algebraic_kernel_d_1<Coeff>} and \ccc{CGAL::Algebraic_kernel_d_2<Coeff>}, 
where initially developed as part of the \exacus~\cite{beh+-eeeafcs-05} project. 
However, the models are now fully integrated into the \cgal~library, 
since also the relevant layers of \exacus\ are now part of \cgal.
The main authors for \ccc{CGAL::Algebraic_kernel_d_1<Coeff>} and \ccc{CGAL::Algebraic_kernel_d_2<Coeff>} are 
Michael Hemmer and Michael Kerber, respectively. Notwithstanding, the authors also want to emphasize the 
contribution of all authors of the \exacus\ project, 
particularly the contribution of Arno Eigenwillig, Sebastian Limbach and Pavel Emeliyanenko. 

The two univariate kernels that interface the library \rs~\cite{cgal:r-rs} were 
written by Luis Pe\~{n}aranda and Sylvain Lazard. 
Both models interface the library \rs~\cite{cgal:r-rs} by Fabrice Rouillier. 
The authors want to thank Fabrice Rouillier and Elias Tsigaridas for 
strong support and many useful discussions that lead to the integration of \rs.

