% TODO: remove references to Gmpfr and Gmpfi, since they will be part of CGAL.

\subsection{Models}

\subsubsection{Algebraic kernels based on \rs}

The package offers two univariate algebraic kernels that are based on the
library \rs{} \cite{cgal:r-rs}, namely
\ccc{CGAL::Algebraic_kernel_rs_gmpz_1} and
\ccc{CGAL::Algebraic_kernel_rs_gmpq_1}. As the names indicate, the kernels
are based on the library \rs{} \cite{cgal:r-rs} and support univariate
polynomials over \ccc{CGAL::Gmpz} or \ccc{CGAL::Gmpq}, respectively.

In general we encourage to use \ccc{CGAL::Algebraic_kernel_rs_gmpz_1}
instead of \ccc{CGAL::Algebraic_kernel_rs_gmpq_1}. This is caused by the
fact that the most efficient way to compute operations (such as gcd) on
polynomials with rational coefficients is to use the corresponding
implementation for polynomials with integer coefficients.  That is, the
\ccc{CGAL::Algebraic_kernel_rs_gmpq_1} is slightly slower due to overhead
caused by the necessary conversions.  However, since this may not always be
a major issue the \ccc{CGAL::Algebraic_kernel_rs_gmpq_1} is provided for
convenience.

The core of both kernels is the implementation of the interval Descartes
algorithm~\cite{cgal:rz-jcam-04} of the library \rs~\cite{cgal:r-rs}, which
is used to isolate the roots of the polynomial.  The \rs~library restricts
its attention to univariate integer polynomials and some substantial gain
of efficiency can be made by using a kernel that does not follow the
generic programming paradigm, by avoiding interfaces between layers.
Specifically, the fact of working with only a number type allows to
optimize some polynomial operations as well as memory handling.  The
implementation of these kernels make heavy use of the \mpfr~
\cite{cgal:mt-mpfr} and \mpfi~\cite{cgal:r-mpfi} libraries, and of their
CGAL interfaces, \ccc{Gmpfr} and \ccc{Gmpfi}.  The algebraic numbers (roots
of the polynomials) are represented in the two \rs~kernels by a \ccc{Gmpfi}
interval and a pointer to the polynomial of which they are roots.  See
\cite{cgal:lpt-wea-09} for more details on the implementation, tests of
these kernels, comparisons with other algebraic kernels and discussions
about the efficiency.
