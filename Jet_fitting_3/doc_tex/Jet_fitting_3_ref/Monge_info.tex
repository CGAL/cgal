% +------------------------------------------------------------------------+
% | Reference manual page: Monge_info.tex
% +------------------------------------------------------------------------+
% | 09.02.2006   Marc Pouget and Fr�d�ric Cazals
% | Package: Jet_fitting_3
% | 
\RCSdef{\RCSMongeinfoRev}{$Id$}
\RCSdefDate{\RCSMongeinfoDate}{$Date$}
% |
%%RefPage: end of header, begin of main body
% +------------------------------------------------------------------------+


\begin{ccRefClass}{Monge_info<LocalKernel>}  %% add template arg's if necessary

%% \ccHtmlCrossLink{}     %% add further rules for cross referencing links
%% \ccHtmlIndexC[class]{} %% add further index entries

\ccDefinition
% +--------------------------------------------------------------  
The class \ccRefName\ stores informations on the numerical issues of
the computations performed by the class \ccc{Monge_via_jet_fitting}.
The \ccc{LocalKernel} template parameter must be the same for the
classes \ccc{Monge_info} and \ccc{Monge_via_jet_fitting}. 

\ccInclude{Monge_via_jet_fitting.h}

\ccTypes
% +--------------------------------------------------------------  
\ccTypedef{  typedef typename LocalKernel::FT        LFT;}{}
\ccGlue
\ccTypedef{  typedef typename LocalKernel::Vector_3  LVector;}{}
\ccGlue

%\ccNestedType{TYPE}{some nested types}
\ccHeading{Data members}
% +--------------------------------------------------------------
\ccVariable{  LFT m_pca_eigen_vals[3]; }{Eigenvalues of the PCA of the
input points, sorted in descending order.} 
\ccVariable{  LVector m_pca_eigen_vecs[3]; }{eigen vectors of the PCA
are sorted in accordance.}  
\ccVariable{  LFT m_cond_nb; }{Condition number of the least square system.} 
%\ccVariable{ }{} 

\ccCreation
% +--------------------------------------------------------------  
\ccCreationVariable{monge_info}  %% choose variable name

\ccConstructor{Monge_info();}{default constructor.}

\ccAccessFunctions
% +--------------------------------------------------------------
\ccMemberFunction{const LFT* pca_eigen_vals();}{}
\ccMemberFunction{LFT* pca_eigen_vals();}{}
\ccGlue
\ccMemberFunction{const LVector* pca_eigen_vecs();}{}
\ccMemberFunction{LVector* pca_eigen_vecs();}{}
\ccGlue
\ccMemberFunction{const LFT cond_nb();}{}
\ccMemberFunction{LFT& cond_nb();}{}


%\ccOperations

%\ccMethod{void foo();}{some member functions}

\ccSeeAlso
% +--------------------------------------------------------------  
\ccc{Monge_via_jet_fitting}.
%\ccc{some_other_function}.

%\ccExample

%A short example program.
%Instead of a short program fragment, a full running program can be
%included using the 
%\verb|\ccIncludeExampleCode{Jet_fitting_3/Monge_info.C}| 
%macro. The program example would be part of the source code distribution and
%also part of the automatic test suite.

%\begin{ccExampleCode}
%void your_example_code() {
%}
%\end{ccExampleCode}

%% \ccIncludeExampleCode{Jet_fitting_3/Monge_info.C}

\end{ccRefClass}

% +------------------------------------------------------------------------+
%%RefPage: end of main body, begin of footer
% EOF
% +------------------------------------------------------------------------+

