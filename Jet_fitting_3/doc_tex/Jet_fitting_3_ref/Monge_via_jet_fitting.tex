% +------------------------------------------------------------------------+
% | Reference manual page: Monge_via_jet_fitting.tex
% +------------------------------------------------------------------------+
% | 09.02.2006   Marc Pouget and Fr�d�ric Cazals
% | Package: Jet_fitting_3
% | 
\RCSdef{\RCSMongeviajetfittingRev}{$Id$}
\RCSdefDate{\RCSMongeviajetfittingDate}{$Date$}
% |
%%RefPage: end of header, begin of main body
% +------------------------------------------------------------------------+


\begin{ccRefClass}{Monge_via_jet_fitting<DataKernel, LocalKernel, SvdTraits>} 
 %% add template arg's if necessary

%% \ccHtmlCrossLink{}     %% add further rules for cross referencing links
%% \ccHtmlIndexC[class]{} %% add further index entries

\ccDefinition
  
The class \ccRefName\ is designed to perform the estimation of the
local differential quantities at a given point.  The point range is
given by a pair of input iterators, and it is assumed that the point
where the calculation is carried out is the point that the begin
iterator refers to.
%%
The results are stored in an instance of the nested class \ccc{Monge_form},
the particular information returned depending on the degrees specified
for the polynomial fitting and for the Monge form.

The default for the template \ccc{LocalKernel} is
\ccc{Cartesian<double>} and the default for \ccc{SvdTraits} is \ccc{Lapack_svd}.

\ccInclude{CGAL/Monge_via_jet_fitting.h}

\ccParameters
The class \ccRefName\ has three template parameters. Parameter
\ccc{DataKernel} provides  the geometric classes and tools
corresponding to the input points, and also members of the
\ccc{Monge_form} class. Parameter  \ccc{LocalKernel} provides
the geometric classes and tools required by local
computations. Parameter \ccc{SvdTraits} features the linear
algebra algorithm required by the fitting method.

\ccTypes

%\ccNestedType{TYPE}{some nested types}
\ccTypedef{  typedef DataKernel   Data_kernel;}{}
\ccGlue
\ccTypedef{  typedef LocalKernel  Local_kernel;}{}
\ccGlue
\ccTypedef{typedef typename Local_kernel::FT       FT; }{}
\ccGlue
\ccTypedef{  typedef typename Local_kernel::Vector_3 Vector_3;}{}

\ccNestedType{Monge_form}{see the page  \ccc{Monge_form} }

\ccCreation
\ccCreationVariable{monge_fitting}  %% choose variable name, given by \ccVar

\ccConstructor{Monge_via_jet_fitting();} {default constructor} 

\ccOperations \ccMethod{ template <class InputIterator> Monge_form
  operator()(InputIterator begin, InputIterator end, size_t d,
  size_t d');}
{This operator performs all the computations. The $N$ input points are
  given by the \ccc{InputIterator} parameters which value-type are
  \ccc{Data_kernel::Point_3}, \ccc{d} is the degree of the fitted
  polynomial, \ccc{d'} is the degree of the expected Monge
  coefficients.  \ccPrecond $N \geq N_{d}:=(d+1)(d+2)/2$, $1 \leq d'
  \leq \min(d,4) $ }


\ccMethod{LFT condition_number();}{condition number of the linear fitting system.}
\ccGlue
\ccMethod{std::pair<LFT, LVector> pca_basis(size_t i);}
{pca eigenvalues and eigenvectors, the pca\_basis has always 3 such pairs.
 Precondition : $i$ ranges from 0 to 2.}


\ccSeeAlso

\ccc{Monge_form},
\ccc{Lapack_svd}.

\end{ccRefClass}

% +------------------------------------------------------------------------+
%%RefPage: end of main body, begin of footer
% EOF
% +------------------------------------------------------------------------+

