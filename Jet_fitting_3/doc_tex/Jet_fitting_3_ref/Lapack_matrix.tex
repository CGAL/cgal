% +------------------------------------------------------------------------+
% | Reference manual page: Lapack_matrix.tex
% +------------------------------------------------------------------------+
% | 09.02.2006   Marc Pouget and Fr�d�ric Cazals
% | Package: Jet_fitting_3
% | 
\RCSdef{\RCSLapack_matrixRev}{$Id$}
\RCSdefDate{\RCSLapack_matrixDate}{$Date$}
% |
%%RefPage: end of header, begin of main body
% +------------------------------------------------------------------------+

\begin{ccRefClass}{Lapack_matrix} %%add template arg's if necessary

%% \ccHtmlCrossLink{}     %% add further rules for cross referencing links
%% \ccHtmlIndexC[class]{} %% add further index entries

\ccDefinition
  
The class \ccRefName\ is a wrapper that enables matricial computations
with the algorithm of the class \ccc{Lapack} and usual data access to
its elements. (Note : in clapack matrices are one-dimensional arrays
and elements are column-major ordered)

\ccInclude{CGAL/Lapack/Linear_algebra_lapack.h}

\ccIsModel
\ccc{LinAlgTraits::Matrix}


\ccCreation
% +--------------------------------------------------------------
\ccCreationVariable{lapack_matrix}  %% choose variable name

\ccConstructor{Lapack_matrix(size_t n1, size_t n2);}
{Create a matrix with n1 lines and n2 columns}

\ccAccessFunctions
% +--------------------------------------------------------------
\ccMemberFunction{const double* matrix();}
{gives access to matrix data usable by clapack. (Note~: in clapack
matrices are one-dimensional arrays and elements are column-major
ordered)}
\ccGlue
\ccMemberFunction{double* matrix(); }{}


\ccMemberFunction{void set_elt(size_t i, size_t j, const double value);}
{sets the element at the line i and column j.}
\ccGlue
\ccMemberFunction{double get_elt(size_t i, size_t j);}
{gets the element at the line i and column j.}


\ccSeeAlso

\ccc{LinAlgTraits}\\
\ccc{Lapack}
%\ccc{some_other_function}.

\end{ccRefClass}

% +------------------------------------------------------------------------+
%%RefPage: end of main body, begin of footer
% EOF
% +------------------------------------------------------------------------+

