\ccRefChapter{Linear and Quadratic Programming Solver}
\label{chapter:QPSolver}
\ccChapterAuthor{Kaspar Fischer \and Bernd G{\"a}rtner \and Sven Sch{\"o}nherr
\and Frans Wessendorp}

\section{Classified Reference Pages}

\ccHeading{Concepts}
\ccRefConceptPage{QuadraticProgramInterface}\\
$\quad$ (for quadratic programs with variable bounds $l\leq x \leq u$) \\
\ccRefConceptPage{LinearProgramInterface} \\
$\quad$(for linear programs with variable bounds $l\leq x \leq u$)\\
\ccRefConceptPage{NonnegativeQuadraticProgramInterface}\\
$\quad$ (for quadratic programs with variable bounds $x\geq 0$) \\
\ccRefConceptPage{NonnegativeLinearProgramInterface}\\
$\quad$ (for linear programs with variable bounds $x\geq 0$)

\ccRefConceptPage{MPSFormat}\\
$\quad$ (the format used for reading and writing linear and quadratic 
programs)

\ccHeading{Classes}

There is a class that represents the solution of a linear
or quadratic program. An instance of this class is returned by any of 
the solution functions below.

\ccRefIdfierPage{Quadratic_program_solution}

We offer a number of predefined models for the above program concepts.
The following two are simultaneously models for all four concepts and
are probably the most convenient models; they allow you to construct
linear or quadratic programs entry by entry, or from streams in
\ccc{MPSFormat}. At any time, you can query these programs for
linearity and nonnegativity and thus select the appropriate 
solution function.

\ccRefIdfierPage{Quadratic_program}\\
$\quad$ (for linear and quadratic programs that own their 
data and are built entry-wise)\\
\ccRefIdfierPage{Quadratic_program_from_mps}\\
$\quad$ (for linear and quadratic programs read from an input stream in 
\ccc{MPSFormat}; the constructed program can also be manipulate entry-wise)

Then there are specific models for any of the four program concepts above; 
these are useful if you want to maintain the program data yourself, since
they simply wrap random access iterators over the program data. 

\ccRefIdfierPage{Quadratic_program_from_iterators}\\
$\quad$ (for quadratic programs that wrap given iterators, without copying 
data) \\
\ccRefIdfierPage{Quadratic_program_from_pointers}\\
$\quad$ (for quadratic programs that wrap given pointers, without copying 
data) 

\ccRefIdfierPage{Linear_program_from_iterators}\\
$\quad$ (for linear programs that wrap given iterators, without copying 
data) \\
\ccRefIdfierPage{Linear_program_from_pointers}\\
$\quad$ (for linear programs that wrap given pointers, without copying 
data) 

\ccRefIdfierPage{Nonnegative_quadratic_program_from_iterators}\\
$\quad$ (for nonnegative quadratic programs, wrapping given iterators)\\
\ccRefIdfierPage{Nonnegative_quadratic_program_from_pointers}\\
$\quad$ (for nonnegative quadratic programs, wrapping given pointers)

\ccRefIdfierPage{Nonnegative_linear_program_from_iterators}\\
$\quad$ (for nonnegative linear programs, wrapping given iterators)\\
\ccRefIdfierPage{Nonnegative_linear_program_from_pointers}\\
$\quad$ (for nonnegative linear programs, wrapping given pointers)

\ccHeading{Functions}

In case you want to construct a program from complicated iterators
(whose types you don't know, or simply don't want to bother with), 
we provide four makers.

\ccRefIdfierPage{make_quadratic_program_from_iterators}\\
\ccRefIdfierPage{make_linear_program_from_iterators}\\
\ccRefIdfierPage{make_nonnegative_quadratic_program_from_iterators}\\
\ccRefIdfierPage{make_nonnegative_linear_program_from_iterators}

There are four functions to solve a program, one for each program
concept. 

\ccRefIdfierPage{solve_quadratic_program}\\
\ccRefIdfierPage{solve_linear_program}\\
\ccRefIdfierPage{solve_nonnegative_quadratic_program}\\
\ccRefIdfierPage{solve_nonnegative_linear_program}

Programs can be written to an output stream in \ccc{MPSFormat}, using
one of the following four functions.

\ccRefIdfierPage{print_quadratic_program}\\
\ccRefIdfierPage{print_linear_program}\\
\ccRefIdfierPage{print_nonnegative_quadratic_program}\\
\ccRefIdfierPage{print_nonnegative_linear_program}
