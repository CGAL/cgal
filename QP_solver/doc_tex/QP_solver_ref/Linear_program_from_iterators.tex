\begin{ccRefClass}{Linear_program_from_iterators<A_it, B_it, R_it, FL_it, L_it, FU_it, U_it, C_it>}

\ccInclude{CGAL/QP_models.h}

\ccDefinition
An object of class \ccRefName\ describes a linear program of the form

%%
\begin{eqnarray*}
\mbox{(QP)}& \mbox{minimize} 
& \qpc^{T}\qpx+c_0 \\
&\mbox{subject to}   & A\qpx\qprel \qpb, \\
&                    & \qpl \leq \qpx \leq \qpu
\end{eqnarray*}
%%
in $n$ real variables $\qpx=(x_0,\ldots,x_{n-1})$.
Here, 
\begin{itemize}
\item $A$ is an $m\times n$ matrix (the constraint matrix), 
\item $\qpb$ is an $m$-dimensional vector (the right-hand side),
\item $\qprel$ is an $m$-dimensional vector of relations 
from $\{\leq, =, \geq\}$, 

\item $\qpl$ is an $n$-dimensional vector of lower
bounds for $\qpx$, where $l_j\in\R\cup\{-\infty\}$ for all $j$
\item $\qpu$ is an $n$-dimensional vector of upper bounds for
$\qpx$, where $u_j\in\R\cup\{\infty\}$ for all $j$

\item $\qpc$ is an $n$-dimensional vector (the linear objective
  function), and 
\item $c_0$ is a constant.

\end{itemize}


This class is simply a wrapper for existing iterators, and it does not
copy the program data.

It frequently happens that all values in one of the vectors from
above are the same, for example if the system $Ax\qprel b$ is 
actually a system of equations $Ax=b$. To get an iterator over such a 
vector, it is not necessary to store multiple copies of the value in
some container; an instance of the class \ccc{Const_oneset_iterator<T>},
constructed from the value in question, does the job more efficiently.


\ccIsModel
\ccc{QuadraticProgram}\\
\ccc{LinearProgram}

\ccCreation
\ccIndexClassCreation
\ccCreationVariable{lp}

\ccConstructor{Linear_program_from_iterators(int n, int m, 
  const A_it& a, 
  const B_it& b,
  const R_it& r,
  const FL_it& fl,
  const L_it& l,
  const FU_it& fu,
  const U_it& u,
  const C_it& c,
  const std::iterator_traits<C_it>value_type& c0 = 0
  )}{constructs \ccVar\ from given random-access iterators and the constant \ccc{c0}. The passed iterators are merely stored, no copying of the program data takes place. How these iterators are supposed to encode the linear program is
described in \ccc{LinearProgram}. }

\ccExample

\ccReferToExampleCode{QP_solver/first_lp_from_iterators.cpp}

The following example for the simpler model
\ccc{Nonnegative_linear_program_from_iterators<A_it, B_it, R_it, C_it>} 
should give you a flavor of the use of this 
model in practice.

\ccReferToExampleCode{QP_solver/solve_convex_hull_containment_lp.h}\\
\ccReferToExampleCode{QP_solver/convex_hull_containment.cpp}

\ccSeeAlso
\ccc{LinearProgram}
\ccc{Quadratic_program<NT>}\\
\ccc{Quadratic_program_from_mps<NT>}

\end{ccRefClass}
