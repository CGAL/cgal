\begin{ccRefClass}{Quadratic_program<NT>}

\ccInclude{CGAL/QP_models.h}

\ccDefinition
An object of class \ccRefName\ describes a convex quadratic program of the form
%%
\begin{eqnarray*}
\mbox{(QP)}& \mbox{minimize} & x^{T}Dx+c^{T}x+c_0 \\
&\mbox{subject to}   & Ax\qprel b, \\
&                    & l \leq x \leq u
\end{eqnarray*}
%%
in $n$ real variables $x=(x_0,\ldots,x_{n-1})$. If $D=0$, the program is
a linear program; if the variable bounds are $x\geq 0$, we have a 
nonnegative program. More specifically,  
\begin{itemize}
\item $A$ is an $m\times n$ matrix (the constraint matrix), 
\item $b$ is an $m$-dimensional vector (the right-hand side),
\item $\qprel$ is an $m$-dimensional vector of relations 
from $\{\leq, =, \geq\}$, 
\item $l$ is an $n$-dimensional vector of lower
bounds for $x$,
\item $u$ is an $n$-dimensional vector of upper bounds for
$x$, 
\item $D$ is a symmetric positive-semidefinite $n\times n$ matrix (the
  quadratic objective function),
\item $c$ is an $n$-dimensional vector (the linear objective
  function), and 
\item $c_0$ is a constant.
\end{itemize}

This class allows you to build your program entry by entry, using
the set-methods below. 

If you only need to wrap existing (random-access)
iterators over your own data, then you may use any of the four models
\ccc{Quadratic_program_from_iterators<A_it, B_it, R_it, FL_it, L_it, FU_it, U_it, D_it, C_it>}, 
\ccc{Linear_program_from_iterators<A_it, B_it, R_it, FL_it, L_it, FU_it, U_it, C_it>}, 
\ccc{Nonnegative_quadratic_program_from_iterators<A_it, B_it, R_it, D_it, C_it>}, and
\ccc{Nonnegative_linear_program_from_iterators<A_it, B_it, R_it, C_it>}. 

If you want to read a quadratic program in \ccc{MPSFormat} from a stream, 
please use the model \ccc{Quadratic_program_from_mps<NT>}.

\ccIsModel
\ccc{QuadraticProgramInterface}\\
\ccc{LinearProgramInterface}\\
\ccc{NonnegativeQuadraticProgramInterface}\\
\ccc{NonnegativeLinearProgramInterface}

\ccTypes

\ccNestedType{NT}{The number type of the program entries.}

\ccCreation
\ccIndexClassCreation
\ccCreationVariable{qp}

\ccConstructor{
  Quadratic_program
  (CGAL::Comparison_result default_r = CGAL::EQUAL,
   bool default_fl = true,
   const NT& default_l = 0,
   bool default_fu = false,
   const NT& default_u = 0);}
{constructs a quadratic program with no variables and no constraints, ready
for data to be added.  Unless relations are explicitly set, they will 
be of type \ccc{default_r}. Unless bounds are explicitly set, they
will be as specified by \ccc{default_fl} (finite lower bound?), 
\ccc{default_l} (lower bound value if lower bound is finite),
\ccc{default_fu} (finite upper bound?), and
\ccc{default_l} (upper bound value if upper bound is finite). If all
parameters take their default values, we thus get equality constraints 
and bounds $x\geq 0$ by default. Numerical entries that are not 
explicitly set will default to $0$.\ccPrecond if 
  $\ccc{default_fl}=\ccc{default_fu}=\ccc{true}$, then
  $\ccc{default_l}\leq\ccc{default_u}$}

\ccOperations

\ccCreationVariable{qp}

\ccMethod{bool is_linear() const;}{returns \ccc{true} if and only if 
\ccVar\ is a linear program.}

\ccMethod{bool is_nonnegative() const;}{returns \ccc{true} if and only if  
\ccVar\ is a nonnegative program.}

\ccMethod{void set_a (int j, int i, const NT& val);}{sets the entry $A_{ij}$
in column $j$ and row $i$ of the constraint matrix $A$ of \ccVar\ to 
\ccc{val}. An existing entry is overwritten. \ccVar\ is enlarged if
necessary to accomodate this entry.}

\ccMethod{void set_b (int i, const NT& val);}{sets the entry $b_i$
of \ccVar\ to \ccc{val}. An existing entry is overwritten. 
\ccVar\ is enlarged if necessary to accomodate this entry.}

\ccMethod{void set_l (int j, bool is_finite, const NT& val = NT(0));}
{if \ccc{is_finite}, this sets the entry $l_j$ of \ccVar\ to \ccc{val},
otherwise it sets $l_j$ to $-\infty$. An existing entry is overwritten. 
\ccVar\ is enlarged if necessary to accomodate this entry.}

\ccMethod{void set_u (int j, bool is_finite, const NT& val = NT(0));}
{if \ccc{is_finite}, this sets the entry $u_j$ of \ccVar\ to \ccc{val},
otherwise it sets $u_j$ to $\infty$. An existing entry is overwritten. 
\ccVar\ is enlarged if necessary to accomodate this entry.}

\ccMethod{void set_c (int j, const NT& val);}{sets the entry $c_j$
of \ccVar\ to \ccc{val}. An existing entry is overwritten. 
\ccVar\ is enlarged if necessary to accomodate this entry.}

\ccMethod{void set_c0 (const NT& val);}{sets the entry $c_0$
of \ccVar\ to \ccc{val}. An existing entry is overwritten.}

\ccMethod{void set_d (int i, int j, const NT& val);}{sets the entries 
$2D_{ij}$ and $2D_{ji}$ of \ccVar\ to \ccc{val}. Existing entries are 
overwritten. \ccVar\ is enlarged if necessary to accomodate these entries.
\ccPrecond \ccc{j <= i}}

\ccExample

\ccReferToExampleCode{QP_solver/first_qp.cpp}\\
\ccReferToExampleCode{QP_solver/first_lp.cpp}\\
\ccReferToExampleCode{QP_solver/first_nonnegative_qp.cpp}\\
\ccReferToExampleCode{QP_solver/first_nonnegative_lp.cpp}\\
\ccReferToExampleCode{QP_solver/invert_matrix.cpp}
\ccSeeAlso
\ccc{Quadratic_program_from_iterators<A_it, B_it, R_it, FL_it, L_it, FU_it, U_it, D_it, C_it>}\\
\ccc{Linear_program_from_iterators<A_it, B_it, R_it, FL_it, L_it, FU_it, U_it, C_it>}\\
\ccc{Nonnegative_quadratic_program_from_iterators<A_it, B_it, R_it, D_it, C_it>}\\
\ccc{Nonnegative_linear_program_from_iterators<A_it, B_it, R_it, C_it>}\\ 
\ccc{Quadratic_program_from_mps<NT>}

\end{ccRefClass}
