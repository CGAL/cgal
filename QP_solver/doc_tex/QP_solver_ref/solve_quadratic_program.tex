\begin{ccRefFunction}{solve_quadratic_program}

\ccInclude{CGAL/QP_functions.h}

This function solves a quadratic program, using some exact
Integral Domain \ccc{ET} for its computations.   

\ccFunction{template <QuadraticProgramInterface, ET>
  Quadratic_program_solution<ET> solve_quadratic_program 
  (const QuadraticProgramInterface& qp, const ET&);}
{returns the solution of the quadratic program \ccc{qp}, solved
with exact number type \ccc{ET}.}

\ccHeading{Requirements}
\ccc{ET} is a model of the concepts \ccc{IntegralDomain} and
\ccc{RealEmbeddable}; it must
be an exact type, and all entries of \ccc{qp} are convertible to 
\ccc{ET}.

Here are some recommended combinations of input type (the type of
the \ccc{qp} entries) and \ccc{ET}.

\begin{tabular}{lll} 
input type        &| &  \ccc{ET} \\ \hline
\ccc{double}      &| & \ccc{MP_Float}, \ccc{Gmpzf}, or \ccc{Gmpq} \\
\ccc{int}         &| & \ccc{MP_Float}, or \ccc{Gmpz} \\
any exact type \ccc{NT} &|&  \ccc{NT}
\end{tabular}

{\bf Note:} by default, this function performs a large number of 
runtime-checks to ensure consistency during the solution process.
However, these checks slow down the computations by a considerable
factor. For maximum efficiency, it is advisable to define the macros
\texttt{CGAL\_QP\_NO\_ASSERTIONS} or \texttt{NDEBUG}.

\ccExample
\ccReferToExampleCode{QP_solver/first_qp.cpp}

\ccSeeAlso

The models of \ccRefIdfierPage{QuadraticProgramInterface}:

\ccc{Quadratic_program<NT>}\\
\ccc{Quadratic_program_from_mps<NT>}\\
\ccc{Quadratic_program_from_iterators<A_it, B_it, R_it, FL_it, L_it, FU_it, U_it, D_it, C_it>}
\end{ccRefFunction}
