\begin{ccRefClass}{Quadratic_program_solution<ET>}

\ccInclude{CGAL/QP_solution.h}

\ccDefinition
An object of class \ccRefName\ represents the solution of a linear or 
convex quadratic program of the general form
%%
\begin{eqnarray*}
\mbox{(QP)}& \mbox{minimize} & x^{T}Dx+c^{T}x+c_0 \\
&\mbox{subject to}   & Ax\qprel b, \\
&                    & l \leq x \leq u
\end{eqnarray*}
%%
in $n$ real variables $x=(x_0,\ldots,x_{n-1})$. If $D=0$, the program is
a linear program; if the variable bounds are $x\geq 0$, we have a 
nonnegative program. Objects of type \ccRefName\ are returned by any of 
the four functions 
\ccc{solve_quadratic_program}, \ccc{solve_linear_program},
\ccc{solve_nonnegative_quadratic_program}, and
\ccc{solve_nonnegative_linear_program}. 

\ccHeading{Terminology}
If there is no $x$ that satisfies all the (in)equalities,
the program is called \emph{infeasible}, otherwise, it is \emph{feasible},
and any $x$ that satisfies all (in)equalities is called a \emph{feasible
solution}. 

If the objective function value becomes arbitrarily small over the
\emph{feasible region} (the set of feasible solutions), the program
is called \emph{unbounded}, and \emph{bounded} otherwise. 

Any program that is both feasible and bounded has at least one
feasible solution $x^*$ whose objective function value is not larger
than that of any other feasible solution. This is called an 
\emph{optimal solution}.

Every convex quadratic program (even if it is infeasible or unbounded)
has a 'solution' in form of an object of the class \ccRefName.

\ccTypes

\ccNestedType{ET}{The exact number type that was used to solve the 
program.}

\ccNestedType{QP_status}{An enumeration type containing the values
\ccc{QP_OPTIMAL}, \ccc{QP_INFEASIBLE}, and \ccc{QP_UNBOUNDED}.}

\ccNestedType{Variable_value_iterator}{An iterator type with value type 
\ccc{Quotient<ET>} to go over the values of the variables in the solution.} 

\ccNestedType{Variable_numerator_iterator}{An iterator type with value type 
\ccc{ET} to go over the numerators of the variable values
with respect to a common denominator.}

\ccNestedType{Index_iterator}{An iterator type with value type \ccc{int}
to go over the indices of the basic variables and the basic constraints.}

\ccNestedType{Optimality_certificate_iterator}{An iterator type with 
value type \ccc{Quotient<ET>} to go over an $m$-vector $\lambda$ that proves 
optimality of the solution.} 

\ccNestedType{Optimality_certificate_numerator_iterator}{An iterator type 
with value type \ccc{ET} to go over the numerators of the vector $\lambda$
with respect to a common denominator.}

\ccNestedType{Infeasibility_certificate_iterator}{An iterator type with 
value type \ccc{ET} to go over an $m$-vector $\lambda$ that proves 
infeasibility of the solution.} 

\ccNestedType{Unboundedness_certificate_iterator}{An iterator type with 
value type \ccc{ET} to go over an $n$-vector $w$ that proves unboundedness 
of the solution.} 

\ccCreation
\ccIndexClassCreation
\ccCreationVariable{sol}

\ccConstructor{Quadratic_program_solution();}
{constructs a void instance of \ccRefName\ that is associated to 
no program.}

Objects of type \ccRefName\ can be copied and assigned.

Objects of type \ccRefName\ that are associated to an actual
program are returned by any of the four functions 
\ccc{solve_quadratic_program}, \ccc{solve_linear_program},
\ccc{solve_nonnegative_quadratic_program}, and
\ccc{solve_nonnegative_linear_program}. 

\ccOperations

\ccMethod{bool is_void() const;}{returns \ccc{true} iff \ccVar\ is not
associated to a program. The condition 
!\ccVar\ccc{.is_void()} is a precondition for all access methods below.} 

\ccHeading{Solution status}
Here are the access methods for the status of the solution.

\ccMethod{bool is_optimal() const;}{returns \ccc{true} iff \ccVar\
is an optimal solution of the associated program.}

\ccMethod{bool is_infeasible() const;}{returns \ccc{true} iff the 
associated program is infeasible.}

\ccMethod{bool is_unbounded() const;}{returns \ccc{true} iff the 
associated program is unbounded.}

\ccMethod{QP_status status() const;}{returns the status of the solution;
this is one of the values \ccc{QP_OPTIMAL}, \ccc{QP_INFEASIBLE}, and 
\ccc{QP_UNBOUNDED}, depending on whether the quadratic program asociated
to \ccVar\ has an optimal solution, is infeasible, or is unbounded.}

\ccHeading{Solution values}
The actual solution (variable and objective function values) can be
accessed as follows.

\ccMethod{Quotient<ET> objective_value() const;}{returns the objective 
function value of \ccVar.}

\ccMethod{ET objective_value_numerator() const;}{returns the numerator
of the objective function value of \ccVar.}

\ccMethod{ET objective_value_denominator() const;}{returns the denominator
of the objective function value of \ccVar.}

\ccMethod{Variable_value_iterator variable_values_begin() const;}{returns
a random-access iterator over the values of the
variables in \ccVar. The value type is \ccc{Quotient<ET>}, and the valid
iterator range has length $n$.}

\ccMethod{Variable_value_iterator variable_values_end() const;}{returns
the corresponding past-the-end iterator.}

\ccMethod{Variable_numerator_iterator variable_numerators_begin() const;}
{returns a random-access iterator \ccc{it} over the values 
of the variables in \ccVar, with respect to a common
denominator of all variables. The value type is \ccc{ET}, and the valid
iterator range has length $n$.}

\ccMethod{Variable_numerator_iterator variable_numerators_end() const;}
{returns the corresponding past-the-end iterator.}

\ccMethod{const ET& variables_common_denominator() const;}
{returns the common denominator of the variable values as referred to
by the previous two methods.}

\ccHeading{Basic variables and constraints}
The solution of a linear or quadratic program distinguishes 'important'
variables (the ones not attaining one of their bounds), and 'important'
constraints (the ones being satisfied with equality). The following
methods grant access to them.

\ccMethod{Index_iterator basic_variable_indices_begin() const;}
{returns a random access iterator over the indices of the basic
variables. The value type is \ccc{int}. It is guaranteed that any 
variable that is not basic in \ccVar\ attains one of its bounds. 
In particular, if the bounds are of type $x\geq 0$, all non-basic 
variables have value $0$.}

\ccMethod{Index_iterator basic_variable_indices_end() const;}
{returns the corresponding past-the-end iterator.}

\ccMethod{int number_of_basic_variables() const;}
{returns the number of basic variables, equivalently the length
of the range determined by the previous two iterators.}

\ccMethod{Index_iterator basic_constraint_indices_begin() const;}
{returns a random access iterator over the indices of the basic
constraints in the system $Ax\qprel b$. The value type is \ccc{int}. 
It is guaranteed that any basic constraint is satisfied with equality.
In particular, if the system is of type $Ax=b$, all constraints are
basic.}

\ccMethod{Index_iterator basic_constraint_indices_end() const;}
{returns the corresponding past-the-end iterator.}

\ccMethod{int number_of_basic_constraints() const;}
{returns the number of basic constraint, equivalently the length
of the range determined by the previous two iterators.}

\ccHeading{Certificates}

A certificate is a vector that admits a simple proof for the 
correctness of the solution. Any non-void object of \ccRefName\ 
comes with such a certificate.

{\bf Lemma 1 (optimality certificate):} A feasible $n$-vector $x^*$ 
is an optimal solution of (QP) if an $m$-vector $\lambda$ with the
following properties exist.
\begin{enumerate}
\item if the $i$-th constraint is of type $\leq$ ($\geq$, respectively), 
then $\lambda_i\geq 0$ ($\lambda_i\leq 0$, respectively).
\item $\lambda^T(Ax^*-b) = 0$.
\item \[
\begin{array}{llll}
&&\geq 0, & \mbox{if $x^*_j = l_j < u_j$} \\
(c^T + \lambda^T A + 2{x^*}^TD)_j& \quad  &= 0, & \mbox{if $l_j < x^*_j < u_j$} \\
&&\leq 0, & \mbox{if $l_j < u_j = x^*_j$.}
\end{array}\]
\end{enumerate}

\begin{ccAdvanced}
{\bf Proof:} Let $x$ be any feasible solution. We need to prove that
\[c^Tx + x^TDx \geq c^Tx^* + {x^*}^TDx^*.\]

For this, we argue as follows.
\[
\begin{array}{lcll}
c^Tx + 2{x^*}^TDx &\geq& c^Tx + 2{x^*}^TDx + \lambda^T(Ax-b) &  
\mbox{(by $Ax\qprel b$ and 1.)} \\
                  &=& (c^T + \lambda^T A + 2{x^*}^TD)x - \lambda^Tb \\
                  &\geq& (c^T + \lambda^T A + 2{x^*}^TD)x^* - \lambda^Tb &
\mbox{(by $l\leq x \leq u$ and 3.)} \\
                  &=& c^Tx^* + 2{x^*}^TDx^* &
\mbox{(by 2.)}
\end{array}
\]

After adding $x^TDx - x^TDx - {x^*}^TDx^* = -{x^*}^TDx^*$ to both sides of
this inequality, we get
\[
c^Tx + x^TDx - (x-x^*)^TD(x-x^*) \geq c^Tx^* + {x^*}^TDx^*,
\] 
and since $D$ is positive semidefinite, we have
$(x-x^*)^TD(x-x^*)\geq 0$ and the lemma follows.
\end{ccAdvanced}

\ccMethod{Optimality_certificate_iterator 
optimality_certifcate_begin() const;}
{returns a random access iterator over the optimality certificate 
$\lambda$ as given in Lemma 1, with respect to the solution $x^*$ 
obtained from \ccVar\ccc{.variable_values_begin()}. The value type
is \ccc{Quotient<ET>}, and the valid iterator range has length $m$.
\ccPrecond \ccVar\ccc{.is_optimal()}}

\ccMethod{Optimality_certificate_iterator 
  optimality_certificate_end() const;}
{returns the corresponding past-the-end iterator.}

\ccMethod{Optimality_certificate_numerator_iterator 
optimality_certifcate_numerator_begin() const;}
{returns a random access iterator over the numerator values
of the optimality certificate $\lambda$, with respect to a 
common denominator of all values. The value type
is \ccc{ET}, and the valid iterator range has length $m$.}

\ccMethod{Optimality_certificate_numerator_iterator 
  optimality_certificate_numerator_end() const;}
{returns the corresponding past-the-end iterator.}

\ccMethod{ET optimality_certificate_denominator() const;}
{returns the common denominator of the certificate values 
$\lambda$ as referred to by the previous two methods.}

{\bf Lemma 2 (infeasibility certificate):} The program (QP) is
infeasible if an $m$-vector $\lambda$ with the
following properties exist.
\begin{enumerate}
\item if the $i$-th constraint is of type $\leq$ ($\geq$, respectively), 
then $\lambda_i\geq 0$ ($\lambda_i\leq 0$, respectively).
\item 
\[
\begin{array}{llll}
&&\geq 0 & \mbox{if $u_j=\infty$} \\
\lambda^T A_j &\quad  \\
&&\leq 0 & \mbox{if $l_j=-\infty$.}
\end{array}
\]
\item \[\lambda^Tb \quad<\quad \sum_{j: \lambda^TA_j <0} \lambda^TA_j u_j 
\quad+\quad  \sum_{j: \lambda^TA_j >0} \lambda^TA_j l_j.\]
\end{enumerate}

\begin{ccAdvanced}
{\bf Proof:} Let us assume for the purpose of obtaining a contradiction
that there is a feasible solution $x$. Then we get
\[
\begin{array}{lcll}
0 &\geq& \lambda^T(Ax -b) &  \mbox{(by $Ax\qprel b$ and 1.)} \\
  &=& \sum_{j: \lambda^TA_j <0} \lambda^TA_j x_j 
\quad+\quad  \sum_{j: \lambda^TA_j >0} \lambda^TA_j x_j - \lambda^T b \\
  &\geq& \sum_{j: \lambda^TA_j <0} \lambda^TA_j u_j 
\quad+\quad  \sum_{j: \lambda^TA_j >0} \lambda^TA_j l_j - \lambda^T b &
\mbox{(by $l\leq x \leq u$ and 2.)} \\
  &>& 0 & \mbox{(by 3.)}, 
\end{array}
\]
and this is the desired contradiction $0>0$.
\end{ccAdvanced}

\ccMethod{Infeasibility_certificate_iterator 
  infeasibility_certificate_begin() const;}
{returns a random access iterator over the infeasibility certificate 
$\lambda$ as given in Lemma 2. The value type
is \ccc{ET}, and the valid iterator range has length $m$.
\ccPrecond \ccVar\ccc{.is_infeasible()}}

\ccMethod{Infeasibility_certificate_iterator 
  infeasibility_certificate_end() const;}
{returns the corresponding past-the-end iterator.}

{\bf Lemma 3 (unboundedness certificate:)} Let $x^*$ be a feasible
solution of (QP). The program (QP) is unbounded if an $n$-vector 
$w$ with the following properties exist.
\begin{enumerate}
\item if the $i$-th constraint is of type $\leq$ ($\geq, =$, respectively),
then $(Aw)_j\leq 0$ ($(Aw)_j\geq 0, (Aw)_j=0$, respectively).
\item 
\[
\begin{array}{llll}
&&\geq 0 & \mbox{if $l_j$ is finite} \\
w_j &\quad  \\
&&\leq 0 & \mbox{if $u_j$ is finite.}
\end{array}
\]
\item $w^TDw=0$ and $(c^T+2{x^*}^TD)w<0$.
\end{enumerate}

The vector $w$ is called an \emph{unbounded direction}.

\begin{ccAdvanced}
{\bf Proof:} For a real number $t$, consider the vector $x(t):=x^*+tw$. By 1.
and 2., $x(t)$ is feasible for all $t\geq 0$. The objective function value
of $x(t)$ is
\begin{eqnarray*}
c^T x(t) + x(t)^TD x(t) &=& 
c^Tx^* + tc^Tw + {x^*}^TDx^* +  2t{x^*}^TDw + t^2 w^TDw  \\
&=& c^Tx^* + {x^*}^TDx^* + t(c^T + 2{x^*}^TD)w + t^2w^TDw.
\end{eqnarray*}
By condition 3., this tends to $-\infty$ for $t\rightarrow\infty$, so
the problem is indeed unbounded. 
\end{ccAdvanced}

\ccMethod{Unboundedness_certificate_iterator 
unboundedness_certificate_begin() const;}
{returns a random acess iterator over the unbounded direction $w$
as given in Lemma 3,with respect to the solution $x^*$ 
obtained from \ccVar\ccc{.variable_values_begin()}. The value type
is \ccc{ET}, and the valid iterator range has length $n$.
\ccPrecond \ccVar\ccc{.is_unbounded()}}

\ccMethod{Unboundedness_certificate_iterator 
unboundedness_certificate_end();}
{returns the corresponding past-the-end iterator.}

\ccSeeAlso

The program concepts 

\ccRefIdfierPage{QuadraticProgramInterface}\\
\ccRefIdfierPage{LinearProgramInterface}\\
\ccRefIdfierPage{NonnegativeQuadraticProgramInterface}\\
\ccRefIdfierPage{NonnegativeLinearProgramInterface}

and the functions that compute objects of class \ccRefName\ from
models of these concepts:

\ccRefIdfierPage{solve_quadratic_program}\\
\ccRefIdfierPage{solve_linear_program}\\
\ccRefIdfierPage{solve_nonnegative_quadratic_program}\\
\ccRefIdfierPage{solve_nonnegative_linear_program}
\end{ccRefClass}
