% +------------------------------------------------------------------------+
% | Reference manual page: AABBPrimitive.tex
% +------------------------------------------------------------------------+
% | 21.02.2009   Author
% | Package: Package
% |
\RCSdef{\RCSAABBPrimitiveRev}{$Id: header.tex 40270 2007-09-07 15:29:10Z lsaboret $}
\RCSdefDate{\RCSAABBPrimitiveDate}{$Date: 2007-09-07 17:29:10 +0200 (Ven, 07 sep 2007) $}
% |
\ccRefPageBegin
%%RefPage: end of header, begin of main body
% +------------------------------------------------------------------------+


\begin{ccRefConcept}{AABBPrimitive}

%% \ccHtmlCrossLink{}     %% add further rules for cross referencing links
%% \ccHtmlIndexC[concept]{} %% add further index entries

\ccDefinition
  
The concept \ccRefName\ describes the requirements for the primitives stored in the AABB data structure. The concept encapsulates a type for the input datum (a geometric object) and an identifier type by which those primitives are referred to. The concept \ccRefName\ also refines the concepts DefaultConstructible and Assignable.

\ccTypes

\ccNestedType{Point}{3D point type.}

\ccNestedType{Datum}{Type of input datum.}

\ccNestedType{Id}{Type of identifiers by which the input objects are referred to.}


% variable name
\ccCreationVariable{primitive}  

\ccOperations

\ccMethod{Datum datum();}
{Returns the datum (geometric object) wrapped inside the primitive.}

\ccMethod{Id id();}
{Returns the corresponding identifier. This identifier is only used as a reference for the objects in the output of the \ccc{AABB_tree} methods.}

\ccMethod{Point reference_point();}
{Returns a 3D point located on the geometric object wrapped by the primitive. This function is used to sort the primitives during the AABB tree construction and to construct the search KD-tree internal to the AABB tree in order to accelerate projection queries.}

\ccSeeAlso
\ccc{AABB_tree<Traits>}

\ccExample

The \ccc{Primitive} type can be, e.g., a wrapper around a \ccc{Handle}. Assume for instance that the input objects are the triangle faces of a mesh stored as a \ccc{CGAL::Polyhedron}. The \ccc{Datum} would be a \ccc{Triangle_3} and the \ccc{Id} would be a polyhedron \ccc{Face_handle}. Function \ccc{datum()} can return either a \ccc{Triangle_3} constructed on the fly (memory saving) from the id (face handle) or a \ccc{Triangle_3} stored internally (faster but memory consuming). 

\end{ccRefConcept}

% +------------------------------------------------------------------------+
%%RefPage: end of main body, begin of footer
\ccRefPageEnd
% EOF
% +------------------------------------------------------------------------+

