% +------------------------------------------------------------------------+
% | Reference manual page: AABBPrimitive.tex
% +------------------------------------------------------------------------+
% | 21.02.2009   Author
% | Package: Package
% |
\RCSdef{\RCSAABBPrimitiveRev}{$Id: header.tex 40270 2007-09-07 15:29:10Z lsaboret $}
\RCSdefDate{\RCSAABBPrimitiveDate}{$Date: 2007-09-07 17:29:10 +0200 (Ven, 07 sep 2007) $}
% |
\ccRefPageBegin
%%RefPage: end of header, begin of main body
% +------------------------------------------------------------------------+


\begin{ccRefConcept}{AABBPrimitive}

%% \ccHtmlCrossLink{}     %% add further rules for cross referencing links
%% \ccHtmlIndexC[concept]{} %% add further index entries

\ccDefinition
  
The concept \ccRefName\ describes the requirements for the primitives stored in the AABB data structure. The concept encapsulates a type for the geometric primitives and an identifier type by which those primitives are referred to. The concept \ccRefName\ also refines the concepts DefaultConstructible and Assignable.

\ccConstructor{AABBPrimitive(Handle h);}
{Constructs a primitive whose geometric primitive is the one h points to.}

% \ccGeneralizes

% ThisConcept \\
% ThatConcept

\ccTypes

\ccNestedType{Data_type}{The geometric primitive type.}

\ccNestedType{Id_type}{The type of identifiers by which the primitives are referred to.}


\ccCreationVariable{primitive}  %% choose variable name
\ccOperations

%\ccMethod{void foo();}{some member functions}
\ccMethod{Data_type get_data();}
{Returns the geometric data wrapped inside the Primitive.}

\ccMethod{Id_type get_id();}
{Returns the corresponding identifier. This identifier is only used as a representant for the primitive in the output of the \ccc{AABB_tree} methods.}

% \ccHasModels

% \ccc{AABB_primitive_base<ID>}


\ccSeeAlso
\ccc{AABB_tree<Traits>}

\ccExample

Usually, the \ccc{Primitive} type is a wrapper around a \ccc{Handle}. Assume for instance, that geometric primitives 
are to be the faces of a mesh stored as a \ccc{CGAL::Polyhedron}, the Primitive type would be a wrapper around the \ccc{Face_handle}  type of the polyhedron. The \ccc{Data_type} would be \ccc{Face},
 the \ccc{Id_type} would be \ccc{Face_handle}, \ccc{get_data()} would dereference the \ccc{Face_handle}, and \ccc{get_id()} would return the \ccc{Face_handle} itself. 

% A short example program.
% Instead of a short program fragment, a full running program can be
% included using the 
% \verb|\ccIncludeExampleCode{Package/AABBPrimitive.C}| 
% macro. The program example would be part of the source code distribution and
% also part of the automatic test suite.

% \begin{ccExampleCode}
% void your_example_code() {
% }
% \end{ccExampleCode}

%% \ccIncludeExampleCode{Package/AABBPrimitive.C}

\end{ccRefConcept}

% +------------------------------------------------------------------------+
%%RefPage: end of main body, begin of footer
\ccRefPageEnd
% EOF
% +------------------------------------------------------------------------+

