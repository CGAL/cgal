% +------------------------------------------------------------------------+
% | Reference manual page: AABBTrianglePrimitive.tex
% +------------------------------------------------------------------------+
% | 25.02.2009   Author
% | Package: Package
% |
\RCSdef{\RCSAABBTrianglePrimitiveRev}{$Id: header.tex 40270 2007-09-07 15:29:10Z lsaboret $}
\RCSdefDate{\RCSAABBTrianglePrimitiveDate}{$Date: 2007-09-07 17:29:10 +0200 (Ven, 07 sep 2007) $}
% |
\ccRefPageBegin
%%RefPage: end of header, begin of main body
% +------------------------------------------------------------------------+


\begin{ccRefConcept}{AABBTrianglePrimitive}

%% \ccHtmlCrossLink{}     %% add further rules for cross referencing links
%% \ccHtmlIndexC[concept]{} %% add further index entries

\ccDefinition
  
The concept \ccRefName\ describes the requirement for the primitive used in the class \ccc{AABB_traits<GeomTraits,TrianglePrimitive>} (which implements a model of \ccc{AABBTraits}) when the objects to be stored in the tree are triangles.

The concept \ccRefName\ refines the concept \ccc{AABBPrimitive} with \ccc{Data} being \ccc{GeomTraits::Triangle_3}.

\ccGeneralizes

\ccc{AABBPrimitive}\\


\ccTypes

\ccNestedType{Triangle_3}{Type for triangles. Must match the triangle type \ccc{GeomTraits::Triangle_3} provided by the geometric traits.}

\ccCreation
\ccCreationVariable{primitive}  %% choose variable name

\ccConstructor{AABBTrianglePrimitive(Id id);}
{Constructs a primitive whose triangle is the one \ccc{id} refers to.}

\ccOperations

\ccMethod{Triangle_3 datum();}
{Builds a triangle from the primitive.}

% \ccHasModels

% \ccc{Some_class},
% \ccc{Some_other_class}.

\ccSeeAlso
\ccc{AABBPrimitive}\\
\ccc{AABB_traits<GeomTraits,TrianglePrimitive>}\\

\end{ccRefConcept}

% +------------------------------------------------------------------------+
%%RefPage: end of main body, begin of footer
\ccRefPageEnd
% EOF
% +------------------------------------------------------------------------+

