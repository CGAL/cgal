\begin{ccRefClass}{Pure_complex_simplex<PCTraits, Data, PCDSSimplex>}

\ccDefinition

The class \ccRefName\ is a model of the concept \ccc{PureComplexSimplex}. It
is used by default for representing simplices in the class
\ccc{Pure_complex<PCTraits, PCDS>}.

A \ccRefName\ stores handles to the vertices of the simplex as well as handles
to its neighboring simplices.

\ccInclude{CGAL/Pure_complex_simplex.h}

\ccParameters

\ccc{PCTraits} must be a model of the concept \ccc{PureComplexTraits}. It
provides geometric types and predicates for use in the
\ccc{Pure_complex<PCTraits, PCDS>} class.

\ccc{Data} is an optional type of data to be stored in the vertex class. The
class template \ccRefName\ accepts that no second parameter be specified. In
this case, \ccc{Data} defaults to \ccc{CGAL::No_vertex_data}.

\ccc{PCDSSimplex} must be a model of the concept \ccc{PureComplexDSSimplex}.
The class template \ccRefName\ accepts that no third parameter be specified.
It also accepts the tag \ccc{CGAL::Default} as third parameter. In both
cases, \ccc{PCDSSimplex} defaults to \ccc{CGAL::Pure_complex_ds_simplex<>}.

\ccInheritsFrom

\ccc{PCDSSimplex} (the third template parameter)

\ccIsModel

\ccc{PureComplexSimplex}

Additionally, the class \ccRefName\ also provides the following type,
constructors and methods:

\ccTypes

\ccTypedef{typedef Data Data;}{The type of the additional data stored in the
simplex. If you read a \ccRefName\ from a stream (a file) or write a \ccRefName
to a stream, then streaming operators \ccc{<<} and \ccc{>>} must be provided for this
type.}

\ccCreation
\ccCreationVariable{s}

\ccConstructor{template< typename T> Pure_complex_simplex(int dmax, const T
& t);}{Sets the maximum possible dimension of the simplex to \ccc{dmax}.
The parameter \ccc{t} is passed to the \ccc{Data} constructor.}

\ccHeading{Data access}

\ccMethod{const Data & data() const;}{Returns a reference to the stored data.}
\ccGlue\ccMethod{Data & data();}{Returns a reference to the stored data.}

\ccHeading{Input/Output}

\ccFunction{istream & operator>>(istream & is, Pure_complex_simplex & v);}%
{Inputs the non-combinatorial information given by the simplex, \emph{i.e.},
the point and other possible information. The data of type \ccc{Data} is
\textbf{also} read.}

\ccFunction{ostream & operator<<(ostream & os, const Pure_complex_simplex & v);}%
{Outputs the non-combinatorial information given by the simplex, \emph{i.e.},
the point and other possible information. The data of type \ccc{Data} is
\textbf{also} written.}

\ccSeeAlso

\ccc{Pure_complex_vertex<PCTraits, Data, PCDSVertex>}

\end{ccRefClass}
