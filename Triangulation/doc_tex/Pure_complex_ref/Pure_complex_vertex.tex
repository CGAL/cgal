\begin{ccRefClass}{Pure_complex_vertex<PCTraits, Data, PCDSVertex>}

\ccDefinition

The class \ccRefName\ is a model of the concept \ccc{PureComplexVertex}. It is
used by default for representing vertices in the class
\ccc{Pure_complex<PCTraits, PCDS>}.

A \ccRefName\ stores a point and an incident simplex.

\ccInclude{CGAL/Pure_complex_vertex.h}

\ccParameters

\ccc{PCTraits} must be a model of the concept \ccc{PureComplexTraits}. It
provides geometric types and predicates for use in the
\ccc{Pure_complex<PCTraits, PCDS>} class. It is of interest here for its
declaration of the \ccc{Point} type.

\ccc{Data} is an optional type of data to be stored in the vertex class. The
class template \ccRefName\ accepts that no second parameter be specified. In
this case, \ccc{Data} defaults to \ccc{CGAL::No_vertex_data}.

\ccc{PCDSVertex} must be a model of the concept \ccc{PureComplexDSVertex}. The
class template \ccRefName\ accepts that no third parameter be specified. It
also accepts the tag \ccc{CGAL::Default} as third parameter. In both cases,
\ccc{PCDSVertex} defaults to \ccc{CGAL::Pure_complex_ds_vertex<>}.

\ccInheritsFrom

\ccc{PCDSVertex} (the third template parameter)

\ccIsModel

\ccc{PureComplexVertex}

Additionally, the class \ccRefName\ also provides the following type,
constructors and methods:

\ccTypes

\ccTypedef{typedef Data Data;}{The type of the additional data stored in the
vertex. If you read a \ccRefName\ from a stream (a file) or write a \ccRefName
to a stream, then streaming operators \ccc{<<} and \ccc{>>} must be provided for this
type.}

\ccCreation
\ccCreationVariable{v}

\ccConstructor{template< typename T> Pure_complex_vertex(Simplex_handle s,
const Point & p, const T & t);}{Constructs a vertex with adjacent simplex
\ccc{s}. The vertex is embedded at point \ccc{p} and the parameter \ccc{t} is
passed to the \ccc{Data} constructor.}

\ccGlue\ccConstructor{template< typename T> Pure_complex_vertex(const Point
& p, const T & t);}{Same as above, but without adjacent simplex.}

\ccGlue\ccConstructor{Pure_complex_vertex();}%
{Same as above, but with default-constructed \ccc{Point} and \ccc{Data}.}

\ccHeading{Data access}

\ccMethod{const Data & data() const;}{Returns a reference to the stored data.}
\ccGlue\ccMethod{Data & data();}{Returns a reference to the stored data.}

\ccHeading{Input/Output}

\ccFunction{istream & operator>>(istream & is, Pure_complex_vertex & v);}%
{Inputs the non-combinatorial information given by the vertex, \emph{i.e.},
the point and other possible information. The data of type \ccc{Data} is
\textbf{also} read.}

\ccFunction{ostream & operator<<(ostream & os, const Pure_complex_vertex & v);}%
{Outputs the non-combinatorial information given by the vertex, \emph{i.e.},
the point and other possible information. The data of type \ccc{Data} is
\textbf{also} written.}

\ccSeeAlso

\ccc{Pure_complex_simplex<PCTraits, Data, PCDSSimplex>}

\end{ccRefClass}
