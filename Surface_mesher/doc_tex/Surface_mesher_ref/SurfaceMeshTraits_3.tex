% +------------------------------------------------------------------------+
% | Reference manual page: SurfaceMeshTraits_3.tex
% +------------------------------------------------------------------------+
% | 02.12.2005   Author
% | Package: Package
% | 
\RCSdef{\RCSSurfaceOracleRev}{$Id$}
\RCSdefDate{\RCSSurfaceOracleDate}{$Date$}
% |
%%RefPage: end of header, begin of main body
% +------------------------------------------------------------------------+


\begin{ccRefConcept}{SurfaceMeshTraits_3}

%% \ccHtmlCrossLink{}     %% add further rules for cross referencing links
%% \ccHtmlIndexC[concept]{} %% add further index entries

\ccDefinition
  
The concept \ccRefName\ describes the knowledge that is required on the
surface to be meshed. A model of this concept
implements an oracle that is able to tell wether a segment
(or a ray,  or a line) intersects the surface or not
and to compute some intersection
points if any. The concept \ccRefName\ also includes a constructor able  to provide
a small set of initial points on the surface.

%\ccGeneralizes
%ThisConcept \\
%ThatConcept

\ccTypes

\ccNestedType{Point_3}{The type of points. 
When a model of this concept is plugged as \ccc{Surface} parameter
in the surface mesher,
\ccc{make_surface_mesh<C2T3,Surface,Criteria,Tag>},
the  type \ccc{Surface::Point_3} is required to match
the point type of the 
three dimensional embedding triangulation 
\ccc{C2T3::Triangulation_3}.}
\ccGlue
\ccNestedType{Segment_3}{The type of segments.}
\ccGlue
\ccNestedType{Ray_3}{The type of rays.}
\ccGlue
\ccNestedType{Line_3}{The type of lines.}


\ccNestedType{Intersect_3}
{A model of this type provides the operator \\
\ccc{CGAL::object operator()(Type1 type1, Surface_3 surface)}
to compute the intersection of the surface  
with an object of type \ccc{Type1} which may be  any of 
\ccc{Segment_3, Ray_3. Line_3} .}


\ccNestedType{Construct_initial_points}
{A model of this type provides the following operators
to construct initial points on the surface \\
\ccc{template <class OutputIteratorPoints>
OutputIteratorPoints ()(OutputIteratorPoints pts);}
which {outputs a set of points on the surface.}
\ccGlue
\ccc{template <class OutputIteratorPoints>
OutputIteratorPoints () (OutputIteratorPoints pts, int n);}
which {outputs a set of \ccc{n} points on the surface.}


%\ccCreation
%\ccCreationVariable{surfor}  %% choose variable name



\ccOperations
The following functions give access to the construction objects:

\ccMethod{Intersect_3 intersect_3_object();}
\ccGlue
\ccMethod{Construct_initial_points construct_initial_points_object()}


\ccHasModels

\ccc{Surface_mesh_traits_3<Surface>}


\ccSeeAlso
\ccc{make_surface_mesh<C2T3,Surface,Criteria,Tag>}\\
%\ccc{ImplicitFunction}\\
%\ccc{ImplicitSurfaceOracleTraits_3}

%Some\_other\_concept,
%\ccc{some_other_function}.

%\ccExample

%A short example program.
%Instead of a short program fragment, a full running program can be
%included using the 
%\verb|\ccIncludeExampleCode{Package/SurfaceOracle.C}| 
%macro. The program example would be part of the source code distribution and
%also part of the automatic test suite.

%\begin{ccExampleCode}
%void your_example_code() {
%}
%\end{ccExampleCode}

%%% \ccIncludeExampleCode{Package/SurfaceOracle.C}

\end{ccRefConcept}

% +------------------------------------------------------------------------+
%%RefPage: end of main body, begin of footer
% EOF
% +------------------------------------------------------------------------+

