% +------------------------------------------------------------------------+
% | Reference manual page: Implicit_surface_3.tex
% +------------------------------------------------------------------------+
% | 09.12.2005   Mariette Yvinec
% | Package: Surface_mesher
% | 
\RCSdef{\RCSImplicitsurface3Rev}{$Id$}
\RCSdefDate{\RCSImplicitsurface3Date}{$Date$}
% |
%%RefPage: end of header, begin of main body
% +------------------------------------------------------------------------+


\begin{ccRefClass}{Implicit_surface_3<Traits, Function>}

%% \ccHtmlCrossLink{}     %% add further rules for cross referencing links
%% \ccHtmlIndexC[class]{} %% add further index entries

\ccDefinition
  
The class \ccRefName\  implements a surface described 
as the zero level
set  of a function \begin{math}f : \R^3 \longrightarrow \R\end{math}.


For this type of surface, the library provides a partial specialization
of the  surface mesher traits generator:
\ccc{Surface_mesh_traits_generator_3<Implicit_surface_3<Traits,
Function> >},
that provides a traits class, model of the concept
\ccc{SurfaceMeshTraits_3},
to be used by the surface mesher.


The parameter \ccc{Traits} is a traits class 
that has to be implemented with a model of 
\ccc{ImplicitSurfaceTraits_3}.
Actually, this traits class implements the oracle needed by the
surface mesher:
the types, predicates and constructors  provided
in \ccc{Traits} are
passed by the surface mesher traits generator
to the generated the traits class
used by the surface mesh generator.
%used in 
%the partial specialisation 
%\ccc{Surface_mesh_traits_generator_3<Implicit_surface_3<Traits, Function>
%  >::Type}
%to generate the model of  \ccc{SurfaceMeshTraits_3}.

The template parameter \ccc{Function}  stands for a model
of the concept \ccc{ImplicitFunction}.
The number type \ccc{Function::FT} has to match
the type \ccc{Traits::FT}.

\ccInclude{CGAL/Implicit_surface_3.h}

\ccCreation
\ccCreationVariable{surface}

\ccConstructor{
Implicit_surface_3(Function f,
                   Sphere_3 bounding_sphere,
                   FT error_bound = FT(1e-3));}
{\ccc{f} is the object of type \ccc{Function} that represents the implicit
  surface.\\
 \ccc{bounding_sphere} is a bounding sphere of the implicit surface. The
 evaluation of \ccc{f} at the center \ccc{c} of this sphere must be
 negative: $f(c)<0$.\\
 \ccc{error_bound} is a relative error bound 
used to  compute intersection points between the implicit surface
and  query segments. This bound is used in  the default generated traits class.
In this traits class, the intersection points between
the surface and segments/rays/line are constructed by dichotomy. The
dichotomy is stopped when the size of the intersected
segment  is less than the  product of \ccc{error_bound} by the
radius of \ccc{bounding_sphere}.}

\ccSeeAlso
\ccc{make_surface_mesh},\\
\ccc{Surface_3} \\
\ccc{Surface_mesh_traits_generator_3<Surface>},\\
\ccc{ImplicitSurfaceTraits}, \\
\ccc{ImplicitFunction}.

\end{ccRefClass}

% +------------------------------------------------------------------------+
%%RefPage: end of main body, begin of footer
% EOF
% +------------------------------------------------------------------------+

