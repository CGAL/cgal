% +------------------------------------------------------------------------+
% | Reference manual page: surface_mesher.tex
% +------------------------------------------------------------------------+
% | 01.12.2005   Author
% | Package: Package
% | 
\RCSdef{\RCSsurfacemesherRev}{$Id$}
\RCSdefDate{\RCSsurfacemesherDate}{$Date$}
% |
%%RefPage: end of header, begin of main body
% +------------------------------------------------------------------------+


\begin{ccRefFunction}{make_surface_mesh}  %% add template arg's if necessary

%% \ccHtmlCrossLink{}     %% add further rules for cross referencing links
%% \ccHtmlIndexC[function]{} %% add further index entries

\ccInclude{CGAL/make_surface_mesh.h}

\ccDefinition
  
The function \ccRefName\ is a surface mesh generator,
that is a function to build a two dimensional mesh 
approximating  a surface.

The library provides two overloaded version 
of this function:

\ccThree{2,5cm}{4cm}{}


\ccGlobalFunction{template <class SurfaceMeshC2T3,
                            class Surface,
                            class FacetsCriteria,
                            class Tag >
void make_surface_mesh(SurfaceMeshC2T3& c2t3,
                       Surface surface,
                       FacetsCriteria criteria,
                       Tag,
		       int initial_number_of_points = 20) ;}


\ccGlobalFunction{template <class SurfaceMeshC2T3,
                            class SurfaceMeshTraits,
                            class FacetsCriteria,
                            class Tag >
void make_surface_mesh(SurfaceMeshC2T3& c2t3,
                       SurfaceMeshTraits::Surface_3  surface,
		       SurfaceMeshTraits traits,
                       FacetsCriteria criteria,
                       Tag,
		       int initial_number_of_points = 20 );}



\ccParameters
The template parameter  \ccc{SurfaceMeshC2T3}
is required to be a model of the concept
\ccc{SurfaceMeshComplex_2InTriangulation_3},
a data structure able to represent a two dimensional
complex  embedded in a three dimensional triangulation.
The argument \ccc{c2t3} of type \ccc{SurfaceMeshC2T3},  passed by reference
to the surface mesh generator, 
is used to maintain  the current approximating mesh and it stores
the final mesh at the end of the procedure.
The type \ccc{SurfaceMeshC2T3} is in particular required to
provide a type \ccc{SurfaceMeshC2T3::Triangulation_3}
for the three dimensional triangulation
embedding the surface mesh.
The vertex and cell base classes of the triangulation
\ccc{SurfaceMeshC2T3::Triangulation_3} are required
to be  models  of the concepts 
\ccc{SurfaceMeshVertexBase_3} and 
\ccc{SurfaceMeshCellBase_3} respectively.


The template parameter \ccc{Surface}  stands for the surface type.
This type has to be a model of the concept \ccc{Surface_3}.

The knowledge on the surface, required by the surface mesh generator
is  encapsulated in a
traits class. Actually, the mesh generator accesses the surface to be meshed
through this traits class only. 
The traits class is required to be a model
of the concept \ccc{SurfaceMeshTraits_3}.

In the first  version
of  \ccRefName\, the surface type is a template parameter \ccc{Surface}
and the surface mesh generator traits type 
is  automatically generated form the surface type  through
the class 
\ccc{Surface_mesh_traits_generator_3<Surface>}.

The difference between the two overloaded versions of
\ccc{make_surface_mesh}
can be explained as follows
\begin{itemize}
\item
In the first  overloaded version of
of \ccc{make_surface_mesh},  the surface type  is given  
as template parameter  (\ccc{Surface}) and the \ccc{surface}
to be meshed is passed as parameter to the mesh generator.
In that case the surface mesh generator traits type 
is  automatically generated form the surface type
by an auxiliary class called  the \ccc{Surface_mesh_traits_generator_3}.
  %through
%the class 
%\ccc{Surface_mesh_traits_generator_3<Surface>}.
%(This mechanism is similar to the 
%\ccc{Kernel_traits}  mechanism.) \\
\item In the second overloaded version of \ccc{make_surface_mesh}, 
the surface mesh generator traits type is provided
by the  template parameter \ccc{SurfaceMeshTraits_3}
and the surface type is obtained from this traits type.
Both the surface and the traits 
are passed to the mesh generator as arguments. 
\end{itemize}


The first overloaded version can be used
whenever the surface type either provides  a nested type
\ccc{Surface::Surface_mesher_traits_3} 
that is  a model of \ccc{SurfaceMeshTraits_3}
or is a surface type for which a specialization
of the traits generator \ccc{Surface_mesh_traits_generator_3<Surface>}
is provided.
Currently, the library provides partial specializations
of  \ccc{Surface_mesher_traits_generator_3<Surface>}
for implicit surfaces (\ccc{Implicit_surface_3<Traits, Function>}) and 
gray level images (\ccc{Gray_level_image_3<FT, Point>}).



The template parameter \ccc{FacetsCriteria} has to be a model
of the concept \ccc{SurfaceMeshFacetsCriteria_3}.
The argument of type  \ccc{FacetsCriteria} passed to the surface
mesh generator specifies the size and shape  requirements
on the output surface mesh.

The template parameter \ccc{Tag}
is a tag whose type affects the behavior of the
meshing algorithm. The function \ccRefName\  has specialized versions
for the following  tag types: \\
- \ccc{Manifold_tag}: the output mesh is guaranteed to be a manifold
surface without boundary.\\
- \ccc{Manifold_with_boundary_tag}: the output mesh is guaranteed to be
manifold but may have boundaries.\\
- \ccc{Non_manifold_tag}: the algorithm relies on the given criteria and
guarantees nothing else.



The Delaunay refinement
process is started with an initial set of points which is the union 
of two sets: the
set of vertices in the initial  triangulation pointed to by the
\ccc{c2t3} argument   and a set of
points provided by the traits class.
The optional parameter  \ccc{initial_number_of_points}
allows to monitor the number of points in this second set.
(This parameter is passed to the \ccc{operator()} of 
the constructor object \ccc{Construct_initial_points} 
in the traits class.)
The meshing algorithm requires that the initial set of points
includes at least one point
on each connected components of the surface to be meshed.
one.


 

%\ccIsModel

%Concept

\ccSeeAlso
%\ccc{Complex2InTriangulation3} \\
\ccc{SurfaceMeshComplex_2InTriangulation_3} \\
\ccc{SurfaceMeshCellBase_3} \\
\ccc{SurfaceMeshVertexBase_3} \\
\ccc{Surface_3} \\
\ccc{SurfaceMeshFacetsCriteria_3} \\
\ccc{Surface_mesh_default_triangulation_3}


%\ccExample

%A short example program.
%Instead of a short program fragment, a full running program can be
%included using the 
%\verb|\ccIncludeExampleCode{Package/surface_mesher.cpp}| 
%macro. The program example would be part of the source code distribution and
%also part of the automatic test suite.

%\begin{ccExampleCode}
%void your_example_code() {
%}
%\end{ccExampleCode}

%% \ccIncludeExampleCode{Package/make_surface_mesh.cpp}

\end{ccRefFunction}

% +------------------------------------------------------------------------+
%%RefPage: end of main body, begin of footer
% EOF
% +------------------------------------------------------------------------+

