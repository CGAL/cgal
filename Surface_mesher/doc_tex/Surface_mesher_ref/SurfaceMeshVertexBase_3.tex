% +------------------------------------------------------------------------+
% | Reference manual page: SurfaceMeshVertexBase_3.tex
% +------------------------------------------------------------------------+
% | 09.12.2005   Author
% | Package: Package
% | 
\RCSdef{\RCSSurfaceMeshVertexBaseRev}{$Id$}
\RCSdefDate{\RCSSurfaceMeshVertexBaseDate}{$Date$}
% |
%%RefPage: end of header, begin of main body
% +------------------------------------------------------------------------+

\begin{ccRefConcept}{SurfaceMeshVertexBase_3}

%% \ccHtmlCrossLink{}     %% add further rules for cross referencing links
%% \ccHtmlIndexC[concept]{} %% add further index entries

\ccDefinition
  
The concept \ccRefName\ describes the vertex base type
of the three dimensional triangulation used
to embed the surface mesh.

More precisely,
the first template parameter \ccc{C2T3} of the surface mesher
\ccc{make_surface_mesh}
is a model of the concept 
\ccc{SurfaceMeshComplex_2InTriangulation_3} 
which describes a data structure to store
a pure two dimensional complex 
embedded in a three dimensional triangulation.
In particular, the type \ccc{C2T3} is required to provide
a three dimensional triangulation type
\ccc{C2T3::Triangulation_3}
The concept \ccRefName\ describes the vertex base type
required in this triangulation type.





\ccGeneralizes

\ccc{TriangulationVertexBase_3}



The class \ccRefName\ includes two boolean  mark,
\ccc{in_complex_mark}  and \ccc{regular_or_boundary_mark}, 
to store some information about the status of the vertex.
To allow a caching mechanism, 
the concept \ccRefName\ stores two additionnal boolean mark
to mark the validity of the previous ones.




%\ccTypes

%\ccNestedType{TYPE}{some nested types}

\ccCreation
\ccCreationVariable{vb}  %% choose variable name


\ccOperations

\ccMethod{bool regular_or_boundary_mark();}
{Returns the value of the \ccc{regular_or_boundary_mark}.}
\ccGlue
\ccMethod{void set_regular_or_boundary_mark(bool b);}
{ Sets the value of the \ccc{regular_or_boundary_mark} to \ccc{b}.}
\ccGlue
\ccMethod{bool regular_or_boundary_validity_mark();}
{Returns the value of the \ccc{regular_or_boundary_vaidity_mark}.}
\ccGlue
\ccMethod{void set_regular_or_boundary_validity_mark(bool b);}
{Sets the value of the \ccc{regular_or_boundary_validity_mark} to \ccc{b}.}

\ccMethod{bool in_complex_mark();}
{Returns the value of the \ccc{in_complex_mark}.}
\ccGlue
\ccMethod{void  set_in_complex_mark(bool b);}
{Sets the value of the \ccc{in_complex_mark} to \ccc{b}.}
\ccGlue
\ccMethod{bool in_complex_validity_mark();}
{Returns the value of the \ccc{in_complex_validity_mark}.}
\ccGlue
\ccMethod{void set_in_complex_validity_mark(bool b);}
{Sets the value of the \ccc{in_complex_validity_mark} to \ccc{b}.}


\ccHasModels

\ccc{Surface_mesh_vertex_base_3<Gt,Vb>}


\ccSeeAlso

\ccc{SurfaceMesherComplex_2InTriangulation_3} \\
\ccc{Surface_mesh_complex_2_in_triangulation_3<Tr>}.

%\ccExample

%A short example program.
%Instead of a short program fragment, a full running program can be
%included using the 
%\verb|\ccIncludeExampleCode{Package/SurfaceMeshVertexBase_3.C}| 
%macro. The program example would be part of the source code distribution and
%also part of the automatic test suite.

%\begin{ccExampleCode}
%void your_example_code() {
%}
%\end{ccExampleCode}

%%% \ccIncludeExampleCode{Package/SurfaceMeshVertexBase_3.C}

\end{ccRefConcept}

% +------------------------------------------------------------------------+
%%RefPage: end of main body, begin of footer
% EOF
% +------------------------------------------------------------------------+

