% +------------------------------------------------------------------------+
% | Reference manual page: SurfaceMeshVertexBase_3.tex
% +------------------------------------------------------------------------+
% | 09.12.2005   Author
% | Package: Package
% | 
\RCSdef{\RCSSurfaceMeshVertexBaseRev}{$Id$}
\RCSdefDate{\RCSSurfaceMeshVertexBaseDate}{$Date$}
% |
%%RefPage: end of header, begin of main body
% +------------------------------------------------------------------------+


\begin{ccRefConcept}{SurfaceMeshVertexBase_3}

%% \ccHtmlCrossLink{}     %% add further rules for cross referencing links
%% \ccHtmlIndexC[concept]{} %% add further index entries

\ccDefinition
  
The concept \ccRefName\ describes the vertex base type
of the three dimensional triangulation used
to embed the surface mesh.

More precisely,
the first template parameter \ccc{C2T3} of the surface mesher
\ccc{make_surface_mesh}
is a model of the concept 
\ccc{SurfaceMeshComplex_2InTriangulation_3} 
which describes a data structure to store
a pure two dimensional complex 
embedded in a three dimensional triangulation.
In particular, the type \ccc{C2T3} is required to provide
a three dimensional triangulation type
\ccc{C2T3::Triangulation_3}
The concept \ccRefName\ describes the vertex base type
required in this triangulation type.





\ccGeneralizes

\ccc{TriangulationVertexBase_3}


To allow a caching mechanism, 
the concept \ccRefName\ 
add functions to store and retrieve three boolean marks.
The mark \ccc{in_complex_mark} marks if the vertex is in the complex,
the mark \ccc{regular_or_boundary_mark} marks if 
status of the vertex is either \ccc{REGULAR} or \ccc{BOUNDARY}
and the third one, \ccc{status_mark}, marks if the two other
marks are valid, i. e. conform to the actual status of the vertex with
respect to the complex.



%\ccTypes

%\ccNestedType{TYPE}{some nested types}

\ccCreation
\ccCreationVariable{vb}  %% choose variable name


\ccOperations

\ccMethod{bool is_regular_or_boundary_mark_true();}
{Returns \ccc{true} if the \ccc{regular_or_boundary_mark} of the
vertex  is set to \ccc{true}, \ccc{false} otherwise.} 
\ccGlue
\ccMethod{void set_regular_or_boundary_mark(bool b);}
{ Sets the \ccc{regular_or_boundary_mark} of the vertex  according to \ccc{b}.}

\ccMethod{bool is_in_complex_mark_true();}
{Returns \ccc{true} if the \ccc{in_complex_mark} is set to \ccc{true},
false otherwise.}
\ccGlue
\ccMethod{void  set_in_complex_mark(bool b);}
{Sets the \ccc{set_in_complex_mark} as \ccc{b}.}

\ccMethod{bool is_status_mark_valid();}{returns \ccc{true} if the
status marks are valid, i. e. coherent to the actual status of the vertex.}
\ccGlue
\ccMethod{void set_status_cached(bool b);}{sets the validity of
status marks  as b.}


\ccHasModels

\ccc{Surface_mesh_vertex_base_3<Gt,Vb>}


\ccSeeAlso

\ccc{SurfaceMesherComplex_2InTriangulation_3} \\
\ccc{Surface_mesh_complex_2_in_triangulation_3<Tr>}.

%\ccExample

%A short example program.
%Instead of a short program fragment, a full running program can be
%included using the 
%\verb|\ccIncludeExampleCode{Package/SurfaceMeshVertexBase_3.C}| 
%macro. The program example would be part of the source code distribution and
%also part of the automatic test suite.

%\begin{ccExampleCode}
%void your_example_code() {
%}
%\end{ccExampleCode}

%%% \ccIncludeExampleCode{Package/SurfaceMeshVertexBase_3.C}

\end{ccRefConcept}

% +------------------------------------------------------------------------+
%%RefPage: end of main body, begin of footer
% EOF
% +------------------------------------------------------------------------+

