% +------------------------------------------------------------------------+
% | Reference manual page: arr_locate.tex
% +------------------------------------------------------------------------+
% | 
% | Package: Arrangement_2
% | 
% +------------------------------------------------------------------------+

\ccRefPageBegin

\begin{ccRefFunction}{locate}

\ccInclude{CGAL/Arr_batched_point_location.h}

\ccFunction{template<class Arrangement, 
                     class PointsIterator, class OutputIterator>
            OutputIterator locate (const Arrangement& arr,
                                   PointsIterator points_begin,
                                   PointsIterator points_end,
                                   OutputIterator oi);}
   {Performs a batched point-location operation on an arrangement.
    The function accepts a range of query points, defined by
    \ccc{[points_begin, points_end)} and locates each point in the
    arrangement. The query-results are returned in through the output iterator,
    whose value type is \ccc{std::pair<Point_2,Object>}. Namely, each
    result is given as a point and an object represneting the arrangement
    feature that contains it (an \ccc{Object} that may be either
    \ccc{Face_const_handle}, \ccc{Halfedge_const_handle} or
    \ccc{Vertex_const_hanlde}). The result pair in output sequence are sorted
    by an increasing $xy$-lexicographical order on the query points.
    The function returns a past-the-end iterator for the output sequence.
    \ccPrecond{The value-type of \ccc{PointsIterator} is 
               \ccc{Traits::Point_2}.}
    \ccPrecond{The value-type of \ccc{OutputIterator} is 
               \ccc{std::pair<Traits::Point_2,Object>}.}}

\end{ccRefFunction}

\ccRefPageEnd
