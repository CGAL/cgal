\ccRefPageBegin

\begin{ccRefConcept}{ArrangementMergingXMonotoneTraits_2}

\ccThree{ArrangementMergingXMonotoneTraits::Has_merge_category}{}{}
\ccThreeToTwo

\ccDefinition

This concept refines the \ccc{ArrangementXMonotoneTraits_2} concept.
A model of this concept is able to merge $x$-monotone curves. This
is necessary, for example, by the \ccc{remove_vertex} free function
that removed a given vertex from a given arrangement.

\ccRefines
\ccc{ArrangementXMonotoneTraits_2}

\ccHeading{Tags}
%===============

\ccNestedType{Has_merge_category}{indicates whether the nested functors
                                  \ccc{Are_mergeable_2} and \ccc{Merge_2}
                                  are provided.}


\ccHeading{Functor Types}
%========================

\ccThree{Compare_y_at_x_2}{}{\hspace*{14cm}}
\ccThreeToTwo

\ccNestedType{Are_mergeable_2}
{provides the operator~:
 \begin{itemize}
 \item \ccc{bool operator() (X_monotone_curve_2 c1, X_monotone_curve_2 c2)} \\
 which accepts two $x$-monotone curves \ccc{c1} and \ccc{c2} that share
 a common endpoint, and determines whether they can be merged to form a single
 continuous $x$-monotone curve.
 \end{itemize}
 This functor is required only if the traits class defines the
 \ccc{Has_merge_category} tag as \ccc{Tag_true}).}

\ccNestedType{Merge_2}
{provides the operator~:
 \begin{itemize}
 \item \ccc{void operator() (X_monotone_curve_2 c1, X_monotone_curve_2 c2,
                             X_monotone_curve_2& c)} \\
 which accepts two {\sl mergeable} $x$-monotone curves \ccc{c1} and \ccc{c2}
 (see above), and sets \ccc{c} to be the merged curve.
 \end{itemize}
 This functor is required only if the traits class defines the
 \ccc{Has_merge_category} tag as \ccc{Tag_true}).} 

\ccCreation
\ccCreationVariable{traits}
%==========================

\ccThree{Construct_x_monotone_curve_2~~~}{}{\hspace*{7cm}}
\ccThreeToTwo

\ccConstructor{ArrangementMergingXMonotoneTraits_2();}{default constructor.}
\ccGlue
\ccConstructor{ArrangementMergingXMonotoneTraits_2(ArrangementMergingXMonotoneTraits_2 other);}
{copy constructor}
\ccGlue
\ccMethod{ArrangementMergingXMonotoneTraits_2  operator=(other);}{assignment operator.}

\ccHeading{Accessing Functor Objects}
%====================================

\ccMethod{Are_mergeable_2 are_mergeable_2_object() const;} {}
\ccGlue
\ccMethod{Merge_2 merge_2_object();} {}

\ccHasModels
%===========

\ccc{CGAL::Arr_segment_traits_2<Kernel>} \\
\ccc{CGAL::Arr_non_caching_segment_traits_2<Kernel>} \\
\ccc{CGAL::Arr_polyline_traits_2<SegmentTraits>} \\
\ccc{CGAL::Arr_circle_segment_traits_2<Kernel>} \\
\ccc{CGAL::Arr_conic_traits_2<RatKernel,AlgKernel,NtTraits>} \\
\ccc{CGAL::Arr_rational_arc_traits_2<AlgKernel,NtTraits>} \\
\ccc{CGAL::Arr_curve_data_traits_2<Tr,XData,Mrg,CData,Cnv>}\\
\ccc{CGAL::Arr_consolidated_curve_data_traits_2<Traits,Data>}

\ccSeeAlso
%=========

\ccc{ArrangementXMonotoneTraits_2}\lcTex{
  (\ccRefPage{ArrangementXMonotoneTraits_2})}

\end{ccRefConcept}

\ccRefPageEnd
