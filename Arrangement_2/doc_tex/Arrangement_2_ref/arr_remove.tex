\ccRefPageBegin

\begin{ccRefFunction}{remove_edge}

\ccDefinition

Removes an edge given by one of the twin halfedges \ccc{e} that forms it,
from the arrangement \ccc{arr}. Using this function is equivalent to
invoking \ccc{arr.remove_edge (e, true, true)} --- that is, to remove the
edge and its end-vertices, in case they become isolated. However, this
free function requires that \ccc{Traits} be a model of the refined concept
\ccc{ArrangementXMonotoneTraits_2}, which requires merge operations
on $x$-monotone curves. If one of the end-vertices of \ccc{e} becomes
redundant after \ccc{e} is removed (see \ccc{remove_vertex()} for the
definition of a redundant vertex), it is removed and its
incident edges are merged.
If the edge-removal operation causes two faces to merge, the merged face
is returned. Otherwise, the face to which the edge was incident is
returned.

\ccInclude{CGAL/Arrangement_2.h}

\ccGlobalFunction{template<typename Traits, typename Dcel>
        typename Arrangement_2<Traits,Dcel>::Face_handle
        remove_edge (Arrangement_2<Traits,Dcel>& arr,
                     typename Arrangement_2<Traits,Dcel>::Halfedge_handle e);}

\end{ccRefFunction}

\ccRefPageEnd
