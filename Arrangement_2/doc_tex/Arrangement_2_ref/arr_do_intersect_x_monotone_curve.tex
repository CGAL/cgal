\ccRefPageBegin

\begin{ccRefFunction}{do_intersect_x_monotone_curve}

\ccDefinition

The function \ccRefName\ checks if a given $x$-monotone curve
intersects an existing arrangement's edges or vertices.
The function uses the zone algorithm to check if the curve intersects
the arrangement. First, the curve's left endpoint is located. Then, 
its zone is computed starting from its left endpoint location. The
zone computation terminates when an intersection with an arrangement's
edge/vertex is found or when the right endpoint is reached. 
A given point-location object is used for locating the left endpoint 
of the given curve in the existing arrangement. By default, the function 
uses the ``walk along line'' point-location strategy --- namely an 
instance of the class 
\ccc{Arr_walk_along_line_point_location<Arrangement_2<Traits,Dcel> >}.

\ccInclude{CGAL/Arrangement_2.h}

\ccGlobalFunction{template <class Traits, class Dcel, class PointLocation>
  bool do_intersect_x_monotone_curve (Arrangement_2<Traits,Dcel>& arr, 
  const typename Traits::X_monotone_curve_2& c,
  const PointLocation& pl);}
Checks if the given $x$-monotone curve \ccc{xc} intersects edges or
vertices of the existing arrangement \ccc{arr}. It uses the
point-location object \ccc{pl} to locate the left endpoint of
\ccc{xc} in \ccc{arr}.
\ccPrecond{If provided, \ccc{pl} must be attached to the given arrangement
\ccc{arr}.}

%%%%

\ccRequirements
\begin{itemize}
\item The instantiated \ccc{Traits} class must model the
  \ccc{ArrangementXMonotoneTraits_2} concept.
\item The point-location object \ccc{pl}, must model the
  \ccc{ArrangementPointLocation_2} concept.
\end{itemize}
			 
\end{ccRefFunction}

\ccRefPageEnd
