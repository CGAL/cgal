% +------------------------------------------------------------------------+
% | Reference manual page: Is_vacuously_valid.tex
% +------------------------------------------------------------------------+
% | 26.07.2000   Susan Hert
% | Package: Partition_2
% | 
% +------------------------------------------------------------------------+


\begin{ccRefFunctionObjectClass}{Is_vacuously_valid<Traits>}  


\ccDefinition
  
Function object class that indicates all sequences of points are valid.
\ccIndexSubitem{polygon partitioning}{valid}
\ccIndexSubitem{polygon}{valid}

\ccInclude{CGAL/polygon_function_objects.h}

\ccIsModel

\ccRefConceptPage{PolygonIsValid}%
\ccIndexSubitem[c]{PolygonIsValid}{model}

\ccCreation
\ccCreationVariable{f}  %% choose variable name

\ccConstructor{Is_vacuously_valid(const Traits& t);}{}
\ccThree{boolxxx}{operator()(InputIterator first, InputIterator beyond)xxx}{}
\ccOperations

\ccMethod{
template<class InputIterator>
bool operator()(InputIterator first, InputIterator beyond);}
{
  returns \ccc{true}.
}

\ccSeeAlso

\ccRefIdfierPage{CGAL::partition_is_valid_2} \\
\ccRefIdfierPage{CGAL::Partition_is_valid_traits_2<Traits, PolygonIsValid>}

\ccImplementation

This test requires $O(1)$ time.

\end{ccRefFunctionObjectClass}

% +------------------------------------------------------------------------+
%%RefPage: end of main body, begin of footer
% EOF
% +------------------------------------------------------------------------+

