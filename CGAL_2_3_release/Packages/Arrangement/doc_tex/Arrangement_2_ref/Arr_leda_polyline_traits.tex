% +------------------------------------------------------------------------+
% | Reference manual page: Arr_polyline_traits.tex (Arrangement)
% +------------------------------------------------------------------------+
% | 
% | Package: arr (Arrangement_2)
% | 
% +------------------------------------------------------------------------+

\ccRefPageBegin

%%RefPage: end of header, begin of main body
% +------------------------------------------------------------------------+
% +========================================================================+
%  Arr_polyline_traits<NT>
% +========================================================================+
\begin{ccRefClass}{Arr_leda_polyline_traits<Container>}

\ccDefinition
    The class \ccRefName\ is a traits class for polyline curves. Like
    in \ccc{Arr_leda_segment_exact_traits} the traits class uses
    \leda\/'s rational kernel to benefit its floating point ``filters'' and
    normalizes points to achieve better performance.

    The template parameter \ccc{Container} corresponds to the
    container type for holding the polyline curve points. It must be a
    model of the \ccc{Reversible Container} concept and it must be
    templated with \ccc{leda_rat_point}. \ccc{Container} has a default
    value which is \ccc{std::vector<leda_rat_point>}. The first point
    in the container is considered as the source point of the
    polyline while the last point is considered the target one.



%See example ***SHOW WHERE.
\ccInclude{CGAL/Arr_polyline_traits.h}


\ccIsModel
    \ccc{ArrangementTraits_2} \lcTex{(\ccRefPage{ArrangementTraits_2}).}

\ccSeeAlso
    \ccc{Arr_polyline_traits<R, Container>} 
    \lcTex{(\ccRefPage{CGAL::Arr_polyline_traits<R, Container>}).}\\
    \ccc{Arr_leda_segment_exact_traits}
    \lcTex{(\ccRefPage{CGAL::Arr_leda_segment_exact_traits}).}

\end{ccRefClass}


% EOF -----------------------------------------------------------------------80

% +------------------------------------------------------------------------+
%%RefPage: end of main body, begin of footer
\ccRefPageEnd
% EOF
% +------------------------------------------------------------------------+
