
\cleardoublepage
\chapter{Intersections}

An important relation between geometrical objects is the intersection
relation.
Two objects \ccStyle{obj1} and \ccStyle{obj2} intersect if there is a point
\ccStyle{p} that is part of both \ccStyle{obj1} and \ccStyle{obj2}.
The intersection region of those two objects is defined as the set of all
points \ccStyle{p} that are part of both \ccStyle{obj1} and \ccStyle{obj2}.

Note that for objects like triangles and polygons that enclose a
bounded region, this region is part of the object.
If a segment lies completely inside a triangle, then those two objects
intersect and the intersection region is the complete segment.

In the following sections we describe two families of functions.
One that checks whether two objects intersect; one that computes the
intersection region.

There are two ways to use those functions.
The simplest way is to include the header file
\ccStyle{CGAL/intersections.h}. Here all intersection routines are declared.

The drawback of this approach is that a lot is included, which results
in long compilation times.
It is also possible to include only the intersections that are of interest.
The naming scheme of the header files is as follows.
Intersections of types \ccStyle{CGAL_Type1<R>} and \ccStyle{CGAL_Type2<R>}
are declared in header file
\ccStyle{CGAL/Type1_Type2_intersection.h}.
So, intersection routines of segments and lines in 2D are declared in
\ccStyle{CGAL/Segment_2_Line_2_intersection.h}
The order of the type names does not matter.
It is also possible to include
\ccStyle{CGAL/Line_2_Segment_2_intersection.h}

For intersections of two objects of the same type, the type name should be
mentioned twice:\\
\ccStyle{#include <CGAL/Segment_2_Segment_2_intersection.h>}

Every intersection header file declares both the checking routines and the
routines to compute the intersection region.


\section{Checking}

In order to check if two objects of type \ccStyle{Type1} and \ccStyle{Type2}
intersect, there are routines of the following form:

\ccUnchecked{
\ccGlobalFunctionTemplate{R}
{bool CGAL_do_intersect(Type1<R> obj1, Type2<R> obj2);}
}


The types \ccStyle{Type1} and \ccStyle{Type2} can be any of the
following:
\begin{itemize}
\item \ccStyle{CGAL_Point_2}
\item \ccStyle{CGAL_Line_2}
\item \ccStyle{CGAL_Ray_2}
\item \ccStyle{CGAL_Segment_2}
\item \ccStyle{CGAL_Triangle_2}
\item \ccStyle{CGAL_Iso_rectangle_2}
%\item \ccStyle{CGAL_Polygon_2}
\end{itemize}



\section{Computing the intersection region}


The intersection region of two geometrical objects of type \ccStyle{Type1}
and \ccStyle{Type2} can be computed with the following family of routines:

\ccUnchecked{
\ccGlobalFunctionTemplate{R}
{CGAL_Object CGAL_intersection(Type1<R> obj1, Type2<R> obj2);}
}

The return value is \ccStyle{CGAL_Object}. This is a special kind of object
that can contain an object of any type.
A previous section
%Section \ref{CGAL-Object}
describes the use of this class.

The types \ccStyle{Type1} and \ccStyle{Type2} can be any of the
following:
\begin{itemize}
\item \ccStyle{CGAL_Point_2}
\item \ccStyle{CGAL_Line_2}
\item \ccStyle{CGAL_Ray_2}
\item \ccStyle{CGAL_Segment_2}
\item \ccStyle{CGAL_Triangle_2}
\item \ccStyle{CGAL_Iso_rectangle_2}
\end{itemize}

The family of routines that compute the intersection region is not as complete
as the family of intersection checking routines.
Not all combinations of types are possible.
For instance, for more complicated types like polygons and discs those
routines are not available.
The result must be representable by a single geometric object.
The intersection of a line segment with a polygon can result in
several line segments. The intersection of a triangle and a disc can
result in an object bounded by curved and straight edges.  The
implementation of boolean operations in the basic library provides more
functionality for computing intersections of this kind.


\ccExample

The first example demonstrates the most common use of intersection routines.
\begin{verbatim}
#include <CGAL/Segment_2_Line_2_intersection.h>

template <class R>
void foo(CGAL_Segment_2<R> seg, CGAL_Line_2<R> line)
{
    CGAL_Object result;
    CGAL_Point_2<R> ipoint;
    CGAL_Segment_2<R> iseg;

    result = CGAL_intersection(seg, line);
    if (CGAL_assign(ipoint, result)) {
        // handle the point intersection case.
    } else
        if (CGAL_assign(iseg, result)) {
            // handle the segment intersection case.
        } else {
            // handle the no intersection case.
        }
}
\end{verbatim}




\subsection{Possible Result Values}
\label{all_intersection_results}

Here we describe the types that can be contained in the \ccStyle{CGAL_Object}
return value for every possible pair of geometric objects.

\begin{ccTexOnly}
\begin{center}
\begin{tabular}{|l|l|l|}
\hline
type A      & type B                       &    return type \\
\hline
\raisebox{-2.5mm}[0ex][0ex]{\ccStyle{CGAL_Line_2}} &
\raisebox{-2.5mm}[0ex][0ex]{\ccStyle{CGAL_Line_2}} &
\ccStyle{CGAL_Point_2} \\
 &  & \ccStyle{CGAL_Line_2} \\
\hline
\raisebox{-2.5mm}[0ex][0ex]{\ccStyle{CGAL_Segment_2}} &
\raisebox{-2.5mm}[0ex][0ex]{\ccStyle{CGAL_Line_2}} &
\ccStyle{CGAL_Point_2} \\
 &  & \ccStyle{CGAL_Segment_2} \\
\hline
\raisebox{-2.5mm}[0ex][0ex]{\ccStyle{CGAL_Segment_2}} &
\raisebox{-2.5mm}[0ex][0ex]{\ccStyle{CGAL_Segment_2}} &
\ccStyle{CGAL_Point_2} \\
 &  & \ccStyle{CGAL_Segment_2} \\
\hline
\raisebox{-2.5mm}[0ex][0ex]{\ccStyle{CGAL_Ray_2}} &
\raisebox{-2.5mm}[0ex][0ex]{\ccStyle{CGAL_Line_2}} &
\ccStyle{CGAL_Point_2} \\
 & & \ccStyle{CGAL_Ray_2} \\
\hline
\raisebox{-2.5mm}[0ex][0ex]{\ccStyle{CGAL_Ray_2}} &
\raisebox{-2.5mm}[0ex][0ex]{\ccStyle{CGAL_Segment_2}} &
\ccStyle{CGAL_Point_2} \\
 &  & \ccStyle{CGAL_Segment_2} \\
\hline
 &  & \ccStyle{CGAL_Point_2} \\
\ccStyle{CGAL_Ray_2} & \ccStyle{CGAL_Ray_2} &
\ccStyle{CGAL_Segment_2} \\
 &  & \ccStyle{CGAL_Ray_2} \\
\hline
\raisebox{-2.5mm}[0ex][0ex]{\ccStyle{CGAL_Triangle_2}} &
\raisebox{-2.5mm}[0ex][0ex]{\ccStyle{CGAL_Line_2}} & \ccStyle{CGAL_Point_2} \\
 & & \ccStyle{CGAL_Segment_2} \\
\hline
\raisebox{-2.5mm}[0ex][0ex]{\ccStyle{CGAL_Triangle_2}} &
\raisebox{-2.5mm}[0ex][0ex]{\ccStyle{CGAL_Segment_2}} & \ccStyle{CGAL_Point_2} \\
 & & \ccStyle{CGAL_Segment_2} \\
\hline
\raisebox{-2.5mm}[0ex][0ex]{\ccStyle{CGAL_Triangle_2}} &
\raisebox{-2.5mm}[0ex][0ex]{\ccStyle{CGAL_Ray_2}} & \ccStyle{CGAL_Point_2} \\
 & & \ccStyle{CGAL_Segment_2} \\
\hline
 & & \ccStyle{CGAL_Point_2} \\
\raisebox{-2.5mm}[0ex][0ex]{\ccStyle{CGAL_Triangle_2}} &
\raisebox{-2.5mm}[0ex][0ex]{\ccStyle{CGAL_Triangle_2}} & \ccStyle{CGAL_Segment_2} \\
 & & \ccStyle{CGAL_Triangle_2} \\
 & & \ccStyle{vector<CGAL_Point_2>} \\
\hline
 & & \ccStyle{CGAL_Point_2} \\
\raisebox{2.5mm}[0ex][0ex]{\ccStyle{CGAL_Iso_rectangle_2}} &
\raisebox{2.5mm}[0ex][0ex]{\ccStyle{CGAL_Line_2}} & \ccStyle{CGAL_Segment_2} \\
\hline
 & & \ccStyle{CGAL_Point_2} \\
\raisebox{2.5mm}[0ex][0ex]{\ccStyle{CGAL_Iso_rectangle_2}} &
\raisebox{2.5mm}[0ex][0ex]{\ccStyle{CGAL_Segment_2}} & \ccStyle{CGAL_Segment_2} \\
\hline
 & & \ccStyle{CGAL_Point_2} \\
\raisebox{2.5mm}[0ex][0ex]{\ccStyle{CGAL_Iso_rectangle_2}} &
\raisebox{2.5mm}[0ex][0ex]{\ccStyle{CGAL_Ray_2}} & \ccStyle{CGAL_Segment_2} \\
\hline
\ccStyle{CGAL_Iso_rectangle_2} &
\ccStyle{CGAL_Iso_rectangle_2} & \ccStyle{CGAL_Iso_rectangle_2} \\
\hline
\end{tabular}
\end{center}
\end{ccTexOnly}

\begin{ccHtmlOnly}
<DIV ALIGN="CENTER">
<TABLE CELLPADDING=3 BORDER="1">
<TR> <TH> type A </TH>
 <TH> type B </TH>
 <TH> return type </TH>
</TR>
<TR>
    <TD VALIGN="CENTER" > CGAL_Line_2 </TD>
    <TD VALIGN="CENTER" > CGAL_Line_2 </TD>
    <TD><TABLE>
	<TR><TD>CGAL_Point_2</TD></TR>
	<TR><TD>CGAL_Line_2</TD></TR>
        </TABLE></TD>
</TR>
<TR>
    <TD VALIGN="CENTER" > CGAL_Segment_2 </TD>
    <TD VALIGN="CENTER" > CGAL_Line_2 </TD>
    <TD><TABLE>
	<TR><TD>CGAL_Point_2</TD></TR>
	<TR><TD>CGAL_Segment_2</TD></TR>
      </TABLE></TD>
</TR>
<TR>
    <TD VALIGN="CENTER" > CGAL_Segment_2 </TD>
    <TD VALIGN="CENTER" > CGAL_Segment_2 </TD>
    <TD><TABLE>
	<TR><TD>CGAL_Point_2</TD></TR>
	<TR><TD>CGAL_Segment_2</TD></TR>
      </TABLE></TD>
</TR>
<TR>
    <TD VALIGN="CENTER" > CGAL_Ray_2 </TD>
    <TD VALIGN="CENTER" > CGAL_Line_2 </TD>
    <TD><TABLE>
	<TR><TD>CGAL_Point_2</TD></TR>
	<TR><TD>CGAL_Ray_2</TD></TR>
      </TABLE></TD>
</TR>
<TR>
    <TD VALIGN="CENTER" > CGAL_Ray_2 </TD>
    <TD VALIGN="CENTER" > CGAL_Segment_2 </TD>
    <TD><TABLE>
	<TR><TD>CGAL_Point_2</TD></TR>
	<TR><TD>CGAL_Segment_2</TD></TR>
      </TABLE></TD>
</TR>
<TR>
    <TD VALIGN="CENTER" > CGAL_Ray_2 </TD>
    <TD VALIGN="CENTER" > CGAL_Ray_2 </TD>
    <TD><TABLE>
	<TR><TD>CGAL_Point_2</TD></TR>
	<TR><TD>CGAL_Segment_2</TD></TR>
	<TR><TD>CGAL_Ray_2</TD></TR>
      </TABLE></TD>
</TR>
<TR>
    <TD VALIGN="CENTER" > CGAL_Triangle_2 </TD>
    <TD VALIGN="CENTER" > CGAL_Line_2 </TD>
    <TD><TABLE>
	<TR><TD>CGAL_Point_2</TD></TR>
	<TR><TD>CGAL_Segment_2</TD></TR>
      </TABLE></TD>
</TR>
<TR>
    <TD VALIGN="CENTER" > CGAL_Triangle_2 </TD>
    <TD VALIGN="CENTER" > CGAL_Segment_2 </TD>
    <TD><TABLE>
	<TR><TD>CGAL_Point_2</TD></TR>
	<TR><TD>CGAL_Segment_2</TD></TR>
      </TABLE></TD>
</TR>
<TR>
    <TD VALIGN="CENTER" > CGAL_Triangle_2 </TD>
    <TD VALIGN="CENTER" > CGAL_Ray_2 </TD>
    <TD><TABLE>
	<TR><TD>CGAL_Point_2</TD></TR>
	<TR><TD>CGAL_Segment_2</TD></TR>
      </TABLE></TD>
</TR>
<TR>
    <TD VALIGN="CENTER" > CGAL_Triangle_2 </TD>
    <TD VALIGN="CENTER" > CGAL_Triangle_2 </TD>
    <TD><TABLE>
	<TR><TD>CGAL_Point_2</TD></TR>
	<TR><TD>CGAL_Segment_2</TD></TR>
	<TR><TD>CGAL_Triangle_2</TD></TR>
	<TR><TD>vector&lt;CGAL_Point_2&gt;</TD></TR>
      </TABLE></TD>
</TR>
<TR>
    <TD VALIGN="CENTER" > CGAL_Iso_rectangle_2 </TD>
    <TD VALIGN="CENTER" > CGAL_Line_2 </TD>
    <TD><TABLE>
	<TR><TD>CGAL_Point_2</TD></TR>
	<TR><TD>CGAL_Segment_2</TD></TR>
      </TABLE></TD>
</TR>
<TR>
    <TD VALIGN="CENTER" > CGAL_Iso_rectangle_2 </TD>
    <TD VALIGN="CENTER" > CGAL_Segment_2 </TD>
    <TD><TABLE>
	<TR><TD>CGAL_Point_2</TD></TR>
	<TR><TD>CGAL_Segment_2</TD></TR>
      </TABLE></TD>
</TR>
<TR>
    <TD VALIGN="CENTER" > CGAL_Iso_rectangle_2 </TD>
    <TD VALIGN="CENTER" > CGAL_Ray_2 </TD>
    <TD><TABLE>
	<TR><TD>CGAL_Point_2</TD></TR>
	<TR><TD>CGAL_Segment_2</TD></TR>
      </TABLE></TD>
</TR>
<TR>
    <TD VALIGN="CENTER" > CGAL_Iso_rectangle_2 </TD>
    <TD VALIGN="CENTER" > CGAL_Iso_rectangle_2 </TD>
    <TD><TABLE>
	<TR><TD>CGAL_Iso_rectangle_2</TD></TR>
      </TABLE></TD>
</TR>
</TABLE>
</DIV>
\end{ccHtmlOnly}


