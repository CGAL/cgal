\RCSdef{\IntervalskiplistRev}{$Id$}
\RCSdefDate{\IntervalskiplistDate}{$Date$}
% +------------------------------------------------------------------------+

\ccParDims

\ccUserChapter{Interval Skip List\label{chapter_Interval_skip_list}}

\begin{ccPkgDescription}{3D Triangulations}
\ccPkgSummary{
This package  allows to build and handle
triangulations for point sets in three dimensions.
Any CGAL  triangulation covers the convex hull of its
vertices. Triangulations are build incrementally 
and can be modified by insertion or removal of vertices. 
They offer point location facilities.

The package provides plain triangulation (whose faces
depends on the  insertion order of the vertices) and
Delaunay triangulations.  Regular triangulations are
also provided for sets of weighted points.
Delaunay and regular
triangulations offer nearest neighbor queries
and primitives to build the dual Voronoi and power diagrams.}

%\ccPkgDependsOn{}
\ccPkgMaturity{Introduced in \cgal\ 3.1}

\end{ccPkgDescription}

\ccChapterRelease{\IntervalskiplistRev. \ \IntervalskiplistDate}
\ccChapterAuthor{Andreas Fabri}

\minitoc



% +========================================================================+
\section{Definition}
% +========================================================================+
  
An interval skip list is a data structure for finding all intervals 
that contain a point, and for stabbing queries, that is for answering 
the question whether a given point is contained in an interval or not. 
The implementation we provide is dynamic, that is the user can freely
mix calls  to the methods \ccc{insert(..)}, \ccc{remove(..)}, 
\ccc{find_intervals(..)}, and \ccc{is_contained(..)}

The interval skip list class is parameterized with an interval class.

The data structure was introduced by Hanson~\cite{h-islds-91}, and it is called
interval skip list, because it is an extension of the randomized list
structure known as skip list~\cite{p-slpab-90}.
 
% +========================================================================+
\section{Example Programs}
% +========================================================================+
\label{sectionIntervalskiplistExamples}

We give two examples. The first one uses a basic interval class.  In
the second example we create an interval skip list for the $z$-ranges
of the faces of a terrain, allowing to answer level queries.

% +-------------------------------------------------------------+
\subsection{First Example with Simple Interval}

The first example reads two numbers \ccc{n} and \ccc{d} from \ccc{std::cin}.
It creates \ccc{n} intervals of length \ccc{d} with the left endpoint at \ccc{n}.
It then reads out the intervals for the 1-dimensional points with
coordinates $0 ... n+d$. 

The interval skip list class has as template argument an interval
class. In this example we use the class \ccc{Interval_skip_list_interval}.

\newpage
\ccIncludeExampleCode{Interval_skip_list/intervals.cpp}

% +-------------------------------------------------------------+
\subsection{Example with Faces of a Triangulated Terrain}



The second example creates an interval skip list that allows to find all the faces
of a terrain intersected by an horizontal plane at a given height.
The data points for the terrain are  read from a file. 

As model for the interval concept, we use a class that is a wrapper
around a face handle of a triangulated terrain. Lower and upper bound
of the interval are smallest and largest $z$-coordinate of the face.

\ccIncludeExampleCode{Interval_skip_list/isl_terrain.cpp}


% +--------------------------------------------------------+

%% =============================================================================
% The CGAL Reference Manual
% Chapter: Geometric Optimisation
% -----------------------------------------------------------------------------
% file   : doc_tex/basic/Optimisation/Optimisation_ref/main.tex
% package: Optimisation_doc
% author : Sven Sch�nherr <sven@inf.ethz.ch>
% -----------------------------------------------------------------------------
% $Revision$
% $Date$
% =============================================================================

\section{Reference Pages}

% =============================================================================
% The CGAL Reference Manual
% Chapter: Geometric Optimisation
% -----------------------------------------------------------------------------
% file   : doc_tex/basic/Optimisation/Optimisation_ref/reference_part.tex
% package: Optimisation_doc
% author : Sven Sch�nherr <sven@inf.ethz.ch>
% -----------------------------------------------------------------------------
% $Revision$
% $Date$
% =============================================================================

\newcommand{\inputOpt}[1]{\input{Optimisation_ref/#1.tex}}

\newcommand{\linebreakByHand}{\ccTexHtml{\linebreak[4]}{}}
\newcommand{  \newlineByHand}{\ccTexHtml{\\}{}}

% cross references
\index{minimum enclosing|see{{smallest enclosing}}}
\index{minimum spanning|see{{smallest enclosing}}}
\index{concentric spheres|see{{annulus}}}

% -----------------------------------------------------------------------------
\section*{Introduction}

This chapter describes concepts, classes, and functions for solving
geometric optimisation problems. They are divided into four categories.

\paragraph{Bounding Areas and Volumes.}
Smallest enclosing circle and ellipse (2D), smallest enclosing rectangle,
parallelogram, and strip (2D), rectangular $p$-center (2D), smallest
enclosing sphere and annulus (dD).

\paragraph{Inscribed Areas.}
Maximum area and perimeter inscribed $k$-gon (2D), extremal inscribed
$k$-gon (2D).

\paragraph{Optimal Distances.}
All furthest neigbors (2D), width of point set (3D), polytope distance (dD).

\paragraph{Advanced Techniques.}
Monotone and sorted matrix search.

\section*{Assertions}
The optimisation code uses infix \ccc{OPTIMISATION} in the assertions,
e.g.\ defining the compiler flag
\ccc{CGAL_OPTIMISATION_NO_PRECONDITIONS} switches precondition
checking off, cf.~\cgalReferToAssertions


% -----------------------------------------------------------------------------
\subsection*{Bounding Areas and Volumes}

\ccRefIdfierPage{CGAL::Min_circle_2<Traits>}\\[1ex]
\ccRefIdfierPage{CGAL::Min_circle_2_traits_2<K>}\\[1ex]
\ccRefConceptPage{MinCircle2Traits}

\smallskip

\ccRefIdfierPage{CGAL::Min_ellipse_2<Traits>}\\[1ex]
\ccRefIdfierPage{CGAL::Min_ellipse_2_traits_2<K>}\\[1ex]
\ccRefConceptPage{MinEllipse2Traits}

\smallskip

\ccRefIdfierPage{CGAL::min_rectangle_2}\\
\ccRefIdfierPage{CGAL::min_parallelogram_2}\\
\ccRefIdfierPage{CGAL::min_strip_2}\\[1ex]
\ccRefIdfierPage{CGAL::Min_quadrilateral_default_traits_2<R>}\\[1ex]
\ccRefConceptPage{MinQuadrilateralTraits_2}

\smallskip

\ccRefIdfierPage{CGAL::rectangular_p_center_2}\\[1ex]
\ccRefIdfierPage{CGAL::Rectangular_p_center_default_traits_2<R>}\\[1ex]
\ccRefConceptPage{RectangularPCenterTraits_2}

\bigskip

\ccRefIdfierPage{CGAL::Min_sphere_d<Traits>}\\
\ccRefIdfierPage{CGAL::Min_annulus_d<Traits>}\\[1ex]
\ccRefIdfierPage{CGAL::Optimisation_d_traits_2<K,ET,NT>}\\
\ccRefIdfierPage{CGAL::Optimisation_d_traits_3<K,ET,NT>}\\
\ccRefIdfierPage{CGAL::Optimisation_d_traits_d<K,ET,NT>}\\[1ex]
\ccRefConceptPage{OptimisationDTraits}

% -----------------------------------------------------------------------------
\subsection*{Inscribed Areas}

\ccRefIdfierPage{CGAL::maximum_area_inscribed_k_gon_2}\\
\ccRefIdfierPage{CGAL::maximum_perimeter_inscribed_k_gon_2}\\
\ccRefIdfierPage{CGAL::extremal_polygon_2}\\[1ex]
\ccRefIdfierPage{CGAL::Extremal_polygon_area_traits_2<K>}\\
\ccRefIdfierPage{CGAL::Extremal_polygon_perimeter_traits_2<K>}\\[1ex]
\ccRefConceptPage{ExtremalPolygonTraits_2}

% -----------------------------------------------------------------------------
\subsection*{Optimal Distances}

%\ccRefIdfierPage{CGAL::width_2}%\\[1ex]
%\ccRefIdfierPage{CGAL::Min_quadrilateral_default_traits_2<K>}\\[1ex]
%\ccRefConceptPage{MinQuadrilateralTraits_2}

%\smallskip

\ccRefIdfierPage{CGAL::all_furthest_neighbors_2}\\[1ex]
%\ccRefIdfierPage{CGAL::All_furthest_neighbors_default_traits_2<R>}\\[1ex]
\ccRefConceptPage{AllFurthestNeighborsTraits_2}

\smallskip

\ccRefIdfierPage{CGAL::Width_3<Traits>}\\[1ex]
\ccRefIdfierPage{CGAL::Width_default_traits_3<K>}\\[1ex]
\ccRefConceptPage{WidthTraits_3}

\smallskip

\ccRefIdfierPage{CGAL::Polytope_distance_d<Traits>}\\[1ex]
\ccRefIdfierPage{CGAL::Optimisation_d_traits_2<K,ET,NT>}\\
\ccRefIdfierPage{CGAL::Optimisation_d_traits_3<K,ET,NT>}\\
\ccRefIdfierPage{CGAL::Optimisation_d_traits_d<K,ET,NT>}\\[1ex]
\ccRefConceptPage{OptimisationDTraits}

% -----------------------------------------------------------------------------
\subsection*{Advanced Techniques}

\ccRefIdfierPage{CGAL::monotone_matrix_search}\\[1ex]
\ccRefIdfierPage{CGAL::Dynamic_matrix<M>}\\[1ex]
\ccRefConceptPage{MonotoneMatrixSearchTraits}\\
\ccRefConceptPage{BasicMatrix}

\smallskip

\ccRefIdfierPage{CGAL::sorted_matrix_search}\\[1ex]
\ccRefIdfierPage{CGAL::Sorted_matrix_search_traits_adaptor<F,M>}\\[1ex]
\ccRefConceptPage{SortedMatrixSearchTraits}

\smallskip

% =============================================================================

% Bounding Areas and Volumes

\inputOpt{main_Min_circle_2}
\inputOpt{main_Min_ellipse_2}
\inputOpt{main_Min_quadrilateral_2}
\inputOpt{main_Rectangular_p_centers}

\inputOpt{main_Min_sphere_d}
\inputOpt{main_Min_annulus_d}
\inputOpt{main_Optimisation_d_traits}

% Inscribed Areas

\inputOpt{main_Extremal_polygons}

% Optimal Distances

\inputOpt{main_All_furthest_neighbors}

\inputOpt{main_Width_3}

\inputOpt{main_Polytope_distance_d}


% Advanced Techniques

\inputOpt{main_Matrix_search}


% ===== EOF ===================================================================


% ===== EOF ===================================================================


% EOF


