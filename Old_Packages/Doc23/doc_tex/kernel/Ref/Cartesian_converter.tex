\begin{ccRefClass}{Cartesian_converter<K1, K2,
                             Converter = NT_converter<K1::FT, K2::FT> >}
%\ccTexHtml{\ccSetThreeColumns{Point_2< us<RT> > }{}{\hspace*{8.5cm}}}{}

\KernelRefLayout\gdef\ccTagOperatorLayout{\ccFalse}

\ccDefinition

\ccClassTemplateName converts objects from the kernel traits \ccc{K1} to
the kernel traits \ccc{K2}.  Those traits must be of the form
\ccc{Cartesian<FT1>} and \ccc{Cartesian<FT2>} (or the equivalent with
\ccc{Simple_cartesian}).  It then provides the following operators to convert
objects from \ccc{K1} to \ccc{K2}.

\ccInclude{CGAL/Cartesian_converter.h}

\ccTypes

The third template parameter \ccc{Converter} is a function object that must
satisfy:

\ccMemberFunction{K2::FT operator()(const K1::FT &n);}
{ converts \ccc{n} to an \ccc{K2::FT} which has the same value.}

The default value of this parameter uses the conversion operator from
\ccc{K1::FT} to \ccc{K2::FT}.

\ccCreation
\ccCreationVariable{conv}

\ccConstructor{Cartesian_converter<K1, K2, Converter>();}{Default constructor.}

\ccOperations

\ccMemberFunction{K2::Point_2 operator()(const K1::Point_2&p);}
{ returns a \ccc{K2::Point_2} which coordinates are those of \ccc{p},
converted by \ccc{Converter}.}

Similar operators are defined for the other kernel traits types \ccc{Point_3},
\ccc{Vector_2}...


%\ccTexHtml{\KernelRefLayout}{}
\end{ccRefClass}

