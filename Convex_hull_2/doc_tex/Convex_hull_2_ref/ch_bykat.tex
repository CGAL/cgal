% +------------------------------------------------------------------------+
% | Reference manual page: ch_bykat.tex
% +------------------------------------------------------------------------+
% | 09.05.2001   Susan Hert and Stefan Schirra
% | Package: Convex_hull_2
% | 
% +------------------------------------------------------------------------+

%\renewcommand{\ccRefPageBegin}{\begin{ccAdvanced}}
%\renewcommand{\ccRefPageEnd}{\end{ccAdvanced}}
\begin{ccRefFunction}{ch_bykat}  
\ccIndexSubitemBegin{convex hull, 2D}{Bykat algorithm}

\ccDefinition
  
The function \ccRefName\ generates the counterclockwise sequence of extreme
points from a given set of input points.


\ccInclude{CGAL/ch_bykat.h}

\ccFunction{template <class InputIterator, class OutputIterator, class Traits>
            OutputIterator
            ch_bykat( InputIterator  first,
                              InputIterator  beyond,
                              OutputIterator result,
                              const Traits & ch_traits = Default_traits);}
            {generates the counterclockwise sequence of extreme points
            of the points in the range [\ccc{first},\ccc{beyond}).
            The resulting sequence is placed starting at position
            \ccc{result}, and the past-the-end iterator for the resulting
            sequence is returned. It is not specified at which point the
            cyclic sequence of extreme points is cut into a linear sequence.
            \ccPrecond %\ccIndexSubitem[C]{ch_bykat}{preconditions}
            The source range [\ccc{first},\ccc{beyond}) does not contain
            \ccc{result}.}

The default traits class \ccc{Default_traits} is the kernel in which the
type \ccc{ForwardIterator::value_type} is defined.


\ccHeading{Requirements}
\begin{enumerate}
   \item    \ccc{InputIterator::value_type} and 
            \ccc{OutputIterator::value_type}
            are equivalent to \ccc{Traits::Point_2}.
   \item    \ccc{Traits} defines the following subset of types from
            the concept \ccc{ConvexHullTraits_2} and their corresponding member
            %\ccIndexMainItem[c]{ConvexHullTraits_2}
            functions that return instances of these types:
            \begin{itemize}
                \item \ccc{Traits::Point_2},
                \item \ccc{Traits::Less_signed_distance_to_line_2},
                \item \ccc{Traits::Left_turn_2}, 
                \item \ccc{Traits::Less_xy_2},
		\item \ccc{Traits::Equal_2}.
            \end{itemize}
\end{enumerate}

\ccSeeAlso

\ccRefIdfierPage{CGAL::ch_akl_toussaint} \\
\ccRefIdfierPage{CGAL::ch_eddy} \\
\ccRefIdfierPage{CGAL::ch_graham_andrew} \\
\ccRefIdfierPage{CGAL::ch_jarvis} \\
\ccRefIdfierPage{CGAL::ch_melkman} \\
\ccRefIdfierPage{CGAL::convex_hull_2} 

\ccImplementation
This function implements the non-recursive variation of
Eddy's algorithm \cite{e-nchap-77} described in \cite{b-chfsp-78}.
This algorithm requires $O(n h)$ time 
in the worst case for $n$ input points with $h$ extreme points.  

\ccIndexSubitemEnd{convex hull, 2D}{Bykat algorithm}
\end{ccRefFunction}
%\renewcommand{\ccRefPageBegin}{}
%\renewcommand{\ccRefPageEnd}{}

% +------------------------------------------------------------------------+
%%RefPage: end of main body, begin of footer
% EOF
% +------------------------------------------------------------------------+

