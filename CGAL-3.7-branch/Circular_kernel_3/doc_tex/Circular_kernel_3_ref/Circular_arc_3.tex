\begin{ccRefClass}{Circular_arc_3<SphericalKernel>}

\ccInclude{CGAL/Circular_arc_3.h}

\ccIsModel

\ccc{SphericalKernel::CircularArc_3}

\ccCreation
\ccCreationVariable{ca}

\ccThree{Circular_arc_point_3}{ca.is_x_monotone()}{}
\ccThreeToTwo

\ccConstructor{Circular_arc_3(const Circle_3<SphericalKernel> &c)}
{Constructs an arc from a full circle.}

\ccConstructor{Circular_arc_3(const Circle_3<SphericalKernel> &c, const Circular_arc_point_3& pt)}
{Constructs an arc from a full circle, using pt as source and target.}

\ccConstructor{Circular_arc_3(const Circle_3<SphericalKernel> &c,
		const Circular_arc_point_3<SphericalKernel> &p,
		const Circular_arc_point_3<SphericalKernel> &q)}
{Constructs the circular arc supported by \ccc{c}, whose source and target 
are \ccc{p} and \ccc{q}, respectively.
\ccPrecond{\ccc{p} and \ccc{q} lie on \ccc{c} and are different.}}

The circular arc constructed from a circle, a source, and a target, is
defined as the set of points of the circle that lie between the source \ccc{p1}
and the target \ccc{p2}, when traversing the circle counterclockwise
seen from the side of the plane of the circle pointed by its \textit{positive} normal 
vectors.

In this definition, we say that a normal vector (a,b,c) is \textit{positive} if 
$(a,b,c)>(0,0,0)$ (i.e. $(a>0) || (a==0) \&\& (b>0) || (a==0)\&\&(b==0)\&\&(c>0)$).

\ccConstructor{Circular_arc_3(const Point_3<SphericalKernel> &p,
                              const Point_3<SphericalKernel> &q,
                              const Point_3<SphericalKernel> &r)}
{Constructs an arc that is supported by the circle of type
  \ccc{Circle_3<SphericalKernel>} passing through the points \ccc{p},
  \ccc{q} and \ccc{r}. The source and target are respectively \ccc{p}
  and \ccc{r}, when traversing the supporting circle in the
  counterclockwise direction 
  seen from the side of the plane containing the circle pointed by its \textit{positive}
  normal vectors.
  Note that, depending on the orientation of the point triple
  \ccc{(p,q,r)}, \ccc{q} may not lie on the arc. 
\ccPrecond{\ccc{p}, \ccc{q}, and \ccc{r} are not collinear.}}

\ccAccessFunctions

\ccThree{SphericalKernel::Circular_arc_point_3}{ca.is_x_monotone()}{}
\ccThreeToTwo

\ccMethod{Circle_3<SphericalKernel> supporting_circle();}{}

\ccMethod{Point_3<SphericalKernel> const& center( ) const;}{
        returns the center of the supporting circle.}
\ccGlue
\ccMemberFunction{SphericalKernel::FT const& squared_radius( ) const;}{
        returns the squared radius of the supporting circle.}

\ccMethod{Plane_3<SphericalKernel> supporting_plane();}{}
\ccGlue
\ccMethod{Sphere_3<SphericalKernel> diametral_sphere();}{}

\ccMethod{Circular_arc_point_3<SphericalKernel> source();}{}
\ccGlue
\ccMethod{Circular_arc_point_3<SphericalKernel> target();}{}

When the methods \ccc{source} and \ccc{target} return the same point, then 
the arc is in fact a full circle. %\footnote{so far, arcs of zero length are
%points, not arcs. But see Menelaos' remark: add functors to test whether an
%arc is degenerate... to be fixed}

When the arc was constructed from its (full) underlying circle, then
source and target both return the smallest $x$-extremal point of the
circle if the circle is not contained in a plane $x=A$, and the smallest
$y$-extremal point otherwise.

\ccOperations

\ccFunction{bool operator==(const Circular_arc_3<SphericalKernel> &a1,
			const Circular_arc_3<SphericalKernel> &a2);}
{Test for equality. Two arcs are equal, iff their non-oriented
  supporting planes are equal, and the centers and squared
  radii of their respective supporting circles are equal, and their
  sources and targets are equal.} 

\ccFunction{bool operator!=(const Circular_arc_3<SphericalKernel> &a1,
			const Circular_arc_3<SphericalKernel> &a2);}
{Test for nonequality.} 

\ccHeading{I/O}

\ccFunction{istream& operator>> (std::istream& is, Circular_arc_3 & ca);}{}
\ccGlue
\ccFunction{ostream& operator<< (std::ostream& os, const Circular_arc_3 & ca);}{}

The input/output format of a circular arc consists of the supporting circle 
represented as a \ccc{Circle_3} object followed by the source and target 
points of the arc represented as two \ccc{Circular_arc_point_3} objects. 
The defined arc is the unique arc constructed from such three objects. 

\ccSeeAlso

\ccRefIdfierPage{CGAL::Circular_arc_point_3<SphericalKernel>}\\
\ccRefIdfierPage{CGAL::Line_arc_3<SphericalKernel>}

\end{ccRefClass}
