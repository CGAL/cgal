
\begin{ccRefFunctionObjectConcept}{SphericalKernel::ConstructPlane_3}
\ccCreationVariable{fo}

\ccRefines

\ccc{Kernel::ConstructPlane_3}

A model \ccVar\ of this concept must provide:

\ccMemberFunction{SphericalKernel::Plane_3 operator()
	(const SphericalKernel::Circular_arc_3 &a);}
{Constructs the plane containing the arc.}

\ccMemberFunction{SphericalKernel::Plane_3 operator()
	(const SphericalKernel::Polynomial_1_3 &p);}
{Constructs a plane from an equation.}

\ccSeeAlso

\ccRefIdfierPage{SphericalKernel::GetEquation}

\end{ccRefFunctionObjectConcept}
\begin{ccRefFunctionObjectConcept}{SphericalKernel::ConstructSphere_3}
\ccCreationVariable{fo}

\ccRefines

\ccc{Kernel::ConstructSphere_3}

A model \ccVar\ of this concept must provide:

\ccMemberFunction{SphericalKernel::Sphere_3
	operator()(const SphericalKernel::Circular_arc_3 & a);}
{Returns the diametral sphere of the supporting circle of arc \ccc{a}.}

\ccMemberFunction{SphericalKernel::Sphere_3 operator()
	(const SphericalKernel::Polynomial_2_3 &p);}
{Constructs a sphere from an equation.}

\ccSeeAlso

\ccRefIdfierPage{SphericalKernel::GetEquation}

\end{ccRefFunctionObjectConcept}
\begin{ccRefFunctionObjectConcept}{SphericalKernel::ConstructLine_3}
\ccCreationVariable{fo}

\ccRefines

\ccc{Kernel::ConstructLine_3}

A model \ccVar\ of this concept must provide:

\ccMemberFunction{SphericalKernel::Line_3 operator()
	(const SphericalKernel::Line_arc_3 &s);}
{Constructs the line containing the segment.}

\ccMemberFunction{SphericalKernel::Line_3 operator()
	(const SphericalKernel::Polynomials_for_lines_3 &p);}
{Constructs a line from an equation.}

\ccSeeAlso

\ccRefIdfierPage{SphericalKernel::GetEquation}

\end{ccRefFunctionObjectConcept}
\begin{ccRefFunctionObjectConcept}{SphericalKernel::ConstructCircle_3}
\ccCreationVariable{fo}

A model \ccVar\ of this concept must provide:

\ccMemberFunction{SphericalKernel::Circle_3 operator()
	(const SphericalKernel::Circular_arc_3 &a);}
{Constructs the circle containing the arc.}

\ccMemberFunction{SphericalKernel::Circle_3 operator()
	(const SphericalKernel::Polynomials_for_circles_3 &p);}
{Constructs a circle from an equation.}

\ccSeeAlso

\ccRefIdfierPage{SphericalKernel::GetEquation}\\


\end{ccRefFunctionObjectConcept}

\begin{ccRefFunctionObjectConcept}{SphericalKernel::ConstructCircularArcPoint_3}
\ccCreationVariable{fo}

A model \ccVar\ of this concept must provide:

\ccMemberFunction{SphericalKernel::Circular_arc_point_3 operator()
	(const SphericalKernel::Root_for_spheres_2_3 & r);}
{}

\ccMemberFunction{SphericalKernel::Circular_arc_point_3 operator()
	(const SphericalKernel::Point_3 & p);}
{}

\end{ccRefFunctionObjectConcept}

\begin{ccRefFunctionObjectConcept}{SphericalKernel::ConstructLineArc_3}
\ccCreationVariable{fo}

A model \ccVar\ of this concept must provide:

\ccMemberFunction{SphericalKernel::Line_arc_3 operator()
	(const SphericalKernel::Line_3 &l,
	const SphericalKernel::Circular_arc_point_3 &p,
	const SphericalKernel::Circular_arc_point_3 &q);}
{Constructs the line segment supported by \ccc{l}, whose source 
is \ccc{p} and whose target is \ccc{q}.
\ccPrecond{\ccc{p} and \ccc{q} lie on \ccc{l} and are different.}}

%\ccMemberFunction{SphericalKernel::Line_arc_3 operator()
%	(const SphericalKernel::Line_3 &l,
%	const SphericalKernel::Point_3 &p1,
%	const SphericalKernel::Point_3 &p2);}
%{Same.}

\ccMemberFunction{SphericalKernel::Line_arc_3 operator()
	(const SphericalKernel::Segment_3 &s);}
{}

\ccMemberFunction{SphericalKernel::Line_arc_3 operator()
	(const SphericalKernel::Point_3 &p,
	const SphericalKernel::Point_3 &q);}
{}

\end{ccRefFunctionObjectConcept}
\begin{ccRefFunctionObjectConcept}{SphericalKernel::ConstructCircularArc_3}
\ccCreationVariable{fo}

A model \ccVar\ of this concept must provide:

\ccMemberFunction{SphericalKernel::Circular_arc_3 operator()
	(const SphericalKernel::Circle_3 &c);}
{Constructs an arc from a full circle.}

\ccMemberFunction{SphericalKernel::Circular_arc_3 operator()
	(const SphericalKernel::Circle_3 &c,
	const SphericalKernel::Circular_arc_point_3 &p,
	const SphericalKernel::Circular_arc_point_3 &q);}
{Constructs the circular arc supported by \ccc{c}, whose source and target 
are \ccc{p} and \ccc{q}, respectively.
\ccPrecond{\ccc{p} and \ccc{q} lie on \ccc{c} and they are different.}}

The circular arc constructed from a circle, a source, and a target, is
defined as the set of points of the circle that lie between the source \ccc{p1}
and the target \ccc{p2}, when traversing the circle counterclockwise
seen from the side of the plane of the circle pointed by its \textit{positive} normal 
vectors.

In this definition, we say that a normal vector (a,b,c) is \textit{positive} if 
$(a,b,c)>(0,0,0)$ (i.e. $(a>0) || (a==0) \&\& (b>0) || (a==0)\&\&(b==0)\&\&(c>0)$).


\ccMemberFunction{SphericalKernel::Circular_arc_3 operator()
                 (const SphericalKernel::Point_3 &p,
                 const SphericalKernel::Point_3 &q,
                 const SphericalKernel::Point_3 &r);}
{Constructs an arc that is supported by the circle of type
  \ccc{SphericalKernel::Circle_3} passing through the points \ccc{p},
  \ccc{q} and \ccc{r}. The source and target are respectively \ccc{p}
  and \ccc{r}, when traversing the supporting circle in the
  counterclockwise direction   
  seen from the side of the plane containing the circle pointed by its \textit{positive}
  normal vectors.
  the circle. 
  Note that, depending on the orientation of the point triple
  \ccc{(p,q,r)}, \ccc{q} may not lie on the arc. 
\ccPrecond{\ccc{p}, \ccc{q}, and \ccc{r} are not collinear.}}

\end{ccRefFunctionObjectConcept}

\begin{ccRefFunctionObjectConcept}{SphericalKernel::ConstructCircularMinVertex_3} 

\ccCreationVariable{fo}

A model \ccVar\ of this concept must provide:

%\ccMemberFunction{SphericalKernel::Circular_arc_point_3 operator()
%	(const SphericalKernel::Circular_arc_3 & c);}
%{Constructs the $x$-minimal vertex of \ccc{c}.
%\ccPrecond{The arc \ccc{c} is $x$-monotone.}}

\ccMemberFunction{SphericalKernel::Circular_arc_point_3 operator()
	(const SphericalKernel::Line_arc_3 & l);}
{Constructs the minimal vertex of \ccc{l} with lexicographically 
smallest coordinates.}

\end{ccRefFunctionObjectConcept}
\begin{ccRefFunctionObjectConcept}{SphericalKernel::ConstructCircularMaxVertex_3} 

\ccCreationVariable{fo}

A model \ccVar\ of this concept must provide:

%\ccMemberFunction{SphericalKernel::Circular_arc_point_3 operator()
%	(const SphericalKernel::Circular_arc_3 & c);}
%{Constructs the $x$-maximal vertex of \ccc{c}.
%\ccPrecond{The arc \ccc{c} is $x$-monotone.}}

\ccMemberFunction{SphericalKernel::Circular_arc_point_3 operator()
	(const SphericalKernel::Line_arc_3 & l);}
{Constructs the maximal vertex of \ccc{l} with lexicographically 
largest coordinates.}

\end{ccRefFunctionObjectConcept}
\begin{ccRefFunctionObjectConcept}{SphericalKernel::ConstructCircularSourceVertex_3} 

\ccCreationVariable{fo}

A model \ccVar\ of this concept must provide:

\ccMemberFunction{SphericalKernel::Circular_arc_point_3 operator()
	(const SphericalKernel::Circular_arc_3 & a);}
{Constructs the source vertex of \ccc{a}.}

\ccMemberFunction{SphericalKernel::Circular_arc_point_3 operator()
	(const SphericalKernel::Line_arc_3 & l);}
{Same, for a line segment.}

\end{ccRefFunctionObjectConcept}
\begin{ccRefFunctionObjectConcept}{SphericalKernel::ConstructCircularTargetVertex_3} 

\ccCreationVariable{fo}

A model \ccVar\ of this concept must provide:

\ccMemberFunction{SphericalKernel::Circular_arc_point_3 operator()
	(const SphericalKernel::Circular_arc_3 & a);}
{Constructs the target vertex of \ccc{a}.}

\ccMemberFunction{SphericalKernel::Circular_arc_point_3 operator()
	(const SphericalKernel::Line_arc_3 & l);}
{Same, for a line segment.}

\end{ccRefFunctionObjectConcept}
\begin{ccRefFunctionObjectConcept}{SphericalKernel::ConstructBbox_3}

\ccCreationVariable{fo}

A model \ccVar\ of this concept must provide operators to construct 
a bounding box of geometric objects:

\ccMemberFunction{CGAL::Bbox_3 operator()
	(const SphericalKernel::Circular_arc_point_3 & p);}
{}

\ccMemberFunction{CGAL::Bbox_3 operator()
	(const SphericalKernel::Line_arc_3 & l);}
{}

\ccMemberFunction{CGAL::Bbox_3 operator()
	(const SphericalKernel::Circular_arc_3 & a);}
{}

\end{ccRefFunctionObjectConcept}

