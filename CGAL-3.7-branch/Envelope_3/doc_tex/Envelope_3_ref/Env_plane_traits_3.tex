% +------------------------------------------------------------------------+
% | Reference manual page: Env_triangle_traits.tex
% +------------------------------------------------------------------------+
% | 
% | Package: Envelope_3
% | 
% +------------------------------------------------------------------------+

\ccRefPageBegin

\begin{ccRefClass}{Env_plane_traits_3<Kernel>}
    
\ccDefinition 
%============

The traits class \ccRefName\ models the \ccc{EnvelopeTraits_3} concept,
and is used for the construction of lower and upper envelopes of planes
and half planes in the space. It is parameterized by a \cgal-kernel model,
which is parameterized in turn by a number type. The number type should 
support exact rational arithmetic, to avoid numerical errors and
robustness problems. In particular, the number type should support the 
arithmetic operations $+$, $-$, $\times$, and $\div$ without loss of 
precision. For optimal performance, we recommend instantiating the traits 
class with the predefined
\ccc{Exact_predicates_exact_constructions_kernel} provided by \cgal.
Using this kernel guarantees exactness and robustness, while it incurs
only a minor overhead (in comparison to working with a fast, inexact number
type) for most inputs.

Note that an entire plane has no boundaries, and the projection of a 
half-plane is an (unbounded) line. Naturally, rays and segments may occur as
a result of overlaying projections of several half planes. Indeed,
\ccRefName\ inherits from the \ccc{Arr_linear_traits_2<Kernel>} traits
class, and extends it by adding operations on planes and half planes. 
The nested \ccc{Xy_monotone_surface_3} and \ccc{Surface_3} types refer
to the same type. They are constructible from a \ccc{Kernel::Plane_3} 
in case of an entire plane, or from \ccc{Kernel::Plane_3} and 
\ccc{Kernel::Line_2} in case of a half-plane. The line orientation 
determines which half is considered.

\ccInclude{CGAL/Env_plane_traits_3.h}
 
\ccIsModel
    \ccc{EnvelopeTraits_3}

\ccInheritsFrom
    \ccc{Arr_linear_traits_2<Kernel>}

\end{ccRefClass}

\ccRefPageEnd
