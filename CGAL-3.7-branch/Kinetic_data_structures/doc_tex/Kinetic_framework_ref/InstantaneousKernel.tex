% +------------------------------------------------------------------------+
% | Reference manual page: Event.tex
% +------------------------------------------------------------------------+
% | 20.03.2005   Author
% | Package: Kinetic_data_structures
% | 
\RCSdef{\RCSEventRev}{$Id$}
\RCSdefDate{\RCSEventDate}{$Date$}
% |
%%RefPage: end of header, begin of main body
% +------------------------------------------------------------------------+


\begin{ccRefConcept}{Kinetic::InstantaneousKernel}

%% \ccHtmlCrossLink{}     %% add further rules for cross referencing links
%% \ccHtmlIndexC[concept]{} %% add further index entries

\ccDefinition
  
The concept \ccRefName\ covers models that act as adaptors allowing
CGAL static data structures to act on snapshots of kinetic
data. Different methods for evaluating predicates are used depending
on whether time is set using an \ccc{NT} or a \ccc{Time}
object. Evaluating predicates when time is the former is much cheaper.


\ccTypes 

\ccNestedType{NT}{A number type which can be used to represent the current time. This must be a ring or field type.}

\ccNestedType{Time}{The type used to represent the current time. This type must be comparable.}

\ccCreationVariable{a}  %% choose variable name

\ccOperations

\ccMethod{Time time();}{Return the current time.}

\ccMethod{NT time_as_nt();}{Return the current time as an \ccc{NT}. As a precondition, \ccc{time_is_nt} must be true.}

\ccMethod{bool time_is_nt();}{Return true if the last time time was set, it was using an object of type \ccc{NT}.}

\ccMethod{void set_time(Time);}{Set the current time to have a certain value. All existing predicates are updated automatically.}


\ccMethod{void set_time(NT);}{The the current time to be an instance of \ccc{NT}. With this more efficient techniques can be used. \ccc{time_is_nt()} must be true.}

\ccMethod{Static_object static_object(Key);}{Return a static object corresponding to the kinetic object at this instant in time. \ccc{time_is_nt()} must be true.}

\ccHasModels

\ccc{Kinetic::Default_instantaneous_kernel}





\end{ccRefConcept}

% +------------------------------------------------------------------------+
%%RefPage: end of main body, begin of footer
% EOF
% +------------------------------------------------------------------------+

