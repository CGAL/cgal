% +------------------------------------------------------------------------+
% | Reference manual page: Regular_trianglation_3.tex
% +------------------------------------------------------------------------+
% | 20.03.2005   Author
% | Package: Kinetic_data_structures
% | 
\RCSdef{\RCSRegulartrianglationRev}{$Id$}
\RCSdefDate{\RCSRegulartrianglationDate}{$Date$}
% |
%%RefPage: end of header, begin of main body
% +------------------------------------------------------------------------+


\begin{ccRefClass}{Kinetic::Regular_triangulation_3<Traits, Visitor, Triangulation>}  %% add template arg's if necessary

%% \ccHtmlCrossLink{}     %% add further rules for cross referencing links
%% \ccHtmlIndexC[class]{} %% add further index entries

\ccDefinition
  
The class \ccRefName\ maintains a triangulation of set of moving
weighted points. Its interface is the same as
\ccc{Kinetic::Delaunay_triangulation_3<Traits, Visitor, Triangulation>}.

Note that the regular triangulation tracks as points are added to the \ccc{Kinetic::ActiveObjectsTable}, but not removed from it.


The optional \ccc{Triangulation} template argument must be a model of
\ccc{CGAL::RegularTriangulation_3} which has
\ccc{Kinetic::Regular_triangulation_cell_base_3<Traits, Base>} as a
cell base and
\ccc{Kinetic::Regular_triangulation_vertex_base_3<Traits, Base>} as a
vertex base.

\ccInclude{CGAL/Kinetic/Regular_triangulation_3.h}


\ccSeeAlso
\ccc{Kinetic::Delaunay_triangulation_3<Traits, Visitor, Triangulation>}.
\ccc{Kinetic::RegularTriangulationVisitor_3}.

\ccExample

\ccIncludeExampleCode{Kinetic_data_structures/Kinetic_regular_triangulation_3.cpp}

\end{ccRefClass}

% +------------------------------------------------------------------------+
%%RefPage: end of main body, begin of footer
% EOF
% +------------------------------------------------------------------------+

