\section{Quick Hints \label{sec:quick_hints}}

This section gives quick answers to some questions people might have.
It presumes knowledge of kinetic data structures and this framework.

\subsubsection{How do I store extra information allow with, for example, a kinetic Point\_2?}

See the example
\ccReferToExampleCode{Kinetic_framework/defining_a_simulation_traits.cpp} to see how
to define a new \ccc{SimulationTraits} class where the
\ccc{Active\-Objects\-Table} contains extra data along with the point.

\subsubsection{Where is the best place to look if I want to write my own kinetic data structure?}
We provide two simple kinetic data structures, first most trivial is
\ccReferToExampleCode{Kinetic_framework/trivial_kds.cpp} and a slightly more
complicated one is:

\ccInclude{CGAL/Kinetic/Sort.h}

\subsubsection{How can I use kinetic data structures to update Delaunay triangulations?}
We are working on that one, but you will have to wait.



%%% Local Variables: 
%%% mode: latex
%%% TeX-master: t
%%% End: 
