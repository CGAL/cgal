\section{Definitions}
This section presents $d$-dimensional range and segment trees.
A one-dimensional range tree is a binary search tree on 
{\em{one-dimensional point data}}. 
Here we call all one-dimensional data types having a strict ordering
(like integer and double) {\em point data}. 
{\em{$d$-dimensional point data}} are $d$-tuples of one-dimensional 
point data.

A one-dimensional segment tree is a binary search tree as well, but with
{\em{one-dimensional interval data}} as input data.
One-dimensional interval data is a pair (i.e., 2-tuple) $(a,b)$, where $a$ 
and $b$ are one-dimensional point data of the same type and $a< b$. 
The pair $(a,b)$ represents a half open interval $[a,b)$.
Analogously, a $d$-dimensional interval  is represented by a $d$-tuple of
one-dimensional intervals.

The {\em{input data type}} for a $d$-dimensional tree is a container 
class consisting of a $d$-dimensional point data type, interval data type 
or a mixture of both, and optionally a {\em{value type}}, which 
can be used to store arbitrary data. 
E.g., the $d$--dimensional bounding box of a $d$--dimensional polygon 
may define the interval data of a $d$--dimensional segment tree and
the polygon itself can be stored as its value.   
An {\em{input data item}} is an instance of an input data type.

The range and segment tree classes are fully generic in the sense that they 
can be used to define {\em{multilayer trees}}. 
A multilayer tree of dimension (number of layers) $d$ is a simple tree in 
the $d$-th layer, whereas the $k$-th layer, $1\leq k\leq d-1$, of the tree 
defines a tree where each (inner) vertex contains a multilayer tree of 
dimension $d-k+1$.
The $k-1$-dimensional tree which is nested in the $k$-dimensional tree 
($T$) is called the {\em sublayer tree} (of $T$).
For example, a $d$-dim tree can be a range tree on the first layer, 
constructed with respect to the first dimension of $d$-dimensional data 
items.
On all the data items in each subtree, a $(d-1)$-dimensional tree is built,
either a range or a segment tree, with respect to the second dimension of 
the data items.
And so on.
%E.g., for each inner vertex of a range tree, a sublayer tree is created 
%according to all data items of the subtree of that vertex. 
%For each vertex of a segment tree an instance (tree)  is created according 
%to all data items of that vertex.
%E.g., one can define a segment tree for which each vertex contains a 
%range tree.
\lcTex{
Figures~\ref{User:fig:range.eps}, \ref{User:fig:d-range.eps} and
\ref{User:fig:d-segment.eps} illustrate the meaning of a sublayer tree
graphically.
}
\lcHtml{
The figures in Sections~\ref{sec:range_trees} and~\ref{sec:segment_trees}
illustrate the means of a sublayer tree graphically.
}

After creation of the tree, further insertions or deletions of data items 
are disallowed.  
The tree class does neither depend on the type of data nor on the concrete
physical representation of the data items.
E.g., let a multilayer tree be a segment tree for which each vertex
defines a range tree. 
We can choose the data items to consist of intervals of type \ccStyle{double} 
and the point data of type \ccStyle{integer}. 
As value type we can choose  \ccStyle{string}.

For this generality we have to
define what the tree of each dimension looks like and how the
input data is organized.
For dimension $k$, $1\le k\le 4$, \cgal\/ provides ready-to-use
range and segment trees that can store k-dimensional keys
(intervals resp.). 
Examples illustrating the use of these classes are given in
Sections~\ref{sec:range_tree_ex}
and~\ref{sec:segment_tree_ex}.
The description of the functionality of these classes as well as
the definition of higher dimensional trees and mixed multilayer
trees is given in the reference manual.

In the following two sections we give short definitions of the version of
the range tree and segment tree implemented here together with some
examples. The presentation closely follows~\cite{bkos-cgaa-97}.
