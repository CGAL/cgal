% +------------------------------------------------------------------------+
% | Reference manual page: hilbert_sort.tex
% +------------------------------------------------------------------------+
% | 05.06.2006   Christophe Delage
% | Package: Spatial_sorting
% | 
% |
%%RefPage: end of header, begin of main body
% +------------------------------------------------------------------------+


\begin{ccRefFunction}{hilbert_sort}  %% add template arg's if necessary

%% \ccHtmlCrossLink{}     %% add further rules for cross referencing links
%% \ccHtmlIndexC[function]{} %% add further index entries

\ccDefinition
  
The function \ccRefName\ sorts an iterator range of points
along a Hilbert curve.

\ccInclude{CGAL/hilbert_sort.h}

\ccGlobalFunction{template <class RandomAccessIterator, class Traits>
                  void
                  hilbert_sort( RandomAccessIterator begin,
                                RandomAccessIterator end,
                                const Traits& traits = Default_traits);}%
                 {sorts the range [\ccc{begin},\ccc{end}) in place.}
                 
The default traits class \ccc{Default_traits} is the kernel in which the type
\ccc{RandomAccessIterator::value_type} is defined.

\ccHeading{Requirements}
\begin{enumerate}
\item  \ccc{RandomAccessIterator::value_type} is convertible to
\ccc{Traits::Point_2} or \ccc{Traits::Point_3}.
\item    \ccc{Traits} is a model for concept \ccc{SpatialSortingTraits_2} or \ccc{SpatialSortingTraits_3}.
\end{enumerate}

\ccImplementation

Creates an instance of \ccc{Hilbert_sort_2<Traits>} or
\ccc{Hilbert_sort_3<Traits>} and calls its \ccc{operator()}.

%\ccc{Some_other_class},
%\ccc{some_other_function}.


\end{ccRefFunction}

% +------------------------------------------------------------------------+
%%RefPage: end of main body, begin of footer
% EOF
% +------------------------------------------------------------------------+

