% +------------------------------------------------------------------------+
% | Reference manual page: parameters_exude.tex
% +------------------------------------------------------------------------+
% | 20.10.2009   Stephane Tayeb
% | Package: Mesh_3
% |
\RCSdef{\RCSexudeRev}{$Id$}
\RCSdefDate{\RCSexudeDate}{$Date$}
% |
\ccRefPageBegin
%%RefPage: end of header, begin of main body
% +------------------------------------------------------------------------+


\begin{ccRefFunction}{parameters::exude}  %% add template arg's if necessary

%% \ccHtmlCrossLink{}     %% add further rules for cross referencing links
%% \ccHtmlIndexC[function]{} %% add further index entries

\ccDefinition
  
The function \ccRefName\ allows the user to trigger a call to \ccc{make_mesh_3} in
\ccc{make_mesh_3} and \ccc{refine_mesh_3} mesh generation functions.
It also allows the user to pass parameters
to the optimization function \ccc{exude_mesh_3} through these mesh generation functions.


\ccInclude{CGAL/refine_mesh_3.h}

\ccGlobalFunction{parameters::internal::Exude parameters::exude(
  double time_limit = 0, 
  double sliver_bound = 0);}

\ccParameters

The parameters are named parameters. They are the same (i.e. they have the same
name and the same default values) as the parameters of \ccc{exude_mesh_3}
function. See its manual page for further details.


\ccSeeAlso

\ccc{no_exude} \\
\ccc{exude_mesh_3} \\
\ccc{make_mesh_3} \\
\ccc{refine_mesh_3} 


\ccExample

\begin{ccExampleCode}
// Mesh generation with an exudation step
C3t3 c3t3 = make_mesh_3<c3t3>(domain, criteria, parameters::exude());
refine_mesh_3(c3t3, domain, criteria, parameters::exude(parameters::time_limit=10));
\end{ccExampleCode}


\end{ccRefFunction}

% +------------------------------------------------------------------------+
%%RefPage: end of main body, begin of footer
\ccRefPageEnd
% EOF
% +------------------------------------------------------------------------+

