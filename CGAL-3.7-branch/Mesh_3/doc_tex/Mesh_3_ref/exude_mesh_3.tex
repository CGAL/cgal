% +------------------------------------------------------------------------+
% | Reference manual page: exude_mesh_3.tex
% +------------------------------------------------------------------------+
% | 28.07.2009   Stephane Tayeb
% | Package: Mesh_3
% |
\RCSdef{\RCSexudemeshRev}{$Id$}
\RCSdefDate{\RCSexudemeshDate}{$Date$}
% |
\ccRefPageBegin
%%RefPage: end of header, begin of main body
% +------------------------------------------------------------------------+


\begin{ccRefFunction}{exude_mesh_3}  %% add template arg's if necessary

%% \ccHtmlCrossLink{}     %% add further rules for cross referencing links
%% \ccHtmlIndexC[function]{} %% add further index entries

\ccDefinition
  
The function \ccRefName\ performs a sliver exudation  process on a  Delaunay mesh.

The sliver exudation process consists in turning the Delaunay triangulation
into a weighted Delaunay triangulation and  optimizing the weights
of vertices in such a way that  slivers disappear and
the  quality of the mesh improves.

% The exuder tries to improve the dihedral angles of the mesh degree by degree: at the
% end of step \emph{n}, the worst tetrahedron of the mesh has a minimal dihedral
% angle larger than \emph{n}. The exuder exits if it is not the case.

\ccInclude{CGAL/exude_mesh_3.h}

\ccGlobalFunction{
	template<typename C3T3>
        Mesh_optimization_return_code
        exude_mesh_3(C3T3& c3t3,
                     double time_limit=0,
                     double sliver_bound=0);}{
 \ccPrecond{\ccc{time_limit} $\geq$ 0 and 0 $\leq$ \ccc{sliver_bound} $\leq$ 180} }

\ccParameters

Parameter \ccc{C3T3} is required to be a model of the concept
\ccc{MeshComplex_3InTriangulation_3}.
The argument \ccc{c3t3}, passed by
reference, provides the initial mesh  
and is modified  by the algorithm 
to represent the final optimized mesh.

The function has two optional parameters which are named parameters (we use the Boost.Parameters library). 
Therefore, when calling the function,  the parameters can be provided in any order 
provided that the names of the parameters are used
 (see example at the bottom of this page). 
\begin{itemize}
\item
Parameter \ccc{time_limit}, whose name is \ccc{parameters::time_limit},
 is used to set up, in seconds,
 a CPU time limit after which the optimization process is stopped. This time is
 measured using the \ccc{CGAL::Timer} class.
The default value is \ccc{0} and means that there is no time limit.
\item 
Parameter \ccc{sliver_bound}, whose name is \ccc{parameters::sliver_bound},
is designed to give, in degree,  a targeted
lower bound on dihedral angles of mesh cells.
The exudation process considers in turn all the mesh cells 
that have a smallest dihedral angle less than \ccc{sliver_bound}
and tries to make them disappear by weighting their vertices.
The optimization process
stops when every cell in the mesh achieves this quality.
The default value is \ccc{0} and means that  there is no targeted bound : 
the exuder then runs as long as
it can improve the smallest dihedral angles of the set of cells
incident to  some vertices.
\end{itemize}




\ccHeading{Return Values}
The function \ccRefName{} returns a value of type \ccc{Mesh_optimization_return_code}
which is:
\begin{itemize}
\item \ccc{BOUND_REACHED} when the targeted bound for the smallest dihedral angle in the mesh is reached.
\item\ccc{TIME_LIMIT_REACHED} when the time limit is reached.
\item \ccc{CANT_IMPROVE_ANYMORE} when exudation process stops because it can no longer improve
the smallest dihedral angle of the set of cells incident to some vertex in the mesh.
\end{itemize}

\ccSeeAlso

\ccc{Mesh_optimization_return_code} \\
\ccc{make_mesh_3} \\
\ccc{refine_mesh_3} \\
\ccc{perturb_mesh_3} \\
\ccc{lloyd_optimize_mesh_3} \\
\ccc{odt_optimize_mesh_3} \\

\ccExample

\begin{ccExampleCode}
// Exude without sliver_bound, using at most 10s CPU time
exude_mesh_3(c3t3, parameters::time_limit=10);
\end{ccExampleCode}


\end{ccRefFunction}

% +------------------------------------------------------------------------+
%%RefPage: end of main body, begin of footer
\ccRefPageEnd
% EOF
% +------------------------------------------------------------------------+

