
\begin{ccRefClass}{Edge_index_property_map_stored<Graph>}

%% add template arg's if necessary

%% \ccHtmlCrossLink{}     %% add further rules for cross referencing links
%% \ccHtmlIndexC[class]{} %% add further index entries
\ccDefinition

The class \ccRefName\ provides a model for the concept 
\ccAnchor{http://www.boost.org/libs/property_map/ReadablePropertyMap.html}{ReadablePropertyMap} 
that maps an edge in a {\sc Bgl}
\ccAnchor{http://www.boost.org/libs/graph/doc/Graph.html}{graph}
to the integer numbers in the range \ccc{[0,boost::num_edges(graph))}
by accessing the index directly from the edge.

The template parameter \ccc{Graph} must be a model of a {\sc Bgl}
\ccAnchor{http://www.boost.org/libs/graph/doc/Graph.html}{graph}

\ccInclude{CGAL/boost/graph/Edge_index_property_map_stored.h}

\ccTypes
  \ccTypedef{std::size_type value_type;}
    {The type of the property.}
\ccGlue
  \ccTypedef{std::size_type reference;}
    {The result type of the map operator.}
\ccGlue
  \ccNestedType{Graph}{The Graph template parameter.}
\ccGlue
  \ccTypedef{typename boost::graph_traits<Graph>::edge_descriptor key_type;}
  {The type of {\sc Bgl} edge descriptor used as key.\\
   This edge must have a member function \ccc{std::size_t id() const;}
   returning the actual index, which is required to be unique for each edge
   and in the range \ccc{[0,num_edges(graph))}.
  }

\ccCreation
\ccCreationVariable{pm}  %% choose variable name

\ccConstructor{CGAL::Edge_index_property_map_stored<Graph>(); }
{Default constructor.}

\ccOperations

\ccMethod
  {reference operator[]( key_type const& edge ) const;}
  {Returns \ccc{edge->id()}.}  
    
\ccIsModel
\ccc{EdgeIndexPropertyMap}


\end{ccRefClass}

% +------------------------------------------------------------------------+
%%RefPage: end of main body, begin of footer
% EOF
% +------------------------------------------------------------------------+

