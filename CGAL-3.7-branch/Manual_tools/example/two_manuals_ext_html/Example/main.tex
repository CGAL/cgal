% +------------------------------------------------------------------------+
% | main.tex
% +------------------------------------------------------------------------+
% | chapter title, some introduction and a kind of table-of-contents
% +------------------------------------------------------------------------+

\chapter{Example User Manual}
\ccChapterRelease{\ccRevision. \ \ccDate}\\
\ccChapterAuthor{Lutz Kettner and Susan Hert}

This small example illustrates the creation of two manuals -- a user manual
and a reference manual in one document.  See the \texttt{Makefile} for the
commands used to create the manuals.

The file \texttt{html\_wrapper.tex} is used to define formatting commands
used throughout the manual.  There is a link to the file in the parent
directory from each subidirectory so a global change in formatting can
be accomplished by editing a single file.  This wrapper file is
required for such global settings since each file given as an argument to
\texttt{cc\_manual\_to\_html} is converted independently when the
\texttt{-extended} option is used until the end, when information is gathered 
together about all files processed in order to produce hyperlinks and such. 

The files \texttt{user\_part.tex} and \texttt{ref\_part.tex} contain simply
the \verb+\part+ commands for the two manuals.  It would also be possible to
put these commands in the files containing the first chapter commands for each
manual.

See the directory \texttt{../one\_manual\_ext\_html} for an
example of the creation of a single manual with the reference pages
simply contained in a chapter of the manual.
