\begin{ccRefFunction}{connect_holes}
\label{ref_bso_connect_holes}

\ccThree{OutputIterator}{complement}{}
\ccThreeToTwo

\ccDefinition

\ccInclude{CGAL/connect_holes.h}

\ccGlobalFunction{template <class Kernel, class Container,
                            class OutputIterator>
OutputIterator
connect_holes(const Polygon_with_holes_2<Kernel,Container>& pwh,
              OutputIterator oi);}
{Connects the holes of \ccc{pwh} with its outer boundary. This is done
 by locating the topmost vertex in each hole in the polygon with holes
 \ccc{pwh}, and connecting it by a vertical segment to the polygon
 feature located directly above it (a vertex or an edge of the outer
 boundary, or of another hole). The function produces an output
 sequence of points, which corresponds to the traversal of the vertices
 of the input polygon; this traversal starts from the outer boundary
 and moves to the holes using the auxiliary vertical segments that
 were added to connect the polygon with its holes. The value-type
 of \ccc{oi} is \ccc{Kernel::Point_2}.
 \ccPrecond{The input polygon with holes \ccc{pwh} is bounded
            (namely it has a valid outer boundary).}}

\end{ccRefFunction}

\ccRefPageEnd
