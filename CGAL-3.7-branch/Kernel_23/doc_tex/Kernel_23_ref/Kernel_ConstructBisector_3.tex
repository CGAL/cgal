\begin{ccRefFunctionObjectConcept}{Kernel::ConstructBisector_3}
A model for this must provide:

\ccCreationVariable{fo}

\ccMemberFunction{Kernel::Plane_3 operator()(const Kernel::Point_3&p, 
                                             const Kernel::Point_3&q );}
{constructs the bisector plane of $p$ and $q$.
The bisector is oriented in such a way that \ccc{p} lies on its
positive side. \ccPrecond{\ccc{p} and \ccc{q} are not equal.}}

\ccMemberFunction{Kernel::Plane_3 operator()(const Kernel::Plane_3&h1,
                                             const Kernel::Plane_3&h2);}
{constructs the bisector of the two planes $h1$ and $h2$.
In the general case, the bisector has a normal vector which has the same
direction as the sum of the normalized normal vectors of the two planes, and
passes through the intersection of \ccc{h1} and \ccc{h2}.
If \ccc{h1} and \ccc{h2} are parallel, then the bisector is defined as the
plane which has the same oriented normal vector as \ccc{l1}, and which is at
the same distance from \ccc{h1} and \ccc{h2}.
This function requires that \ccc{Kernel::RT} supports the \ccc{sqrt()}
operation.}

\ccRefines
\ccc{AdaptableFunctor} (with two arguments)

\ccSeeAlso
\ccRefIdfierPage{CGAL::bisector}

\end{ccRefFunctionObjectConcept}
