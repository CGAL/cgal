\begin{ccRefClass}{Circle_3<Kernel>}

\ccDefinition

An object of type \ccRefName\ is a circle in the
three-dimensional Euclidean space $\E^3$. Note that the
circle can be degenerate, i.e.\ the squared radius may be zero.

% -----------------------------------------------------------------------------
\ccCreation
\ccCreationVariable{c}

\ccHidden
\ccConstructor{ Circle_3( );}{
        introduces an uninitialized variable \ccVar\ of type
        \ccClassTemplateName.}

\ccConstructor{Circle_3(Point_3<Kernel> const& center, 
		Kernel::FT const& sq_r, 
		Plane_3<Kernel> const& plane);}
	{introduces a variable \ccVar\ of type \ccClassTemplateName.
        It is initialized to the circle of center \ccc{center} and 
	squared radius \ccc{sq_r} in plane \ccc{plane}.
	\ccPrecond{\ccc{center} lies in \ccc{plane} and
		\ccc{sq_r} $\geq$ 0.}}

\ccConstructor{Circle_3(Point_3<Kernel> const& center, 
		Kernel::FT const& sq_r, 
		Vector_3<Kernel> const& n);}
	{introduces a variable \ccVar\ of type \ccClassTemplateName.
        It is initialized to the circle of center \ccc{center} and 
	squared radius \ccc{sq_r} in a plane normal to
	the vector \ccc{n}.
	\ccPrecond{\ccc{sq_r} $\geq$ 0.}}

\ccConstructor{Circle_3(Point_3<Kernel> const& p, 
               Point_3<Kernel> const& q, Point_3<Kernel> const& r);}
	{introduces a variable \ccVar\ of type \ccClassTemplateName.
        It is initialized to the circle passing through the three points.
        \ccPrecond{The three points are not collinear.}}

\ccConstructor{Circle_3(Sphere_3<Kernel> const& sphere1, 
		Sphere_3<Kernel> const& sphere2);}
	{introduces a variable \ccVar\ of type \ccClassTemplateName.
        It is initialized to the circle along which the two spheres intersect.
	\ccPrecond{The two spheres intersect along a circle.}}

\ccConstructor{Circle_3(Sphere_3<Kernel> const& sphere, 
		Plane_3<Kernel> const& plane);}
	{introduces a variable \ccVar\ of type \ccClassTemplateName.
        It is initialized to the circle along which the sphere and the 
	plane intersect.
	\ccPrecond{The sphere and the plane intersect along a circle.}}

\ccConstructor{Circle_3(Plane_3<Kernel> const& plane, 
		Sphere_3<Kernel> const& sphere);}
	{introduces a variable \ccVar\ of type \ccClassTemplateName.
        It is initialized to the circle along which the sphere and the 
	plane intersect.
	\ccPrecond{The sphere and the plane intersect along a circle.}}

\ccHidden
\ccConstructor{ Circle_3( Circle_3<Kernel> const&);}{
        copy constructor.}

\ccHidden
\ccMemberFunction{ Circle_3<Kernel>& operator = ( Circle_3<Kernel> const&);}{
        assignment.}

% -----------------------------------------------------------------------------
\ccAccessFunctions

\ccMemberFunction{Point_3<Kernel> const& center( ) const;}{
        returns the center of \ccVar.}
\ccGlue
\ccMemberFunction{Kernel::FT const& squared_radius( ) const;}{
        returns the squared radius of \ccVar.}
\ccGlue
\ccMemberFunction{Plane_3<Kernel> const& supporting_plane( ) const;}{
        returns the supporting plane of \ccVar.}
\ccGlue
\ccMemberFunction{Sphere_3<Kernel> const& diametral_sphere( ) const;}{
	returns the diametral sphere of \ccVar.}


\ccMemberFunction{Kernel::FT const& area_divided_by_pi( ) const;}{
	returns the area of \ccVar, divided by $\pi$. }
\ccGlue
\ccMemberFunction{double const& approximate_area( ) const;}{
	returns an approximation of the area of \ccVar. }
\ccGlue
\ccMemberFunction{Kernel::FT const& squared_length_divided_by_pi_square( ) const;}{
	returns the squared length of \ccVar, divided by $\pi^2$. }
\ccGlue
\ccMemberFunction{double const& approximate_squared_length( ) const;}{
	returns an approximation of the squared length (i.e. perimeter) of \ccVar. }


\ccPredicates

\ccMethod{bool has_on(Point_3<Kernel> const& p) const;}
       {}

\ccOperations

\ccFunction{ bool operator == (Circle_3<Kernel> const& c1,
		           Circle_3<Kernel> const& c2);}
	{returns \ccc{true}, iff \ccc{c1} and \ccc{c2} are equal,
        i.e.\ if they have the same center, the same squared radius
	and the same supporting plane.}

\ccFunction{ bool operator != (Circle_3<Kernel> const& c1,
		           Circle_3<Kernel> const& c2);}
	{}

\ccMemberFunction{ Bbox_3 bbox() const;}{
        returns a bounding box containing \ccVar.}

\ccSeeAlso

\ccRefConceptPage{Kernel::Circle_3}

\end{ccRefClass}
