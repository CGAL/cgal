% +------------------------------------------------------------------------+
% | Reference manual page: Triangulation_2::Vertex_base.tex
% +------------------------------------------------------------------------+
% | 11.04.2000   Author
% | Package: Package
% | 
\RCSdef{\RCSTriangulation::VertexbaseRev}{$Id$}
\RCSdefDate{\RCSTriangulation::VertexbaseDate}{$Date$}
% |
%%RefPage: end of header, begin of main body
% +------------------------------------------------------------------------+


\begin{ccRefConcept}{TriangulationVertexBase_2}

%% \ccHtmlCrossLink{}     %% add further rules for cross referencing links
%% \ccHtmlIndexC[concept]{} %% add further index entries

\ccDefinition
  
The concept \ccRefName\ describes the requirements for the
vertex base class of a triangulation data structure
to be plugged in a basic, Delaunay or constrained
triangulations.

The concept \ccRefName\ refines the concept
\ccc{TriangulationDSVertexBase_2}
adding geometric information :
the vertex base of a triangulation stores a point.

\ccRefines{\ccc{TriangulationDSVertexBase_2}}

\ccTypes
\ccNestedType{Point}
{Must be the same as the point type \ccc{TriangulationTraits_2::Point_2}
defined by the geometric traits class of the triangulation.} 

\ccCreation
\ccCreationVariable{v}  %% choose variable name


\ccConstructor{TriangulationVertexBase_2(Point p);}
{constructs a vertex embedded in point \ccc{p}.}
\ccConstructor{TriangulationVertexBase_2(Point p, Face_handle f);}
{constructs a vertex embedded in point \ccc{p} and pointing on face \ccc{f}.}

\ccAccessFunctions
\ccThree{istream&}{istream& is << & v }{}
\ccMethod{Point point() const;}
{returns  the point.}
%\ccGlue
%\ccMethod{void* face() const;}{ returns a pointer to an incident face.}

\ccHeading{Setting}
\ccMethod{void set_point(Point p);}
{sets the point.}

\ccHeading{I/O}

\ccFunction{istream& operator>>
(istream& is, TriangulationVertexBase_2 & v);}
{Inputs the non-combinatorial information given by the vertex: 
the point and other possible information.}
%\ccPrecond{The point and the other information have a corresponding
%operator \ccc{>>}. \textit{This precondition is optional for the
%triangulation data structure alone.}}}  

\ccFunction{ostream& operator<< (ostream& os, 
			const TriangulationVertexBase_2 & v);}
{Outputs the non combinatorial operation given by the vertex: the
point and other possible information.}
%\ccPrecond{The point and the other information have a corresponding
%operator \ccc{<<}. \textit{This precondition is optional for the
%triangulation data structure alone.}}} 




\ccHasModels

\ccc{CGAL::TriangulationVertexBase_2<Traits>}.


\ccSeeAlso

\ccc{TriangulationDataStructure_2} \\
\ccc{TriangulationDataStructure_2::Vertex} \\
\ccc{CGAL::Triangulation_vertex_base_2<Traits>}


\end{ccRefConcept}

% +------------------------------------------------------------------------+
%%RefPage: end of main body, begin of footer
% EOF
% +------------------------------------------------------------------------+

