% +------------------------------------------------------------------------+
% | Reference manual page: Triangulation_vertex_base_2.tex
% +------------------------------------------------------------------------+
% | 11.04.2000   Author
% | Package: Package
% | 
\RCSdef{\RCSTriangulationvertexbaseRev}{$Id$}
\RCSdefDate{\RCSTriangulationvertexbaseDate}{$Date$}
% |
%%RefPage: end of header, begin of main body
% +------------------------------------------------------------------------+


\begin{ccRefClass}{Triangulation_vertex_base_2<Traits,Vb>}  
%% add template arg's if necessary

%% \ccHtmlCrossLink{}     %% add further rules for cross referencing links
%% \ccHtmlIndexC[class]{} %% add further index entries

\ccDefinition
  
The class \ccRefName\ is the default model for the concept
\ccc{TriangulationVertexBase_2}.

 \ccRefName\ can be simply plugged in the triangulation data
structure
of a triangulation, or used  as a base  class to derive
other base vertex classes tuned for specific applications.

\ccInclude{CGAL/Triangulation_vertex_base_2.h}

\ccParameters
\ccRefName\  is  templated by a geometric traits class which provide the type
\ccc{Point}. It is strongly recommended to instantiate this
traits class with the model used for the triangulation traits class.
This ensures that the point type defined by \ccRefName\
is the same as the point type defined by 
the triangulation.

The second template parameter of \ccRefName\ 
has to be a model of the concept \ccc{TriangulationDSVertexBase_2}
By default this parameter is
 instantiated by \ccc{CGAL::Triangulation_ds_vertex_base_2<>}.

\ccIsModel
\ccc{TriangulationVertexBase_2}

\ccInheritsFrom
\ccc{Vb}

\ccSeeAlso
\ccc{CGAL::Triangulation_ds_vertex_base_2<Tds>} \\
\ccc{CGAL::Triangulation_face_base_2<Traits,Fb>} \\
\ccc{CGAL::Regular_triangulation_vertex_base_2<Traits,Vb>} \\
\ccc{CGAL::Triangulation_vertex_base_with_info_2<Info,Traits,Vb>}

\end{ccRefClass}

% +------------------------------------------------------------------------+
%%RefPage: end of main body, begin of footer
% EOF
% +------------------------------------------------------------------------+

