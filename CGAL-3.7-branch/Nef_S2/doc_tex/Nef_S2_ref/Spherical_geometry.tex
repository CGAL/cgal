\ccHeading{Restricted Spherical Geometry}

We introduce geometric objects that are part of the spherical surface
$S_2$ and operations on them. We define types \ccc{Sphere_point},
\ccc{Sphere_circle}, \ccc{Sphere_segment}, and \ccc{Sphere_direction}.
\ccc{Sphere_point}s are points on $S_2$, \ccc{Sphere_circle}s are
oriented great circles of $S_2$, \ccc{Sphere_segment}s are oriented
parts of \ccc{Sphere_circles} bounded by a pair of
\ccc{Sphere_point}s, and \ccc{Sphere_direction}s are directions that
are part of great circles. (a direction is usually defined to be a
vector without length, that floats around in its underlying space and
can be used to specify a movement at any point of the underlying
space; in our case we use directions only at points that are part of
the great circle that underlies also the direction.)

Note that we have to consider special geometric properties of the
objects. For example two points that are part of a great circle define
two \ccc{Sphere_segment}s, and two arbitrary \ccc{Sphere_segment}s can
intersect in two points.

If we restrict our geometric objects to a so-called perfect hemisphere
of $S_2$\footnote{A perfect hemisphere of $S_2$ is an open half-sphere
  plus an open half-circle in the boundary of the open half-sphere
  plus one endpoint of the half-circle.} then the restricted objects
behave like in classical geometry, e.g., two points define exactly one
segment, two segments intersect in at most one interior point
(non-degenerately), or three non-cocircular sphere points can be
qualified as being positively or negatively oriented.

