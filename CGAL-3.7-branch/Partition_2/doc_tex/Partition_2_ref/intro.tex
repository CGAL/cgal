% +------------------------------------------------------------------------+
% | Reference manual page: Partition_2_ref/intro.tex
% +------------------------------------------------------------------------+
% | 26.6.2000   Susan Hert
% | Package: Partition_2
% | 
% |
%%RefPage: end of header, begin of main body
% +------------------------------------------------------------------------+

%\clearpage
%\section{Reference Pages for 2D Polygon Partitioning}

\ccRefChapter{2D Polygon Partitioning\label{chap:partition_2_ref}}
\ccChapterAuthor{Susan Hert}

\ccEnableRawListOfRefpages


\begin{ccTexOnly}
\ifnum\ccNewRefManualStyle=\ccTrue
\end{ccTexOnly}
A {\em partition}\ccIndexMainItemDef{partition} of a polygon is a set 
of polygons such that the interiors of the polygons do not intersect and 
the union of the polygons is equal to the interior of the original polygon.
Functions are available for partitioning planar polygons into two 
types of subpolygons --- $y$-monotone polygons and convex polygons.  

The function that produces a $y$-monotone partitioning is based on the 
algorithm presented in \cite{bkos-cgaa-97} which requires $O(n \log n)$ time 
and $O(n)$ space for a polygon with $n$ vertices and guarantees nothing 
about the number of polygons produced with respect to the optimal number.
Three functions are provided for producing
convex partitions. Two of these functions produce approximately optimal 
partitions and one results in an optimal partition, where ``optimal'' is
defined in terms of the number of partition polygons.   The two functions
that implement approximation algorithms are guaranteed to produce no more 
than four times the optimal number of convex pieces.  The optimal partitioning
function provides an implementation of Greene's dynamic programming algorithm
\cite{g-dpcp-83}, which requires $O(n^4)$ time and $O(n^3)$ space to produce a 
convex partitioning. One of the approximation algorithms is also due to 
Greene \cite{g-dpcp-83} and requires $O(n \log n)$ time and $O(n)$ space
to produce a convex partitioning given a $y$-monotone partitioning.  The
other approximation algorithm is a result of Hertel and
Mehlhorn \cite{hm-ftsp-83}, which requires $O(n)$ time and space to produce
a convex partitioning from a triangulation of a polygon.
Each of the partitioning functions uses a traits class to supply the
primitive types and predicates used by the algorithms.
\begin{ccTexOnly}
\fi
\end{ccTexOnly}



\ccHeading{Assertions}

\begin{ccPackage}{polygon partitioning}
\ccIndexAssertionFlag[polygon partitioning]
The assertion flags for this package use \ccc{PARTITION} in their names
({\em e.g.}, \ccc{CGAL_PARTITION_NO_POSTCONDITIONS}).
The precondition checks for the planar polygon partitioning functions
are:  counterclockwise ordering of the input vertices and simplicity of the 
polygon these vertices represent.
\ccIndexSubitem[C]{approx_convex_partition_2}{postconditions}
\ccIndexSubitem[C]{greene_approx_convex_partition_2}{postconditions}
\ccIndexSubitem[C]{optimal_convex_partition_2}{postconditions}
\ccIndexSubitem[C]{y_monotone_partition_2}{postconditions}
The postcondition checks are:  simplicity, counterclockwise orientation,
and convexity (or $y$-monotonicity) of the partition polygons
and validity of the partition ({\em i.e.}, the partition polygons are 
nonoverlapping and the union of these polygons is the same as the
original polygon)
\ccIndexSubitemDef{partition}{valid}.
\end{ccPackage}

\section{Classified Reference Pages}

\subsection*{Concepts}

\lcTex{
\ccRefConceptPage{ConvexPartitionIsValidTraits_2}\\
\ccRefConceptPage{IsYMonotoneTraits_2}\\
\ccRefConceptPage{OptimalConvexPartitionTraits_2}\\
\ccRefConceptPage{PartitionTraits_2}\\
\ccRefConceptPage{PartitionIsValidTraits_2}\\
\ccRefConceptPage{YMonotonePartitionIsValidTraits_2}\\
\ccRefConceptPage{YMonotonePartitionTraits_2} \\
}

\lcHtml{
% +------------------------------------------------------------------------+
% | Reference manual page: ConvexPartitionIsValidTraits_2.tex
% +------------------------------------------------------------------------+
% | 10.05.2000   Susan Hert
% | Package: Partition_2
% | 
% |
%%RefPage: end of header, begin of main body
% +------------------------------------------------------------------------+


\begin{ccRefConcept}{ConvexPartitionIsValidTraits_2}
\ccIndexSubitemBegin[C]{approx_convex_partition_2}{traits class}
\ccIndexSubitemBegin[C]{greene_approx_convex_partition_2}{traits class}
\ccIndexSubitemBegin[C]{optimal_convex_partition_2}{traits class}
\ccIndexSubitemBegin[C]{convex_partition_is_valid_2}{traits class}


%% \ccHtmlCrossLink{}     %% add further rules for cross referencing links
%% \ccHtmlIndexC[concept]{} %% add further index entries

\ccDefinition
  
Requirements of a traits class used 
by \ccc{convex_partition_is_valid_2} for testing the validity of a
convex partition of a polygon.

\ccTypes

All types required by the concept PartitionIsValidTraits\_2 are required
except the function object type \ccc{Is_valid}. The following type is
required instead:

\ccNestedType{Is_convex_2}{Model of the concept PolygonIsValid that tests if
                           a sequence of points is convex or not.}

\ccCreationVariable{traits}
\ccOperations

The following function that creates an instance of the above predicate object
type must exist instead of the function \ccc{is_valid_object} required by
PartitionIsValidTraits\_2.

\ccMethod{Is_convex_2 is_convex_2_object(ConvexPartitionIsValidTraits_2 t);}{}

\ccHasModels

\ccc{Partition_traits_2}

\ccSeeAlso

\ccc{approx_convex_partition_2},
\ccc{greene_approx_convex_partition_2},
\ccc{Is_convex_2},
\ccc{is_convex_2},
\ccc{optimal_convex_partition_2}

\ccIndexSubitemEnd[C]{approx_convex_partition_2}{traits class}
\ccIndexSubitemEnd[C]{greene_approx_convex_partition_2}{traits class}
\ccIndexSubitemEnd[C]{optimal_convex_partition_2}{traits class}
\ccIndexSubitemEnd[C]{convex_partition_is_valid_2}{traits class}
\end{ccRefConcept}

% +------------------------------------------------------------------------+
%%RefPage: end of main body, begin of footer
% EOF
% +------------------------------------------------------------------------+


% +------------------------------------------------------------------------+
% | Reference manual page: IsYMonotoneTraits_2.tex
% +------------------------------------------------------------------------+
% | 10.05.2000   Susan Hert
% | Package: Partition_2
% | 
% |
% +------------------------------------------------------------------------+


\begin{ccRefConcept}{IsYMonotoneTraits_2}
\ccIndexSubitemBegin[C]{is_y_monotone_2}{traits class}

\ccDefinition
  
Requirements of a traits class to be
used with the function \ccc{is_y_monotone_2} that tests whether a sequence of
2D points defines a $y$-monotone polygon or not.  

\ccTypes

The following two types are required:

\ccNestedType{Point_2}{The point type of the polygon vertices.}

\ccNestedType{Less_yx_2}{
Predicate object type that compares \ccc{Point_2}s lexicographically.
Must provide \ccc{bool operator()(Point_2 p, Point_2 q)} where \ccc{true}
is returned iff $p <_{xy} q$.
We have $p<_{xy}q$, iff $p_x < q_x$ or $p_x = q_x$ and $p_y < q_y$,
where $p_x$ and $p_y$ denote $x$ and $y$ coordinate of point $p$ resp.
}


\ccCreation
\ccCreationVariable{traits}  %% choose variable name

Only a copy constructor is required.

\ccConstructor{IsYMonotoneTraits_2(IsYMonotoneTraits_2& tr)}{}

\ccOperations

The following function that creates an instance of the above predicate
object type must exist: 

\ccMethod{Less_yx_2 less_yx_2_object();}{}

\ccHasModels

\ccRefIdfierPage{CGAL::Partition_traits_2<R>} \\
\ccc{CGAL::Kernel_traits_2}

\ccSeeAlso

\ccRefIdfierPage{CGAL::Is_y_monotone_2<Traits>} \\
\ccRefIdfierPage{CGAL::y_monotone_partition_2} \\
\ccRefIdfierPage{CGAL::y_monotone_partition_is_valid_2}

\ccIndexSubitemEnd[C]{is_y_monotone_2}{traits class}
\end{ccRefConcept}

% +------------------------------------------------------------------------+
%%RefPage: end of main body, begin of footer
% EOF
% +------------------------------------------------------------------------+


% +------------------------------------------------------------------------+
% | Reference manual page: OptimalConvexPartitionTraits_2.tex
% +------------------------------------------------------------------------+
% | 10.05.2000   Susan Hert
% | Package: Partition_2
% | 
% |
% +------------------------------------------------------------------------+


\begin{ccRefConcept}{OptimalConvexPartitionTraits_2}
\ccIndexSubitemBegin[C]{optimal_convex_partition_2}{traits class}

\ccDefinition
  
Requirements of a traits class to be
used with the function \ccc{optimal_convex_partition_2} that computes
an optimal convex partition of a polygon.

\ccRefines

\ccRefConceptPage{PartitionTraits_2}%
\ccIndexMainItem[c]{PartitionTraits_2}

\ccTypes

\ccIndexMainItem[c]{PartitionTraits_2}
In addition to the types listed with the concept \ccc{PartitionTraits_2}, the
following types are required:

\ccNestedType{Segment_2}{A segment type}
\ccNestedType{Ray_2}{A ray type}
\ccNestedType{Object_2}{A general object type that can be either a point or a segment}
\ccNestedType{Construct_segment_2}{Function object type that provides
\ccc{Segment_2 operator()(Point_2 p, Point_2 q)}, which constructs and
returns the segment defined by the points $p$ and $q$.}
\ccNestedType{Construct_ray_2}{Function object type that provides
\ccc{Ray_2 operator()(Point_2 p, Point_2 q)}, which constructs and returns
the ray from point $p$ through point $q$.}
\ccNestedType{Collinear_are_ordered_along_line_2}{Predicate object type that
determines orderings of \ccc{Point_2}s on a line.  Must provide
\ccc{bool operator()(Point_2 p, Point_2 q, Point_2 r)} that 
returns \ccStyle{true}, iff \ccStyle{q} lies between \ccStyle{p}
and \ccStyle{r} and \ccc{p}, \ccc{q}, and \ccc{r} satisfy the precondition
that they are collinear.}

\ccNestedType{Are_stritcly_ordered_along_line_2}{Predicate object type that
determines orderings of \ccc{Point_2}s.  Must provide
\ccc{bool operator()(Point_2 p, Point_2 q, Point_2 r)} that 
returns \ccStyle{true}, iff the three points are collinear and 
\ccStyle{q} lies strictly between \ccStyle{p}
and \ccStyle{r}.  Note that \ccc{false} should be returned if
\ccStyle{q==p} or \ccStyle{q==r}.}

\ccNestedType{Intersect_2}{Function object type that provides
\ccc{Object_2 operator()(Segment_2 s1, Segment_2 s2)} that returns
the intersection of two segments (which may be either a segment or
a point).}

\ccNestedType{Assign_2}{Function object type that provides
\ccc{bool operator()(Segment_2 s1, Object_2 o)} that returns 
\ccc{true} if \ccc{o} is a segment and assigns the value of \ccc{o}
to \ccc{s1}; returns \ccc{false} otherwise.}

\ccCreationVariable{traits}  %% choose variable name
\ccCreation

Only a copy constructor is required.

\ccConstructor{OptimalConvexPartitionTraits_2(OptimalConvexPartitionTraits_2& tr
)}{}

\ccOperations

\ccIndexMainItem[c]{PartitionTraits_2}
In addition to the functions required by \ccc{PartitionTraits_2}, the following
functions that create instances of the above function object types
must exist:

\ccSetThreeColumns{Collinear_are_ordered_along_line_2xxxx}{traits.collinear_are_ordered_along_line_2_object()}{}
\ccMethod{Collinear_are_ordered_along_line_2 collinear_are_ordered_along_line_2_object() const;}{}
\ccMethod{Construct_segment_2 construct_segment_2_object() const;}{}
\ccMethod{Construct_ray_2 construct_ray_2_object() const;}{}
\ccMethod{Are_strictly_ordered_along_line_2
    are_strictly_ordered_along_line_2_object() const;}{}


\ccHasModels

%\ccc{Partition_optimal_convex_traits_2},
\ccRefIdfierPage{CGAL::Partition_traits_2<R>}

\ccSeeAlso

\ccRefIdfierPage{CGAL::convex_partition_is_valid_2} \\
\ccRefIdfierPage{CGAL::Partition_is_valid_traits_2<Traits, PolygonIsValid>}

\ccIndexSubitemEnd[C]{optimal_convex_partition_2}{traits class}
\end{ccRefConcept}

% +------------------------------------------------------------------------+
%%RefPage: end of main body, begin of footer
% EOF
% +------------------------------------------------------------------------+


% +------------------------------------------------------------------------+
% | Reference manual page: PartitionTraits_2.tex
% +------------------------------------------------------------------------+
% | 10.05.2000   Susan Hert
% | Package: Decomposition_2
% | 
% |
% +------------------------------------------------------------------------+


\begin{ccRefConcept}{PartitionTraits_2}

\ccDefinition
  
The polygon partitioning functions are each parameterized by a traits class 
that defines the primitives used in the algorithms.  Many requirements are 
common
to all traits classes.  The concept \ccRefName\ defines this common set of
requirements.  

\ccTypes

\ccNestedType{Point_2}{The point type on which the partitioning algorithm operates.}

\ccNestedType{Polygon_2}{The polygon type to be created by the partitioning 
algorithm. For testing the validity postcondition of the partition, this 
type should provide a nested type \ccc{Vertex_const_iterator} that is the
type of the iterator over the polygon vertices and member functions
\ccc{Vertex_const_iterator vertices_begin()} and
\ccc{Vertex_const_iterator vertices_end()}.}%
\ccIndexSubitem[C]{approx_convex_partition_2}{postconditions}
\ccIndexSubitem[C]{greene_approx_convex_partition_2}{postconditions}
\ccIndexSubitem[C]{optimal_convex_partition_2}{postconditions}
\ccIndexSubitem[C]{y_monotone_partition_2}{postconditions}

\ccNestedType{Less_xy_2}{
Predicate object type that compares \ccc{Point_2}s lexicographically.
Must provide \ccc{bool operator()(Point_2 p, Point_2 q)} where \ccc{true}
is returned iff $p <_{xy} q$.
We have $p<_{xy}q$, iff $p_x < q_x$ or $p_x = q_x$ and $p_y < q_y$,
where $p_x$ and $p_y$ denote the $x$ and $y$ coordinates of point $p$, 
respectively.
}

\ccNestedType{Less_yx_2}{
Same as \ccc{Less_xy_2} with the roles of $x$ and $y$ interchanged.}


\ccNestedType{Leftturn_2}{
Predicate object type that provides 
\ccc{bool operator()(Point_2 p,Point_2 q,Point_2 r)}, which
returns \ccc{true} iff \ccc{r} lies to the left of the 
oriented line through \ccc{p} and \ccc{q}.}

\ccNestedType{Orientation_2}{Predicate object type that provides
\ccc{CGAL::Orientation operator()(Point_2 p, Point_2 q, Point_2 r)} that
returns \ccStyle{CGAL::LEFTTURN}, if $r$ lies to the left of the oriented 
line $l$ defined by $p$ and $q$, returns \ccStyle{CGAL::RIGHTTURN} if $r$ 
lies to the right of $l$, and returns \ccStyle{CGAL::COLLINEAR} if $r$ lies
on $l$.}

\ccNestedType{Compare_y_2}{Predicate object type that provides
\ccc{CGAL::Comparision_result operator()(Point_2 p, Point_2 q)} to compare
the $y$ values of two points.  The operator must return
\ccc{CGAL::SMALLER} if $p_y < q_y$, \ccc{CGAL::LARGER} if $p_y > q_y$ and
\ccc{CGAL::EQUAL} if $p_y = q_y$.}

\ccNestedType{Compare_x_2}{The same as \ccc{Compare_y_2}, except that $x$ 
coordinates are compared instead of $y$.}

%\ccNestedType{Construct_line_2}{Predicate object type that provides
%\ccc{Line_2 operator()(Point_2 p, Point_2 q)}, which constructs and
%returns the line defined by the points $p$ and $q$.}


\ccCreation
\ccCreationVariable{traits}  %% choose variable name

A copy constructor and default constructor are required.

\ccConstructor{PartitionTraits_2()}{}

\ccConstructor{PartitionTraits_2(PartitionTraits_2& tr)}{}

\ccOperations

The following functions that create instances of the above predicate object
types must exist.

\ccMethod{Less_yx_2 less_yx_2_object();}{}

\ccMethod{Less_xy_2 less_xy_2_object();}{}

\ccMethod{Leftturn_2 leftturn_2_object();}{}

\ccMethod{Orientation_2 orientation_2_object();}{}

\ccMethod{Compare_y_2 compare_y_2_object();}{}

\ccMethod{Compare_x_2 compare_x_2_object();}{}

%\ccMethod{Construct_line_2 construct_line_2_object();}{}

%The following functions are required by the functions \ccc{is_simple_2} and
%\ccc{orientation_2}, which are used to test the simplicity and CCW order 
%preconditions of the partitioning functions. 
%
%\ccMemberFunction{
%    Comparison_result compare_x(const Point_2 &p, const Point_2 &q) const;
%}
%{
%\lcTex{
%Returns
%$
%  \left\{
%  \begin{array}{lll}
%     \ccStyle{SMALLER} & \mbox{if} & p_x < q_x \\
%     \ccStyle{EQUAL}   & \mbox{if} & p_x = q_x \\
%     \ccStyle{LARGER}  & \mbox{if} & p_x > q_x
%  \end{array}
%  \right.%}
%$
%}
%\lcHtml{
%Returns SMALLER if p_x < q_x, EQUAL if p_x = q_x and LARGER if p_x > q_x
%}
%}
%
%
%\ccMemberFunction{
%    Comparison_result compare_y(const Point_2 &p, const Point_2 &q) const;
%}
%{
%\lcTex{
%Returns
%$
%  \left\{
%  \begin{array}{lll}
%     \ccStyle{SMALLER} & \mbox{if} & p_y < q_y \\
%     \ccStyle{EQUAL}   & \mbox{if} & p_y = q_y \\
%     \ccStyle{LARGER}  & \mbox{if} & p_y > q_y
%  \end{array}
%  \right.%}
%$
%}
%\lcHtml{
%Returns SMALLER if p_y < q_y, EQUAL if p_y = q_y and LARGER if p_y > q_y
%}
%}
%
%\ccMemberFunction{
%    FT cross_product_2(const Vector_2& p, const Vector_2& q) const;
%}
%{
%Returns $p_x q_y - p_y q_x$.
%}
%
%\ccMemberFunction{
%    bool do_intersect(const Point_2& p1,
%                      const Point_2& q1,
%                      const Point_2& p2,
%                      const Point_2& q2) const;
%}
%{
%Returns \ccc{true} iff the segments \ccStyle{[p1,q1]} and \ccStyle{[p2,q2]}
%intersect.
%}
%
%\ccMemberFunction{
%    bool have_equal_direction(const Vector_2& v1,
%                              const Vector_2& v2 ) const;
%}
%{
%Returns \ccc{true} iff the vectors \ccStyle{v1} and \ccStyle{v2} have the
%same direction.
%}
%
%\ccMemberFunction{
%    bool is_negative(const FT& x) const;
%}
%{
%Returns \ccStyle{true} iff \ccStyle{x<0}.
%}
%
%\ccMemberFunction{
%    bool lexicographically_yx_smaller_or_equal(const Point_2& p,
%                                               const Point_2& q) const;
%}
%{
%Returns \ccc{true} iff $p_y \leq q_y$ or $p_y = q_y$ and $p_x \leq q_x$.
%}
%
%\ccMethod{
%    Orientation
%    orientation(const Point_2& p, const Point_2& q, const Point_2& r) const;
%}
%{
%Returns \ccStyle{LEFTTURN}, if $r$ lies to the left of the oriented
%line $l$ defined by $p$ and $q$, returns \ccStyle{RIGHTTURN} if $r$
%lies to the right of $l$, and returns \ccStyle{COLLINEAR} if $r$ lies
%on $l$.
%%}



\ccHasModels

\ccRefIdfierPage{CGAL::Partition_traits_2<R>}

\ccSeeAlso

\ccRefIdfierPage{CGAL::approx_convex_partition_2} \\
\ccRefIdfierPage{CGAL::greene_approx_convex_partition_2} \\
\ccRefIdfierPage{CGAL::optimal_convex_partition_2} \\
\ccRefIdfierPage{CGAL::y_monotone_partition_2}

\end{ccRefConcept}

% +------------------------------------------------------------------------+
%%RefPage: end of main body, begin of footer
% EOF
% +------------------------------------------------------------------------+


% +------------------------------------------------------------------------+
% | Reference manual page: PartitionIsValidTraits_2.tex
% +------------------------------------------------------------------------+
% | 10.05.2000   Susan Hert
% | Package: Partition_2
% | 
% |
% +------------------------------------------------------------------------+

\renewcommand\ccRefPageBegin{\ccParDims\cgalColumnLayout\begin{ccAdvanced}}
\renewcommand\ccRefPageEnd{\ccParDims\cgalColumnLayout\end{ccAdvanced}}
\begin{ccRefConcept}{PartitionIsValidTraits_2}

\ccDefinition
  
Requirements of a traits class that is used 
by \ccc{partition_is_valid_2}, \ccc{convex_partition_is_valid_2},
and \ccc{y_monotone_partition_is_valid_2} for testing if a given set of 
polygons are nonoverlapping and if their union is a polygon that is the 
same as a polygon represented by a given sequence of points.  Note that the 
traits class for \ccc{partition_is_valid_2} may have to satisfy additional
requirements if each partition polygon is to be tested for having a
particular property; see, for example, the descriptions of the 
function \ccc{is_convex_2}
and the concept YMonotonePartitionTraits\_2 for the additional requirements
for testing for convexity and $y$-monotonicity, respectively.

\ccTypes

\ccNestedType{Point_2}{The point type on which the partitioning algorithm operates.}

\ccNestedType{Polygon_2}{The polygon type created by the partitioning 
function. This type should provide a nested type \ccc{Vertex_const_iterator} 
that is the type of the non-mutable iterator over the polygon vertices.}%

\ccNestedType{Is_valid}{A model of the concept PolygonIsValid}

\ccNestedType{Less_xy_2}{
Predicate object type that compares \ccc{Point_2}s lexicographically.
Must provide \ccc{bool operator()(Point_2 p, Point_2 q)} where \ccc{true}
is returned iff $p <_{xy} q$.
We have $p<_{xy}q$, iff $p_x < q_x$ or $p_x = q_x$ and $p_y < q_y$,
where $p_x$ and $p_y$ denote the $x$ and $y$ coordinates of point $p$, 
respectively.
}

\ccNestedType{Leftturn_2}{
Predicate object type that provides 
\ccc{bool operator()(Point_2 p,Point_2 q,Point_2 r)}, which
returns \ccc{true} iff \ccc{r} lies to the left of the 
oriented line through \ccc{p} and \ccc{q}.}

\ccNestedType{Orientation_2}{Predicate object type that provides
\ccc{CGAL::Orientation operator()(Point_2 p, Point_2 q, Point_2 r)} that
returns \ccStyle{CGAL::LEFTTURN}, if $r$ lies to the left of the oriented 
line $l$ defined by $p$ and $q$, returns \ccStyle{CGAL::RIGHTTURN} if $r$ 
lies to the right of $l$, and returns \ccStyle{CGAL::COLLINEAR} if $r$ lies
on $l$.}



\ccCreation
\ccCreationVariable{traits}  %% choose variable name

Only a copy constructor is required.

\ccConstructor{PartitionIsValidTraits_2(PartitionIsValidTraits_2& tr)}{}

\ccOperations

The following functions that create instances of the above predicate object
types must exist.

\ccMethod{Orientation_2 is_valid_object();}{}

\ccMethod{Less_xy_2 less_xy_2_object();}{}

\ccMethod{Leftturn_2 leftturn_2_object();}{}

\ccMethod{Orientation_2 orientation_2_object();}{}


\ccHasModels

\ccRefIdfierPage{CGAL::Partition_is_valid_traits_2<Traits, PolygonIsValid>}

\ccSeeAlso

\ccRefIdfierPage{CGAL::approx_convex_partition_2} \\
\ccRefIdfierPage{CGAL::greene_approx_convex_partition_2} \\
\ccRefIdfierPage{CGAL::optimal_convex_partition_2} \\
\ccRefIdfierPage{CGAL::y_monotone_partition_2}

\end{ccRefConcept}
\renewcommand\ccRefPageBegin{\ccParDims\cgalColumnLayout}
\renewcommand\ccRefPageEnd{\ccParDims\cgalColumnLayout}

% +------------------------------------------------------------------------+
%%RefPage: end of main body, begin of footer
% EOF
% +------------------------------------------------------------------------+


% +------------------------------------------------------------------------+
% | Reference manual page: YMonotonePartitionIsValidTraits_2.tex
% +------------------------------------------------------------------------+
% | 10.05.2000   Susan Hert
% | Package: Partition_2
% | 
% +------------------------------------------------------------------------+

\renewcommand\ccRefPageBegin{\ccParDims\cgalColumnLayout\begin{ccAdvanced}}
\renewcommand\ccRefPageEnd{\ccParDims\cgalColumnLayout\end{ccAdvanced}}
\begin{ccRefConcept}{YMonotonePartitionIsValidTraits_2}
\ccIndexSubitemBegin[C]{y_monotone_partition_2}{traits class}
\ccIndexSubitemBegin[C]{y_monotone_partition_is_valid}{traits class}


\ccDefinition
  
Requirements of a traits class that is used 
by \ccc{y_monotone_partition_is_valid_2} for testing the validity of a
$y$-monotone partition of a polygon.

\ccTypes

All types required by the concept PartitionIsValidTraits\_2 are required
except the function object type \ccc{Is_valid}. The following type is
required instead:

\ccNestedType{Is_y_monotone_2}{Model of the concept PolygonIsValid that tests if
                               a sequence of points is $y$-monotone or not.}

\ccCreationVariable{traits}
\ccCreation

Only a copy constructor is required.

\ccConstructor{YMonotonePartitionIsValidTraits_2(YMonotonePartitionIsValidTraits_2& tr
)}{}
\ccOperations

The following function that creates an instance of the above predicate object
type must exist instead of the function \ccc{is_valid_object} required by
PartitionIsValidTraits\_2.

\ccMethod{Is_y_monotone_2 is_y_monotone_2_object(YMonotonePartitionIsValidTraits_2 t);}{}

\ccHasModels

\ccRefIdfierPage{CGAL::Partition_traits_2<R>}

\ccSeeAlso

\ccRefIdfierPage{CGAL::partition_is_valid_2} \\
\ccRefIdfierPage{CGAL::y_monotone_partition_2}

\ccIndexSubitemEnd[C]{y_monotone_partition_2}{traits class}
\ccIndexSubitemEnd[C]{y_monotone_partition_is_valid}{traits class}
\end{ccRefConcept}
\renewcommand\ccRefPageBegin{\ccParDims\cgalColumnLayout}
\renewcommand\ccRefPageEnd{\ccParDims\cgalColumnLayout}

% +------------------------------------------------------------------------+
%%RefPage: end of main body, begin of footer
% EOF
% +------------------------------------------------------------------------+


% +------------------------------------------------------------------------+
% | Reference manual page: YMonotonePartitionTraits_2.tex
% +------------------------------------------------------------------------+
% | 10.05.2000   Susan Hert
% | Package: Partition_2
% | 
% |
%%RefPage: end of header, begin of main body
% +------------------------------------------------------------------------+


\begin{ccRefConcept}{YMonotonePartitionTraits_2}
\ccIndexSubitemBegin[C]{y_monotone_partition_2}{traits class}

\ccDefinition
  
Requirements of a traits class to be
used with the function \ccc{y_monotone_partition_2}.  

\ccRefines

\ccRefConceptPage{PartitionTraits_2}

\ccTypes

\ccIndexMainItem[c]{PartitionTraits_2}
In addition to the types defined for the concept \ccc{PartitionTraits_2}, the
following types are also required:


\ccNestedType{Line_2}{}
%                      that has a function \ccc{FT x_at_y(FT x)} that returns
%                      the $y$-coordinate of the point on non-vertical line 
%                      $l$ with the given $x$-coordinate.}

\ccNestedType{Compare_x_at_y_2}{Predicate object type that provides
\ccc{CGAL::Comparision_result operator()(Point_2 p, Line_2 h)} to compare
the $x$ coordinate of \ccc{p} and the horizontal projection of \ccc{p}
on \ccc{h}.}

\ccNestedType{Construct_line_2}{Function object type that provides
\ccc{Line_2 operator()(Point_2 p, Point_2 q)}, which constructs and
returns the line defined by the points $p$ and $q$.}

\ccNestedType{Is_horizontal_2}{Function object type that provides
\ccc{bool operator()(Line_2 l)}, which returns \ccc{true} iff the
line \ccc{l} is horizontal.}

\ccCreation
\ccCreationVariable{traits}  %% choose variable name

A copy constructor and default constructor are required.

\ccConstructor{YMonotonePartitionTraits();}{}

\ccConstructor{YMonotonePartitionTraits(const YMonotonePartitionTraits& tr);}{}

\ccOperations

In addition to the functions required for the concept \ccc{PartitionTraits_2},
\ccIndexMainItem[c]{PartitionTraits_2}
the following functions that create instances of the above function 
object types must exist.

\ccMethod{Construct_line_2 construct_line_2_object();}{}

\ccMethod{Compare_x_at_y_2 compare_x_at_y_2_object();}{}

\ccMethod{Is_horizontal_2 is_horizontal_2_object();}{}

\ccHasModels

\ccRefIdfierPage{CGAL::Partition_traits_2<R>}

%\ccSeeAlso
%
%\ccc{y_monotone_partition_2}

\ccIndexSubitemEnd[C]{y_monotone_partition_2}{traits class}
\end{ccRefConcept}

% +------------------------------------------------------------------------+
%%RefPage: end of main body, begin of footer
% EOF
% +------------------------------------------------------------------------+


}

\subsection*{Function Object Concepts}

\lcTex{\ccRefConceptPage{PolygonIsValid} \\}
\lcHtml{% +------------------------------------------------------------------------+
% | Reference manual page: PolygonIsValid.tex
% +------------------------------------------------------------------------+
% | 10.05.2000   Susan Hert
% | Package: Partition_2
% | 
% |
%%RefPage: end of header, begin of main body
% +------------------------------------------------------------------------+

\renewcommand\ccRefPageBegin{\ccParDims\cgalColumnLayout\begin{ccAdvanced}}
\renewcommand\ccRefPageEnd{\ccParDims\cgalColumnLayout\end{ccAdvanced}}
\begin{ccRefFunctionObjectConcept}{PolygonIsValid}

\ccDefinition
  
Function object that determines if a sequence of points represents a
valid partition polygon or not, where ``valid'' can assume any of several
meanings ({\it e.g.}, convex or $y$-monotone).
\ccIndexSubitem{polygon partitioning}{valid}


\ccCreation
\ccCreationVariable{f}  %% choose variable name

\ccConstructor{PolygonIsValid(const Traits& t);}{
\ccc{Traits} is a model of the concept required by the function that checks
for validity of the polygon.
}

\ccOperations

\ccMethod{
template<class InputIterator>
bool operator()(InputIterator first, InputIterator beyond);}
{ 
  returns \ccc{true} iff the points of type \ccc{Traits::Point_2}
  in the range [\ccc{first},\ccc{beyond}) define a valid polygon.
}

\ccHasModels

\ccRefIdfierPage{CGAL::Is_convex_2<Traits>}  \\
\ccRefIdfierPage{CGAL::Is_y_monotone_2<Traits>}

\ccSeeAlso

\ccRefIdfierPage{CGAL::approx_convex_partition_2} \\
\ccRefIdfierPage{CGAL::convex_partition_is_valid_2} \\
\ccRefIdfierPage{CGAL::greene_approx_convex_partition_2} \\
\ccRefIdfierPage{CGAL::optimal_convex_partition_2} \\
\ccRefIdfierPage{CGAL::partition_is_valid_2} \\
\ccRefIdfierPage{CGAL::y_monotone_partition_2} \\
\ccRefIdfierPage{CGAL::y_monotone_partition_is_valid_2}

\end{ccRefFunctionObjectConcept}
\renewcommand\ccRefPageBegin{\ccParDims\cgalColumnLayout}
\renewcommand\ccRefPageEnd{\ccParDims\cgalColumnLayout}


% +------------------------------------------------------------------------+
%%RefPage: end of main body, begin of footer
% EOF
% +------------------------------------------------------------------------+

}

\subsection*{Classes}

\lcTex{
\ccRefIdfierPage{CGAL::Partition_is_valid_traits_2<Traits, PolygonIsValid>}\\
\ccRefIdfierPage{CGAL::Partition_traits_2<R>} \\
}

\lcHtml{
% +------------------------------------------------------------------------+
% | Reference manual page: Partition_is_valid_traits_2.tex
% +------------------------------------------------------------------------+
% | 27.07.2000   Susan Hert
% | Package: Partition_2
% | 
% +------------------------------------------------------------------------+

\def\ccRefPageBegin{\ccParDims\cgalColumnLayout\begin{ccAdvanced}}
\def\ccRefPageEnd{\ccParDims\cgalColumnLayout\end{ccAdvanced}}
\begin{ccRefClass}{Partition_is_valid_traits_2<Traits, PolygonIsValid>}  
\ccIndexSubitemBegin[C]{partition_is_valid_2}{traits class}

\ccDefinition
  
Class that derives a traits class for \ccc{partition_is_valid_2} from
a given traits class by defining the validity testing function object
in terms of a supplied template parameter.

\ccInclude{CGAL/Partition_is_valid_traits_2.h}

\ccInheritsFrom

\ccc{Traits}

\ccIsModel

PartitionIsValidTraits\_2\ccIndexSubitem[c]{PartitionIsValidTraits_2}{model}

\ccTypes
\ccSetThreeColumns{typedef Traits::Orientation_2xxxx}{Orientation_2xxxx}{}
\ccAutoIndexingOff
\ccTypedef{typedef PolygonIsValid          Is_valid;}{}
\ccGlue
\ccTypedef{typedef Traits::Point_2         Point_2;}{}
\ccGlue
\ccTypedef{typedef Traits::Polygon_2       Polygon_2;}{}
\ccGlue
\ccTypedef{typedef Traits::Less_xy_2       Less_xy_2;}{}
\ccGlue
\ccTypedef{typedef Traits::Leftturn_2      Leftturn_2;}{}
\ccGlue
\ccTypedef{typedef Traits::Orientation_2   Orientation_2;}{}
\ccGlue
\ccTypedef{typedef Traits::Less_xy         Less_xy;}{}
\ccGlue
\ccTypedef{typedef Traits::Vector_2        Vector_2;}{}
\ccGlue
\ccTypedef{typedef Traits::FT              FT;}{}

\ccAutoIndexingOn

\ccCreationVariable{traits}
\ccOperations

\ccMethod{
   Is_valid
   is_valid_object(const Traits& traits) const;
}{function returning an instance of \ccc{Is_valid}}

\ccSeeAlso

\ccRefIdfierPage{CGAL::Is_convex_2<Traits>} \\
\ccRefIdfierPage{CGAL::Is_vacuously_valid<Traits>} \\
\ccRefIdfierPage{CGAL::Is_y_monotone_2<Traits>} \\
\ccRefIdfierPage{CGAL::Partition_traits_2<R>}

\ccExample

The following code fragment will compute an optimal
convex partitioning of the simple polygon \ccc{P}
and store the partition polygons in the list \ccc{partition_polys}.
It then asserts that the partition produced is valid.  The
traits class used for testing the validity is derived from the
traits class used to produce the partition with the function object
class \ccc{CGAL::Is_convex_2}\ccIndexMainItem[C]{Is_convex_2} used 
to define the required \ccc{Is_valid} type.  (Note that
this assertion is superfluous unless postcondition checking for
\ccc{optimal_convex_partition_2} has been turned off during compilation.)

\begin{verbatim}
   #include <CGAL/basic.h>
   #include <CGAL/Cartesian.h>
   #include <CGAL/Partition_traits_2.h>
   #include <CGAL/Partition_is_valid_traits_2.h>
   #include <CGAL/polygon_function_objects.h>
   #include <CGAL/partition_2.h>
   #include <list>

   typedef CGAL::Cartesian<double>                           R;
   typedef CGAL::Partition_traits_2<R>                       Traits;
   typedef CGAL::Is_convex_2<Traits>                         Is_convex_2;
   typedef Traits::Polygon_2                                 Polygon_2;
   typedef Traits::Point_2                                   Point_2;
   typedef Polygon_2::Vertex_const_iterator                  Vertex_iterator;
   typedef std::list<Polygon_2>                              Polygon_list;
   typedef CGAL::Partition_is_valid_traits_2<Traits, Is_convex_2>
                                                             Validity_traits;
   Polygon_2             P;
   Polygon_list          partition_polys;
   Traits                partition_traits;
   Validity_traits       validity_traits;

   // ...
   // insert vertices into P to create a simle, CCW-oriented polygon
   // ...
   CGAL::optimal_convex_partition_2(P.vertices_begin(),
                                    P.vertices_end(),
                                    std::back_inserter(partition_polys),
                                    partition_traits);
   assert(partition_is_valid_2(P.vertices_begin(), P.vertices_end(),
                               partition_polys.begin(), partition_polys.end(),
                               validity_traits));
\end{verbatim}


\ccIndexSubitemEnd[C]{partition_is_valid_2}{traits class}
\end{ccRefClass}
\def\ccRefPageBegin{\ccParDims\cgalColumnLayout}
\def\ccRefPageEnd{\ccParDims\cgalColumnLayout}

% +------------------------------------------------------------------------+
%%RefPage: end of main body, begin of footer
% EOF
% +------------------------------------------------------------------------+


% +------------------------------------------------------------------------+
% | Reference manual page: Partition_traits_2.tex
% +------------------------------------------------------------------------+
% | 10.05.2000   Susan Hert
% | Package: Partition_2
% | 
% |
% +------------------------------------------------------------------------+


\begin{ccRefClass}{Partition_traits_2<R>}  %% add template arg's if necessary
\ccIndexTraitsClassDefault{approx_convex_partition_2}
\ccIndexTraitsClassDefault{greene_approx_convex_partition_2}
\ccIndexTraitsClassDefault{optimal_convex_partition_2}
\ccIndexTraitsClassDefault{y_monotone_partition_2}

\ccDefinition
  
Traits class that can be used with all the
2-dimensional polygon partitioning algorithms.  It is parameterized by
a representation class \ccc{R}.

\ccInclude{CGAL/Partition_traits_2.h}

\ccIsModel

ConvexPartitionIsValidTraits\_2,\ccIndexSubitem[c]{ConvexPartitionIsValidTraits_2}{model}
IsYMonotoneTraits\_2,\ccIndexSubitem[c]{ConvexPartitionIsValidTraits_2}{model}
OptimalConvexPartitionTraits\_2,\ccIndexSubitem[c]{OptimalConvexPartitionTraits_2}{model}
PartitionTraits\_2,\ccIndexSubitem[c]{PartitionTraits_2}{model}
YMonotonePartitionIsValidTraits\_2,\ccIndexSubitem[c]{YMonotonePartitionIsValidTraits_2}{model}
YMonotonePartitionTraits\_2\ccIndexSubitem[c]{YMonotonePartitionTraits_2}{model}

\ccTypes

\ccSetThreeColumns{typedef CGAL::Polygon_2<Poly_Traits, Container>}{Polygon_2;}{}
\ccAutoIndexingOff
\ccTypedef{typedef R::Line_2                      Line_2;}{}
\ccGlue
\ccTypedef{typedef R::Segment_2                   Segment_2;}{}
\ccGlue
\ccTypedef{typedef R::Ray_2                       Ray_2;}{}
\ccGlue
\ccTypedef{typedef R::Less_yx_2                   Less_yx_2;}{}
\ccGlue
\ccTypedef{typedef R::Less_xy_2                   Less_xy_2;}{}
\ccGlue
\ccTypedef{typedef R::Leftturn_2                  Leftturn_2;}{}
\ccGlue
\ccTypedef{typedef R::Orientation_2               Orientation_2;}{}
\ccGlue
\ccTypedef{typedef R::Compare_y_2                 Compare_y_2;}{}
\ccGlue
\ccTypedef{typedef R::Compare_x_2                 Compare_x_2;}{}
\ccGlue
\ccTypedef{typedef R::Construct_line_2            Construct_line_2;}{}
\ccGlue
\ccTypedef{typedef R::Construct_ray_2             Construct_ray_2;}{}
\ccGlue
\ccTypedef{typedef R::Construct_segment_2         Construct_segment_2;}{}
\ccGlue
\ccTypedef{typedef R::Collinear_are_ordered_along_line_2  Collinear_are_ordered_along_line_2;}{}
\ccGlue
\ccTypedef{typedef R::Are_strictly_ordered_along_line_2  Are_strictly_ordered_along_line_2;}{}
\ccGlue
\ccTypedef{typedef CGAL::Polygon_traits_2<R>      Poly_Traits;}{}
\ccGlue
\ccTypedef{typedef Poly_Traits::Point_2           Point_2;}{}
\ccGlue
\ccTypedef{typedef std::list<Point_2>                      Container;}{}
\ccGlue
\ccTypedef{typedef CGAL::Polygon_2<Poly_Traits, Container> Polygon_2;}{}
\ccGlue
\ccTypedef{typedef R::Less_xy_2                Less_xy;}{}
\ccGlue
\ccTypedef{typedef Poly_Traits::Vector_2    Vector_2;}{}
\ccGlue
\ccTypedef{typedef R::FT                    FT;}{}
\ccGlue
\ccTypedef{typedef Partition_traits_2<R>  Self;}{}
\ccGlue
\ccTypedef{typedef CGAL::Is_convex_2<Self>          Is_convex_2;}{}
\ccGlue
\ccTypedef{typedef CGAL::Is_y_monotone_2<Self>      Is_y_monotone_2;}{}
\ccAutoIndexingOn


\ccCreation
\ccCreationVariable{traits}  %% choose variable name

A default constructor and copy constructor are defined.

\ccOperations

For each predicate object type \ccc{Pred_object_type} listed above
({\em i.e.}, \ccc{Less_yx_2}, \ccc{Less_xy_2}, \ccc{Leftturn_2},
\ccc{Orientation_2}, \ccc{Compare_y_2}, \ccc{Compare_x_2},
\ccc{Construct_line_2}, \ccc{Construct_ray_2}, \ccc{Construct_segment_2},
\ccc{Collinear_are_ordered_along_line_2},
\ccc{Are_strictly_ordered_along_line_2}, \ccc{Is_convex_2},
\ccc{Is_y_monotone_2}) there is a
corresponding function of the following form defined:

\ccSetThreeColumns{Comarison_resultxx}{traits.compare_x( Point_2 p, Point_2 q)}{
}
\ccAutoIndexingOff
\ccMemberFunction{
Pred_object_type pred_object_type_object();
}
{
Returns an instance of \ccc{Pred_object_type}.
}
\ccAutoIndexingOn

In addition, the following functions are defined:

\ccMemberFunction{
    Comparison_result compare_x(const Point_2 &p, const Point_2 &q) const;
}
{
  Returns \ccc{Compare_x_2()(p, q)}.
}

\ccMemberFunction{
    Comparison_result compare_y(const Point_2 &p, const Point_2 &q) const;
}
{
  Returns \ccc{Compare_y_2()(p, q)}.
}

\ccSetThreeColumns{Comarison_resultxx}{traits.cross_product_2xx}{}
\ccMemberFunction{FT
    cross_product_2(const Vector_2& p, const Vector_2& q) const;
}{ Returns \ccc{p.x() * q.y() - q.x() * p.y()}.}

\ccMemberFunction{
    bool do_intersect(const Point_2& p1,
                      const Point_2& q1,
                      const Point_2& p2,
                      const Point_2& q2) const;
}
{
  Returns \ccc{do_intersect(Segment_2(p1,q1), Segment_2(p2,q2)}.
}

\ccMemberFunction{
    bool have_equal_direction(const Vector_2& v1,
                              const Vector_2& v2 ) const;
}
{
  Returns the value of the comparison \ccc{R::Direction_2(v1) == R::Direction_2(v2)}.
}

\ccMemberFunction{
    Orientation orientation(const Point_2& p,
                                 const Point_2& q,
                                 const Point_2& r) const;
}
{
   Returns \ccc{Orientation_2()(p,q,r)}.
}

\ccMemberFunction{
    bool lexicographically_xy_smaller(const Point_2& p, const Point_2& q) const;
}
{
Returns \ccc{Less_xy_2()(p,q)}.
}

\ccSeeAlso

\ccc{approx_convex_partition_2},
\ccc{convex_partition_is_valid_2},
\ccc{greene_approx_convex_partition_2},
\ccc{optimal_convex_partition_2},
\ccc{partition_is_valid_2},
\ccc{Partition_is_valid_traits_2},
\ccc{y_monotone_partition_2},
\ccc{y_monotone_partition_is_valid_2}


\end{ccRefClass}

% +------------------------------------------------------------------------+
%%RefPage: end of main body, begin of footer
% EOF
% +------------------------------------------------------------------------+


}

\subsection*{Function Object Classes}

\lcTex{
\ccRefIdfierPage{CGAL::Is_convex_2<Traits>}\\
\ccRefIdfierPage{CGAL::Is_vacuously_valid<Traits>}\\
\ccRefIdfierPage{CGAL::Is_y_monotone_2<Traits>} \\
}

\lcHtml{
% +------------------------------------------------------------------------+
% | Reference manual page: Is_convex_2.tex
% +------------------------------------------------------------------------+
% | 26.07.2000   Susan Hert
% | Package: Partition_2
% | 
% +------------------------------------------------------------------------+


\begin{ccRefFunctionObjectClass}{Is_convex_2<Traits>} 

\ccDefinition
  
Function object class for testing if a sequence of points represents
a convex polygon or not.
\ccIndexSubitem{convex polygon}{function object}

\ccInclude{CGAL/polygon_function_objects.h}

\ccIsModel

\ccRefConceptPage{PolygonIsValid}%
\ccIndexSubitem[c]{PolygonIsValid}{model}

\ccCreation
\ccCreationVariable{f}  %% choose variable name

\ccConstructor{Is_convex_2(const Traits& t);}{\ccc{Traits} satisfies the
requirements of the function \ccc{is_convex_2}}

\ccOperations

\ccMethod{
template<class InputIterator>
bool operator()(InputIterator first, InputIterator beyond);}
{
  returns \ccc{true} iff the points of type \ccc{Triats::Point_2}
  in the range [\ccc{first},\ccc{beyond}) define a convex polygon. 
}

\ccSeeAlso

\ccRefIdfierPage{CGAL::convex_partition_is_valid_2} \\
\ccRefIdfierPage{CGAL::Partition_is_valid_traits_2<Traits, PolygonIsValid>}

\ccImplementation

This test requires $O(n)$ time for a polygon with $n$ vertices.

\end{ccRefFunctionObjectClass}

% +------------------------------------------------------------------------+
%%RefPage: end of main body, begin of footer
% EOF
% +------------------------------------------------------------------------+


% +------------------------------------------------------------------------+
% | Reference manual page: Is_vacuously_valid.tex
% +------------------------------------------------------------------------+
% | 26.07.2000   Susan Hert
% | Package: Partition_2
% | 
% +------------------------------------------------------------------------+


\begin{ccRefFunctionObjectClass}{Is_vacuously_valid<Traits>}  


\ccDefinition
  
Function object class that indicates all sequences of points are valid.
\ccIndexSubitem{polygon partitioning}{valid}
\ccIndexSubitem{polygon}{valid}

\ccInclude{CGAL/polygon_function_objects.h}

\ccIsModel

\ccRefConceptPage{PolygonIsValid}%
\ccIndexSubitem[c]{PolygonIsValid}{model}

\ccCreation
\ccCreationVariable{f}  %% choose variable name

\ccConstructor{Is_vacuously_valid(const Traits& t);}{}
\ccThree{boolxxx}{operator()(InputIterator first, InputIterator beyond)xxx}{}
\ccOperations

\ccMethod{
template<class InputIterator>
bool operator()(InputIterator first, InputIterator beyond);}
{
  returns \ccc{true}.
}

\ccSeeAlso

\ccRefIdfierPage{CGAL::partition_is_valid_2} \\
\ccRefIdfierPage{CGAL::Partition_is_valid_traits_2<Traits, PolygonIsValid>}

\ccImplementation

This test requires $O(1)$ time.

\end{ccRefFunctionObjectClass}

% +------------------------------------------------------------------------+
%%RefPage: end of main body, begin of footer
% EOF
% +------------------------------------------------------------------------+


% +------------------------------------------------------------------------+
% | Reference manual page: Is_y_monotone_2.tex
% +------------------------------------------------------------------------+
% | 26.07.2000   Susan Hert
% | Package: Partition_2
% | 
% +------------------------------------------------------------------------+


\begin{ccRefFunctionObjectClass}{Is_y_monotone_2<Traits>} 

\ccDefinition
  
Function object class that tests whether a sequence of points represents
a $y$-monotone polygon or not.
\ccIndexSubitem{y-monotone polygon}{function object}

\ccInclude{CGAL/polygon_function_objects.h}

\ccIsModel

\ccRefConceptPage{PolygonIsValid}%
\ccIndexSubitem[c]{PolygonIsValid}{model}

\ccCreation
\ccCreationVariable{f}  %% choose variable name

\ccConstructor{Is_y_monotone_2(const Traits& t);}{\ccc{Traits} is a model of
the concept IsYMonotoneTraits\_2}

\ccOperations

\ccMethod{
template<class InputIterator>
bool operator()(InputIterator first, InputIterator beyond);}
{
  returns \ccc{true} iff the points of type \ccc{Traits::Point_2} in the 
  range [\ccc{first},\ccc{beyond}) define a $y$-monotone polygon.
}

\ccSeeAlso

\ccRefIdfierPage{CGAL::convex_partition_is_valid_2} \\
\ccRefIdfierPage{CGAL::Partition_is_valid_traits_2<Traits, PolygonIsValid>}

\ccImplementation

This test requires $O(n)$ time for a polygon with $n$ vertices.

\end{ccRefFunctionObjectClass}

% +------------------------------------------------------------------------+
%%RefPage: end of main body, begin of footer
% EOF
% +------------------------------------------------------------------------+


}

\subsection*{Functions}

\lcTex{
\ccRefIdfierPage{CGAL::approx_convex_partition_2} \\
\ccRefIdfierPage{CGAL::convex_partition_is_valid_2} \\
\ccRefIdfierPage{CGAL::greene_approx_convex_partition_2} \\
\ccRefIdfierPage{CGAL::is_y_monotone_2} \\
\ccRefIdfierPage{CGAL::optimal_convex_partition_2} \\
\ccRefIdfierPage{CGAL::partition_is_valid_2} \\
\ccRefIdfierPage{CGAL::y_monotone_partition_2} \\
\ccRefIdfierPage{CGAL::y_monotone_partition_is_valid_2} \\
}

\lcHtml{
\begin{ccRefFunction}{approx_convex_partition_2}

\ccDefinition
Function that produces a set of 
convex polygons that represent a partitioning of a polygon defined
on a sequence of points.  The number of convex polygons produced is 
no more than four times the minimal number.%
\ccIndexSubsubitem{polygon partitioning}{convex}{approximately optimal}


\ccInclude{CGAL/partition_2.h}

\ccFunction{
template <class InputIterator, class OutputIterator, class Traits>
OutputIterator approx_convex_partition_2(InputIterator first,
                                         InputIterator beyond,
                                         OutputIterator result,
                                         const Traits& traits = Default_traits);
}
{
computes a partition of the polygon defined 
by the points in the range [\ccc{first}, \ccc{beyond}) into convex 
polygons. The counterclockwise-oriented partition polygons are written to
the sequence starting at position \ccc{result}.  The past-the-end iterator for 
the resulting sequence of polygons is returned.
\ccPrecond The points in the range [\ccc{first}, \ccc{beyond}) define a simple 
counterclockwise-oriented polygon.
%\ccIndexSubitem[C]{approx_convex_partition_2}{preconditions}
}

\ccHeading{Requirements}
\begin{enumerate}
    \item \ccc{Traits} is a model of the concept 
          PartitionTraits\_2\ccIndexMainItem[c]{PartitionTraits_2}
          and, for the purposes of checking the postcondition that the partition
          produced is valid, it should also be a model of
          the concept ConvexPartitionIsValidTraits\_2%
          \ccIndexMainItem[c]{ConvexPartitionIsValidTraits_2}.
    \item \ccc{OutputIterator::value_type} should be \ccc{Traits::Polygon_2}.
    \item \ccc{InputIterator::value_type} should be \ccc{Traits::Point_2},
          which should also be the type of the points stored in an object
          of type \ccc{Traits::Polygon_2}.
    \item Points in the range $[first, beyond)$ must define a simple polygon
          whose vertices are oriented counterclockwise.
\end{enumerate}

The default traits class \ccc{Default_traits} is \ccc{Partition_traits_2},
%\ccIndexTraitsClassDefault{approx_convex_partition_2} 
with the representation type determined by \ccc{InputIterator1::value_type}.

\ccSeeAlso

\ccRefIdfierPage{CGAL::convex_partition_is_valid_2} \\
\ccRefIdfierPage{CGAL::greene_approx_convex_partition_2} \\
\ccRefIdfierPage{CGAL::optimal_convex_partition_2} \\
\ccRefIdfierPage{CGAL::partition_is_valid_2} \\
\ccRefIdfierPage{CGAL::Partition_is_valid_traits_2<Traits, PolygonIsValid>} \\
\ccRefIdfierPage{CGAL::y_monotone_partition_2}

\ccImplementation
This function implements the algorithm of Hertel and Mehlhorn
\cite{hm-ftsp-83} and is based on the class 
\ccc{CGAL::Constrained_triangulation_2}.  Given a triangulation of
the polygon, the function requires $O(n)$ time and
space for a polygon with $n$ vertices.

\ccExample

The following program computes an approximately optimal
convex partitioning of a polygon using the default
traits class and stores the partition polygons in the list 
\ccc{partition_polys}.

\ccIncludeExampleCode{Partition_2/approx_convex_ex.C}

\end{ccRefFunction}

\renewcommand\ccRefPageBegin{\ccParDims\cgalColumnLayout\begin{ccAdvanced}}
\renewcommand\ccRefPageEnd{\ccParDims\cgalColumnLayout\end{ccAdvanced}}
\begin{ccRefFunction}{convex_partition_is_valid_2}

\ccDefinition
Function that determines if a given set of polygons represents
a valid convex partitioning for a given sequence of points that represent a
simple, counterclockwise-oriented polygon.  
A convex partition is valid if the 
polygons do not overlap, the union of the polygons is the same as the original
polygon given by the sequence of points, and if each partition polygon is 
convex. 
\ccIndexSubsubitem{polygon partitioning}{convex}{valid}

\ccInclude{CGAL/partition_is_valid_2.h}

\ccFunction{
template<class InputIterator, class ForwardIterator, class Traits>
bool
convex_partition_is_valid_2 (InputIterator point_first, 
                             InputIterator point_last,
                             ForwardIterator poly_first, 
                             ForwardIterator poly_last,
                             const Traits& traits = Default_traits);
}
{
determines if the polygons in the range [\ccc{poly_first}, \ccc{poly_last})
define a valid convex partition of the polygon defined by the points in the 
range [\ccc{point_first}, \ccc{point_last}). 
The function returns \ccc{true} iff the partition is valid and otherwise
returns \ccc{false}.
}

\ccHeading{Preconditions}
\ccIndexSubitem[C]{convex_partition_is_valid_2}{preconditions}
\begin{enumerate}
    \item The points in the range [\ccc{point_first}, \ccc{point_last}) define
          a simple, counterclockwise-oriented polygon.
    \item \ccc{Traits} is a model of the concept
          ConvexPartitionIsValidTraits\_2%
          \ccIndexMainItem[c]{ConvexPartitionIsValidTraits_2}.
    \item \ccc{InputIterator::value_type} should be \ccc{Traits::Point_2},
          which should also be the type of the points stored in an object
          of type \ccc{Traits::Polygon_2}.
    \item \ccc{ForwardIterator::value_type} should be 
          \ccc{Traits::Polygon_2}.
\end{enumerate}

The default traits class \ccc{Default_traits} is \ccc{Partition_traits_2},%
\ccIndexTraitsClassDefault{convex_partition_is_valid_2} 
with the representation type determined by \ccc{InputIterator::value_type}.

\ccSeeAlso

\ccRefIdfierPage{CGAL::approx_convex_partition_2} \\
\ccRefIdfierPage{CGAL::greene_approx_convex_partition_2} \\
\ccRefIdfierPage{CGAL::optimal_convex_partition_2}\\
\ccRefIdfierPage{CGAL::partition_is_valid_2} \\
\ccc{CGAL::is_convex_2}

\ccExample

The following code fragment will compute an optimal
convex partitioning of the simple polygon \ccc{P} 
and store the partition polygons in the list \ccc{partition_polys}.
It then asserts that the partition produced is valid. 
(Note that this assertion is superfluous unless the postcondition checking
done in \ccc{optimal_convex_partition_2} has been turned off during
compilation.)

\begin{verbatim}
   #include <CGAL/basic.h>
   #include <CGAL/Cartesian.h>
   #include <CGAL/Partition_traits_2.h>
   #include <CGAL/partition_2.h>
   #include <list>

   typedef CGAL::Cartesian<double>                           R;
   typedef CGAL::Partition_traits_2<R>                       Traits;
   typedef Traits::Polygon_2                                 Polygon_2;
   typedef Polygon_2::Vertex_const_iterator                  Vertex_iterator;

   Polygon_2            P;
   std::list<Polygon_2> partition_polys;
   Traits               partition_traits;

   // ...
   // insert vertices into P to create a simple CCW-oriented polygon
   // ...
   CGAL::optimal_convex_partition_2(P.vertices_begin(),
                                   P.vertices_end(),
                                   std::back_inserter(partition_polys),
                                   partition_traits);
   assert(convex_partition_is_valid_2(P.vertices_begin(), P.vertices_end(),
                                      partition_polys.begin(), 
                                      partition_polys.end(),
                                      partition_traits));
\end{verbatim}

\end{ccRefFunction}
\renewcommand\ccRefPageBegin{\ccParDims\cgalColumnLayout}
\renewcommand\ccRefPageEnd{\ccParDims\cgalColumnLayout}

\begin{ccRefFunction}{greene_approx_convex_partition_2}

\ccDefinition
Function that produces a set of 
convex polygons that represent a partitioning of a polygon defined
on a sequence of points using the approximation algorithm of 
Greene \cite{g-dpcp-83}.  The number of convex polygons produced is 
no more than four times the minimal number.%
\ccIndexSubsubitem{polygon partitioning}{convex}{approximately optimal}

\ccInclude{CGAL/partition_2.h}

\ccFunction{
template <class InputIterator, class OutputIterator, class Traits>
OutputIterator greene_approx_convex_partition_2(InputIterator first,
                                                InputIterator beyond,
                                                OutputIterator result,
                                                const Traits& traits = Default_traits);
}
{
computes a partition of the simple, counterclockwise-oriented polygon defined 
by the points in the range [\ccc{first}, \ccc{last}) into convex 
polygons. The counterclockwise-oriented partition polygons are written to
the sequence starting at position \ccc{result}.  The past-the-end iterator for 
the resulting sequence of polygons is returned.
}

\ccHeading{Preconditions}
\ccIndexSubitem[C]{greene_approx_convex_partition_2}{preconditions}
\begin{enumerate}
    \item \ccc{Traits} is a model of the concepts PartitionTraits\_2%
          \ccIndexMainItem[c]{PartitionTraits_2} and 
          YMonotonePartitionTraits\_2% 
          \ccIndexMainItem[c]{YMonotonePartitionTraits_2}. 
          For the purpose of 
          checking the validity of the $y$-monotone partition produced as
          a preprocessing step for the convex partitioning, it must also 
          be a model of YMonotonePartitionIsValidTraits\_2
          \ccIndexMainItem[c]{YMonotonePartitionIsValidTraits_2}. 
          For the purpose of checking
          the postcondition that the convex partition is valid, \ccc{Traits}
          must also be a model of ConvexPartitionIsValidTraits\_2%
          \ccIndexMainItem[c]{ConvexPartitionIsValidTraits_2}. 
    \item \ccc{OutputIterator::value_type} should be \ccc{Traits::Polygon_2}.
    \item \ccc{InputIterator::value_type} should be \ccc{Traits::Point_2},
          which should also be the type of the points stored in an object
          of type \ccc{Traits::Polygon_2}.
    \item Points in the range $[first, beyond)$ must define a simple polygon
          whose vertices are oriented counterclockwise.
\end{enumerate}

The default traits class \ccc{Default_traits} is \ccc{Partition_traits_2}.
\ccIndexTraitsClassDefault{greene_approx_convex_partition_2}

\ccSeeAlso
\ccc{approx_convex_partition_2},
\ccc{convex_partition_is_valid_2},
\ccc{optimal_convex_partition_2},
\ccc{partition_is_valid_2},
\ccc{y_monotone_partition_2}

\ccExample
The following code fragment will compute an approximately optimal
convex partitioning of the simple polygon \ccc{P} using the default
traits class and store the partition polygons in the list 
\ccc{partition_polys}.

\begin{verbatim}
   typedef CGAL::Cartesian<double>                           R;
   typedef CGAL::Polygon_traits_2<R>                         Traits;
   typedef Traits::Point_2                                   Point_2;
   typedef std::list<Point_2>                                Container;
   typedef CGAL::Polygon_2<Traits, Container>                Polygon_2;

   Polygon_2            P;
   std::list<Polygon_2> partition_polys;

   // ...
   // insert vertices into P to create a simple CCW-oriented polygon
   // ...
   CGAL::greene_approx_convex_partition_2(P.vertices_begin(),
                                          P.vertices_end(),
                                          std::back_inserter(partition_polys));
\end{verbatim}


\end{ccRefFunction}

\begin{ccRefFunction}{is_y_monotone_2}

\ccDefinition

Function for testing the $y$-monotonicity of a sequence of points.
\ccIndexSubitem{polygon partitioning}{y-monotone}
\ccIndexMainItem{y-monotone polygon}

\ccInclude{CGAL/is_y_monotone_2.h}

\ccFunction{
template<class InputIterator, class Traits>
bool
is_y_monotone_2(InputIterator first, InputIterator beyond,
                const Traits& traits);
}
{
Determines if the sequence of points in the range 
[\ccc{first}, \ccc{beyond}) define a $y$-monotone 
polygon or not. If so, the function returns \ccc{true}, otherwise it
returns \ccc{false}. 
}

\ccHeading{Requirements}
\ccIndexSubitem[C]{is_y_monotone_2}{preconditions}
\begin{enumerate}
    \item \ccc{Traits} is a model of the concept IsYMonotoneTraits\_2.%
          \ccIndexMainItem[c]{IsYMonotoneTraits_2}
    \item \ccc{InputIterator::value_type} should be \ccc{Traits::Point_2}.
\end{enumerate}

The default traits class \ccc{Default_traits} is the kernel in which the
type \ccc{InputIterator::value_type} is defined.%
\ccIndexTraitsClassDefault{is_y_monotone_2}

\ccSeeAlso
\ccRefIdfierPage{CGAL::Is_y_monotone_2<Traits>} \\
\ccRefIdfierPage{CGAL::y_monotone_partition_2} \\
\ccRefIdfierPage{CGAL::y_monotone_partition_is_valid_2}

\ccImplementation

This function requires $O(n)$ time for a polygon with $n$ vertices.

\ccExample

The following program computes a $y$-monotone partitioning
of a polygon using the default
traits class and stores the partition polygons in the list 
\ccc{partition_polys}.  It then asserts that each of the partition 
polygons is, in fact, a $y$-monotone polygon and that the partition
is valid.  (Note that the
assertions are superfluous unless the postcondition checking done
by \ccc{y_monotone_partition_2} has been turned off during compilation.)

\ccIncludeExampleCode{Partition_2/y_monotone_ex.C}



\end{ccRefFunction}

\begin{ccRefFunction}{optimal_convex_partition_2}

Function that produces a set of convex polygons 
that represent a partitioning of a polygon defined on a sequence of 
points. 
The number of convex polygons produced is minimal.
\ccIndexSubsubitem{polygon partitioning}{convex}{optimal}

\ccInclude{CGAL/partition_2.h}

\ccFunction{
template <class InputIterator, class OutputIterator, class Traits>
OutputIterator optimal_convex_partition_2(InputIterator first,
                                          InputIterator beyond,
                                          OutputIterator result,
                                          const Traits& traits = Default_traits);
}
{
computes a partition of the polygon defined
by the points in the range [\ccc{first}, \ccc{beyond}) into convex
polygons. The counterclockwise-oriented partition polygons are written to
the sequence starting at position \ccc{result}.  The past-the-end iterator for
the resulting sequence of polygons is returned.
\ccPrecond The points in the range [\ccc{first}, \ccc{beyond}) define a
simple, counterclockwise-oriented polygon.
}

\ccHeading{Requirements}
%\ccIndexSubitem[C]{optimal_convex_partition_2}{preconditions}
\begin{enumerate}
    \item \ccc{Traits} is a model of the concept OptimalConvexPartitionTraits\_2%
          \ccIndexMainItem[c]{OptimalConvexPartitionTraits_2}.
          For the purposes of checking the
          postcondition that the partition is valid, \ccc{Traits} should
          also be a model of ConvexPartitionIsValidTraits\_2.
          \ccIndexMainItem[c]{ConvexPartitionIsValidTraits_2}
    \item \ccc{OutputIterator::value_type} should be
          \ccc{Traits::Polygon_2}.
    \item \ccc{InputIterator::value_type} should be \ccc{Traits::Point_2},
          which should also be the type of the points stored in an object
          of type \ccc{Traits::Polygon_2}.
\end{enumerate}

The default traits class \ccc{Default_traits} is \ccc{Partition_traits_2},
%\ccIndexTraitsClassDefault{optimal_convex_partition_2}
with the representation type determined by \ccc{InputIterator::value_type}.

\ccSeeAlso
\ccRefIdfierPage{CGAL::approx_convex_partition_2} \\
\ccRefIdfierPage{CGAL::convex_partition_is_valid_2} \\
\ccRefIdfierPage{CGAL::greene_approx_convex_partition_2} \\
\ccRefIdfierPage{CGAL::partition_is_valid_2} \\
\ccRefIdfierPage{CGAL::Partition_is_valid_traits_2<Traits, PolygonIsValid>}


\ccImplementation
This function implements the dynamic programming algorithm of Greene 
\cite{g-dpcp-83}, which requires $O(n^4)$ time and $O(n^3)$ space to
produce a partitioning of a polygon with $n$ vertices.  

\ccExample

The following program computes an optimal
convex partitioning of a polygon using the default
traits class and stores the partition polygons in the list 
\ccc{partition_polys}.  
It then asserts that the partition produced is valid.  The
traits class used for testing the validity is derived from the
traits class used to produce the partition with the function object
class \ccc{CGAL::Is_convex_2}\ccIndexMainItem[C]{Is_convex_2} used
to define the required \ccc{Is_valid} type. 
(Note that this assertion is superfluous unless the 
postcondition checking for \ccc{optimal_convex_partition_2} has been
turned off.)

\ccIncludeExampleCode{Partition_2/optimal_convex_ex.C}


\end{ccRefFunction}
 
\renewcommand\ccRefPageBegin{\ccParDims\cgalColumnLayout\begin{ccAdvanced}}
\renewcommand\ccRefPageEnd{\ccParDims\cgalColumnLayout\end{ccAdvanced}}
\begin{ccRefFunction}{partition_is_valid_2}

\ccDefinition

Function that determines if a given set of polygons represents
a valid partition for a given sequence of points that 
define a simple, counterclockwise-oriented polygon.  A valid partition is one in
which the polygons are nonoverlapping and the union of the polygons is the 
same as the original polygon.
\ccIndexSubitem{polygon partitioning}{valid}
\ccIndexSubsubitem{polygon partitioning}{convex}{valid}
\ccIndexSubsubitem{polygon partitioning}{y-monotone}{valid}

\ccInclude{CGAL/partition_is_valid_2.h}

\ccFunction{
template<class InputIterator, class ForwardIterator, class Traits>
bool
partition_is_valid_2 (InputIterator point_first, InputIterator point_last,
                      ForwardIterator poly_first, ForwardIterator poly_last,
                      const Traits& traits = Default_traits);
}
{
returns \ccc{true} iff the polygons in the range [\ccc{poly_first}, 
\ccc{poly_last}) define a valid partition of the polygon defined by the 
points in the range [\ccc{point_first}, \ccc{point_last}) and 
\ccc{false} otherwise.  
Each polygon must also satisfy the property 
tested by \ccc{Traits::Is_valid()}. 
}

\ccHeading{Preconditions}
\ccIndexSubitem[C]{partition_is_valid_2}{preconditions}
\begin{enumerate}
    \item Points in the range [\ccc{point_first}, \ccc{point_last}) define
          a simple, counterclockwise-oriented polygon.
    \item \ccc{Traits} is a model of the concept 
          \ccc{PartitionIsValidTraits_2}%
          \ccIndexMainItem[c]{PartitionIsValidTraits_2} and the
          concept defining the requirements for the validity test 
          implemented by \ccc{Traits::Is_valid()}.
    \item \ccc{InputIterator::value_type} should be \ccc{Traits::Point_2},
          which should also be the type of the points stored in an object
          of type \ccc{Traits::Polygon_2}.
    \item \ccc{ForwardIterator::value_type} should be 
          \ccc{Traits::Polygon_2}.
\end{enumerate}

The default traits class \ccc{Default_traits} is \ccc{Partition_traits_2},%
\ccIndexTraitsClassDefault{partition_is_valid_2}
with the representation type determined by \ccc{InputIterator::value_type}.

\ccSeeAlso

\ccRefIdfierPage{CGAL::approx_convex_partition_2} \\
\ccRefIdfierPage{CGAL::greene_approx_convex_partition_2} \\
\ccRefIdfierPage{CGAL::is_y_monotone_2} \\
\ccRefIdfierPage{CGAL::optimal_convex_partition_2} \\
\ccRefIdfierPage{CGAL::Partition_is_valid_traits_2<Traits, PolygonIsValid>} \\
\ccRefIdfierPage{CGAL::y_monotone_partition_2} \\
\ccc{CGAL::is_convex_2} 

\ccExample

The following code fragment will compute an optimal
convex partitioning of the simple polygon \ccc{P} 
and store the partition polygons in the list \ccc{partition_polys}.
It then asserts that the partition produced is valid.  (Note that
this assertion is superfluous unless postcondition checking for 
\ccc{optimal_convex_partition_2} has been turned off during compilation.)

\begin{verbatim}
   #include <CGAL/basic.h>
   #include <CGAL/Cartesian.h>
   #include <CGAL/Partition_traits_2.h>
   #include <CGAL/Partition_is_valid_traits_2.h>
   #include <CGAL/polygon_function_objects.h>
   #include <CGAL/partition_2.h>
   #include <list>

   typedef CGAL::Cartesian<double>                           R;
   typedef CGAL::Partition_traits_2<R>                       Traits;
   typedef CGAL::Is_convex_2<Traits>                         Is_convex_2;
   typedef Traits::Polygon_2                                 Polygon_2;
   typedef Traits::Point_2                                   Point_2;
   typedef Polygon_2::Vertex_const_iterator                  Vertex_iterator;
   typedef std::list<Polygon_2>                              Polygon_list;
   typedef CGAL::Partition_is_valid_traits_2<Traits, Is_convex_2>
                                                             Validity_traits;
   Polygon_2             P;
   Polygon_list          partition_polys;
   Traits                partition_traits;
   Validity_traits       validity_traits;

   // ...
   // insert vertices into P to create a simple, CCW-oriented polygon
   // ...
   CGAL::optimal_convex_partition_2(P.vertices_begin(),
                                    P.vertices_end(),
                                    std::back_inserter(partition_polys),
                                    partition_traits);
   assert(partition_is_valid_2(P.vertices_begin(), P.vertices_end(),
                               partition_polys.begin(), partition_polys.end(),
                               validity_traits));
\end{verbatim}

\end{ccRefFunction}
\renewcommand\ccRefPageBegin{\ccParDims\cgalColumnLayout}
\renewcommand\ccRefPageEnd{\ccParDims\cgalColumnLayout}
 
\begin{ccRefFunction}{y_monotone_partition_2}

\ccDefinition

Function that produces a set of $y$-monotone polygons that 
represent a partitioning of a polygon defined on a sequence of points.
\ccIndexSubitem{polygon partitioning}{y-monotone}

\ccInclude{CGAL/partition_2.h}

\ccFunction{
template <class InputIterator, class OutputIterator, class Traits>
OutputIterator y_monotone_partition_2(InputIterator first, 
                                      InputIterator beyond,
                                      OutputIterator result, 
                                      const Traits& traits = Default_traits);
}
{
computes a partition of the simple, counterclockwise-oriented polygon defined 
by the points in the range [\ccc{first}, \ccc{last}) into $y$-monotone 
polygons. The counterclockwise-oriented partition polygons are written to
the sequence starting at position \ccc{result}.  The past-the-end iterator for 
the resulting sequence of polygons is returned.
}

\ccHeading{Preconditions}
\ccIndexSubitem[C]{y_monotone_partition_2}{preconditions}
\begin{enumerate}
    \item \ccc{Traits} is a model of the concept YMonotonePartitionTraits\_2.%
          \ccIndexMainItem[c]{YMonotonePartitionTraits_2} and, for the purposes
          of checking the postcondition that the partitionis valid, it should 
          also be a model of YMonotonePartitionIsValidTraits\_2%
          \ccIndexMainItem[c]{YMonotonePartitionIsValidTraits_2}.
    \item \ccc{OutputIterator::value_type} should be 
          \ccc{Traits::Polygon_2}.
    \item \ccc{InputIterator::value_type} should be \ccc{Traits::Point_2},
          which should also be the type of the points stored in an object
          of type \ccc{Traits::Polygon_2}.
    \item Points in the range $[first, beyond)$ must define a simple polygon
          whose vertices are oriented counterclockwise.
\end{enumerate}

The default traits class \ccc{Default_traits} is \ccc{Partition_traits_2}.
\ccIndexTraitsClassDefault{y_monotone_partition_2}

\ccSeeAlso
\ccc{approx_convex_partition_2},
\ccc{greene_approx_convex_partition_2},
\ccc{optimal_convex_partition_2},
\ccc{partition_is_valid_2},
\ccc{y_monotone_partition_is_valid_2}

\ccExample


The following code fragment will compute a $y$-monotone partitioning
of the simple polygon \ccc{P} using the default
traits class and store the partition polygons in the list 
\ccc{partition_polys}.

\begin{verbatim}
   typedef CGAL::Cartesian<double>                           R;
   typedef CGAL::Polygon_traits_2<R>                         Traits;
   typedef Traits::Point_2                                   Point_2;
   typedef std::list<Point_2>                                Container;
   typedef CGAL::Polygon_2<Traits, Container>                Polygon_2;

   Polygon_2            P;
   std::list<Polygon_2> partition_polys;

   // ...
   // insert vertices into P to create a simple CCW-oriented polygon
   // ...
   CGAL::y_monotone_partition_2(P.vertices_begin(),
                                P.vertices_end(),
                                std::back_inserter(partition_polys));
\end{verbatim}



\end{ccRefFunction}
 
\renewcommand\ccRefPageBegin{\ccParDims\cgalColumnLayout\begin{ccAdvanced}}
\renewcommand\ccRefPageEnd{\ccParDims\cgalColumnLayout\end{ccAdvanced}}
\begin{ccRefFunction}{y_monotone_partition_is_valid_2}

\ccDefinition
Function that determines if a given set of polygons represents
a valid $y$-monotone partitioning for a given sequence of points that 
define a simple, counterclockwise-oriented polygon.  
A valid partition is one in
which the polygons are nonoverlapping and the union of the polygons is the 
same as the original polygon and each polygon is $y$-monotone
\ccIndexSubsubitem{polygon partitioning}{y-monotone}{valid}

\ccInclude{CGAL/partition_is_valid_2.h}

\ccFunction{
template<class InputIterator, class ForwardIterator, class Traits>
bool
y_monotone_partition_is_valid_2 (InputIterator point_first, 
                                 InputIterator point_beyond,
                                 ForwardIterator poly_first, 
                                 ForwardIterator poly_beyond,
                                 const Traits& traits = Default_traits);
}
{
determines if the polygons in the range [\ccc{poly_first}, \ccc{poly_beyond})
define a valid $y$-monotone partition of the polygon represented by the points 
in the range [\ccc{point_first}, \ccc{point_beyond}). 
The function returns \ccc{true} iff the partition is valid and otherwise
returns false.
\ccPrecond{Points in the range [\ccc{point_first}, \ccc{point_beyond}) define
a simple, counterclockwise-oriented polygon.}
%\ccIndexSubitem[C]{y_monotone_partition_is_valid_2}{preconditions}
}

\ccHeading{Requirements}
\begin{enumerate}
    \item \ccc{Traits} is a model of the concept 
          \ccc{YMonotonePartitionIsValidTraits_2}%
          \ccIndexMainItem[c]{YMonotonePartitionIsValidTraits_2}.
    \item \ccc{InputIterator::value_type} should be \ccc{Traits::Point_2},
          which should also be the type of the points stored in an object
          of type \ccc{Traits::Polygon_2}.
    \item \ccc{ForwardIterator::value_type} should be 
          \ccc{Traits::Polygon_2}.
\end{enumerate}

The default traits class \ccc{Default_traits} is \ccc{Partition_traits_2},%
\ccIndexTraitsClassDefault{y_monotone_partition_is_valid_2}
with the representation type determined by \ccc{InputIterator::value_type}.

\ccSeeAlso

\ccRefIdfierPage{CGAL::y_monotone_partition_2} \\
\ccRefIdfierPage{CGAL::is_y_monotone_2} \\
\ccRefIdfierPage{CGAL::partition_is_valid_2} \\
\ccRefIdfierPage{CGAL::Partition_is_valid_traits_2<Traits, PolygonIsValid>}

\ccImplementation

This function uses the function \ccc{partition_is_valid_2} together with
the function object \ccc{Is_y_monotone_2} to determine if each polygon
is $y$-monotone or not. Thus the time required is $O(n \log n + e \log e)$ 
where $n$ is the total number of vertices of the partition polygons and
$e$ is the total number of edges.

\ccExample

See the example presented with the function \ccc{y_monotone_partition_2}
for an illustration of the use of this function.


\end{ccRefFunction}
\renewcommand\ccRefPageBegin{\ccParDims\cgalColumnLayout}
\renewcommand\ccRefPageEnd{\ccParDims\cgalColumnLayout}

}

\ccPrintSortedListOfRefpages

\clearpage
%% EOF %%
