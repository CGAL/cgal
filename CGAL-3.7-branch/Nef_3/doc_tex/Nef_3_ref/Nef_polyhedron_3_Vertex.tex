% +------------------------------------------------------------------------+
% | Reference manual page: Halfedge.tex
% +------------------------------------------------------------------------+
% | 14.05.2004   Peter Hachenberger
% | Package: Nef_3
% | 
\RCSdef{\RCSVertexRev}{$Id$}
\RCSdefDate{\RCSVertexDate}{$Date$}
% +------------------------------------------------------------------------+

\ccRefPageBegin

%%RefPage: end of header, begin of main body
% +------------------------------------------------------------------------+


\begin{ccRefClass}[Nef_polyhedron_3<Traits>::]{Vertex}

\ccDefinition

A vertex is a point in the 3-dimensional space. Its incidence
structure can be accessed creating a sphere map of the vertex. 
This is done by the member function \ccc{get_sphere_map} of
the class \ccc{Nef_polyhedron_3}.

\ccInclude{CGAL/Nef_polyhedron_3.h}

\ccTypes
\ccThree{Vertex_const_handle}{h.halfedge() const;;}{}
\ccThreeToTwo

The following types are the same as in \ccc{Nef_polyhedron_3<Traits>}.

\ccNestedType{Mark}{type of mark.}
\ccNestedType{Point_3}{point type stored in Vertex.}

\ccCreation
\ccCreationVariable{v}

There is no need for a user to create a \ccc{Vertex} explicitly. The
class \ccc{Nef_polyhedron_3<Traits>} manages the needed vertices internally.

%\ccConstructor{Vertex();}{default constructor.}

%\ccConstructor{Vertex(Mark m);}
%{creates Vertex with an initial value for its mark.}

\ccOperations

\ccMethod{const Mark& mark() const;}{the mark of \ccVar\ .}

\ccMethod{const Point_3& point() const;}{the point of \ccVar\ .}

%\ccSeeAlso

\ccRefIdfierPage{CGAL::Nef_polyhedron_3<Traits>}\\
\ccRefIdfierPage{CGAL::Nef_polyhedron_S2<Traits>}

\ccTagDefaults
\end{ccRefClass}

% +------------------------------------------------------------------------+
%%RefPage: end of main body, begin of footer
\ccRefPageEnd
% EOF
% +------------------------------------------------------------------------+
