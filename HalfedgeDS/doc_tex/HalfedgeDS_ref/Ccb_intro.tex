% +------------------------------------------------------------------------+
% | Reference manual page: HalfedgeDS.tex
% +------------------------------------------------------------------------+
% | 22.03.1999   Lutz Kettner
% | Package: HalfedgeDS
% | 
\RCSdef{\RCSHalfedgeDSRev}{$Id: HalfedgeDS.tex 38221 2007-04-17 16:31:42Z spion $}
\RCSdefDate{\RCSHalfedgeDSDate}{$Date: 2007-04-17 13:31:42 -0300 (Tue, 17 Apr 2007) $}
% +------------------------------------------------------------------------+

\section{Faces with several boundaries}

{\bf\ttfamily
In some geometric structures, such as a \ccc{Polyhedron_3} or a \ccc{Straight_skeleton_2},
faces are bounded by a single connected component. That is, just one cycle of halfedges bounds
any face. But in other structures, like an arrangement of curves in a surface, faces
might have holes or even multiple outer connected components of the boundary. 
In these later cases, a face can be bounded by more than just one cycle of halfedges. 

Therefore, the current \ccc{HDS} design supports faces
whose boundaries are composed of multiple outer and inner halfedge cycles. Furthermore, faces can contain isolated vertices in its interior.

This design supports two {\em mutually exclusive} methods for 
storing the incidence relationship between faces and halfedges:
direct-mutual-reference (having faces and halfedges reference 
each other directly) or through a common halfedge cycle.



\ccHeading{Direct Mutual Reference:}

Prior designs of the \ccc{HDS} only supported faces bounded by
a single cycle of halfedges.
This is still possible in the current design, not only for backward compatibility 
but to optimize the structure if multiple boundary components are not needed:
as in \cgal\ version 3.3.1, a type tag is used to allow a face to directly store 
a reference to a halfedge (implicitly representing the cycle) and vice versa. In this case, \ccc{Face::halfedge()}, \ccc{Face::set_halfedge()}, \ccc{Halfedge::face()} 
and \ccc{Halfedge::set_face()} are defined

\ccHeading{Ccbs items:}

Using this method a cycle of halfedges is collectively represented 
by a fourth first-class \ccc{HDS} item named \ccc{ccb} (for Connected Component of the Boundary) 

A face stores sequences of outer and inner ccb handles, while each ccb in turn stores
a reference to a halfedge\footnote{In other hole-supporting HDS designs, faces stores a sequence of halfedges (not ccbs), each for a different cycle. In our design a cycle is given by a fourth HDS item instead of a halfedge to allow the centralization of cycle-related information}. 
A face also stores another sequence of ccb handles for isolated vertices where each such vertex is represented by a ccb composed of a single-halfedge single-vertex self-loop.

When using ccb items, incident faces and halfedges are not directly cross referenced. Instead,
they are indirectly related via a common ccb object: faces store ccb handles and each halfedge in the ccb  stores the handle to its hosting ccb.

\ccc{Face::halfedge()} {\em is not} defined if ccbs are used because a face can have many halfedge cycles, not just one, or it could have no outer ccb but inner ccbs.

\ccc{Halfedge::face()} on the other hand is well defined as a shortcut for \ccc{Halfedge::ccb()->face()}
since every halfedge belongs to a ccb.

A \ccc{ccb} is a first-class \ccc{HDS} item, just like vertices, halfedges and faces.
However, a ccb does not contribute {\em by itself} to the incidence information 
maintained by the \ccc{HDS}. That is, the incidence is betwen vertices, edges and faces,
which are all well defined mathematical concepts belonging to the field of algebraic topology.
This new design does not introduce a new topological concept but a new
auxiliary object in a similar way halfedges are auxiliary objects that 
relate to the topological concept of {\em edge} only when considered in pairs.

A \ccc{ccb} item allows upper level structures to centralize boundary-related information
(combinatorial or geometric) in a single object. This is the reason why 
connected components of the boundary are represented by this first-class \ccc{HDS} item 
and not just by a halfedge implicitly representing the cycle (as in other designs).

For instance, a derived \ccc{ccb} can store the iterator that identifies the position of the ccb 
in the face for fast hole migration across different faces, or cache curve-level geometric
information such as orientation even if not supported by the curve type itself.
}

