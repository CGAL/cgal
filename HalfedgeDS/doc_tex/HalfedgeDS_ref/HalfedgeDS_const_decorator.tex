% +------------------------------------------------------------------------+
% | Reference manual page: HalfedgeDS_const_decorator.tex
% +------------------------------------------------------------------------+
% | 22.03.1999   Lutz Kettner
% | Package: HalfedgeDS
% | 
\RCSdef{\RCSHalfedgeDSconstdecoratorRev}{$Revision$}
\RCSdefDate{\RCSHalfedgeDSconstdecoratorDate}{$Date$}
% +------------------------------------------------------------------------+

\ccRefPageBegin

%%RefPage: end of header, begin of main body
% +------------------------------------------------------------------------+


\begin{ccRefClass}{HalfedgeDS_const_decorator<HDS>}
\label{pageHalfedgeDSconstDecoratorRef}

\ccDefinition
  
The classes \ccc{CGAL::HalfedgeDS_items_decorator<HDS>},
\ccc{CGAL::HalfedgeDS_decorator<HDS>}, and
\ccc{CGAL::HalfedgeDS_const_decorator<HDS>} provide additional functions
to examine and to modify a halfedge data structure \ccc{HDS}. The class
\ccc{CGAL::HalfedgeDS_items_decorator<HDS>} provides additional functions
for vertices, halfedges, and faces of a halfedge data structure
without knowing the containing halfedge data structure. The class
\ccc{CGAL::HalfedgeDS_decorator<HDS>} stores a reference to the halfedge
data structure and provides functions that modify the halfedge data
structure, for example Euler-operators. The class
\ccc{CGAL::HalfedgeDS_const_decorator<HDS>} stores a const reference to
the halfedge data structure. It contains non-modifying functions, for
example the test for validness of the data structure.

All these additional functions take care of the different capabilities
a halfedge data structure may have or may not have.  The functions
evaluate the type tags of the halfedge data structure to decide on the
actions. If a particular feature is not supported nothing is done.
Note that for example the creation of new halfedges is mandatory for
all halfedge data structures and will not appear here again.

\ccInclude{CGAL/HalfedgeDS_const_decorator.h}

\ccInheritsFrom

\ccRefIdfierPage{CGAL::HalfedgeDS_items_decorator<HDS>}

\ccCreation
\ccCreationVariable{D}

\ccTagFullDeclarations
\ccConstructor{HalfedgeDS_const_decorator( const HDS& hds);}
    {keeps internally a const reference to \ccc{hds}.}
\ccTagDefaults


% -----------------------------------------
\ccHeading{Validness Checks}
\ccThree{void}{D.is_val}{}

A halfedge data structure has no definition of validness of its own,
but a useful set of tests is defined with the following levels:
%
\begin{description}
\item[Level 0] 
  The number of halfedges is even. All pointers except
  the vertex pointer and the face pointer for border halfedges are
  unequal to their respective default construction value.  For all
  halfedges $h$: The opposite halfedge is different from $h$ and the
  opposite of the opposite is equal to $h$. The next of the previous
  halfedge is equal to $h$. For all vertices $v$: the incident vertex
  of the incident halfedge of $v$ is equal to $v$. The halfedges
  around $v$ starting with the incident halfedge of $v$ form a cycle.
  For all faces $f$: the incident face of the incident halfedge of $f$
  is equal to $f$. The halfedges around $f$ starting with the incident
  halfedge of $f$ form a cycle.  Redundancies among internal variables
  are tested, e.g., that iterators enumerate as many items as the
  related size value indicates.
\item[Level 1] 
  All tests of level 0. For all halfedges $h$: The
  incident vertex of $h$ exists and is equal to the incident vertex of
  the opposite of the next halfedge. The incident face (or hole) of
  $h$ is equal to the incident face (or hole) of the next halfedge.
\item[Level 2]
  All tests of level 1. The sum of all halfedges that can
  be reached through the vertices must be equal to the number of all
  halfedges, i.e., all halfedges incident to a vertex must form a single
  cycle.
\item[Level 3]
  All tests of level 2. The sum of all halfedges that can
  be reached through the faces must be equal to the number of all
  halfedges, i.e., all halfedges surrounding a face must form a single
  cycle (no holes in faces).
\item[Level 4]
  All tests of level 3 and \ccc{normalized_border_is_valid}.
\end{description}

\ccMethod{bool is_valid( bool verbose = false, int level = 0) const;}{%
    returns \ccc{true} if the halfedge data structure \ccc{hds} is
    valid with respect to the \ccc{level} value as defined above. 
    If \ccc{verbose} is \ccc{true}, statistics are written to \ccc{cerr}.}

\ccMethod{bool normalized_border_is_valid( bool verbose = false) const;}{%
    returns \ccc{true} if the border halfedges are in normalized 
    representation, which is when enumerating all halfedges with the
    halfedge iterator the following holds: The non-border edges precede the 
    border edges. For border edges, the second halfedge is a border halfedge. 
    (The first halfedge may or may not be a border halfedge.) The halfedge
    iterator \ccc{border_halfedges_begin()} denotes the first border
    edge. If \ccc{verbose} is \ccc{true}, statistics are written to \ccc{cerr}.
}



\ccSeeAlso

\ccRefIdfierPage{CGAL::HalfedgeDS_items_decorator<HDS>}\\
\ccRefIdfierPage{CGAL::HalfedgeDS_decorator<HDS>}

\ccExample

The following program fragment illustrates the implementation of a
\ccc{is_valid()} member function for a simplified polyhedron class.
We assume here that the level three check is the appropriate default
for polyhedral surfaces.

\begin{ccExampleCode}
namespace CGAL {
    template <class Traits>
    class Polyhedron {
        typedef HalfedgeDS_default<Traits> HDS;
        HDS hds;
    public:
        // ...
        bool is_valid( bool verb = false, int level = 0) const {
            Verbose_ostream verr(verb);
            verr << "begin Polyhedron::is_valid( verb=true, level = " << level 
                 << "):" << std::endl;
            HalfedgeDS_const_decorator<HDS> decorator(hds);
            bool valid = decorator.is_valid( verb, level + 3);
            // further checks ...
        }
    };
}
\end{ccExampleCode}

\end{ccRefClass}

% +------------------------------------------------------------------------+
%%RefPage: end of main body, begin of footer
\ccRefPageEnd
% EOF
% +------------------------------------------------------------------------+
