% +------------------------------------------------------------------------+
% | Reference manual page: HalfedgeDSFace.tex
% +------------------------------------------------------------------------+
% | 22.03.1999   Lutz Kettner
% | Package: HalfedgeDS
% | 
\RCSdef{\RCSFaceRev}{$Id$}
\RCSdefDate{\RCSFaceDate}{$Date$}
% +------------------------------------------------------------------------+

\ccRefPageBegin

%%RefPage: end of header, begin of main body
% +------------------------------------------------------------------------+


\begin{ccRefConcept}{HalfedgeDSFace}
\label{pageHalfedgeDSItemsFaceRef}

\ccDefinition
  
The concept \ccRefName\ defines the requirements for the local \ccc{Face} 
type in the \ccc{HalfedgeDS} concept. It is also required in 
the \ccc{Face_wrapper<Refs,Traits>} member class template of an
items class, see the \ccc{HalfedgeDSItems} concept.

% +------------------------------------------------------------------------+
% | Reference manual page: HalfedgeDS.tex
% +------------------------------------------------------------------------+
% | 22.03.1999   Lutz Kettner
% | Package: HalfedgeDS
% | 
\RCSdef{\RCSHalfedgeDSRev}{$Id: HalfedgeDS.tex 38221 2007-04-17 16:31:42Z spion $}
\RCSdefDate{\RCSHalfedgeDSDate}{$Date: 2007-04-17 13:31:42 -0300 (Tue, 17 Apr 2007) $}
% +------------------------------------------------------------------------+

{\bf\ttfamily
Beginning with \cgal\ version 3.4 this package has been extended
to efficiently support the orientable-2-manifolds that correspond to the topology 
of the arrangement of curves on surfaces.
The added features are purely an extension so the extended structure is completely 
backward compatible with version 3.3.1.
}


% +------------------------------------------------------------------------+
% | Reference manual page: HalfedgeDS.tex
% +------------------------------------------------------------------------+
% | 22.03.1999   Lutz Kettner
% | Package: HalfedgeDS
% | 
\RCSdef{\RCSHalfedgeDSRev}{$Id: HalfedgeDS.tex 38221 2007-04-17 16:31:42Z spion $}
\RCSdefDate{\RCSHalfedgeDSDate}{$Date: 2007-04-17 13:31:42 -0300 (Tue, 17 Apr 2007) $}
% +------------------------------------------------------------------------+

{\XHDS
In some geometric structures, such as a \ccc{Polyhedron_3} or a \ccc{Straight_skeleton_2},
faces are bounded by a single connected component. That is, just one cycle of halfedges bounds
any face. But in other structures, like an arrangement of curves in a surface, faces
might have holes or even multiple outer connected components of the boundary. 
In these later cases, a face can be bounded by more than just one cycle of halfedges. 

Therefore, this design supports faces
whose boundaries are composed of multiple outer and inner halfedge cycles. Furthermore, faces can contain isolated vertices in its interior.

This design supports two {\em mutually exclusive} methods for 
storing the incidence relationship between faces and halfedges:
direct-mutual-reference (having faces and halfedges reference 
each other directly) or through a common halfedge cycle.



\ccHeading{Direct Mutual Reference:}

If multiple boundary components are not needed a type tag is used to allow a face to directly store 
a reference to a halfedge (implicitly representing the cycle) and vice versa. In this case, \ccc{Face::halfedge()}, \ccc{Face::set_halfedge()}, \ccc{Halfedge::face()} 
and \ccc{Halfedge::set_face()} are defined

\ccHeading{\ccc{Halfedge_cycle} Items:}

Using this method a cycle of halfedges is collectively represented 
by a fourth first-class \ccc{HDS} item named \ccc{halfedge_cycle} 

A face stores sequences of outer and inner \ccc{halfedge_cycle} handles, while each 
\ccc{halfedge_cycle} in turn stores a reference to a halfedge\footnote{In other
hole-supporting HDS designs, faces stores a sequence of halfedges (each for a different cycle).
In our design a cycle is given by a \ccc{halfedge_cycle} instead of a \ccc{halfedge} to allow 
the centralization of cycle-related information}. 
A face also stores another sequence of \ccc{halfedge_cycle} handles for isolated vertices where
each such vertex is represented by a cycle composed of a single-halfedge single-vertex self-loop.

When using \ccc{halfedge_cycles}, incident faces and halfedges are not directly cross referenced. Instead,
they are indirectly related via a common \ccc{halfedge_cycle} object: faces store 
\ccc{halfedge_cycle} handles and each halfedge in the cycle stores the handle to the 
\ccc{halfedge_cycle} it belongs to.

\ccc{Face::halfedge()} {\em is not} defined if \ccc{halfedge_cycles} are used because a face
can have many cycles, not just one; or it could have only inner cycles (holes).

\ccc{Halfedge::face()} on the other hand is well defined as a shortcut for \ccc{Halfedge::halfedge_cycle()->face()}
since every halfedge belongs to a \ccc{halfedge_cycle}.

A \ccc{halfedge_cycle} is a first-class \ccc{HDS} item, just like vertices, halfedges and faces.
However, a \ccc{halfedge_cycle} does not contribute {\em by itself} to the incidence information 
maintained by the \ccc{HDS}. That is, the incidence is betwen vertices, edges and faces,
which are all well defined mathematical concepts belonging to the field of algebraic topology.
This design does not introduce a new topological concept but a new
auxiliary object in a similar way halfedges are auxiliary objects that 
relate to the topological concept of {\em edge} only when considered in pairs.

A \ccc{halfedge_cycle} allows upper level structures to centralize boundary-related information
(combinatorial or geometric) in a single object. This is the reason why 
connected components of the boundary are represented by this first-class \ccc{HDS} item 
and not just by a halfedge implicitly representing the cycle (as in other designs).

For instance, a derived \ccc{halfedge_cycle} can store the iterator that identifies the position of
cycle in the face for fast hole migration across different faces, or cache curve-level geometric
information such as orientation even if not supported by the curve type itself.
}



{\bf\ttfamily
If ccbs are used, a face optionally stores begin and end iterators for the outer and inner ccb sequences
as well. It also optionally stores iterators for the sequence of ccbs corresponding to isolated vertices.
If, alternatively, direct mutual reference is used, a face optionally stores a reference 
to an incident halfedge that points to the face.
A type tag indicates whether the related member functions are supported. 
}

{\bf\ttfamily

The choice between using ccbs or direct mutual reference
(to define the incidence relation between faces and halfedges) is selected 
via a combination of type tags:

For direct mutual reference (the only method available in \cgal\ 3.3.1) use

\ccc{Supports_face_halfedge} $\equiv$  \ccc{CGAL::Tag_true}
and
\ccc{Supports_face_ccb} $\equiv$  \ccc{CGAL::Tag_false}

For ccb items use:

\ccc{Supports_face_halfedge} $\equiv$  \ccc{CGAL::Tag_false}
and
\ccc{Supports_face_ccb} $\equiv$  \ccc{CGAL::Tag_true}
 
}

\begin{ccAdvanced}
{\bf\ttfamily
The storage for \ccc{ccb} items is controlled by the \ccc{HDS} itself. Therefore, 
there is a \ccc{Ccb_iterator} type local to \ccc{HalfedgeDS} corresponding to a 
container whose elements are of type \ccc{HalfedgeDSCcb}.

Faces maintain their own local sequences of ccbs. These sequences are
containers of \ccc{Ccb_handle} elements, not of \ccc{Ccb} elements (like the
container in the HDS). Thus, a Face defines a local \ccc{Ccb_handle_iterator} type.

While a \ccc{Ccb_iterator} is implicitly convertible to a \ccc{Ccb_handle},
a \ccc{Ccb_handle_iterator} is {\em dereferenceable} as a \ccc{Ccb_handle} 
(i.e. you need to apply the \ccc{operator*} to access the handle)
}
\end{ccAdvanced}

\begin{ccAdvanced}
{\bf\ttfamily
As faces can have more than one outer boundary, an upper level data structure could use a single
face to represent disjoint regions each bounded by a distinct outer ccb. However,
it is highly recommended to restrict faces to correspond to singly connected regions, and
to use multiple outer ccbs to represent disjoint boundary components around the same 
connected region. For example, the surface of a cross section of a cylinder is a single
connected region with two disjoint outer boundaries. This would be given by 1 face with
2 outer ccbs.
}
\end{ccAdvanced}

Figure~\ccTexHtml{\ref{figureHalfedgeDSOptionalMethods} 
on page \pageref{figureHalfedgeDSOptionalMethods}}{}\begin{ccHtmlOnly}
  <A HREF="Concept_HalfedgeDS.html#figureHalfedgeDSOptionalMethods"><IMG 
  SRC="cc_ref_up_arrow.gif" ALT="reference arrow" WIDTH="10" HEIGHT="10"></A>
\end{ccHtmlOnly}
depicts the relationship between a halfedge and its incident
halfedges, vertices, and faces.

For the protection of the integrity of the data structure classes such as
\ccc{CGAL::Polyhedron_3} are allowed to redefine the modifying member 
functions to be private. In order to make them accessible for the 
halfedge data structure they must be derived from a base class \ccc{Base}
where the modifying member functions are still public. (The protection
can be bypassed by the user, but not by accident.)

\ccTypes

\ccThree{Ccb_const_handle_const_iterator}{v.set_halfedge( Halfedge_handle h);}{}
\ccThreeToTwo
%\ccTwo{HalfedgeDSFace:: Halfedge_const_handleXX}{}

\ccNestedType{HalfedgeDS}
    {instantiated \ccc{HalfedgeDS} ( $\equiv$ \ccc{Refs}).}
\ccGlue
\ccNestedType{size_type}{size type.}
\ccGlue
\ccNestedType{Base}{base class that allows modifications.}
\ccGlue
\ccNestedType{Vertex}{model of \ccc{HalfedgeDSVertex}.}
\ccGlue
\ccNestedType{Halfedge}{model of \ccc{HalfedgeDSHalfedge}.}
\ccGlue
\ccNestedType{\bf\ttfamily Ccb}{\bf\ttfamily model of \ccc{HalfedgeDSCcb}.}
\ccGlue
\ccNestedType{Vertex_handle}{handle to vertex.}
\ccGlue
\ccNestedType{Halfedge_handle}{handle to halfedge.}
\ccGlue
\ccNestedType{\bf\ttfamily Ccb_handle}{\bf\ttfamily handle to ccb.}
\ccGlue
\ccNestedType{Face_handle}{handle to face.}
\ccGlue
\ccNestedType{Vertex_const_handle}{}
\ccGlue
\ccNestedType{Halfedge_const_handle}{}
\ccGlue
\ccNestedType{Face_const_handle}{}
\ccGlue
\ccNestedType{\bf\ttfamily Ccb_const_handle}{}


\ccNestedType{\bf\ttfamily Ccb_const_handle_const_iterator}
\ccGlue
\ccNestedType{\bf\ttfamily Ccb_handle_iterator}{\bf\ttfamily iterator to the local sequence of \ccc{Ccb_handles}.}

\ccNestedType{Supports_face_halfedge}{\ccc{CGAL::Tag_true} or \ccc{CGAL::Tag_false}.}
\ccGlue
\ccNestedType{\bf\ttfamily Supports_face_ccb}{\bf\ttfamily \ccc{CGAL::Tag_true} or \ccc{CGAL::Tag_false}.}


\ccCreation
\ccCreationVariable{f}

\ccConstructor{Face();}{default constructor.}

\ccTagFullDeclarations

\ccHeading{Operations available if \ccc{Supports_face_ccb} $\equiv$ \ccc{CGAL::Tag_true}}

\ccMethod{\bf\ttfamily size_type size_of_isolated_vertices() const;}{\bf\ttfamily number of isolated  vertices of \ccVar}

\ccMethod{\bf\ttfamily size_type size_of_outer_ccbs() const;}
\ccGlue
\ccMethod{\bf\ttfamily size_type size_of_inner_ccbs() const;}{\bf\ttfamily number of outer/inner ccbs of \ccVar}



\ccMethod{\bf\ttfamily Ccb_handle_iterator isolated_vertices_begin();}{}
\ccGlue
\ccMethod{\bf\ttfamily Ccb_const_handle_const_iterator  isolated_vertices_begin() const;}
         {\bf\ttfamily the begin iterator for the sequence of isolated vertices of \ccVar.}

\ccMethod{\bf\ttfamily Ccb_handle_iterator outer_ccbs_begin();}{}
\ccGlue
\ccMethod{\bf\ttfamily Ccb_const_handle_const_iterator  outer_ccbs_begin() const;}
\ccGlue
\ccMethod{\bf\ttfamily Ccb_handle_iterator inner_ccbs_begin();}{}
\ccGlue
\ccMethod{\bf\ttfamily Ccb_const_handle_const_iterator  inner_ccbs_begin() const;}
         {\bf\ttfamily the begin iterator for the sequence of outer/inner ccbs of \ccVar.}


\ccMethod{\bf\ttfamily Ccb_handle_iterator isolated_vertices_end();}{}
\ccGlue
\ccMethod{\bf\ttfamily Ccb_const_handle_const_iterator isolated_vertices_end() const;}
         {\bf\ttfamily the past-the-end iterator for the sequence of isolated vertices of \ccVar.}


\ccMethod{\bf\ttfamily Ccb_handle_iterator outer_ccbs_end();}{}
\ccGlue
\ccMethod{\bf\ttfamily Ccb_const_handle_const_iterator outer_ccbs_end() const;}
\ccGlue
\ccMethod{\bf\ttfamily Ccb_handle_iterator inner_ccbs_end();}{}
\ccGlue
\ccMethod{\bf\ttfamily Ccb_const_handle_const_iterator inner_ccbs_end() const;}
         {\bf\ttfamily the past-the-end iterator for the sequence of outer/inner ccbs of \ccVar.}


\ccMethod{\bf\ttfamily void isolated_vertex_push_back( Ccb_handle v );}
         {\bf\ttfamily appends $v$ to the isolated vertices sequence}

\ccMethod{\bf\ttfamily void outer_ccb_push_back( Ccb_handle b );}
\ccGlue
\ccMethod{\bf\ttfamily void inner_ccb_push_back( Ccb_handle b );}
         {\bf\ttfamily appends $b$ to the outer/inner ccbs sequence}

\ccMethod{\bf\ttfamily Ccb_handle_iterator isolated_vertex_insert( Ccb_handle_iterator pos, Ccb_handle v );}
         {\bf\ttfamily inserts $v$ before $pos$ in the isolated vertices sequence}

\ccMethod{\bf\ttfamily Ccb_handle_iterator outer_ccb_insert( Ccb_handle_iterator pos, Ccb_handle b );}
\ccGlue
\ccMethod{\bf\ttfamily Ccb_handle_iterator inner_ccb_insert( Ccb_handle_iterator pos, Ccb_handle b );}
         {\bf\ttfamily inserts $b$ before $pos$ in the outer/inner ccbs sequence}

\ccMethod{\bf\ttfamily void isolated_vertex_erase( Ccb_handle_iterator it );}
         {\bf\ttfamily removes the \ccc{Ccb_handle} pointed to be $it$ in the isolated vertices sequence}

\ccMethod{\bf\ttfamily void outer_ccb_erase( Ccb_handle_iterator it );}
\ccGlue
\ccMethod{\bf\ttfamily void inner_ccb_erase( Ccb_handle_iterator it );}
         {\bf\ttfamily removes the \ccc{Ccb_handle} pointed to be $it$ in the outer/inner ccbs sequence}

\ccMethod{\bf\ttfamily void isolated_vertices_clear();}{\bf\ttfamily removes all isolated vertices. }

\ccMethod{\bf\ttfamily void outer_ccbs_clear();}{}
\ccGlue
\ccMethod{\bf\ttfamily void inner_ccbs_clear();}{\bf\ttfamily removes all outer/inner ccbs. }


\ccHeading{Operations available if \ccc{Supports_face_halfedge} $\equiv$ \ccc{CGAL::Tag_true}}

\ccMethod{Halfedge_handle       halfedge();}{}
\ccGlue
\ccMethod{Halfedge_const_handle halfedge() const;}{
    incident halfedge that points to \ccVar.}

\ccMethod{void set_halfedge( Halfedge_handle h);}{
    sets incident halfedge to $h$.}


\ccHasModels

\ccRefIdfierPage{CGAL::HalfedgeDS_face_base<Refs>}\\
\ccRefIdfierPage{CGAL::HalfedgeDS_face_min_base<Refs>}

\ccSeeAlso

\ccRefConceptPage{HalfedgeDS<Traits,Items,Alloc>}\\
\ccRefConceptPage{HalfedgeDSItems}\\
\ccRefConceptPage{HalfedgeDSVertex}\\
\ccRefConceptPage{HalfedgeDSHalfedge}\\
\ccRefConceptPage{HalfedgeDSCcb}

\ccTagDefaults
\end{ccRefConcept}

% +------------------------------------------------------------------------+
%%RefPage: end of main body, begin of footer
\ccRefPageEnd
% EOF
% +------------------------------------------------------------------------+

