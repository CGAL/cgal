% +------------------------------------------------------------------------+
% | CGAL User Manual: 
% +------------------------------------------------------------------------+
% |
% | 28.05.2008   Peter Hachenberger
% | 
\RCSdef{\ConvexDecomposition3Rev}{$Id$}
\RCSdefDate{\ConvexDecomposition3Date}{$Date$}
% +------------------------------------------------------------------------+

\ccParDims

\ccUserChapter{Convex Decomposition of Polyhedra \label{chapterConvexDecomposition3}}
\ccChapterRelease{\ConvexDecomposition3Rev. \ \ConvexDecomposition3Date}
\ccChapterAuthor{Peter Hachenberger}

%
\begin{ccPkgDescription}{3D Triangulations}
\ccPkgSummary{
This package  allows to build and handle
triangulations for point sets in three dimensions.
Any CGAL  triangulation covers the convex hull of its
vertices. Triangulations are build incrementally 
and can be modified by insertion or removal of vertices. 
They offer point location facilities.

The package provides plain triangulation (whose faces
depends on the  insertion order of the vertices) and
Delaunay triangulations.  Regular triangulations are
also provided for sets of weighted points.
Delaunay and regular
triangulations offer nearest neighbor queries
and primitives to build the dual Voronoi and power diagrams.}

%\ccPkgDependsOn{}
\ccPkgMaturity{Introduced in \cgal\ 3.1}

\end{ccPkgDescription}


% +------------------------------------------------------------------------+
\section{Introduction}

For many applications on non-convex polyhedra, there are efficient
solutions that first decompose the polyhedron into convex pieces. As
an example, the Minkowski sum of two polyhedra can be computed by
decomposing both polyhedra into convex pieces, compute pair-wise
Minkowski sums of the convex pieces, and unite the pair-wise sums.

While it is desirable to have a decomposition into a minimum number of
pieces, this problem is know to be NP-hard. Our implementation
decomposes a Nef polyhedron $N$ into $O(r^2)$ convex pieces, where $r$
is the number of edges, which have two adjacent facets that span an
angle of more than 180 degrees with respect to the interior of the
polyhedron. Those edges are also called reflex edges.  The bound of
$O(r^2)$ convex pieces is worst-case optimal.

At the moment our implementation is restricted to the decomposition of
finite point sets. If the input polyhedron is infinite, i.e., the
outer volume is part of the polyhedron, then this volume is ignored
during the decomposition process. An extension to infinite point sets
is planned.

% +------------------------------------------------------------------------+
\section{Usage}

The following example illustrates the usage of the function
\ccc{convex_decomposition_3}. It takes a \ccc{Nef_polyhedron_3}
$N$ as input parameter. The result is the modified polyhedron $N$.
After the execution of the function, it contains additional facets,
such that each marked volume (except for the outer volume) is
subdivided into convex pieces. The convex pieces can then be used by
traversing $N$, or by converting them into separate Nef polyhedra, as
shown in the example code.

\ccIncludeExampleCode{Convex_decomposition_3/list_of_convex_parts.cpp}
