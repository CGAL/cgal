% +------------------------------------------------------------------------+
% | CGAL User Manual: 
% +------------------------------------------------------------------------+
% |
% | 28.05.2008   Peter Hachenberger
% | 
\RCSdef{\ConvexDecomposition3Rev}{$Id$}
\RCSdefDate{\ConvexDecomposition3Date}{$Date$}
% +------------------------------------------------------------------------+

\ccParDims

\ccUserChapter{Convex Decomposition of Polyhedra \label{chapterConvexDecomposition3}}
\ccChapterRelease{\ConvexDecomposition3Rev. \ \ConvexDecomposition3Date}
\ccChapterAuthor{Peter Hachenberger}

%
\begin{ccPkgDescription}{3D Triangulations}
\ccPkgSummary{
This package  allows to build and handle
triangulations for point sets in three dimensions.
Any CGAL  triangulation covers the convex hull of its
vertices. Triangulations are build incrementally 
and can be modified by insertion or removal of vertices. 
They offer point location facilities.

The package provides plain triangulation (whose faces
depends on the  insertion order of the vertices) and
Delaunay triangulations.  Regular triangulations are
also provided for sets of weighted points.
Delaunay and regular
triangulations offer nearest neighbor queries
and primitives to build the dual Voronoi and power diagrams.}

%\ccPkgDependsOn{}
\ccPkgMaturity{Introduced in \cgal\ 3.1}

\end{ccPkgDescription}


% +------------------------------------------------------------------------+
\section{Introduction}

For many applications on non-convex polyhedra, there are efficient
solutions that first decompose the polyhedron into convex pieces. As
an example, the Minkowski sum of two polyhedra can be computed by
decomposing both polyhedra into convex pieces, compute pair-wise
Minkowski sums of the convex pieces, and unite the pair-wise sums.

While it is desirable to have a decomposition into a minimum number of
pieces, this problem is know to be NP-hard~\cite{c-cpplb-84}. Our
implementation decomposes a Nef polyhedron $N$ into $O(r^2)$ convex
pieces, where $r$ is the number of edges, which have two adjacent
facets that span an angle of more than 180 degrees with respect to the
interior of the polyhedron. Those edges are also called reflex edges.
The bound of $O(r^2)$ convex pieces is worst-case
optimal~\cite{c-cpplb-84}.

At the moment our implementation is restricted to the decomposition of
bounded polyhedra. An extension to unbounded polyhedra is planned.

% +------------------------------------------------------------------------+
\section{Interface and Usage}

The \ccc{Nef_polyhedron_3} represents a subdivision of the
three-dimensional space into vertices, edges, facets, and volumes. The
represented polyhedron is the union of the point sets represented by
the selected items. As an example, a cube is represented by 8
vertices, 12 edges, 6 facets, and 2 volumes, where one volume
represents the interior of the cube and the other the volume outside
of the cube. All of these items are selected, except for outer volume.
Read the chapter on 3D Boolean operations on Nef polyhedra for more
details~\ref{chapterNef3}.

The function \ccc{convex_decomposition_3} takes one
\ccc{Nef_polyhedron_3} as input an inserts additional facets into the
selected volumes. These additional facets are also selected and
therefore redundant for the representation of the polyhedron. If some
of these facets split a volume into two parts, each of the two volumes
is represented by a separate volume item. The insertion of facets
stops when all the selected volumes are convex. The modified
polyhedron is the result of the function
\ccc{convex_decomposition_3}. Note that the function 
\ccc{convex_decomposition_3} is restricted to standard kernels. 
The extended kernels, which allow the representation of polyhedra with
an infinite boundary (e.g. halfspaces) only add further unbounded
polyhedra to the domain of representable polyhedra. Using a standard
kernel, unbounded polyhedra can be identified by the selected outer
volume. In such a case, we ignore the outer volume in the
decomposition process.

The convex pieces of the modified polyhedron can be accessed by
traversing $N$~\ref{subsectionNef_3ShellExploration}, or by converting
them into separate Nef polyhedra, as illustrated by the example code.

\ccIncludeExampleCode{Convex_decomposition_3/list_of_convex_parts.cpp}
