% +------------------------------------------------------------------------+
% | Reference manual page: keep_largest_connected_components.tex
% +------------------------------------------------------------------------+
% | 16.04.2009   Pierre Alliez, Laurent Saboret, Gael Guennebaud
% | Package: Surface_reconstruction_points_3
% |
\RCSdef{\RCSerasesmallpolyhedronconnectedcomponentsRev}{$Id$}
\RCSdefDate{\RCSerasesmallpolyhedronconnectedcomponentsDate}{$Date$}
% |
\ccRefPageBegin
%%RefPage: end of header, begin of main body
% +------------------------------------------------------------------------+


\begin{ccRefFunction}{keep_largest_connected_components<Polyhedron>}

%% \ccHtmlCrossLink{}     %% add further rules for cross referencing links
%% \ccHtmlIndexC[function]{} %% add further index entries

\ccDefinition

\ccc{keep_largest_connected_components<Polyhedron>} erases the small connected components of a polyhedron.

% The section below is automatically generated. Do not edit!
%START-AUTO(\ccDefinition)

% Reduce left margin
\ccThree{123456789012345}{6789012}{}

\ccFunction{template<class Polyhedron> unsigned int keep_largest_connected_components(Polyhedron& polyhedron, unsigned int nb_components_to_keep);}
{
Erases small connected components of a polyhedron.
\ccCommentHeading{Template Parameters}  \\
\ccc{Polyhedron}: an instance of \ccc{Polyhedron_3<Traits>} that supports vertices and removal operation. \ccc{nb_components_to_keep}: the number of large connected components to keep.
\ccCommentHeading{Returns} the number of connected components erased.
}
\ccGlue

%END-AUTO(\ccDefinition)

\ccInclude{CGAL/keep_largest_connected_components.h}

\ccSeeAlso

\ccRefIdfierPage{CGAL::Polyhedron_3<Traits>}  \\

\ccExample

See \ccc{APSS_reconstruction.cpp} example.

\end{ccRefFunction}

% +------------------------------------------------------------------------+
%%RefPage: end of main body, begin of footer
\ccRefPageEnd
% EOF
% +------------------------------------------------------------------------+

