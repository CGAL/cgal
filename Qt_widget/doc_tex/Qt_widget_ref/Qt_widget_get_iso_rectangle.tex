% +------------------------------------------------------------------------+
% | CGAL Reference Manual: Reference manual for Qt_widget.tex
% +------------------------------------------------------------------------+
% |
% | 03.01.2001  Radu Ursu
% | 
% | \RCSdef{\qtwidgetRev}{$Id$}
% | \RCSdefDate{\qtwidgetDate}{$Date$}
% +------------------------------------------------------------------------+

% +-----------------------------------------------------+
\begin{ccRefClass}{Qt_widget_get_iso_rectangle<T>}

\ccDefinition
An object of type \ccRefName\ creates a \cgal\ iso\_rectangle this way
one left click will be the first generator point, and second point
will be considered at the coordinates where the left mouse button is
pressed for the second time.
You can always cancel the creation process by pressing the ESC key.
The use of \ccRefName\ requires that the mouse tracking is
enabled for widgets attaching it.

\ccInclude{CGAL/IO/Qt_widget_get_iso_rectangle.h}

\ccParameters
The full template declaration of \ccc{Qt_widget_get_iso_rectangle} states one parameter:
\begin{tabbing}
\ccc{template <} \=\ccc{class T >}\\
        \ccc{class Qt_widget_get_iso_rectangle;}
\end{tabbing}

If T is one of the \cgal\ kernels you don't need additional types. If
not, the parameter T has to provide this types:

\ccTypes
\ccTypedef{typedef T::RT RT;}{This should be a Ring type}

\ccInheritsFrom
\ccc{Qt_widget_layer}

\ccGlue

\ccCreation
\ccCreationVariable{getisor}
\ccSetTwoColumns{Qt_widget_get_iso_rectangle}{}

\ccConstructor{Qt_widget_get_iso_rectangle<T>(const QCursor
c=QCursor(Qt::crossCursor), QObject* parent = 0, const char* name =
0);}{\ccc{c} is the cursor that this layer will use when is
active. \ccc{parent} is the parent widget and \ccc{name} is the name
you give to this layer.}

\end{ccRefClass}

% +-----------------------------------------------------+
% EOF
