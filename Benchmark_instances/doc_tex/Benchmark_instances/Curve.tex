\section{Curve\label{bi_cha:Curve}}

Curves exit in various kind. Currently, we concentrate on
non-linear algebraic curves of different degree. We 
distinguish conics, cubics, and curves of arbitrary degree.

\subsection{Conics\label{bi_sec:Arrangement2Conics}}

A conic is defined by the zero set of a two-variate polynomial
of degree 2. A circle is a special conic.

The following files contain contain randomly generated conics. 
Each instance of one file inlcudes the same number of conics. Not that
a generic conic may also be unbounded.

\begin{itemize}
\item \ccBenchmarkInstance{data/Curve/Conics/random/number10/}
\item \ccBenchmarkInstance{data/Curve/Conics/random/number20/}
\item \ccBenchmarkInstance{data/Curve/Conics/random/number30/}
\item \ccBenchmarkInstance{data/Curve/Conics/random/number40/}
\item \ccBenchmarkInstance{data/Curve/Conics/random/number50/}
\item \ccBenchmarkInstance{data/Curve/Conics/random/number75/}
\item \ccBenchmarkInstance{data/Curve/Conics/random/number100/}
\item \ccBenchmarkInstance{data/Curve/Conics/random/number125/}
\item \ccBenchmarkInstance{data/Curve/Conics/random/number150/}
\item \ccBenchmarkInstance{data/Curve/Conics/random/number175/}
\item \ccBenchmarkInstance{data/Curve/Conics/random/number200/}
\item \ccBenchmarkInstance{data/Curve/Conics/random/number300/}
\item \ccBenchmarkInstance{data/Curve/Conics/random/number400/}
\item \ccBenchmarkInstance{data/Curve/Conics/random/number500/}
\end{itemize}

\subsection{Cubics}

A conic is defined by the zero set of a two-variate polynomial
of degree 3. 

The following files contain contain randomly generated cubics. The
files differ in the number of contained cubics.
SeveraEach instance of one file inlcudes the same number of conics. Note 
that a generic cubic is usally unbounded.
\ccBenchmarkInstance{data/Curve/Cubics/random/rand*.bff}

This set contains instances whose cubics have increasing bitlenght of the
coefficients.
\ccBenchmarkInstance{data/Curve/Cubics/bitgrowth/rand*.bff}


\subsection{Curves of Arbitrary Degree\label{bi_ssec:AlgebraicCurve2}}

There are also initial instances for algebraic curves of arbitrary degree.
The attached archive
lists selected curves that have interesting features like covertical 
$x$-extremal points, several cusps, high-degree singularities, et cetera.

\ccBenchmarkInstance{data/Curve/ArbitraryDegreeCurves/examples/}

Full families of instances will be provided in the future. Until then,
one could combine single curves to pairs or sets to analyse the intersection
or to compute an arrangement of curves of arbitrary degree.

