\section{Polynomial Data}

Polynonomials form fundamenatal mathematical objects, especially
in algebraic computation geometry. Several algorithms need to be implemented
when defining curves and surfaces as zero sets of polynomials. Several
problems can also be reduced to the univariate case, i.e., using elimination
theory. As problems real root isolation, or gcd computations can be
analysed using these data sets.

\subsection{Random}

Univariate random polynomials are provided in two sets. Both sets contain
polynomials of degree up to 8, but with huge coefficients.

The coefficient of this set are integral with bitlengths varying from
200 to 2000. 

\ccBenchmarkInstance{Polynomial/Polynomial_1/Integer/random/}

The coefficient of this sets are square-root extensions (all adjoinded
by the same root). Their bitlengths vary from 100 to 1000 bits.

\ccBenchmarkInstance{Polynomial/Polynomial_1/Sqrt_extension/random/}

\subsection{Resultants}

The following instances collect univariate polynomials
of different size (number of polynomials, bitsize) that appear
when computing intersections of quadrics using resultants.

\ccBenchmarkInstance{Polynomial/Polynomial_1/Sqrt_extension/quadric_resultants/}

\subsection{Pairs of Bivariate Polynomials}

Concerning bivariate polynomials we list instances of various pairs.
\begin{itemize}
\item \ccBenchmarkInstance{Polynomial/Polynomial_2_pair/ci.bff}
\item \ccBenchmarkInstance{Polynomial/Polynomial_2_pair/di.bff}
\item \ccBenchmarkInstance{Polynomial/Polynomial_2_pair/mi.bff}
\item \ccBenchmarkInstance{Polynomial/Polynomial_2_pair/ri.bff}
\item \ccBenchmarkInstance{Polynomial/Polynomial_2_pair/wi.bff}
\end{itemize}

%labels

