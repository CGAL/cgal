\chapter{Arrangement\_2\label{bi_cha:Arrangement2}}

A two-dimensional arrangement is the subdivision of the plane into
maximal connected 0-dimensional (vertices), 1-dimensional (edges), 
and 2-dimensional (faces) cells induced by a set of input objects, like
curves. {\sc Cgal}'s arrangement package provides a mature implementation
to compute arrangements of curves, incrementally and by the sweep-line
paradigm. This chapter collects instances to benchmark the 
{\tt Arrangement\_2} of {\sc Cgal}.

\section{Circles\label{bi_sec:Arrangement2Circles}}

Circles are the simples non-linear objects in 2D. 

\subsection{Random data\label{bi_sec:Arrangement2CirclesRandomdata}}


\section{VLSI-Data\label{bi_sec:Arrangement2VLSI}}

VLSI is used in circuit planning. The involved arrangements contain
circles and segments.

\section{Conics\label{bi_sec:Arrangement2Conics}}

A conic is defined by the zero set of a two-variate polynomial
of degree 2. A circle is a degenerated conic.

\subsection{Random data\label{bi_sec:Arrangement2ConicsRandomdata}}
