% +------------------------------------------------------------------------+
% | Reference manual page: Event.tex
% +------------------------------------------------------------------------+
% | 20.03.2005   Author
% | Package: Kinetic_data_structures
% | 
\RCSdef{\RCSEventRev}{$Id: InstantaneousKernel.tex 29411 2006-03-12 07:28:13Z drussel $}
\RCSdefDate{\RCSEventDate}{$Date: 2006-03-12 08:28:13 +0100 (Sun, 12 Mar 2006) $}
% |
%%RefPage: end of header, begin of main body
% +------------------------------------------------------------------------+


\begin{ccRefConcept}{Kinetic::InstantaneousKernel}

%% \ccHtmlCrossLink{}     %% add further rules for cross referencing links
%% \ccHtmlIndexC[concept]{} %% add further index entries

\ccDefinition
  
The concept \ccRefName\ covers models that act as adaptors allowing
CGAL static data structures to act on snapshots of kinetic data. It
typically only supports one type of moving object.

Currently, each model is created for one particular type of kinetic
primitive. The primitives are identified by
\ccc{Kinetic::ActiveObjectsTable::Key} objects.

\ccTypes 

\ccNestedType{Time}{The type used to represent the current time. This must be a ring or field type.}

\ccCreationVariable{a}  %% choose variable name

\ccOperations

\ccMethod{Time time();}{Return the current time.}

\ccMethod{void set_time(Time);}{Set the current time to have a certain value. All existing predicates are updated automatically.}

\ccMethod{Static_object to_static(Key);}{Return a static object corresponding to the kinetic object at this instant in time.}

\ccHasModels

\ccc{Kinetic::Cartesian_instantaneous_kernel}





\end{ccRefConcept}

% +------------------------------------------------------------------------+
%%RefPage: end of main body, begin of footer
% EOF
% +------------------------------------------------------------------------+

