% +------------------------------------------------------------------------+
% | Reference manual page: Simulator.tex
% +------------------------------------------------------------------------+
% | 20.03.2005   Daniel Russel
% | Package: Kinetic_data_structures
% | 
\RCSdef{\RCSSimulatorRev}{$Id: Simulatorclass.tex 29475 2006-03-14 01:47:13Z drussel $}
\RCSdefDate{\RCSSimulatorDate}{$Date: 2006-03-14 02:47:13 +0100 (Tue, 14 Mar 2006) $}
% |
%%RefPage: end of header, begin of main body
% +------------------------------------------------------------------------+


\begin{ccRefClass}{Kinetic::Simulator<FunctionKernel, EventQueue>}  %% add template arg's if necessary

%% \ccHtmlCrossLink{}     %% add further rules for cross referencing links
%% \ccHtmlIndexC[class]{} %% add further index entries

\ccDefinition
  
The class \ccRefName\ controls kinetic data structures by maintaining
a concept of time and ensuring that events are processed when
necessary. 

\ccInclude{CGAL/Kinetic/Simulator.h}

\ccIsModel

\ccc{Kinetic::Simulator}.


\ccCreation
\ccCreationVariable{sim}  %% choose variable name

\ccConstructor{Simulator(const Time start=Time(0), const Time end= Time::infinity());}{Construct a \ccRefName\ which will process events between times start and end (events outside this window will be discarded).}


\end{ccRefClass}

% +------------------------------------------------------------------------+
%%RefPage: end of main body, begin of footer
% EOF
% +------------------------------------------------------------------------+

