% +------------------------------------------------------------------------+
% | Reference manual page: Event.tex
% +------------------------------------------------------------------------+
% | 20.03.2005   Author
% | Package: Kinetic_data_structures
% | 
\RCSdef{\RCSEventRev}{$Id: Event.tex 28517 2006-02-14 23:14:42Z drussel $}
\RCSdefDate{\RCSEventDate}{$Date: 2006-02-14 15:14:42 -0800 (Tue, 14 Feb 2006) $}
% |
%%RefPage: end of header, begin of main body
% +------------------------------------------------------------------------+


\begin{ccRefConcept}{Kinetic::Certificate}

%% \ccHtmlCrossLink{}     %% add further rules for cross referencing links
%% \ccHtmlIndexC[concept]{} %% add further index entries

\ccDefinition
  
The concept \ccClassName\ represents certificate. Its main purpose is
to provide a way of creating \ccc{Time} objects corresponding to when
the certificate fails and to cache any useful work done in find the
\ccc{Time} for later.

\ccOperations

\ccCreationVariable{c}

\ccMethod{Time failure_time();}{Returns the next failure time.}

\ccMethod{void pop_failure_time();}{Advances to the next failure time (the next root of the certificate functions).}


\ccSeeAlso

Kinetic::Kernel

\end{ccRefConcept}

% +------------------------------------------------------------------------+
%%RefPage: end of main body, begin of footer
% EOF
% +------------------------------------------------------------------------+

