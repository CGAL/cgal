% +------------------------------------------------------------------------+
% | Reference manual page: Event.tex
% +------------------------------------------------------------------------+
% | 20.03.2005   Author
% | Package: Kinetic_data_structures
% | 
\RCSdef{\RCSEventRev}{$Revision$}
\RCSdefDate{\RCSEventDate}{$Date$}
% |
%%RefPage: end of header, begin of main body
% +------------------------------------------------------------------------+


\begin{ccRefConcept}{InstantaneousKernel}

%% \ccHtmlCrossLink{}     %% add further rules for cross referencing links
%% \ccHtmlIndexC[concept]{} %% add further index entries

\ccDefinition
  
The concept \ccRefName\ covers models that act as adaptors allowing CGAL static data structures to act on snapshots of kinetic data. It typically only supports one type of moving object.

\ccTypes 

\ccNestedType{Time}{The type used to represent the current time. This must be a ring or field type.}

\ccCreationVariable{a}  %% choose variable name

\ccOperations

\ccMethod{Time time();}{Return the current time.}
\ccMethod{void set_time(Time);}{Set the current time to have a certain value. All existing predicates are updated automatically.}

Some of the debugging operations involve writing events to standard
out or to a log file. This operation should be supported if you wish
to use the debugging aids.

\ccHasModels

\ccc{CGAL::KDS::Cartesian_instantaneous_kernel}





\end{ccRefConcept}

% +------------------------------------------------------------------------+
%%RefPage: end of main body, begin of footer
% EOF
% +------------------------------------------------------------------------+

