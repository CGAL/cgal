\section{Todo}

\subsection{General thoughts}

\begin{itemize}

\item add reference pages for traits

\item implemented a shared version of polynomials-- copying them all around for the roots is quite expensive

\item current\_time\_nt() instead of rational

\item add motion vector arrows to \ccc{moving_points_2}

\item reread manual

\item restructure examples to illustrate separate parts (listener for example).

\item there might be a bug in Simple\_interval\_root due to its use of
  \ccc{approximate_interval_width} (which uses doubles and so might never

\item there might be a bug in \ccc{Simple_interval_root} due to its use of
  \ccc{approximate_interval_width} (which uses doubles and so might never
  get small enough).

\item The other trajectory modification events need to be documented
  (Set\_moving\_point). 

\item There is no 2d regular triangulation. This is not too much work,
  but a bit (since I will need to refactor my 2d delaunay)

\item the FunctionKernel is not documented much (the options are never
  explained)

\item The FunctionKernel numeric solver I implemented is not as good
  as the one provided by GSL which is shipped on most linux boxes. It
  is GPL, so the user does have to make a decision about using it.
  Currently, there is a traits class that the user can select if they
  want to use it (at the cost of reassembling the SimulationTraits).

\end{itemize}

%%% Local Variables: 
%%% mode: latex
%%% TeX-master: t
%%% End: 
