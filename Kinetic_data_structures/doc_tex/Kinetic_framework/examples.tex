\section{Examples\label{sec:kds_examples}}


We provide a number of examples of different levels of usage of the
kinetic data structures framework, both for kinetic data structures as
well as sweepline algorithms.

To see how to use existing kinetic data structures, look at the
examples in the previous chapter such as
Section~\ref{sec:kds_sort_example}. 

The here we cover implementing kinetic data structures. The examples
explained are
\begin{itemize}
\item A trivial kinetic data structure which has all the parts of a
full kinetic data structure but doesn't do much in
Section~\ref{sec:kds_trivial_example}.
\item Adding a new type of certificate to a kernel in
  Section~\ref{sec:kds_add_cert}.
\end{itemize}

In order to see more detail about how to implement a kinetic data
structure, the best place to start is the source code for the kinetic
sorting data structure, \ccc{Kinetic::Sort<Traits, Visitor>}. Once you
are familiar with that, \ccc{Kinetic::Delaunay_2<Traits,
Triangulation, Visitor>} is the next step in complexity.



\subsection{Implementing a Trivial Kinetic Data Structure}
\label{sec:trivial_kds_example}

 This kinetic data structure maintains one event in the queue which
has a list of the coordinates of all the known moving objects.  Two
classes are defined, the \ccc{Trivial_event}, and the
\ccc{Trivial_kds}. The event classes must be declared outside of the
kinetic data structure so that the \ccc{operator<<} can be defined for
them. 

The kinetic data structure maintains one event containing a list of
the trajectories of all objects in the the simulation. This
event must updated whenever any objects change, in addition, it is
always created to fail one time unit in the future, so it must be
recreated when it fails. As a result, the kinetic data structure has
the main parts of a real one--it responds to changes in trajectories
of the objects and certificate failures (when the event expires).

 The public methods can be grouped into three sets which are shared
 with almost all other kinetic data structures:
\begin{itemize}
\item \ccc{has_certificates} and \ccc{set_has_certificates} which
  checks/sets whether the kinetic data structure is currently
  maintaining certificates.
\item \ccc{insert}, \ccc{set}, \ccc{erase} which are called by the
  \ccc{CGAL::KDS::Notifying_table_listener_helper} in response to the
  addition, modification, or deletion of an object to, in or from the
  simulation.
\item \ccc{audit} which is called periodically by the
  \ccc{CGAL::KDS::Simulator_kds_listener} when kinetic data structures can
  easily audit themselves.
\end{itemize} 

In addition, it has one method which is called when a certificate
fails. The name/existence of such methods depend on the nature of the
kinetic data structure in question.

Like many kinetic data structures, it takes a \ccc{SimulationTraits}
as a template argument. This traits class defines the types needed for
the simulation and is responsible for instantiating them.

 \label{fig:trivial_usage_program}
\ccIncludeExampleCode{Kinetic_data_structures/trivial_kds.cc}
%../../examples/Kinetic_data_structures/ % LocalWords: CGAL

\subsection{Adding a New Certificate Type\label{sec:kds_add_cert}}


The following example shows how to add a new type of certificate to a simulation. 

First we code the actual certificate function generator. It must take
some sort (or sorts) of kinetic primitives, compute some function from
their coordinates.

\begin{ccExampleCode}
template <class KineticKernel>
struct Positive_x_f_2 {
  typedef typename KineticKernel::Certificate_function result_type;
  typedef typename KineticKernel::Point_2 argument_type;
  result_type operator()(const argument_type &p){
    return result_type(p.x()- result_type(0));
  }
};
\end{ccExampleCode}

Then we define a kinetic kernel which includes this predicate. To do
this we wrap the function generator generator in a
\ccc{Kinetic::Certificate_generator<Kernel, Generator>}.
This wrapper uses the generator to create the certificate function and
then the \ccc{Kinetic::FunctionKernel} to solve the certificate
function. The result is wrapped in a \ccc{Kinetic::Certificate}
object.

\begin{ccExampleCode}
template <class FunctionKernel> 
class My_kinetic_kernel:
  public CGAL::Kinetic::Cartesian<FunctionKernel> {
  typedef CGAL::Kinetic::Cartesian<FunctionKernel> P;
  typedef My_kinetic_kernel<FunctionKernel> This;
public:
  typedef CGAL::Kinetic::internal::Certificate_generator<This, Positive_x_f_2<This> > Positive_x_2;
  Positive_x_2 positive_x_2_object() const
  {
    return Positive_x_2(P::function_kernel_object());
  }
};
\end{ccExampleCode}

Now we have the unfortunately rather messy part of assembling a new
\ccc{Kinetic::SimulationTraits} model. This is done in two steps for convenience.

\begin{ccExampleCode}
struct My_simulation_traits {
  typedef My_simulation_traits This;

  typedef CGAL::Exact_predicates_exact_constructions_kernel Static_kernel;
  //typedef CGAL::Regular_triangulation_euclidean_traits_3<Static_kernel_base> Static_kernel;
  typedef CGAL::POLYNOMIAL::Polynomial<Static_kernel::FT> Function;
  typedef CGAL::POLYNOMIAL::Sturm_root_stack_traits<Function> Root_stack_traits;
  typedef CGAL::POLYNOMIAL::Sturm_root_stack<Root_stack_traits> Root_stack;
  typedef CGAL::POLYNOMIAL::Kernel<Function, Root_stack> Function_kernel;

  typedef CGAL::Kinetic::Handle_degeneracy_function_kernel<Function_kernel, false>  Simulator_function_kernel_base;
  struct Simulator_function_kernel: public Simulator_function_kernel_base{};

  typedef My_kinetic_kernel<Simulator_function_kernel> Kinetic_kernel;
  typedef CGAL::Kinetic::Two_list_pointer_event_queue<Function_kernel> Event_queue;
  typedef CGAL::Kinetic::Default_simulator<Simulator_function_kernel, Event_queue > Simulator;

  typedef CGAL::Kinetic::Active_objects_vector<Kinetic_kernel::Point_1> Active_points_1_table;
  typedef CGAL::Kinetic::Active_objects_vector<Kinetic_kernel::Point_2> Active_points_2_table;
  typedef CGAL::Kinetic::Active_objects_vector<Kinetic_kernel::Point_3> Active_points_3_table;
  // typedef Active_objects_vector<Kinetic_kernel::Weighted_point_3> Active_weighted_points_3_table;
 
  typedef CGAL::Kinetic::Default_instantaneous_kernel<This> Instantaneous_kernel;

  Active_points_1_table* active_points_1_table_handle() const { return ap1_.get();}
  Active_points_2_table* active_points_2_table_handle() const {return ap2_.get();}
  Active_points_3_table* active_points_3_table_handle() const {return ap3_.get();}
  //Active_weighted_points_3_table* active_weighted_points_3_table_handle() const {return awp3_.get();}

  Simulator* simulator_handle() const { return sim_.get();}
  const Static_kernel& static_kernel_object() const {return k_;}
  const Kinetic_kernel& kinetic_kernel_object() const {return kk_;}
 
  Instantaneous_kernel instantaneous_kernel_object() const {
    return Instantaneous_kernel(*this);
  }

  My_simulation_traits(const Simulator::Time &lb,
		       const Simulator::Time &ub): sim_(new Simulator(lb, ub)),
						   ap1_(new Active_points_1_table()),
						   ap2_(new Active_points_2_table()),
						   ap3_(new Active_points_3_table())
  {}
 
  
  bool is_exact() const {
    return true;
  }
protected:
  Simulator::Handle sim_;
  Active_points_1_table::Handle ap1_;
  Active_points_2_table::Handle ap2_;
  Active_points_3_table::Handle ap3_;
  //Active_weighted_points_3_table::Handle awp3_;
  Static_kernel k_;
  Kinetic_kernel kk_;
  Function_kernel fk_;
};

\end{ccExampleCode}

Now the simulation traits can be used by a kinetic data structure.
Note that we define active point table for all dimensions. This is
needed by the \ccc{Kinetic::InstantaneousKernel}, even if they are not
used.


