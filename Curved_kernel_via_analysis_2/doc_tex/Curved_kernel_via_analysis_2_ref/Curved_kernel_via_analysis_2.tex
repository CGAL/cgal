% +------------------------------------------------------------------------+
% | Reference manual page: Curved_kernel_via_analysis_2.tex
% +------------------------------------------------------------------------+
% | 27.03.2008   Author
% | Package: Curved_kernel_via_analysis_2
% |
\RCSdef{\RCSCurvedkernelviaanalysisRev}{$Id: header.tex 40270 2007-09-07 15:29:10Z lsaboret $}
\RCSdefDate{\RCSCurvedkernelviaanalysisDate}{$Date: 2007-09-07 17:29:10 +0200 (Fri, 07 Sep 2007) $}
% |
\ccRefPageBegin
%%RefPage: end of header, begin of main body
% +------------------------------------------------------------------------+


\begin{ccRefClass}{Curved_kernel_via_analysis_2}  %% add template arg's if necessary
\ccRefLabel{Curved_kernel_via_analysis_2}

%% \ccHtmlCrossLink{}     %% add further rules for cross referencing links
%% \ccHtmlIndexC[class]{} %% add further index entries

\ccDefinition
  
%The class \ccRefName\ does this and that.

\ccInclude{Curved_kernel_via_analysis_2.h}

% The section below is automatically generated. Do not edit!
%START-AUTO(\ccDefinition)

Kernel for curves in a two-dimensional space, and points and arcs of them.

It expects a \ccc{CurveKernel_2} type that fulfills the \ccc{CurveKernel_2} concept. The other template parameters can be left default. They are used during rebind to exchange certain point and arc type.

Is a model of CGAL's \ccc{ArrangementTraits_2} concept.

%END-AUTO(\ccDefinition)


\ccParameters

% The section below is automatically generated. Do not edit!
%START-AUTO(\ccParameters)

template$<$  \\
class \ccc{CurveKernel_2},   \\
class \ccc{CKvA_} = void,   \\
class \ccc{Point_} = void,   \\
class \ccc{Arc_} = void$>$   \\
class \ccc{Curved_kernel_via_analysis_2};

%END-AUTO(\ccParameters)


\ccIsModel

\ccTypes

%\ccCreation
%\ccCreationVariable{a}  %% choose variable name

%% \ccIncludeExampleCode{Curved_kernel_via_analysis_2/Curved_kernel_via_analysis_2.C}

% The section below is automatically generated. Do not edit!
%START-AUTO(\ccTypes)

\ccNestedType{Curve_kernel_2}
{
this instance's template argument
}
\ccGlue
\ccNestedType{CKvA}
{
[protected] \\
this instance's second template parameter
}
\ccGlue
\ccNestedType{Point}
{
[protected] \\
this instance's third template parameter
}
\ccGlue
\ccNestedType{Arc}
{
[protected] \\
this instance's fourth template parameter
}
\ccGlue
\ccNestedType{Self}
{
[protected] \\
this instance itself
}
\ccGlue
\ccNestedType{CKvA_2}
{
[protected] \\
type of \ccc{CKvA_2} used internally
}
\ccGlue
\ccNestedType{Curve_2}
{
type of curve that can be analyzed
}
\ccGlue
\ccNestedType{Point_2}
{
type of a point on a curve that can be analyzed
}
\ccGlue
\ccNestedType{Arc_2}
{
type of an arc on a curve that can be analyzed
}
\ccGlue
\ccNestedType{X_monotone_curve_2}
{
type of weakly x-monotone arc for \ccc{ArrangementTraits_2}
}
\ccGlue
\ccNestedType{Base_kernel}
{
[protected] \\
class collecting basic types
}
\ccGlue
\ccNestedType{Base_functors}
{
[protected] \\
class collecting basic functors
}
\ccGlue
\ccNestedType{Construct_point_2}
{
functor
}
\ccGlue
\ccNestedType{Construct_point_on_arc_2}
{
functor
}
\ccGlue
\ccNestedType{Construct_arc_2}
{
functor
}
\ccGlue
\ccNestedType{Curved_kernel_via_analysis_2}
{
[inherited] \\
the \ccc{Curved_kernel_via_analysis_2}
}
\ccGlue
\ccNestedType{Has_left_category}
{
[inherited] \\
tag specifies that {\em to the left of} comparisons supported
}
\ccGlue
\ccNestedType{Has_merge_category}
{
[inherited] \\
tag specifies that merge and split functors supported
}
\ccGlue
\ccNestedType{Has_boundary_category}
{
[inherited] \\
tag specifies that unbounded arcs supported
}
\ccGlue
\ccNestedType{Boundary_category}
{
[inherited] \\
tag specifies which boundary functors are implemented
}
\ccGlue
\ccNestedType{Curve_interval_arcno_cache}
{
[inherited] \\
type of inverval arcno cache
}
\ccGlue
\ccNestedType{Curve_analysis_2}
{
[protected, inherited] \\
provides analysis of a single curve
}
\ccGlue
\ccNestedType{Curve_pair_analysis_2}
{
[protected, inherited] \\
provides analysis of a pair of curves
}
\ccGlue
\ccNestedType{Curved_kernel_via_analysis_2}
{
[inherited] \\
this instance's first template parameter
}
\ccGlue
\ccNestedType{Compare_x_2}
{
[inherited] \\
functor
}
\ccGlue
\ccNestedType{Compare_xy_2}
{
[inherited] \\
functor
}
\ccGlue
\ccNestedType{Is_vertical_2}
{
[inherited] \\
functor
}
\ccGlue
\ccNestedType{Is_bounded_2}
{
[inherited] \\
functor
}
\ccGlue
\ccNestedType{Parameter_space_in_x_2}
{
[inherited] \\
functor
}
\ccGlue
\ccNestedType{Parameter_space_in_y_2}
{
[inherited] \\
functor
}
\ccGlue
\ccNestedType{Construct_min_vertex_2}
{
[inherited] \\
functor
}
\ccGlue
\ccNestedType{Construct_max_vertex_2}
{
[inherited] \\
functor
}
\ccGlue
\ccNestedType{Compare_x_near_boundary_2}
{
[inherited] \\
functor
}
\ccGlue
\ccNestedType{Compare_y_near_boundary_2}
{
[inherited] \\
functor
}
\ccGlue
\ccNestedType{Compare_y_at_x_2}
{
[inherited] \\
functor
}
\ccGlue
\ccNestedType{Compare_y_at_x_left_2}
{
[inherited] \\
functor
}
\ccGlue
\ccNestedType{Compare_y_at_x_right_2}
{
[inherited] \\
functor
}
\ccGlue
\ccNestedType{Equal_2}
{
[inherited] \\
functor
}
\ccGlue
\ccNestedType{Is_in_x_range_2}
{
[inherited] \\
functor
}
\ccGlue
\ccNestedType{Do_overlap_2}
{
[inherited] \\
functor
}
\ccGlue
\ccNestedType{Intersect_2}
{
[inherited] \\
functor
}
\ccGlue
\ccNestedType{Trim_2}
{
[inherited] \\
functor
}
\ccGlue
\ccNestedType{Split_2}
{
[inherited] \\
functor
}
\ccGlue
\ccNestedType{Are_mergeable_2}
{
[inherited] \\
functor
}
\ccGlue
\ccNestedType{Merge_2}
{
[inherited] \\
functor
}
\ccGlue
\ccNestedType{Is_on_2}
{
[inherited] \\
functor
}
\ccGlue
\ccNestedType{Make_x_monotone_2}
{
[inherited] \\
functor
}
\ccGlue

%END-AUTO(\ccTypes)

\ccHeading{Variables}

% The section below is automatically generated. Do not edit!
%START-AUTO(\ccHeading{Variables})

\ccVariable{Curve_kernel_2 _m_kernel;}
{
[protected, inherited] \\
an instance of \ccc{Curve_kernel_2}
}
\ccGlue
\ccVariable{Curve_interval_arcno_cache _m_interval_arcno_cache;}
{
[mutable, protected, inherited] \\
an instance of \ccc{Curve_interval_arcno_cache}
}
\ccGlue

%END-AUTO(\ccHeading{Variables})

\ccCreation
\ccCreationVariable{a}  %% choose variable name for \ccMethod below

% The section below is automatically generated. Do not edit!
%START-AUTO(\ccCreation)

\ccConstructor{Curved_kernel_via_analysis_2();}
{
default constructor
}
\ccGlue
\ccConstructor{Curved_kernel_via_analysis_2(const Curve_kernel_2& kernel);}
{
construct from kernel
}
\ccGlue
\begin{description}
\item[Parameters:]
\begin{description}
\item[kernel]Kernel to use internally \end{description}
\end{description}
\ccGlue

%END-AUTO(\ccCreation)

\ccOperations

% The section below is automatically generated. Do not edit!
%START-AUTO(\ccOperations)

\ccMethod{Construct_point_2 construct_point_2_object() const;}
{
returns instance of functor
}
\ccGlue
\ccMethod{Construct_point_on_arc_2 construct_point_on_arc_2_object() const;}
{
returns instance of functor
}
\ccGlue
\ccMethod{Construct_arc_2 construct_arc_2_object() const;}
{
returns instance of functor
}
\ccGlue
\ccMethod{Curved_kernel_via_analysis_2_base();}
{
[inherited] \\
default constructor
}
\ccGlue
\ccMethod{Curved_kernel_via_analysis_2_base(const Curve_kernel_2& kernel);}
{
[inherited] \\
construct using specific \ccc{Curve_kernel_2} instance kernel
}
\ccGlue
\ccMethod{const Curve_interval_arcno_cache& interval_arcno_cache() const;}
{
[inherited] \\
access to static \ccc{Curve_interval_arcno_cache}
}
\ccGlue
\ccMethod{const Curve_kernel_2& kernel() const;}
{
[inherited] \\
instance of internal \ccc{Curve_kernel_2} instance
}
\ccGlue
\begin{description}
\item[Returns:]\end{description}
\ccGlue
\ccMethod{static Curved_kernel_via_analysis_2& instance();}
{
[static, inherited] \\
a default instance of \ccc{Curved_kernel_via_analysis_2}
}
\ccGlue
\begin{description}
\item[Returns:]static instance of \ccc{Curved_kernel_via_analysis_2} \end{description}
\ccGlue
\ccMethod{static Curved_kernel_via_analysis_2& set_instance(const Curved_kernel_via_analysis_2& ckva);}
{
[static, inherited] \\
sets static instance of \ccc{Curved_kernel_via_analysis_2} to ckva
}
\ccGlue
\begin{description}
\item[Parameters:]
\begin{description}
\item[ckva]The instance that should be stored \end{description}
\end{description}
\begin{description}
\item[Returns:]the stored instance \end{description}
\ccGlue
\ccMethod{static void reset_instance();}
{
[static, inherited] \\
resets static instance to original one
}
\ccGlue
\ccMethod{Compare_x_2 compare_x_2_object() const;}
{
[inherited] \\
returns instance of functor
}
\ccGlue
\ccMethod{Compare_xy_2 compare_xy_2_object() const;}
{
[inherited] \\
returns instance of functor
}
\ccGlue
\ccMethod{Is_vertical_2 is_vertical_2_object() const;}
{
[inherited] \\
returns instance of functor
}
\ccGlue
\ccMethod{Is_bounded_2 is_bounded_2_object() const;}
{
[inherited] \\
returns instance of functor
}
\ccGlue
\ccMethod{Parameter_space_in_x_2 parameter_space_in_x_2_object() const;}
{
[inherited] \\
returns instance of functor
}
\ccGlue
\ccMethod{Parameter_space_in_y_2 parameter_space_in_y_2_object() const;}
{
[inherited] \\
returns instance of functor
}
\ccGlue
\ccMethod{Construct_min_vertex_2 construct_min_vertex_2_object() const;}
{
[inherited] \\
returns instance of functor
}
\ccGlue
\ccMethod{Construct_max_vertex_2 construct_max_vertex_2_object() const;}
{
[inherited] \\
returns instance of functor
}
\ccGlue
\ccMethod{Compare_x_near_boundary_2 compare_x_near_boundary_2_object() const;}
{
[inherited] \\
returns instance of functor
}
\ccGlue
\ccMethod{Compare_y_near_boundary_2 compare_y_near_boundary_2_object() const;}
{
[inherited] \\
returns instance of functor
}
\ccGlue
\ccMethod{Compare_y_at_x_2 compare_y_at_x_2_object() const;}
{
[inherited] \\
returns instance of functor
}
\ccGlue
\ccMethod{Compare_y_at_x_left_2 compare_y_at_x_left_2_object() const;}
{
[inherited] \\
returns instance of functor
}
\ccGlue
\ccMethod{Compare_y_at_x_right_2 compare_y_at_x_right_2_object() const;}
{
[inherited] \\
returns instance of functor
}
\ccGlue
\ccMethod{Equal_2 equal_2_object() const;}
{
[inherited] \\
returns instance of functor
}
\ccGlue
\ccMethod{Is_in_x_range_2 is_in_x_range_2_object() const;}
{
[inherited] \\
returns instance of functor
}
\ccGlue
\ccMethod{Do_overlap_2 do_overlap_2_object() const;}
{
[inherited] \\
returns instance of functor
}
\ccGlue
\ccMethod{Intersect_2 intersect_2_object() const;}
{
[inherited] \\
returns instance of functor
}
\ccGlue
\ccMethod{Trim_2 trim_2_object() const;}
{
[inherited] \\
returns instance of functor
}
\ccGlue
\ccMethod{Split_2 split_2_object() const;}
{
[inherited] \\
returns instance of functor
}
\ccGlue
\ccMethod{Are_mergeable_2 are_mergeable_2_object() const;}
{
[inherited] \\
returns instance of functor
}
\ccGlue
\ccMethod{Merge_2 merge_2_object() const;}
{
[inherited] \\
returns instance of functor
}
\ccGlue
\ccMethod{Is_on_2 is_on_2_object() const;}
{
[inherited] \\
returns instance of functor
}
\ccGlue
\ccMethod{Make_x_monotone_2 make_x_monotone_2_object() const;}
{
[inherited] \\
returns instance of functor
}
\ccGlue

%END-AUTO(\ccOperations)

\end{ccRefClass}

% +------------------------------------------------------------------------+
%%RefPage: end of main body, begin of footer
\ccRefPageEnd
% EOF
% +------------------------------------------------------------------------+

