% +------------------------------------------------------------------------+
% | Reference manual page: MeshPolyline_3.tex
% +------------------------------------------------------------------------+
% | 28.07.2009   Stephane Tayeb
% | Package: Mesh_3
% |
\RCSdef{\RCSMeshPolylineRev}{$Id$}
\RCSdefDate{\RCSMeshPolylineDate}{$Date$}
% |
\ccRefPageBegin
%%RefPage: end of header, begin of main body
% +------------------------------------------------------------------------+


\begin{ccRefConcept}{MeshPolyline_3}

%% \ccHtmlCrossLink{}     %% add further rules for cross referencing links
%% \ccHtmlIndexC[concept]{} %% add further index entries

\ccDefinition
  
The concept \ccRefName\ implements a container of points designed to represent a polyline (i.e. a sequence of points).
Types and functions provided in this concept are such as standard template library containers 
are natural models of this concept.

\ccTypes

\ccNestedType{value_type}{Point type. Must match the type \ccc{MeshDomain_3::Point_3}.}
\ccNestedType{const_iterator}{A constant iterator on points. Must be a model of Bidirectional iterator and have \ccc{value_type} as value type.}

\ccCreationVariable{polyline}  %% choose variable name

\ccOperations

\ccMethod{const_iterator begin();}{Returns an iterator on the first point of the polyline.}
\ccMethod{const_iterator end();}{Returns the past-the-end iterator for the above iterator.}

\ccHasModels

\ccc{std::vector<Kernel::Point_3>} for any Kernel of \cgal{} is a natural model of this concept.

\ccSeeAlso

\ccRefIdfierPage{CGAL::Mesh_domain_with_polyline_features_3<MeshDomain>}

\end{ccRefConcept}

% +------------------------------------------------------------------------+
%%RefPage: end of main body, begin of footer
\ccRefPageEnd
% EOF
% +------------------------------------------------------------------------+

