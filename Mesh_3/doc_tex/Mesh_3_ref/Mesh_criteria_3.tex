% +------------------------------------------------------------------------+
% | Reference manual page: Mesh_criteria_3.tex
% +------------------------------------------------------------------------+
% | 16.02.2009   Author
% | Package: Package
% |
\RCSdef{\RCSMeshdefaultcriteriaRev}{$Id: header.tex 40270 2007-09-07 15:29:10Z lsaboret $}
\RCSdefDate{\RCSMeshdefaultcriteriaDate}{$Date: 2007-09-07 17:29:10 +0200 (Ven, 07 sep 2007) $}
% |
\ccRefPageBegin
%%RefPage: end of header, begin of main body
% +------------------------------------------------------------------------+


\begin{ccRefClass}{Mesh_criteria_3<Tr>}  %% add template arg's if necessary

%% \ccHtmlCrossLink{}     %% add further rules for cross referencing links
%% \ccHtmlIndexC[class]{} %% add further index entries

\ccDefinition
  
The class \ccRefName\ is a model of both concepts \ccc{MeshCriteria_3}
and \ccc{MeshCriteriaWithFeatures_3}.
It gathers the refinement criteria for mesh tetrahedra and
surface facets where 
surface facets  are facets in the mesh approximating the domain surface patches.
In addition, for domain with exposed 1-dimensional features, 
the class  \ccRefName\ 
 handles the definition of  a sizing field to guide the discretization of 
 1-dimensional features.

\ccInclude{CGAL/Mesh_criteria_3.h}

\ccParameters
The parameter \ccc{Tr} has to be instantiated with the type used for
\ccc{C3T3::Triangulation},  
where \ccc{C3T3} is the model of \ccc{MeshComplex_3InTriangulation_3}
used in the mesh generation process,
and  \ccc{C3T3::Triangulation} its nested triangulation type.


\ccIsModel
\ccc{MeshCriteria_3}


\ccTypes

\ccTypedef{typedef Mesh_edge_criteria_3<Tr>
  Edge_criteria;}{The criteria for edges.}
\ccGlue
\ccTypedef{typedef Mesh_facet_criteria_3<Tr>
  Facet_criteria;}{The criteria for facets.}
\ccGlue
\ccTypedef{typedef Mesh_cell_criteria_3<Tr> Cell_criteria;}{The
  criteria for cells.}


\ccCreation
\ccCreationVariable{mc}  %% choose variable name

\ccConstructor{Mesh_criteria_3(
         Facet_criteria facet_criteria,
         Cell_criteria  cell_criteria);}
{Construction from facet and cell criteria. The edge criteria are ignored
in this case.}

\ccConstructor{Mesh_criteria_3(
				 Edge_criteria edge_criteria,
         Facet_criteria facet_criteria,
         Cell_criteria  cell_criteria);}
{Construction from edge, facet and cell criteria.}

\ccConstructor{
template<typename FT,
         typename ...Fieldi>
%         typename Field2,
%         typename Field3,
%         typename Field4,
%         typename Field5,
%         typename Field6>
Mesh_criteria_3(
         Field1 parameters::edge_size = ignored,
         FT parameters::facet_angle = ignored,
         Field2 parameters::facet_size = ignored,
         Field3 parameters::facet_distance = ignored,
         Mesh_facet_topology parameters::facet_topology = CGAL::FACET_VERTICES_ON_SURFACE,
         FT parameters::cell_radius_edge_ratio = ignored,
         Field4 parameters::cell_size = ignored);}
%         Field1 parameters::edge_sizing_field = ignored,
%         Field2 parameters::edge_distance_field = ignored,
%         Field3 parameters::facet_sizing_field = ignored,
%         Field4 parameters::facet_distance_field = ignored,
%         Field5 parameters::cell_sizing_field = ignored,
%         Field6 parameters::sizing_field = ignored,
%         Field7 parameters::distance_field = ignored);}
{Construction from  criteria  parameters. This constructor uses named
  parameters (from \emph{Boost.Parameter}) for convenient criteria
  construction. See a complete description of these parameters below.}

The template parameter \ccc{FT} should be a model of concept \ccc{FieldType}. 
The template parameters  \ccc{Fieldi} ($i\in\{1..4\}$) should be either a model
of the concept \ccc{FieldType} or a model of the concept \ccc{MeshDomainField_3}.

The parameters are named parameters 
and can be passed in any order provided their name is given (see example below).
The name of each parameter is the one that is written in the description of
the function (e.g. \ccc{parameters::facet_size}).



The description of each parameter is as follows:

%-\ccc{edge_size}: a uniform  upper bound for the lengths of curve segment edges.\\
-\ccc{edge_size}: a scalar field (resp. a constant) providing a space varying 
  (resp. a uniform)
  upper bound for the lengths of curve segment edges.
%-\ccc{edge_distance}: a uniform upper bound for the distance between the
%center of a  curve segment edge and the corresponding curve segment.\\
%-\ccc{edge_distance_field}: a scalar field  describing  a space varying
%  upper bound for the same distance.\\

-\ccc{facet_angle}: a lower bound for the angles (in degrees) of the
surface mesh facets.\\
%-\ccc{facet_size}: a uniform upper bound for the radii of the surface Delaunay balls.\\
-\ccc{facet_size}: a scalar field (resp. a constant) describing
 a space varying (resp. a uniform) upper-bound or for the radii of the surface Delaunay balls. \\
%-\ccc{facet_distance}: a uniform upper bound for the distance between the center
%of a surface  facet  and the center of its surface Delaunay ball.\\
-\ccc{facet_distance}: a scalar field (resp. a constant) describing a space varying (resp. a uniform)
  upper bound for the same distance. \\
-\ccc{facet_topology}: the set of topological constraints
	which have to be verified by each surface facet. The default value is 
	\ccc{CGAL::FACET_VERTICES_ON_SURFACE}. See \ccc{Mesh_facet_topology} manual page to
	get all possible values.

-\ccc{cell_radius_edge_ratio}: an upper bound for the radius-edge ratio of the  mesh tetrahedra.\\
%-\ccc{cell_size}: a uniform  upper bound for the circumradii of the  mesh tetrahedra.\\
-\ccc{cell_size}: a scalar field (resp. a constant) describing
 a space varying (resp. a uniform) upper-bound for the circumradii of the  mesh tetrahedra.

%-\ccc{sizing_field}: a unique scalar field to be used as a space varying upper bound
%for the lengths of curve segment edges, the circumradii of surface Delaunay balls
%and the circumradii of the  mesh tetrahedra. \\
%-\ccc{distance_field}:  a unique scalar field to be used as a space varying upper bound
%for the distances between centers of a  curve segment edge and the corresponding curve segments,
%and for the distances between the centers
%of surface  facets  and the centers of their surface Delaunay ball.


Note that each size or distance parameter can be specified using two ways: either as 
scalar field or as a numerical value when the field is uniform.
%If both a  numerical value and  a scalar field are provided, only the numerical value
%is taken into account.

%The last two parameters \ccc{sizing_field} and \ccc{distance_field} are provided
%for convenience. They allow the user to set a unique sizing field 
%(resp. a unique  distance field)  to act on mesh faces of all dimensions.
% If both the specialized parameter 
%(e.g. \ccc{facet_sizing_field}) and the global parameter (e.g. \ccc{sizing_field}) are
%specified, then the specialized parameter is prioritized.

Each parameter has a special default value \ccc{ignored} which means that the 
corresponding criteria will be ignored.
 Numerical sizing or distance values, as well as scalar fields
should be given in the unit used  for coordinates of points in the mesh domain class
of the mesh generation process.

\ccExample

\begin{ccExampleCode}
// Create a Mesh_criteria_3<Tr> object with all cell and facet parameters set
Mesh_criteria_3<Tr> criteria (parameters::facet_angle=30,
                              parameters::facet_size=1,
                              parameters::facet_distance=0.1,
                              parameters::cell_radius_edge_ratio=2,
                              parameters::cell_size=1.5);

// Create a Mesh_criteria_3<Tr> object with size ignored (note that the order changed)
Mesh_criteria_3<Tr> criteria (parameters::cell_radius_edge_ratio=2,
                              parameters::facet_angle=30,
                              parameters::facet_distance=0.1);
\end{ccExampleCode}

\ccSeeAlso

\ccRefConceptPage{MeshCriteria_3} \\
\ccRefConceptPage{MeshCriteriaWithFeatures_3}\\
\ccRefConceptPage{MeshCellCriteria_3} \\
\ccRefConceptPage{MeshEdgeCriteria_3} \\
\ccRefConceptPage{MeshFacetCriteria_3}\\
\ccRefConceptPage{MeshDomainField_3}\\
\ccRefIdfierPage{CGAL::Mesh_cell_criteria_3<Tr>} \\
\ccRefIdfierPage{CGAL::Mesh_edge_criteria_3<Tr>} \\
\ccRefIdfierPage{CGAL::Mesh_facet_criteria_3<Tr>} \\
\ccRefIdfierPage{CGAL::Mesh_facet_topology}


%% \ccIncludeExampleCode{Package/Mesh_criteria_3.C}

\end{ccRefClass}

% +------------------------------------------------------------------------+
%%RefPage: end of main body, begin of footer
\ccRefPageEnd
% EOF
% +------------------------------------------------------------------------+

