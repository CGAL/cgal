% +------------------------------------------------------------------------+
% | Reference manual page: MeshComplexWithFeatures_3InTriangulation_3.tex
% +------------------------------------------------------------------------+
% | 28.07.2009   Stephane Tayeb
% | Package: Mesh_3
% |
\RCSdef{\RCSMeshComplexWithFeaturesInTriangulationRev}{$Id$}
\RCSdefDate{\RCSMeshComplexWithFeaturesInTriangulationDate}{$Date$}
% |
\ccRefPageBegin
%%RefPage: end of header, begin of main body
% +------------------------------------------------------------------------+


\begin{ccRefConcept}{MeshComplexWithFeatures_3InTriangulation_3}

%% \ccHtmlCrossLink{}     %% add further rules for cross referencing links
%% \ccHtmlIndexC[concept]{} %% add further index entries

\ccDefinition
  
The concept \ccRefName\ describes a data structure 
to represent and maintain a 3D complex embedded in a 3D triangulation.
The concept \ccRefName\  refines the minimal concept 
\ccc{MeshComplex_3InTriangulation_3}, designed to represent 
3D complexes having only faces with dimension 2 and 3.
Therefore, the concept \ccRefName\   may represent embedded complexes
including {\em features}, i.e. faces with dimension $0$ and $1$.

The data structure includes a 3D triangulation which is itself a 3D complex.
To distinguish the faces of the embedded 3D complex from the
faces of the triangulation,
 we call  
respectively {\em subdomains}, 
{\em surface patches}
{\em curve segments} and {\em corners} the faces
of the complex with respective dimensions $3$, $2$, $1$ and $0$.
The triangulations faces  are called respectively
cells, facets, edges and vertices.

Each subdomain of the embedded 3D complex is a union of 
triangulation cells.
Likewise,  each surface patch  is a union of 
triangulation facets and each curve segment is a union of triangulation edges.
The corners form a subset of the triangulation vertices.
Note that subdomains, surface patches and and curved segments are not
necessarily connected. Likewise each corner may be related to several
mesh vertices.
 Triangulation facets that belong to some
surface patch are  called surface facets.


The concept \ccRefName\  allows us to  mark and retrieve  the 
cells of the triangulation belonging to the subdomains,
the facets of the triangulation belonging to  surface patches,
the edges belonging to curve segments and the vertices that are corners of the embedded complex.

% The concept \ccRefName\  also  includes an index type  for vertices of the triangulation
% and attaches an integer, called the {\em dimension} to each vertex.
% When used for meshing algorithm, 
% the index and the dimension  of each vertex  are used to store repectively
% the lowest dimensional face of the input complex including the vertex
% and the dimension of this face.

Within the mesh generation functions,
the concept \ccRefName\   is the concept describing
the data structure used to maintain the current approximation of the input domain.
At the end of the meshing process, the data structure encodes the resulting mesh.
In particular, each subdomain (resp. surface patch) of the input domain
 is then approximated by a subdomain  (resp. a surface patch) of the embedded complex
while the curve segments and corners represent the $1$ and $0$-dimensional features
of the input complex.
 

\ccRefines

\ccc{MeshComplex_3InTriangulation_3}

\ccTypes

\ccNestedType{Curve_segment_index}{A type for indexes of curve segment. The type must match the type 
\ccc{MeshDomainWithFeatures_3::Curve_segment_index}
when the concept is used for  mesh generation.}

\ccNestedType{Corner_index}{ A type for indexes of corners.
The type must match the type 
\ccc{MeshDomainWithFeatures_3::Corner_index}
when the concept is used for  mesh generation.}

\ccNestedType{Edges_in_complex_iterator}{An iterator type to visit the edges
of the triangulation belonging to curve segments.}
\ccGlue
\ccNestedType{Vertices_in_complex_iterator}{An iterator type to visit the vertices
of the triangulation that are corners of the embedded complex.}

\ccCreationVariable{c3t3}  %% choose variable name

%\ccConstructor{MeshComplexWithFeatures_3InTriangulation_3();}{default constructor.}


\ccModifiers

\ccMethod{void add_to_complex(Edge e, const Curve_segment_index& index);}{
  Adds edge \ccc{e}  as an element of the curve segment with  index \ccc{index}.}
\ccGlue
\ccMethod{void add_to_complex(const Vertex_handle& v1,
  const Vertex_handle& v2, const Curve_segment_index& index);}{
  Same as above with \ccc{e=(v1,v2)}.}

\ccMethod{void add_to_complex(const Vertex_handle& v, const Corner_index& index);}{
  Marks  vertex \ccc{v} as a corner with  index \ccc{index}.}

\ccMethod{void remove_from_complex(const Edge& e);}{
  Removes edge \ccc{e} from the embedded complex.}
\ccGlue
\ccMethod{void remove_from_complex(const Vertex_handle& v1,
  const Vertex_handle& v2);}{
  Same as above with \ccc{e=(v1,v2)}.}

\ccMethod{void remove_from_complex(const Vertex_handle& v);}{
  Removes vertex \ccc{v} from the embedded complex.}

%\ccMethod{void set_curve_segment_index(Edge e, const Curve_segment_index& index)}{
%  Sets the curve index of the edge \ccc{e} to \ccc{index}}




\ccHeading{Queries}
Queries on the 1D complex and 0D complex.

\ccMethod{size_type number_of_edges() const;}{
  Returns the number of edges which belong to curve segments.}
\ccGlue
\ccMethod{size_type number_of_edges(Curve_segment_index index) const;}{
  Returns the number of edges which belong to curve segment with index \ccc{index}.}
\ccGlue
\ccMethod{size_type number_of_corners() const;}{
  Returns the number of vertices which are corners of the complex.}
\ccGlue
\ccMethod{size_type number_of_corners(Corner_index index) const;}{
  Returns the number of vertices which are corners of the complex with index \ccc{index}.}

\ccMethod{bool is_in_complex(const Edge& e) const;}{Returns \ccc{true}
  iff edge \ccc{e} belongs to some curve segment.}
\ccGlue
\ccMethod{bool is_in_complex(const Vertex_handle& v1,
  const Vertex_handle& v2) const;}{
  Same as above with \ccc{e=(v1,v2)}.}
\ccGlue
\ccMethod{bool is_in_complex(const Vertex_handle& v) const;}{
  Returns \ccc{true} if  vertex \ccc{v} is a corner.}

\ccMethod{Curve_segment_index curve_segment_index(const Edge& e);}{
  Returns \ccc{Curve_segment_index} of edge \ccc{e}. The default \ccc{Curve_segment_index}
  value is returned if edge \ccc{e} does not belong to any curve segment.}
\ccGlue
\ccMethod{Curve_segment_index curve_segment_index(const Vertex_handle& v1, const Vertex_handle& v2);}{
  Same as above with \ccc{e=(v1,v2)}.}
\ccGlue
\ccMethod{Corner_index corner_index(const Vertex_handle& v);}{
  Returns \ccc{Corner_index} of vertex \ccc{v}. The default \ccc{Corner_index} value
  is returned if vertex \ccc{v} is not a corner of the complex.}

\ccHeading{Traversal of the complex}

\ccMethod{Edges_in_complex_iterator edges_in_complex_begin() const;}{
  Returns an \ccc{Edges_in_complex_iterator} to visit the edges of the triangulation belonging to curve segments.}
\ccGlue
\ccMethod{Edge_in_complex_iterator edges_in_complex_end() const;}{
  Returns the past-the-end iterator for the above iterator.}
\ccGlue
\ccMethod{Edges_in_complex_iterator edges_in_complex_begin(Curve_segment_index index) const;}{
  Returns an \ccc{Edges_in_complex_iterator} to visit the edges of the triangulation belonging to curve segments
  of index \ccc{index}.}
\ccGlue
\ccMethod{Edge_in_complex_iterator edges_in_complex_end(Curve_segment_index index) const;}{
  Returns the past-the-end iterator for the above iterator.}

\ccMethod{template <typename OutputIterator>
  OutputIterator
  adjacent_vertices_in_complex (const Vertex_handle& v, OutputIterator out) const;}{
  Fills \ccc{out} with the vertices  of the triangulation that are adjacent to vertex \ccc{v} 
  through an edge belonging to some curve segment.
  The value type of \ccc{out}  must be \ccc{std::pair<Vertex_handle,Curve_segment_index>}.
  \ccPrecond{\ccc{c3t3.in_dimension(v) < 2}}}

\ccMethod{Vertices_in_complex_iterator  vertices_in_complex_begin() const;}{
  Returns a \ccc{Vertices_in_complex_iterator} to visit the vertices of the triangulation
that are corners.}
\ccGlue
\ccMethod{Vertices_in_complex_iterator  vertices_in_complex_end() const;}{
  Returns the past-the-end iterator for the above iterator.}
\ccGlue
\ccMethod{Vertices_in_complex_iterator  vertices_in_complex_begin(Corner_index index) const;}{
  Returns a \ccc{Vertices_in_complex_iterator} to visit the vertices of the triangulation
that are corners of index \ccc{index}.}
\ccGlue
\ccMethod{Vertices_in_complex_iterator  vertices_in_complex_end(Corner_index index) const;}{
  Returns the past-the-end iterator for the above iterator.}


\ccHasModels

\ccc{Mesh_complex_3_in_triangulation_3<Tr,CornerIndex,CurveSegmentIndex>}

\ccSeeAlso
\ccRefConceptPage{MeshComplex_3InTriangulation_3}\\
\ccRefConceptPage{MeshDomainWithFeatures_3}

\end{ccRefConcept}

% +------------------------------------------------------------------------+
%%RefPage: end of main body, begin of footer
\ccRefPageEnd
% EOF
% +------------------------------------------------------------------------+

