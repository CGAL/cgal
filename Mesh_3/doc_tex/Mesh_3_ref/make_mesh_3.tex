% +------------------------------------------------------------------------+
% | Reference manual page: make_mesh_3.tex
% +------------------------------------------------------------------------+
% | 13.05.2008   Laurent Rineau
% | Package: Mesh_3
% |
\RCSdef{\RCSmakemeshRev}{$Id$}
\RCSdefDate{\RCSmakemeshDate}{$Date$}
% |
\ccRefPageBegin
%%RefPage: end of header, begin of main body
% +------------------------------------------------------------------------+


\begin{ccRefFunction}{make_mesh_3}  %% add template arg's if necessary

%% \ccHtmlCrossLink{}     %% add further rules for cross referencing links
%% \ccHtmlIndexC[function]{} %% add further index entries

\ccDefinition

The function \ccRefName\ is a 3D
 mesh generator. It produces simplicial meshes which discretize
 3D domains.

The mesh generation algorithm is a Delaunay refinement process
followed by an optimization phase, which is currently implemented
as a sliver exudation process. 
The  criteria driving the Delaunay refinement
process may be tuned to achieve the user needs with respect to
the size of mesh elements, the accuracy of boundaries approximation,
etc.

The function outputs the mesh to an object which provides iterators to
traverse the resulting mesh data structure or can be written to a file
(see~\ref{Mesh_3_section_examples}).

\ccGlobalFunction{
  template <class C3T3,
  class MeshDomain,
  class MeshCriteria>
  C3T3 make_mesh_3(MeshDomain domain, MeshCriteria criteria);}{}


\ccParameters

Parameter \ccc{C3T3} is required to be a model of
the concept 
\ccc{MeshComplex_3InTriangulation_3}, a data structure devised to
represent a 3D complex embedded in a 3D triangulation. This is the return type.

Template parameter \ccc{MeshDomain} is required to be a model of
the concept  \ccc{MeshDomain_3}. The argument \ccc{domain} of type
\ccc{MeshDomain}
 is the link through which the domain
to be discretized is known  by the mesh generation algorithm. 

% The compatibility requirements between the template parameters
% \ccc{C3T3} and \ccc{MeshDomain} are as follows:
% the nested types \ccc{MeshDomain::Subdomain_index},
% \ccc{MeshDomain::Surface_index} and \ccc{MeshDomain::Index} 
% have to be convertible respectively
% to the nested type \ccc{C3T3::Triangulation::Cell::Subdomain_index}
% and \ccc{C3T3::Triangulation::Vertex::Index}.


The argument of
type \ccc{MeshCriteria} passed to the mesh generator specifies the
size and shape requirements for the mesh tetrahedra
and for the triangles in the boundary mesh facets. These criteria
form the rules which drive the refinement process. All mesh elements
satisfy those criteria at the end of the refinement process.
This may not be true anymore after the sliver removal phase although this
last phase is devised to only improve the mesh quality.
The template parameter \ccc{MeshCriteria} has to be a model of the concept
\ccc{MeshCriteria_3}. 
%%%% TODO: Improve discussion about sliver exudation

\ccSeeAlso

\ccc{refine_mesh_3}

\end{ccRefFunction}

% +------------------------------------------------------------------------+
%%RefPage: end of main body, begin of footer
\ccRefPageEnd
% EOF
% +------------------------------------------------------------------------+

