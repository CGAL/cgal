% +------------------------------------------------------------------------+
% | Reference manual page: Implicit_mesh_domain_3.tex
% +------------------------------------------------------------------------+
% | 13.02.2009   Author
% | Package: Package
% |
\RCSdef{\RCSImplicitmeshdomainRev}{$Id: header.tex 40270 2007-09-07 15:29:10Z lsaboret $}
\RCSdefDate{\RCSImplicitmeshdomainDate}{$Date: 2007-09-07 17:29:10 +0200 (Ven, 07 sep 2007) $}
% |
\ccRefPageBegin
%%RefPage: end of header, begin of main body
% +------------------------------------------------------------------------+


\begin{ccRefClass}{Implicit_mesh_domain_3<Function,BGT>}  %% add template arg's if necessary

%% \ccHtmlCrossLink{}     %% add further rules for cross referencing links
%% \ccHtmlIndexC[class]{} %% add further index entries

\ccDefinition
  
The class \ccRefName\ implements a domain whose bounding surface is
described
implicitly as the zero level set of a function.
The domain to be discretized is assumed to be the domain where
the function has negative values.
This class is a model of the concept \ccc{MeshDomain_3}.

\ccParameters
Parameter \ccc{BGT} is a geometric traits which provides
the basic operations to implement
intersection tests and computations 
through a bisection method. This parameter must be instantiated
with a model of the concept \ccc{BisectionGeometricTraits_3}.

 
Parameter \ccc{Function} provides the definition of the function.
This parameter stands for a model of the concept 
\ccc{ImplicitFunction} described in the
surface mesh generation package. 
The number types \ccc{Function::FT}  
and  \ccc{BGT::FT} are required to match.

The constructor of \ccRefName{}
takes as argument a bounding sphere which is required
to circumscribe the surface and to have its center inside the 
 domain. 
This domain constructs intersection points  
between
the surface and segments/rays/lines by 
bisection. It needs an 
\ccc{error_bound} such that the bisection process is stopped
when the query segment is smaller than the error bound.
The \ccc{error_bound} passed as argument to the domain constructor
is a relative error bound expressed as a ratio to the bounding sphere radius.


\ccInclude{CGAL/Implicit_mesh_domain_3.h}

\ccIsModel
\ccc{MeshDomain_3}




\ccCreation
\ccCreationVariable{domain}

\ccConstructor{
Implicit_mesh_domain_3(Function f,
                       BGT::Sphere_3 bounding_sphere,
                       BGT::FT error_bound = FT(1e-3));}
{\ccc{f} is the object of type \ccc{Function} that represents the implicit
  surface.\\
 \ccc{bounding_sphere} is a bounding sphere of the implicit surface. The
 value of \ccc{f} at the sphere center \ccc{c} must be
 negative: $f(c)<0$.\\
 \ccc{error_bound} is the relative error bound 
used to  compute intersection points between the implicit surface
and  query segments.  The
bisection is stopped when the length of the intersected
segment  is less than the  product of \ccc{error_bound} by the
radius of \ccc{bounding_sphere}.}




\ccSeeAlso

\ccRefConceptPage{MeshDomain_3} \\
\ccRefConceptPage{BisectionGeometricTraits_3} \\
\ccRefIdfierPage{CGAL::make_mesh_3}.



\end{ccRefClass}

% +------------------------------------------------------------------------+
%%RefPage: end of main body, begin of footer
\ccRefPageEnd
% EOF
% +------------------------------------------------------------------------+

