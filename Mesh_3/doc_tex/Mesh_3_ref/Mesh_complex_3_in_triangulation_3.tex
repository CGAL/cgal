% +------------------------------------------------------------------------+
% | Reference manual page: Mesh_complex_3_in_triangulation_3.tex
% +------------------------------------------------------------------------+
% | 13.02.2009   Author Mariette Yvinec
% | Package: Package Mesh_3
% |
\RCSdef{\RCSMeshComplexintriangulationRev}{$Id: header.tex 40270 2007-09-07 15:29:10Z lsaboret $}
\RCSdefDate{\RCSMeshComplexintriangulationDate}{$Date: 2007-09-07 17:29:10 +0200 (Ven, 07 sep 2007) $}
% |
\ccRefPageBegin
%%RefPage: end of header, begin of main body
% +------------------------------------------------------------------------+


\begin{ccRefClass}{Mesh_complex_3_in_triangulation_3<Tr,CornerIndex,CurveSegmentIndex>}  %% add template arg's if necessary

%% \ccHtmlCrossLink{}     %% add further rules for cross referencing links
%% \ccHtmlIndexC[class]{} %% add further index entries

\ccDefinition
  

The class \ccRefName\ implements a data structure 
to store the 3D restricted Delaunay triangulation used by a mesh
generation process.

This class is a model of the concept
\ccc{MeshComplexWithFeatures_3InTriangulation_3}.

\ccParameters

The template parameter \ccc{Tr} can be instantiated with any 3D
regular triangulation of \cgal{} provided that its
vertex and cell base class are models of the concepts
\ccc{MeshVertexBase_3} and \ccc{MeshCellBase_3}, respectively. 

The template parameter \ccc{CornerIndex} is the type of the indices for corners and
the template parameter \ccc{CurveSegmentIndex} is the type of the indices for curves segments.
They must match the \ccc{Corner_index} and \ccc{Curve_segment_index} types of the model
of the \ccc{MeshDomainWithFeatures_3} concept used for mesh generation.

Those two last template parameters defaults to \ccc{int}, so that they can be ignored
if the domain used for mesh generation does not include 0 and 1-dimensionnal features (i.e
is a model of the concept \ccc{MeshDomain_3}).


\ccInclude{CGAL/Mesh_complex_3_in_triangulation_3.h}

\ccIsModel
\ccc{MeshComplexWithFeatures_3InTriangulation_3}

\ccTypes

\ccTypedef{typedef Tr::Vertex::Index Index;}{Index type.}
\ccTypedef{typedef Tr::Cell::Surface_patch_index Surface_patch_index;}{Surface index type.}
\ccTypedef{typedef Tr::Cell::Subdomain_index Subdomain_index;}{Subdomain index type.}
\ccTypedef{typedef CornerIndex Corner_index;}{\ccc{Corner_index} type.}
\ccTypedef{typedef CurveSegmentIndex Curve_segment_index;}{\ccc{Curve_segment_index} type.}

\ccOperations
\ccCreationVariable{c3t3}

\ccMethod{void output_to_medit(std::ofstream& os);}{Outputs the mesh to \ccc{os}
    in medit format.}

\ccSeeAlso
\ccRefIdfierPage{CGAL::make_mesh_3} \\
\ccRefIdfierPage{CGAL::refine_mesh_3} \\
\ccRefConceptPage{MeshComplex_3InTriangulation_3} \\
\ccRefConceptPage{MeshComplexWithFeatures_3InTriangulation_3} \\
\ccRefConceptPage{MeshCellBase_3}, \\
\ccRefConceptPage{MeshVertexBase_3}

\end{ccRefClass}

% +------------------------------------------------------------------------+
%%RefPage: end of main body, begin of footer
\ccRefPageEnd
% EOF
% +------------------------------------------------------------------------+

