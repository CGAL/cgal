% +------------------------------------------------------------------------+
% | Reference manual page: mesh_union_of_balls_3.tex
% +------------------------------------------------------------------------+
% | 27.09.2005   Nico Kruithof
% | Package: Skin_surface_3
% | 
\RCSdef{\RCSskinsurfaceRev}{$Id$}
\RCSdefDate{\RCSskinsurfaceDate}{$Date$}
% |
%%RefPage: end of header, begin of main body
% +------------------------------------------------------------------------+

\begin{ccRefFunction}{mesh_union_of_balls_3<SkinSurface_3, Polyhedron_3>}
  \ccDefinition

  The function \ccRefName\ constructs a mesh isotopic to the boundary
  of the union of a set of balls and is based on the algorithm in
  \cite{cgal:kv-mssct-05}. It takes as input a \ccc{SkinSurface_3}
  object, which is a model of the \ccc{SkinSurface_3} concept and
  outputs the mesh in a \ccc{Polyhedron_3} object. 

  The function \ccRefName\ is a wrapper function around
  \ccc{mesh_skin_surface_3} and is provided for convenience's sake.

  \ccInclude{CGAL/mesh_union_of_balls_3.h}

  \ccThree{void}{a}{}
  \ccThreeToTwo

  \ccFunction{void mesh_union_of_balls_3<SkinSurface_3,Polyhedron_3>
    (const SkinSurface_3 &union_of_balls, Polyhedron_3 &p);}{Constructs
    a mesh of the \ccc{union_of_balls} in
    \ccc{p}.\ccPrecond{\ccc{SkinSurface_3} is a model of the concept
      \ccc{SkinSurface_3} and \ccc{Polyhedron_3::HDS} can be used as
      the template argument of the
      \ccc{Polyhedron_incremental_builder_3<HDS>}}.}

\end{ccRefFunction}

% +------------------------------------------------------------------------+
%%RefPage: end of main body, begin of footer
% EOF
% +------------------------------------------------------------------------+

