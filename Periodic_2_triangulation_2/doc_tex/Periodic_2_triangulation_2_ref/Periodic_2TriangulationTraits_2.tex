% +------------------------------------------------------------------------+
% | Reference manual page: Triangulation_traits_2.tex
% +------------------------------------------------------------------------+
% | 23-11-2010 Nico Kruithof
% | Package: Periodic_2_triangulation_2
% | 
\RCSdef{\RCSTriangulationtraitsRev}{$Id$}
\RCSdefDate{\RCSTriangulationtraitsDate}{$Date$}
% |
%%RefPage: end of header, begin of main body
% +------------------------------------------------------------------------+


\begin{ccRefConcept}{Periodic_2TriangulationTraits_2}

%% \ccHtmlCrossLink{}     %% add further rules for cross referencing links
%% \ccHtmlIndexC[concept]{} %% add further index entries

\ccDefinition
%
The concept \ccRefName\ is the first template parameter of the classes
\ccc{Periodic_2_triangulation_2<Traits, Tds>}.  This concept provides the types of
the geometric primitives used in the triangulation and some function
object types for the required predicates on those primitives.

It refines the concept
\ccc{TriangulationTraits_2} from the \cgal\ \ccRef[2D
Triangulation]{Pkg:Triangulation2} package.  It redefines the
geometric objects, predicates and constructions to work with
point-offset pairs. In most cases the offsets will be (0,0) and the
predicates from \ccc{TriangulationTraits_2} can be used
directly. For efficiency reasons we maintain for each functor the
version without offsets.

\ccRefines \ccc{TriangulationTraits_2}

In addition to the requirements described for the traits class
\ccc{TriangulationTraits_2}, the geometric traits class of a
Periodic triangulation must fulfill the following
requirements:

\ccTypes
\ccThree{xxxxxxxxxxxxxxxxxxxxxxxxxxxxxxxxxxxxx}{xxxxxxxxxx}{}
\ccThreeToTwo
%
\ccNestedType{Point_2} {The point type. It must be a model of
  \ccc{Kernel::Point_2}.}
%
\ccGlue \ccNestedType{Segment_2}{The segment type. It must be a model
  of \ccc{Kernel::Segment_2}.}
%
\ccGlue \ccNestedType{Vector_2} {The vector type. It must be a model
  of \ccc{Kernel::Vector_2}.}
%
\ccGlue \ccNestedType{Triangle_2}{The triangle type. It must be a
  model of \ccc{Kernel::Triangle_2}.}
%
\ccGlue \ccNestedType{Iso_rectangle_2} {A type representing an
  axis-aligned rectangle. It must be a model of
  \ccc{Kernel::Iso_rectangle_2}.}
%
\ccGlue \ccNestedType{Periodic_2_offset_2} {The offset type. It must
  be a model of the concept \ccc{Periodic_2Offset_2}.}

% Predicate types
\ccHeading{Predicate types}
\ccTwo{xxxx}{}
%
\ccNestedType{Compare_x_2}{
  A predicate object that must provide the function operators\\
  \ccc{Comparison_result operator()(Point_2 p, Point_2 q)},\\
  which returns \ccc{EQUAL} if the $x$-coordinates of the two points are equal and\\
  \ccc{Comparison_result operator()(Point_2 p, Point_2 q,
    Periodic_2_offset_2 o_p, Periodic_2_offset_2 o_q)},\\
  which returns \ccc{EQUAL} if the $x$-coordinates and $x$-offsets of
  the two point-offset pairs are equal.  Otherwise it must return a
  consistent order for any two points.  \ccPrecond{\ccc{p}, \ccc{q}
    lie inside the domain.}}
%
\ccNestedType{Compare_y_2}{
  A predicate object that must provide the function operators\\
  \ccc{Comparison_result operator()(Point_2 p, Point_2 q)},\\
  which returns \ccc{EQUAL} if the $y$-coordinates of the two points are equal and\\
  \ccc{Comparison_result operator()(Point_2 p, Point_2 q,
    Periodic_2_offset_2 o_p, Periodic_2_offset_2 o_q)},\\
  which returns \ccc{EQUAL} if the $y$-coordinates and $y$-offsets of
  the two point-offset pairs are equal.  Otherwise it must return a
  consistent order for any two points.  \ccPrecond{\ccc{p}, \ccc{q}
    lie inside the domain.}}
%
\ccNestedType{Less_x_2}{
Predicate object. Provides the operators: \\
\ccc{bool operator()(Point p, Point q)} and \\
\ccc{bool operator()(Point p, Point q, Periodic_2_offset_2 o_p, Periodic_2_offset_2 o_q)} \\  
which returns \ccc{true} if \ccc{p} is before \ccc{q}
according to the $x$-ordering of points.\\
%
This predicate is only necessary if the insert function with a range
of points (using Hilbert sorting) is used.}
%
\ccNestedType{Less_y_2}{Predicate object. Provides
the operators: \\
\ccc{bool operator()(Point p, Point q)} and \\ 
\ccc{bool operator()(Point p, Point q, Periodic_2_offset_2 o_p, Periodic_2_offset_2 o_q)} \\
which returns \ccc{true} if \ccc{p} is before \ccc{q}
according to the $y$-ordering of points.\\
%
This predicate is only necessary if the insert function with a range of
points (using Hilbert sorting) is used.
}
%
\ccNestedType{Orientation_2}
{A predicate object that must provide the function operators\\
  \ccc{Orientation operator()(Point_2 p, Point_2 q, Point_2 r)},\\
  which returns \ccc{LEFT_TURN}, \ccc{RIGHT_TURN} or \ccc{COLLINEAR}
  depending on $r$ being, with respect to the oriented line \ccc{pq},
  on the left side, on the right side or on the line.
  and \\
  \ccc{Orientation operator()(Point_2 p, Point_2 q, Point_2 r,
   Periodic_2_offset_2 o_p, Periodic_2_offset_2 o_q,
    Periodic_2_offset_2 o_r)},\\
  which returns \ccc{LEFT_TURN}, \ccc{RIGHT_TURN} or \ccc{COLLINEAR}
  depending on \ccc{(r,o_r)} being, with respect to the oriented line
  defined by \ccc{(p,o_p)(q,o_q)} on the left side, on the right side
  or on the line.}

% Constructors
\ccHeading{Constructor types:}
Note that the traits must provide exact constructions in order to
guarantee exactness of the following construction functors.

\ccNestedType{Construct_point_2} {A constructor object for
\ccc{Point_2}. Provides: \\
\ccc{Point_2 operator()(Point_2 p,Periodic_2_offset_2 p_o)}, \\ 
which constructs a point from a point-offset pair.
%
\ccPrecond{\ccc{p} lies inside the domain.}
}
%
\ccNestedType{Construct_segment_2} {A constructor object for
\ccc{Segment_2}. Provides: \\
\ccc{Segment_2 operator()(Point_2 p,Point_2 q)}, \\ 
which constructs a  segment from two points and\\
\ccc{Segment_2 operator()(Point_2 p,Point_2 q, Periodic_2_offset_2 o_p, Periodic_2_offset_2 o_q)}, \\ 
which constructs a  segment from the points \ccc{(p,o_p)} and  \ccc{(q,o_q)}.
}
%
\ccNestedType{Construct_triangle_2} {A constructor object for
  \ccc{Triangle_2}. Provides: \\
  \ccc{Triangle_2 operator()(Point_2 p,Point_2 q,Point_2 r )}, \\
  which constructs a triangle from three points and\\
  \ccc{Triangle_2 operator()(Point_2 p,Point_2 q,Point_2 r, 
    Periodic_2_offset_2 o_p, Periodic_2_offset_2 o_q, Periodic_2_offset_2 o_r)}, \\
  which constructs a triangle from the three points \ccc{(p,o_p)},
  \ccc{(q,o_q)} and \ccc{(r,o_r)}.  }

\ccCreation %
\ccCreationVariable{traits}  %% choose variable name
%
Only a default constructor, copy constructor and an assignment
operator are required. Note that further constructors can be provided.

\ccThree{xxxxxxxxxxxxxxxxxxxx}{xxxxxxxxxxxxxxxxxxxxxxxxxxxxxxxxxxxxxxxxxxxxxxx}{}
\ccThreeToTwo
\ccConstructor{TriangulationTraits_2();}{default constructor.}
\ccGlue
\ccConstructor{TriangulationTraits_2(TriangulationTraits_2 gtr);}
{Copy constructor}
\ccGlue
\ccMethod{TriangulationTraits_2 operator=(TriangulationTraits_2 gtr);}
{Assignment operator.}

\ccHeading{Predicate functions}
%
The following functions give access to the predicate and constructor
objects.
%
\ccThree{Construct_segment_2}{gt.compare_x(Point p0, Point p1)x}{}
%
\ccMethod{Compare_x_2 compare_x_2_object();}{}
%
\ccGlue
%
\ccMethod{Compare_y_2 compare_y_2_object();}{}
%
\ccGlue
%
\ccMethod{Less_x_2 less_x_2_object();}{}
%
\ccGlue
\ccMethod{Less_y_2 less_y_2_object();}{}
%
\ccGlue
\ccMethod{Orientation_2  orientation_2_object();}{}
%
\ccGlue
\ccMethod{Construct_point_2 construct_point_2_object();}{}
%
\ccMethod{Construct_segment_2 construct_segment_2_object();}{}
%
\ccGlue
\ccMethod{Construct_triangle_2 construct_triangle_2_object();}{}

\ccAccessFunctions
\ccThree{Iso_rectangle_2}{traits}{}
\ccThreeToTwo
\ccMethod{void set_domain(Iso_rectangle_2 domain);}{Sets the
  fundamental domain. This is necessary to evaluate predicates
  correctly.
\ccPrecond{\ccc{domain} represents a square.}}
\ccMethod{Iso_rectangle_2 get_domain() const;}{Returns the
  fundamental domain.}

\ccHasModels
\ccc{CGAL::Periodic_2_triangulation_traits_2}

\ccSeeAlso
\ccc{TriangulationTraits_2}\\
\ccc{CGAL::Periodic_2_triangulation_2<Traits,Tds>}



\end{ccRefConcept}

% +------------------------------------------------------------------------+
%%RefPage: end of main body, begin of footer
% EOF
% +------------------------------------------------------------------------+

