
\chapter{Introduction}

\cgal, the {\em Computational Geometry Algorithms Library}, is written in 
\CC\ and consists of three major parts.

The first part is the kernel, which consists of constant-size non-modifiable 
geometric primitive objects and operations on these objects. 
The objects are represented both as stand-alone classes that are
parameterized by a representation class, which specifies
the underlying number types used for calculations and as members of the
kernel classes, which allows for more flexibility and adaptability of the 
kernel.

The second part is a collection of basic geometric data structures and
algorithms, which are parameterized by traits classes that define the 
interface between the data structure or algorithm and the primitives they use.
In many cases, the kernel classes provided in \cgal\ can be used as traits
classes for these data structures and algorithms.
The collection of basic geometric algorithms and data structures
currently includes polygons, half-edge data structures, polyhedral surfaces, 
topological maps, planar maps, arrangements of curves, triangulations, 
convex hulls, alpha shapes, optimisation algorithms, dynamic
point sets for geometric queries, and multidimensional search trees.

The third part of the library consists of non-geometric support
facilities, such as support for number types, STL extensions for
\cgal, handles, circulators, protected access to internal
representations, geometric object generators, timers, I/O stream
operators and other stream support including PostScript, colors,
windows, and visualization tools GeoWin, Geomview and a Qt widget for
2D \cgal\ objects.

Additional documents accompanying the \cgal\ distribution are the
`Installation Guide' and `The Use of \stl\ and \stl\ Extensions in
\cgal', which gives a manual style introduction to \stl\ constructs
such as iterators and containers, as well an extension, called
circulator, used in many places in \cgal. We also recommend the
standard text book by Austern~\cite{cgal:a-gps-98} for the \stl\ and
its notion of \emph{concepts} and \emph{models}.


Other resources for \cgal\ are the tutorials at
\path|http://www.cgal.org/Tutorials/| and the user support page at
\path|www.cgal.org|.



\begin{ccAdvanced}
Some functionality is considered more advanced.
Such functionality is described in sections such as this one that are bounded 
by horizontal brackets.
\end{ccAdvanced}


\section{Namespace CGAL}

All names introduced by \cgal, especially those documented in these
manuals, are in a namespace called \ccc{CGAL}, which is in global
scope. A user can either qualify names from \cgal\ by adding
\ccc{CGAL::}, e.g., \ccc{CGAL::Point_2< CGAL::Homogeneous< int> >},
make a single name from \cgal\ visible in a scope via a \ccc{using}
statement, e.g., \ccc{using CGAL::Cartesian;}, and then use this name
unqualified in this scope, or even make all names from namespace
\ccc{CGAL} visible in a scope with \ccc{using namespace CGAL;}. The
latter, however, is likely to give raise to name conflicts and is
therefore not recommended.


\section{Inclusion Order of Header files}

Not all compilers fully support standard header names. \cgal\ provides 
workarounds for these problems in \ccc{CGAL/basic.h}. Consequently, as a 
golden rule, you should always inlcude \ccc{CGAL/basic.h} first in your 
programs (or \ccc{CGAL/Cartesian.h}, or \ccc{CGAL/Homogeneous.h}, since they 
include \ccc{CGAL/basic.h} first).






\section{Compile-time Flags to Control Inlining}
\ccIndexMainItem{code optimization}
\ccIndexMainItem{inlining}
\ccIndexMainItem{\tt inline}

Making funcitons inlined can, at times, improve the efficiency of your code.
However this is not always the case and it can differ for a single function
depending on the application in which it is used. Thus \cgal\ defines a set 
of compile-time macros that can be used to control whether certain functions 
are designated as inlined functions or not.  The following table lists the 
macros and their default values, which are set in one of the \cgal\ include
files.  

\begin{tabular}{l|l}
               macro name        & default \\ \hline
\ccc{CGAL_KERNEL_INLINE}         & inline \\
\ccc{CGAL_KERNEL_MEDIUM_INLINE}  &  \\
\ccc{CGAL_KERNEL_LARGE_INLINE}   &  \\
\ccc{CGAL_MEDIUM_INLINE}         & inline \\
\ccc{CGAL_LARGE_INLINE}          &  \\
\ccc{CGAL_HUGE_INLINE}           & 
\end{tabular}

If you wish to change the value of one or more of these macros,
you can simply give it a new value when compiling.  For example, to make
functions that use the macro \ccc{CGAL_KERNEL_MEDIUM_INLINE} inline functions,
you should set the value of this macro to \texttt{inline} instead of the
default blank. 


\section{License Issues}

\cgal\ is open source software, and consists of different parts covered by different licenses.

\subsection{Qpl \label{licenses:QPL}}

More text to come.

\subsection{LGPL \label{licenses:LGPL}}

More text to come.

\subsection{Commercial Licenses \label{licenses:Commercial}}

Users who cannot comply to the open source license terms can buy individual
data structures under various commercial licenses from GeometryFactory (www.geometryfactory.com).

\subsection{License Compatibility \label{licenses:Compatibility}}

GPL ...


\section{Third Party Software}

In this section we list the software that is used by the various
\cgal\ packages.


\subsection{Standard Template Library \label{thirdparty:stl}}

\cgal\ heavily uses the {\sc Stl}, and in particular adopted
many of its design ideas.   The {\sc Stl} comes with the compiler,
but it is possible to use the compiler together with an
alternative {\sc Stl} implementation. You can find online
documentation for the {\sc Stl} at various web sites, e.g., 
\path+http://www.sgi.com/tech/stl/+, \path+http://www.cplusplus.com/reference/stl/+,
or \path+http://msdn.microsoft.com/en-us/library/1fe2x6kt(VS.71).aspx+.


\subsection{Boost \label{thirdparty:Boost}}

Boost is a collection of libraries. \cgal\ needs some of them, that is
it is mandatory.  If Boost is not already on your system, e.g., on
Windows, you can download it from \path'http://www.boost.org'.


\subsection{Blas \label{thirdparty:Blas}}

The \blas\ (Basic Linear Algebra Subprograms) are routines that provide
standard building blocks for performing basic vector and matrix operations.
In \cgal, \blas\ is required by the packages
\ccRef[Estimation of Local Differential Properties]{Pkg:Jet_fitting_3}
and \ccRef[Approximation of Ridges and Umbilics]{Pkg:Ridges_3} only.

You can download the official release from \path'http://www.netlib.org/blas/'
or download optimized implementations from \path'http://www.netlib.org/blas/faq.html#5'.
Alternatively, installing \taucs\ provides \blas.


\subsection{Lapack \label{thirdparty:Lapack}}

\lapack\ provides routines for solving systems of simultaneous linear equations,
least-squares solutions of linear systems of equations, eigenvalue problems,
and singular value problems.
In \cgal, \lapack\ is required by the packages
\ccRef[Estimation of Local Differential Properties]{Pkg:Jet_fitting_3}
and \ccRef[Approximation of Ridges and Umbilics]{Pkg:Ridges_3} only.

You can download the official release from \path'http://www.netlib.org/lapack/'.
Alternatively, installing \taucs\ customized for \cgal\ provides \lapack.


\subsection{GMP \label{thirdparty:GMP}}

A library for multi precision integers and rational numbers.
\cgal\ offers adapters for these number types. The usage
of the {\sc Gmp} library is optional.  If it is not already on your system,
e.g., on Windows, you can download it from \path'http://gmplib.org/'
or from the download section of \path'http://www.cgal.org'.


\subsection{MPFR \label{thirdparty:MPFR}}

A library for multi precision floating point numbers.  The usage of
the {\sc Mpfr} library is optional, and you must install it when you
use {\sc Gmp}.  You can download {\sc Mpfr} from \path'http://www.mpfr.org'
or from the download section of \path'http://www.cgal.org'.

\subsection{RS \label{thirdparty:RS}}

\rs{} stands for Real Solutions and is devoted to the study of the real
roots of polynomial systems with a finite number of complex roots
(including univariate polynomials). \rs{} is used only by one model of the
\ccRef[Algebraic Kernel]{Pkg:AlgebraicKerneld}.

\rs{} is freely distributable for non-commercial use. You can download it
from \rspage{}.

\subsection{Leda \label{thirdparty:Leda}}

A library of efficient data structures and algorithms. \cgal\ offers
adapters to the {\sc Leda} number types. The usage is optional.
It is available commercially from \path'http://www.algorithmic-solutions.com',
and there exists a binary ``free edition''.


\subsection{Taucs \label{thirdparty:Taucs}}

\taucs\ is a library of sparse linear solvers.
In \cgal, it is used to improve the computations within the
\ccRef[Planar Parameterization of Triangulated Surface Meshes]{Pkg:SurfaceParameterization}
package only.

The \taucs\ web site is \path'http://www.tau.ac.il/~stoledo/taucs/'.\\
The latest official version is \taucs\ version 2.2, September 4, 2003.
Copyright (c) 2001, 2002, 2003 by Sivan Toledo, Tel-Aviv University,
stoledo@tau.ac.il. All Rights Reserved.\\
See \path'http://www.tau.ac.il/~stoledo/taucs/' for the license and the availability note.\\
Used by permission of Sivan Toledo.

The \cgal\ project provides a modified version of \taucs\ in the download
section of \path'http://www.cgal.org'. This version fixes some bugs,
supports 64-bit platforms and allows a simplified installation process.
It also contains a complete \lapack\ implementation.\\
{\em CAUTION:} Since version 3.3.1, \cgal\ is no longer compatible with the official
release of \taucs\ (currently 2.2). Make sure to use the modified
version provided in the download section.


\subsection{OpenNL \label{thirdparty:OpenNL}}

OpenNL (Open Numerical Library) is a library to easily construct and solve
sparse linear systems. It is the default solver of the
\ccRef[Surface Mesh Parameterization]{Pkg:SurfaceParameterization} package.

OpenNL's main page is \path'http://www.loria.fr/~levy/software/'.

\cgal\ includes a version of OpenNL in C++, made especially for \cgal\ by Bruno L\'evy.


\subsection{zlib \label{thirdparty:zlib}}

A data compression library.
It is used in the examples of the \ccRef[Surface Mesh Generation]{Pkg:SurfaceMesher3} package.
If it is not already on your system,
e.g., on Windows, you can download it from  \path'http://www.gzip.org/zlib'.

\subsection{Qt \label{thirdparty:Qt}}

Qt is a cross-platform application framework. The usage of Qt is optional, but note that
it is used for many \cgal\ 2D as well as 3D demos.

As Qt is the layer underneath {\sc Kde}, Qt is installed
on many Linux systems. Otherwise you can download it from
 \path'http://qt.nokia.com/'.

\subsection{libQGLViewer \label{thirdparty:libQGLViewer}}

A 3D widget based on \qt~4's \ccc{QGLWidget}. It can be downloaded from
\path'http://www.libqglviewer.com/'.

\subsection{Coin \label{thirdparty:Coin}}

An implementation of Open Inventor.  It is used in the demo
of the \ccRef[Kinetic Data Structures]{Pkg:Kds} package. You can download
it from \path'http://www.coin3d.org/'.





\ccSetThreeColumns{Failure_behaviour }{}{\hspace*{8.5cm}}

\chapter{Checks}

Much of the {\cgal} code contains checks. 
For example, all checks used in the kernel code are prefixed by 
\ccc{CGAL_KERNEL}.
Other packages have their own prefixes, as documented in the corresponding
chapters.
Some are there to check if the kernel behaves correctly, others are there to 
check if the user calls kernel routines in an acceptable manner.

There are four types of checks. 
The first three are errors and lead to a halt of the program if they fail. 
The last only leads to a warning.
\begin{description}
\item[Preconditions] check if the caller of a routine has called it in a
proper fashion. 
If such a check fails it is the responsibility of the caller 
(usually the user of the library).
\item[Postconditions] check if a routine does what it promises to do. 
If such a check fails it is the fault of this routine, so of the library.
\item[Assertions] are other checks that do not fit in the above two 
categories.
\item[Warnings] are checks for which it is not so severe if they fail.
\end{description}

By default, all of these checks are performed. 
It is however possible to turn them off through the use of compile time 
switches.
For example, for the checks in the kernel code, these switches are the 
following:
\ccStyle{CGAL_KERNEL_NO_PRECONDITIONS}, 
\ccStyle{CGAL_KERNEL_NO_POSTCONDITIONS},
\ccStyle{CGAL_KERNEL_NO_ASSERTIONS} and 
\ccStyle{CGAL_KERNEL_NO_WARNINGS}.
So, in order to compile the file \verb~foo.C~ with the postcondition checks
off, you should do:\\
\verb~CC -DCGAL_KERNEL_NO_POSTCONDITIONS foo.C~

Not all checks are on by default.
All four types of checks can be marked as expensive or exactness checks
(or both).
These checks need to be turned on explicitly by supplying one or both of
the compile time switches \ccStyle{CGAL_KERNEL_CHECK_EXPENSIVE} and 
\ccStyle{CGAL_KERNEL_CHECK_EXACTNESS}.

Expensive checks are, as the word says, checks that take a considerable
time to compute. 
Considerable is an imprecise phrase. 
Checks that add less than 10 percent to the execution time of the routine 
they are in are not expensive.
Checks that can double the execution time are. 
Somewhere in between lies the border line.
Checks that increase the asymptotic running time of an algorithm are always 
considered expensive.
Exactness checks are checks that rely on exact arithmetic. 
For example, if the intersection of two lines is computed, the postcondition 
of this routine may state that the intersection point lies on both lines. 
However, if the computation is done with doubles as number type, this may not 
be the case, due to round off errors. 
So, exactness checks should only be turned on if the computation is done 
with some exact number type.

\section{Altering the failure behaviour}

As stated above, if a postcondition, precondition or assertion is
violated, the program will abort (stop and produce a core dump).
This behaviour can be changed by means of the following function.

\ccInclude{CGAL/assertions.h}

\ccGlueBegin
\ccGlobalFunction{Failure_behaviour
set_error_behaviour(Failure_behaviour eb);}
\ccGlueEnd

The parameter should have one of the following values.

\ccGlobalEnum{enum Failure_behaviour 
{ ABORT, EXIT, EXIT_WITH_SUCCESS, CONTINUE };}
The first value is the default. 
If the \ccStyle{EXIT} value is set, the program will stop and return a value 
indicating failure, but not dump the core. 
The last value tells the checks to go on after diagnosing the error.

\begin{ccAdvanced}
If the \ccStyle{EXIT_WITH_SUCCESS} value is set, the program will stop and 
return a value corresponding to successful execution and not dump the core. 
\end{ccAdvanced}

The value that is returned by \ccc{set_error_behaviour} is the value that was in use before.

For warnings there is a separate routine, which works in the same way.
The only difference is that for warnings the default value is
\ccStyle{CONTINUE}.

\ccGlueBegin
\ccGlobalFunction{Failure_behaviour
set_warning_behaviour(Failure_behaviour eb);}
\ccGlueEnd

\section{Control at a finer granularity}

The compile time flags as described up to now all operate on the whole 
library.
Sometimes you may want to have a finer control.
\cgal\ offers the possibility to turn checks on and off with a bit finer
granularity, namely the module in which the routines are defined.
The name of the module is to be appended directly after the \cgal\ prefix.
So, the flag \ccStyle{CGAL_KERNEL_NO_ASSERTIONS} switches off assertions in 
the kernel only, the flag \ccStyle{CGAL_CH_CHECK_EXPENSIVE} turns on
expensive checks in the convex hull module.
The name of a particular module is documented with that module.

\begin{ccAdvanced}

\section{Customising how errors are reported}

Normally, error messages are written to the standard error output.
It is possible to do something different with them.
To that end you can register your own handler.
This function should be declared as follows.

\ccTexHtml{\begin{samepage}}{}
\renewcommand{\ccLongParamLayout}{\ccTrue}

\lcTex{\ccAutoIndexingOff}
\ccGlobalFunction{
void my_failure_function( const char *type, const char *expression,
const char *file, int line, const char *explanation);}
\ccTexHtml{\end{samepage}}{}
\lcTex{\ccAutoIndexingOn}

Your failure function will be called with the following parameters.
\ccStyle{type} is a string that contains one of the words precondition,
postcondition, assertion or warning. 
The parameter \ccStyle{expression} contains the expression that was violated.
\ccStyle{file} and \ccStyle{line} contain the place where the check was made.
The \ccStyle{explanation} parameter contains an explanation of what was 
checked. 
It can be \ccStyle{NULL}, in which case the \ccStyle{expression} is thought
to be descriptive enough.

There are several things that you can do with your own handler.
You can display a diagnostic message in a different way, for instance in 
a pop up window or to a log file (or a combination).
You can also implement a different policy on what to do after an error.
For instance, you can throw an exception or ask the user in a dialogue 
whether to abort or to continue.
If you do this, it is best to set the error behaviour to
\ccStyle{CONTINUE}, so that it does not interfere with your policy.

You can register two handlers, one for warnings and one for errors.
Of course, you can use the same function for both if you want.
When you set a handler, the previous handler is returned, so you can restore
it if you want.

\ccInclude{CGAL/assertions.h}

\ccGlueBegin
\ccGlobalFunction{Failure_function
set_error_handler(Failure_function handler);}

\ccGlobalFunction{Failure_function
set_warning_handler(Failure_function handler);}
\ccGlueEnd

\subsubsection{Example}

\begin{cprog}
#include <CGAL/assertions.h>

void my_failure_handler(
    const char *type,
    const char *expr,
    const char* file,
    int line,
    const char* msg)
{
    /* report the error in some way. */
}

void foo()
{
    CGAL::Failure_function prev;
    prev = CGAL::set_error_handler(my_failure_handler);
    /* call some routines. */
    CGAL::set_error_handler(prev);
}
\end{cprog}

\end{ccAdvanced}

 % extra chapter