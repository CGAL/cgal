%% Copyright (c) 2006  SciSoft.  All rights reserved.
%%
%% This file is part of CGAL (www.cgal.org); you may redistribute it under
%% the terms of the Q Public License version 1.0.
%% See the file LICENSE.QPL distributed with CGAL.
%%
%% Licensees holding a valid commercial license may use this file in
%% accordance with the commercial license agreement provided with the software.
%%
%% This file is provided AS IS with NO WARRANTY OF ANY KIND, INCLUDING THE
%% WARRANTY OF DESIGN, MERCHANTABILITY AND FITNESS FOR A PARTICULAR PURPOSE.
%%
%% $Name:  $
%%
%% Author(s)     : Fernando Cacciola <fernando_cacciola@hotmail.com>


\begin{ccRefClass}{Polygon_offset_builder_2<Ssds,Gt,Container>}

\ccDefinition
The class \ccRefName\  encapsulates the construction of the inward offsetting of a 2D simple polygon based on the straight skeleton of the interior of the polygon. Its first template parameter, \ccc{Ssds}, must be a model of the \ccc{StraightSkeleton_2} concept, its second template parameter, \ccc{Gt}, must be a model of the \ccc{StraightSkeletonBuilderTraits_2} concept, and its third template parameter must be a model of the \ccc{VertexContainer_2} concept.
 
\ccTypes
  \ccNestedType{Ssds}{The straight skeleton data structure (second template parameter)}{}
\ccGlue
  \ccNestedType{Gt}{The geometric traits (first template parameter)}{}
\ccGlue
  \ccNestedType{Container}{The container of 2D vertices that represents each offset polygon generated by the algorithm (third template parameter)}{}
\ccGlue
  \ccNestedType{FT}{A model of the \ccc{SqrtFieldNumberType} used for scalar and Euclidean distance computations (as defined by the geometric traits)}{}
  
\ccCreation
\ccCreationVariable{b}

\ccConstructor{PolygonOffsetBuilder_2( Ssds const& ss );}{Constructs the builder class using the given Straight Skeleton instance.}

\ccHeading{Methods}
\ccMethod{template<class OutputIterator>  OutputIterator construct_offset_polygons( FT t, OutputIterator out );}
{Given the straight skeleton passed in the constructor which corresponds to a certain simple polygon \textit{P}, returns \textit{all} the offset polygons of \textit{P} at the Euclidean distance \ccc{t}.\\
Such offset polygons are Simple Polygons in the interior of P.\\
For any offset distance \ccc{t} there are 0, 1 or more offset polygons.\\
To generate each offset polygon, a default constructed instance of \ccc{Container} type (which must be a model of the \ccc{VertexContainer_2} concept), is dynamically allocated and each offset vertex is added to it.\\
A \ccc{boost::shared_ptr} holding onto the dynamically allocated container is inserted into the output sequence via the OutputIterator \ccc{out}.\\
OutputIterator must be a model of the \textit{OutputIterator} category whose \ccc{value\_type} is a \ccc{boost::shared_ptr} holding the dynamically allocated instances of type Container.\\\\
The method returns an OutputIterator past-the-end of the resulting sequence, which contains each offset polygon generated.\\
You can call \ccc{construct_offset_polygons} with different offset distances (there is no need to construct the builder  again). If you call it with an offset distance so large that there are no offset polygons at that distance, no polygon is inserted into the output sequence and the returned iterator will be equal to \ccc{out}.\\
If the source polygon has no holes, all offset polygons will be oriented counter-clockwise and will have no holes either (just like source polygon).\\
However, if the source polygon has holes, any of the resulting offset polygons can have holes (but there might be no holes in the result if the offset distance is large enough). Any hole contour in an offset polygon will be oriented clockwise.}
   
\ccInclude{CGAL/Polygon_offset_builder_2.h}

\ccSeeAlso
\ccc{VertexContainer_2}\\
\ccc{PolygonOffsetBuilderTraits_2}\\
\ccc{CGAL::Polygon_offset_builder_traits_2<Kernel>}\\
\end{ccRefClass}

% +------------------------------------------------------------------------+
%%RefPage: end of main body, begin of footer
% EOF
% +------------------------------------------------------------------------+
