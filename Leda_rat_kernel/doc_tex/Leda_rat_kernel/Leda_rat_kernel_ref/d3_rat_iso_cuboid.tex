\ccDefGlobalScope{}
\begin{ccRefClass}{d3_rat_iso_cuboid}

\ccDefinition

\ccInclude{CEP/Leda_rat_kernel/LEDA_RATKERNEL/d3_rat_iso_cuboid.h}

The class \ccRefName\  is an iso-oriented cuboid (box)
in 3d space with rational coordinates. It uses the LEDA rational kernel. 
The class is in namespace \ccc{leda} for LEDA versions with a namespace
(4.4 or higher) and else in global scope.


\ccCreationVariable{I}

\ccCreation

\ccConstructor{d3_rat_iso_cuboid();}
{introduces a variable of type \ccRefName\ .}  
  
\ccConstructor{d3_rat_iso_cuboid(leda_integer xmin, leda_integer xmax, leda_integer ymin, leda_integer ymax, leda_integer zmin, leda_integer zmax);}
{introduces a variable with minimal/maximal $x/y/z$ - coordinates $xmin,xmax,ymin,ymax,zmin,zmax$
of type \ccRefName\ . } 

\ccConstructor{d3_rat_iso_cuboid(leda_integer xmin, leda_integer xmax, leda_integer ymin, leda_integer ymax, leda_integer zmin, leda_integer zmax,
                                 leda_integer w);}
{introduces a variable with diagonal opposite vertices $(xmin,ymin,zmin,w)$ and $(xmax,ymax,zmax,w)$
of type \ccRefName\ .}
  
\ccConstructor{d3_rat_iso_cuboid(const leda_d3_rat_point& p,const leda_d3_rat_point& q);}
{introduces a variable of type \ccRefName\ with diagonal opposite vertices $p$ and $q$ . }  



\ccOperations

\ccMethod{leda_d3_rat_point vertex(int i);}
{returns a vertex (corner). Precondition: \ccc{0<=i<=7} .
The ordering of the vertices is compatible to the CGAL class \ccc{Iso\_cuboid\_3}.
The following vertices are returned:
}
\begin{itemize}
\item i==0: lower left front 
\item i==1: lower right front
\item i==2: lower right back
\item i==3: lower left back
\item i==4: upper left front
\item i==5: upper left back
\item i==6: upper right front
\item i==7: upper right back
\end{itemize}

\ccMethod{leda_rational xmin();}
{returns the smallest x-coordinate.}
  
\ccMethod{leda_rational xmax();}
{returns the largest x-coordinate.}  
  
\ccMethod{leda_rational ymin();}
{returns the smallest y-coordinate.} 
  
\ccMethod{leda_rational ymax();}
{returns the largest y-coordinate.} 
  
\ccMethod{leda_rational zmin();}
{returns the smallest z-coordinate.}  
  
\ccMethod{leda_rational zmax();}
{returns the largest z-coordinate.}


\end{ccRefClass} 
