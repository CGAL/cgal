%% Copyright (c) 2004  SciSoft.  All rights reserved.
%%
%% This file is part of CGAL (www.cgal.org); you may redistribute it under
%% the terms of the Q Public License version 1.0.
%% See the file LICENSE.QPL distributed with CGAL.
%%
%% Licensees holding a valid commercial license may use this file in
%% accordance with the commercial license agreement provided with the software.
%%
%% This file is provided AS IS with NO WARRANTY OF ANY KIND, INCLUDING THE
%% WARRANTY OF DESIGN, MERCHANTABILITY AND FITNESS FOR A PARTICULAR PURPOSE.
%%
%% 
%%
%% Author(s)     : Fernando Cacciola <fernando_cacciola@hotmail.com>

\begin{ccRefConcept}{GetPlacement}

%% \ccHtmlCrossLink{}     %% add further rules for cross referencing links
%% \ccHtmlIndexC[concept]{} %% add further index entries

\ccCreationVariable{gp}  %% choose variable name
\ccDefinition

The concept \ccRefName\ describes the requirements for the {\em policy
function object} which gets the {\em collapse placement} of an edge,
that is, the position of the vertex that replaces the edge after the
collapse.

The placement returned is an \ccc{boost::optional} value (i.e., it can
be absent). An absent placement could be the result of a
computational limitation (such as overflow), or can be intentionally
returned to prevent the edge from being collapsed.

A model of this concept can simply return the placement stored in the
passed cache object, if any, or perform the actual computation.

\ccRefines
\ccc{DefaultConstructible}
\ccc{CopyConstructible}

\ccTypes
  \ccNestedType{ECM}{The type of the surface to simplify.
   Must be a model of the \ccc{EdgeCollapsableMesh} concept.}{}
\ccGlue    
  \ccNestedType{Params}{The type of the model-specific parameters needed. Can be \ccc{void}.}{}
\ccGlue    
  \ccTypedef{typename CGAL::halfedge_graph_traits<ECM>::Point Point;}{The point type for the surface vertex.}
\ccGlue    
  \ccTypedef{typename boost::graph_traits<ECM>::edge_descriptor edge_descriptor;}
  {A {\sc Bgl} edge descriptor representing an edge of the surface.}
\ccGlue    
  \ccTypedef{boost::optional<Point> result_type;}{The type of the result (an optional point).}

\ccCreation
\ccCreationVariable{gp}  %% choose variable name

\ccConstructor{GetPlacement();}{Default constructor}

\ccOperations

  \ccMethod{template<class Cache>
            result_type operator()( edge_descriptor const& edge
                                  , ECM&                   surface 
                                  , Cache const&           cache
                                  , Params const*          params
                                  ) const;
           }
  {Computes and returns the placement, that is, the position of the vertex 
  which replaces the collapsing \ccc{edge}, using the model-specific \ccc{params}.\\
  \ccc{edge} is required to be in the \ccc{surface}.\\
  If \ccc{Cache} is a model of \ccc{CostAndPlacementCache}, which means that
  \ccc{cache} stores the placement, a model can simply return that 
  (but is not required to do so).\\
  \ccc{params} can be a \ccc{NULL} pointer.
  }
  
\ccHasModels
\ccRefIdfierPage{CGAL::Surface_mesh_simplification::Cached_cost<ECM>}.
\ccRefIdfierPage{CGAL::Surface_mesh_simplification::Edge_length_cost<ECM>}.
\ccRefIdfierPage{CGAL::Surface_mesh_simplification::LindstromTurk_cost<ECM>}.

\end{ccRefConcept}

% +------------------------------------------------------------------------+
%%RefPage: end of main body, begin of footer
% EOF
% +------------------------------------------------------------------------+
