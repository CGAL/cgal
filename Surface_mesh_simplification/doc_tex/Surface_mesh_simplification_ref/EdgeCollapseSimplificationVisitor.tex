%% Copyright (c) 2004  SciSoft.  All rights reserved.
%%
%% This file is part of CGAL (www.cgal.org); you may redistribute it under
%% the terms of the Q Public License version 1.0.
%% See the file LICENSE.QPL distributed with CGAL.
%%
%% Licensees holding a valid commercial license may use this file in
%% accordance with the commercial license agreement provided with the software.
%%
%% This file is provided AS IS with NO WARRANTY OF ANY KIND, INCLUDING THE
%% WARRANTY OF DESIGN, MERCHANTABILITY AND FITNESS FOR A PARTICULAR PURPOSE.
%%
%% 
%%
%% Author(s)     : Fernando Cacciola <fernando_cacciola@hotmail.com>

\begin{ccRefConcept}{EdgeCollaspeSimplificationVisitor}

%% \ccHtmlCrossLink{}     %% add further rules for cross referencing links
%% \ccHtmlIndexC[concept]{} %% add further index entries

\ccDefinition

The concept \ccRefName\ describes the requirements for the {\em visitor object} which is used to track the edge-collapse simplification algorithm.

The several callbacks given as member functions in the visitor are called from certain places in the algorithm implementation.

\ccTypes
  \ccNestedType{TSM}
  {The type of the surface to simplify. Must be a model of the \ccc{TriangulatedSurfaceMesh} concept.}{}
\ccGlue    
  \ccNestedType{FT}{A \ccc{FieldType} representing the collapse cost}{}
\ccGlue    
  \ccTypedef{typename boost::graph_traits<TSM>::edge_descriptor edge_descriptor;}
  {A {\sc Bgl} edge-descriptor representing an {\em undirected edge} of the surface.}
\ccGlue    
  \ccTypedef{typename CGAL::embeeded_graph_traits<TSM>::Point Point;}
  {The point type of the vertex.}
\ccGlue    
  \ccNestedType{size_type}{An \ccc{IntegerType} representing the number of edges}{}

\ccCreation
\ccCreationVariable{v}  %% choose variable name

\ccOperations

  \ccMethod
  {void OnStarted( TSM& surface );}
  {Called before the algorithm starts.}
  
  \ccMethod
  {void OnFinished ( TSM& surface ) ; }
  {Called after the algorithm finishes.}
  
  \ccMethod
  {void OnStopConditionReached( TSM& surface ) ; } 
  {Called when the \ccc{StopPredicate} returned true
  (but not if the algorithm terminates because the surface couldn't be simplified any futher)
  }
  
  \ccMethod
  {void OnCollected( edge_descriptor const& edge
                   , bool                   is_fixed
                   , TSM&                   surface
                   );
  }                  
  {Called during the {\em collecting phase} (when a cost is assigned to the edges),
  for each \ccc{edge} collected.\\
  \ccc{is_fixed} indicates whether \ccc{edge} is fixed or not.
  If it is fixed it won't be collapsed.
  }
  
  \ccMethod
  {void OnSelected(edge_descriptor const&  edge
                  ,TSM&                    surface
                  ,boost::optional<double> cost
                  ,size_type               initial_count
                  ,size_type               current_count
                  );
  }                 
  {Called during the {\em processing phase} (when edges are collapsed),
  for each \ccc{edge} that is selected.\\
  This method is called {\em before} the algorithm checks 
  if the edge is collapsable.\\
  \ccc{cost} indicates the current collaspe cost for the \ccc{edge}.
  If {\em absent} (meaning that it couldn't be comptuted)
  the edge won't be collapsed.\\
  \ccc{initial_count} and \ccc{current_count} refers to 
  the number of edges.
  }
  
  \ccMethod
  {void OnCollapsing( edge_descriptor const& edge
                    , TSM&                   surface
                    , boost::optional<Point> placement
                    );
  }                  
  {Called when \ccc{edge} is about to be collapsed and replaced by a vertex
  whose position is \ccc{*placement}.\\
  If \ccc{placement} is absent (meaning that it couldn't be computed)
  the edge won't be collapsed.
  }
  
  \ccMethod
  {void OnNonCollapsable( edge_descriptor const& edge
                        , TSM&                   surface
                        );
  }                  
  {Called for each selected \ccc{edge} which cannot be 
  collapsed because doing so would change the topological
  type of the surface. (turn it into a non-manifold
  for instance)
  }
  
\end{ccRefConcept}

% +------------------------------------------------------------------------+
%%RefPage: end of main body, begin of footer
% EOF
% +------------------------------------------------------------------------+
