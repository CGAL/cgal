%% Copyright (c) 2004  SciSoft.  All rights reserved.
%%
%% This file is part of CGAL (www.cgal.org); you may redistribute it under
%% the terms of the Q Public License version 1.0.
%% See the file LICENSE.QPL distributed with CGAL.
%%
%% Licensees holding a valid commercial license may use this file in
%% accordance with the commercial license agreement provided with the software.
%%
%% This file is provided AS IS with NO WARRANTY OF ANY KIND, INCLUDING THE
%% WARRANTY OF DESIGN, MERCHANTABILITY AND FITNESS FOR A PARTICULAR PURPOSE.
%%
%% 
%%
%% Author(s)     : Fernando Cacciola <fernando_cacciola@hotmail.com>

\begin{ccRefConcept}{EdgeCollapsableMesh}

%% \ccHtmlCrossLink{}     %% add further rules for cross referencing links
%% \ccHtmlIndexC[concept]{} %% add further index entries

\ccDefinition

The concept \ccRefName\ describes the requirements for the type of 
triangulated surface mesh that can be passed to the
simplification algorithm.

The surface must be structurally equivalent to a polyhedral surface
having only triangular faces. 
It can have any number of connected components, boundaries 
(borders and holes) and handles (arbitrary genus).

\ccRefines
\ccc{HalfedgeGraph}

\ccHeading{Valid Expressions}

The mesh simplification algorithn requires the free function \ccc{collapse_triangulation_edge}.


  \ccFunction
  {template<class EdgeCollapsableMesh>
  typename boost::graph_traits<EdgeCollapsableMesh>::vertex_descriptor
  collapse_triangulation_edge(typename boost::graph_traits<EdgeCollapsableMesh>::edge_descriptor p_q,
                             EdgeCollapsableMesh& surface);}  
  {Collapses the undirected edge p-q replacing it with one single vertex.\\ 
  Details below ~\ref{SurfaceMeshSimplification:CollapseOperationDetails}.
  }

    
\ccHeading{Collapse Operation Semantics}\label{SurfaceMeshSimplification:CollapseOperationDetails}
  
\subsubsection*{Preconditions}

This function requires the surface to be an {\em open oriented 2-manifold} and it must
never be called in a context were collapsing the undirected edge would result in the surface 
becoming a {\em non-manifold}. That is, as a precondition, undirected edge $(p,q)$ must satisfy
the so-called {\em link-condition}
(whose definition is out of scope for this manual but which is notably explained 
in \cite{degn-tpec-98}).

\subsubsection*{Definitions}

$p$ and $q$ are the source and target vertices of edge $(p,q)$

An undirected edge composed of the pairs $(s,t)$ and $(t,s)$ is denoted $[s,t]$

Removing an undirected edge $[s,t]$ corresponds to removing both directed edges $(s,t)$ and $(t,s)$.

The triangle above $(p,q)$, if any, is called {\em top-triangle}, and the third vertex of
such triangle is called  $t$.

The triangle below $(q,p)$, if any, is called {\em bottom-triangle}, and the third vertex
of such triangle is called $b$.

If $(p,q)$ is not a border edge, hence the top-triangle exists, 
there are edges $[p,t]$ and $[q,t]$ which are incident on such a top-triangle.

If $(p,q)$ is not a border edge, hence the bottom-triangle exists, 
there are edges $[p,b]$ and $[q,b]$ which are incident on such a bottom-triangle.

If $(t,p)$ is not a border edge there exist a {\em top-left-triangle},
adjacent to the top-triangle along $[p,t]$.

If $(b,q)$ is not a border edge there exist a {\em bottom-right-triangle},
adjacent to the bottom-triangle along $[q,b]$.

\subsubsection*{Effects}

The net effect of the operation is equivalent to removing one of the vertices ($p$ or $q$)
and re-triangulating the resulting face. The actual operation is 
required to remove at most 2 triangles, 3 undirected edges and 1 vertex. 


%The preconditions guarantee that the top-left-triangle and top-bottom-triangle
%are disjoint, that $t-p$ and $p-q$ are both incident on the top-triangle and that
%$b-q$ and $q-p$ are both incident on the bottom-triangle.

 
%The triangle adjacent to the top-triangle along $p-t$ is called {\em top-left-triangle}.\\
%The triangle adjacent to the bottom-triangle along $q-b$ is called {\em bottom-right-triangle}.

\subsubsection*{Postconditions}
      
$[p,q]$ is removed from the surface.

The top-triangle, if any, is removed, along with $[p,t]$.

The bottom-triangle, if any, is removed, along with $[q,b]$.

$[q,t]$ and $[p,b]$ are never removed.
  
If the bottom-triangle is removed and neither the top-triangle nor
the bottom-right-triangle exist, vertex $q$ is removed,
otherwise, vertex $p$ is removed.

All those triangles initially defined by the removed vertex 
(other than the top and bottom which are removed) are redefined
with the vertex that is kept. That is, if $p$ is removed then
every triangle $(p,x,y)$ is redefined as $(q,x,y)$. Any
additional properties, such as texture coordinates, planes,
surface patches, etc that these redefined triangles might have
attached are preserved.
\footnote{Hence a model for this concept cannot simply replace
those triangles by new triangles, unless the replacements 
inherit the corresponding properties.}

\subsubsection*{Returns}

The function returns the vertex that is not removed.

\ccHasModels
\ccRefIdfierPage{CGAL::Polyhedron_3<Traits>} via external adaptation.

\ccSeeAlso
\ccRefIdfierPage{CGAL::graph_traits< Polyhedron_3<Traits> >}\\
\ccRefIdfierPage{CGAL::halfedge_graph_traits< Polyhedron_3<Traits> >}

\end{ccRefConcept}

% +------------------------------------------------------------------------+
%%RefPage: end of main body, begin of footer
% EOF
% +------------------------------------------------------------------------+
