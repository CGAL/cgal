\begin{ccRefConcept}{QuadricalKernel_3}

%\ccRefines
%\ccc{SphericalKernel}

%\ccHasModels
%\ccc{CGAL::Spherical_kernel_3<LinearKernel,AlgebraicKernelForSpheres>}

\ccTypes

A model of \ccc{QuadricalKernel_3} is supposed to provide some basic types

\ccNestedType{Linear_kernel}{Model of \ccc{LinearKernel}.}
\ccGlue
\ccNestedType{Spherical_kernel}{Model of \ccc{SphericalKernel}.}

\ccNestedType{Polynomial_3}{Model of \ccc{CGAL::Polynomial_d} specialized to
$d = 3$. \ccc{CGAL::Polynomial_d} is already defined its own \cgal\ package.}

%\ccGlue
%\ccNestedType{Algebraic_kernel}{Model of \ccc{AlgebraicKernelForQuadrics}.}

%\ccNestedType{Polynomial_1_3}{Model of \ccc{AlgebraicKernelForQuadrics::Polynomial_1_3}.}
%\ccGlue
%\ccNestedType{Polynomial_3}{Model of \ccc{AlgebraicKernelForQuadrics::Polynomial_3}.}

%\ccNestedType{Root_of_1}{Model of \ccc{AlgebraicKernelForQuadrics::RootOf_1}.}
%\ccGlue
%\ccNestedType{Root_of_3}{Model of \ccc{AlgebraicKernelForQuadrics::RootOf_3}.}

and to define the following geometric objects

\ccNestedType{Quadric_3}{Model of \ccc{QuadricalKernel_3::Quadric_3}.}
\ccGlue
\ccNestedType{Curve_3}{Model of \ccc{QuadricalKernel_3::Curve_3}.}
\ccGlue
\ccNestedType{Curve_point_3}{Model of \ccc{QuadricalKernel_3::CurvePoint_3}.}
\ccGlue
\ccNestedType{Curve_arc_3}{Model of \ccc{QuadricalKernel_3::CurveArc_3}.}

Note that the \ccRefName\ does not provide triangular patches. 
A triangular patch would consist of three instances of 
\ccc{Curve_arc_3} embedded on an instance of \ccc{Quadric_3}.
But in general it is not possible to define an instance of \ccc{Quadric_3} 
that 
passes through two given instances of \ccc{Curve_point_3}. 
This is due to the fact, that in general each \ccc{Curve_point_3} is of 
algebraic degree 8, while the quadric is defined by rational coefficients. 
This implies that in gerneral there is no way to find a third arc that can 
form together with two given arcs a triangular patch. 
In other words, iterative constructions are not possible. 

A model of \ccc{QuadricalKernel_3} must also provide predicates, 
constructions and other functionalities. 

\ccPredicates

\ccNestedType{Compare_x_3}{Model of \ccc{QuadricalKernel_3::CompareX_3}.}
\ccGlue
\ccNestedType{Compare_y_3}{Model of \ccc{QuadricalKernel_3::CompareY_3}.}
\ccGlue
\ccNestedType{Compare_z_3}{Model of \ccc{QuadricalKernel_3::CompareZ_3}.}
\ccGlue
\ccNestedType{Compare_xy_3}{Model of \ccc{QuadricalKernel_3::CompareXY_3}.}
\ccGlue
\ccNestedType{Compare_xyz_3}{Model of \ccc{QuadricalKernel_3::CompareXYZ_3}.}


\ccNestedType{Equal_3}{Model of \ccc{QuadricalKernel_3::Equal_3}.}
\ccGlue
\ccNestedType{Has_on_3}{Model of \ccc{QuadricalKernel_3::HasOn_3}.}
\ccGlue
\ccNestedType{Do_overlap_3}{Model of \ccc{QuadricalKernel_3::DoOverlap_3}.}

%\ccNestedType{Bounded_side_3}{Model of \ccc{QuadricalKernel_3::BoundedSide_3}.}
%\ccGlue
%\ccNestedType{Has_on_bounded_side_3}{Model of \ccc{QuadricalKernel_3::HasOnBoundedSide_3}.}
%\ccGlue
%\ccNestedType{Has_on_unbounded_side_3}{Model of \ccc{QuadricalKernel_3::HasOnUnboundedSide_3}.}

\ccHeading{Constructions}

\ccNestedType{Construct_quadric_3}{
        Model of
        \ccc{QuadricalKernel_3::ConstructQuadric_3}.}
\ccGlue
\ccNestedType{Construct_curve_3}{
        Model of 
         \ccc{QuadricalKernel_3::ConstructCurve_3}. 
        %Construction from \ccc{Linear_kernel::Line_3} is required.
}
\ccGlue
\ccNestedType{Arcs_of_curve_3}{
        Model of \ccc{QuadricalKernel_3::ArcsOfCurve_3}.}

%\ccGlue
%\ccNestedType{Construct_curve_point_3}{Model of \ccc{QuadricalKernel_3::ConstructCurvePoint_8_3}. Construction from \ccc{Linear_kernel::Point_3} is required.}
%\ccGlue
%\ccNestedType{Construct_curve_arc_3}{Model of \ccc{QuadricalKernel_3::ConstructCurveArc_3}. Construction from \ccc{Linear_kernel::Segment_3} is required.}

\ccNestedType{Construct_curve_arc_min_vertex_3}{
        Model of \\
        \ccc{QuadricalKernel_3::ConstructCurveArcMinVertex_3}.}
\ccGlue
\ccNestedType{Construct_curve_arc_max_vertex_3}{
        Model of  \\
        \ccc{QuadricalKernel_3::ConstructCurveArcMaxVertex_3}.}
\ccGlue
\ccNestedType{Construct_curve_arc_source_vertex_3}{
        Model of \\
        \ccc{QuadricalKernel_3::ConstructCurveArcSourceVertex_3}.}
\ccGlue
\ccNestedType{Construct_curve_arc_target_vertex_3}{
        Model of \\
        \ccc{QuadricalKernel_3::ConstructSpace_CurveArcTargetVertex_3}.}

\ccNestedType{Construct_supporting_curve_3}{
        Model of \\
        \ccc{QuadricalKernel_3::ConstructSupportingCurve_3}.}

%\ccNestedType{Construct_bbox_3}{Model of \ccc{QuadricalKernel_3::ConstructBbox_3}.}

\ccNestedType{Intersect_3}{
        Model of \ccc{QuadricalKernel_3::Intersect_3}.}

%\ccHeading{Computations}

\ccHeading{Link with the algebraic kernel}

\ccNestedType{Get_equation}{Model of \ccc{QuadricalKernel_3::GetEquation}. Its \ccc{return_type} is \ccc{Polynomial_3}.}

\ccOperations

As in the \ccc{Kernel} concept, for each of the function objects
above, there must exist a member function that requires no arguments
and returns an instance of that function object. The name of the
member function is the uncapitalized name of the type returned with
the suffix \ccc{_object} appended. For example, for the function object 
\ccc{QuadricalKernel_3::Construct_quadric_3} 
the following member function must exist: 

\ccCreationVariable{ck}
\ccMethod{Construct_quadric_3 construct_quadric_3_object() const;}{}

%\ccSeeAlso

%\ccRefIdfierPage{SphericalKernel}

\end{ccRefConcept}
