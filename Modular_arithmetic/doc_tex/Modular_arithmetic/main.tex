\cleardoublepage
\ccUserChapter{Modular Arithmetic\label{chap:modular_arithmetic}}
\ccChapterAuthor{Michael Hemmer}


\begin{ccPkgDescription}{3D Triangulations}
\ccPkgSummary{
This package  allows to build and handle
triangulations for point sets in three dimensions.
Any CGAL  triangulation covers the convex hull of its
vertices. Triangulations are build incrementally 
and can be modified by insertion or removal of vertices. 
They offer point location facilities.

The package provides plain triangulation (whose faces
depends on the  insertion order of the vertices) and
Delaunay triangulations.  Regular triangulations are
also provided for sets of weighted points.
Delaunay and regular
triangulations offer nearest neighbor queries
and primitives to build the dual Voronoi and power diagrams.}

%\ccPkgDependsOn{}
\ccPkgMaturity{Introduced in \cgal\ 3.1}

\end{ccPkgDescription}


\section{Introduction}

Modular arithmetic is a fundamental tool in modern algebra systems. In conjunction with the Chinese remainder theorem it serves as the workhorse in several algorithms computing the gcd, resultant etc. Moreover, it can serve as a very efficient filter, since it is often possible to exclude that some value is zero by computing its modular correspondent with respect to one prime only. 

This package introduces a type \ccc{CGAL::Residue}
representing a finite field over some prime. 
The prime can be changed at runtime. 

Moreover, the package introduces \ccc{CGAL::Modular_traits<T>} 
providing a mapping from some algebraic structure \ccc{T} into an algebraic 
structure that is based on the type \ccc{CGAL::Residue}.  
For scalar types, e.g. Integers, this mapping is just the canonical homomorphism
into the type \ccc{CGAL::Residue}. 
For compound types, e.g. Polynomials, the mapping is applied to the 
coefficients of the compound type. 


\section{Software Design}

The class \ccc{CGAL::Modular_traits<T>} is designed such that the concept 
\ccc{Modularizable} can be considered as optional, i.e., 
\ccc{CGAL::Modular_traits<T>} provides a tag that can be used for dispatching. 

\subsection{Examples}

In the following example the modular arithmetic is used as a filter. 
\ccIncludeExampleCode{Modular_arithmetic/modular_filter.cpp}


\section{Design and Implementation History}

The class \ccc{CGAL::Residue} is based on the C-code of Sylvain Pion et. al. 
as it was presented in \cite{bepp-sdrns-99}. 

The remaining part of the package is the result of the integration process
of the NumeriX library of EXACUS \cite{beh+-eeeafcs-05} into CGAL. 

