\cleardoublepage
\ccUserChapter{Modular Arithmetic}
\label{chap:modular_arithmetic}
\ccChapterAuthor{Michael Hemmer}


\begin{ccPkgDescription}{3D Triangulations}
\ccPkgSummary{
This package  allows to build and handle
triangulations for point sets in three dimensions.
Any CGAL  triangulation covers the convex hull of its
vertices. Triangulations are build incrementally 
and can be modified by insertion or removal of vertices. 
They offer point location facilities.

The package provides plain triangulation (whose faces
depends on the  insertion order of the vertices) and
Delaunay triangulations.  Regular triangulations are
also provided for sets of weighted points.
Delaunay and regular
triangulations offer nearest neighbor queries
and primitives to build the dual Voronoi and power diagrams.}

%\ccPkgDependsOn{}
\ccPkgMaturity{Introduced in \cgal\ 3.1}

\end{ccPkgDescription}


\section{Introduction}

This package introduces a type \ccc{CGAL::Modular} 
representing a finite field over some prime. 
This prime can be changed at runtime. From there, the type may serve
as the workhorse for algorithms base on chinese remainder.  

Moreover, the package introduces the \ccc{CGAL::Modular_traits<T>} 
providing a mapping from some algebraic strucutre \ccc{T} into algebraic 
structure that is based on the type \ccc{CGAL::Modular}.  
For scalar types, e.g. Integers, this mapping is just the kanonical homomorphism
into the type \ccc{CGAL::Modular}. 
For compount types, e.g. Polynomials, the mapping is applied to the 
coefficients of the compount type. 

\section{Software Design}

The class \ccc{CGAL::Modular_traits<T>} is designed such that the concept 
\ccc{Modularizable} can be considered as optional, i.e., 
\ccc{CGAL::Modular_traits<T>} provides a tag that can be used for dispatching. 

\subsection{Examples}

In the following example the modular arithmetic is used as a filter. 
\ccIncludeExampleCode{Modular_arithmetic/modular_filter.cpp}

