% +------------------------------------------------------------------------+
% | Reference manual page: average_spacing_3.tex
% +------------------------------------------------------------------------+
% | 02.06.2008   Laurent Saboret, Pierre Alliez
% | Package: Surface_reconstruction_3
% |
\RCSdef{\RCSaveragespacingRev}{$Id$}
\RCSdefDate{\RCSaveragespacingDate}{$Date$}
% |
\ccRefPageBegin
%%RefPage: end of header, begin of main body
% +------------------------------------------------------------------------+


\begin{ccRefFunction}{average_spacing_3}  %% add template arg's if necessary

%% \ccHtmlCrossLink{}     %% add further rules for cross referencing links
%% \ccHtmlIndexC[function]{} %% add further index entries

\ccDefinition

% The section below is automatically generated. Do not edit!
%START-AUTO(\ccDefinition)

\ccFunction{Kernel::FT average_spacing_3(const typename Kernel::Point_3& query, Tree& tree, unsigned int KNN);}
{
Compute average spacing of one query point from K nearest neighbors.
Precondition: KNN $>$= 2.
}
\ccGlue
\begin{description}
\item[Template Parameters:]
\begin{description}
\item[Kernel]Geometric traits class. \item[Tree]KD-tree.\end{description}
\end{description}
\begin{description}
\item[Returns:]average spacing (scalar). \end{description}
\begin{description}
\item[Parameters: ]
\begin{description}
\item[query]3D point whose spacing we want to compute \item[tree]KD-tree \item[KNN]number of neighbors \end{description}
\end{description}
\ccGlue
\ccFunction{Kernel::FT average_spacing_3(InputIterator first, InputIterator beyond, unsigned int KNN, const Kernel& );}
{
Compute average spacing from K nearest neighbors. This variant requires the kernel.
Precondition: KNN $>$= 2.
}
\ccGlue
\begin{description}
\item[Template Parameters:]
\begin{description}
\item[InputIterator]\ccc{value_type} is \ccc{Point_3}. \item[Kernel]Geometric traits class.\end{description}
\end{description}
\begin{description}
\item[Returns:]average spacing (scalar). \end{description}
\begin{description}
\item[Parameters: ]
\begin{description}
\item[first]input points \item[KNN]number of neighbors \end{description}
\end{description}
\ccGlue
\ccFunction{FT average_spacing_3(InputIterator first, InputIterator beyond, unsigned int KNN);}
{
Compute average spacing from K nearest neighbors. This variant deduces the kernel from iterator types.
Precondition: KNN $>$= 2.
}
\ccGlue
\begin{description}
\item[Template Parameters:]
\begin{description}
\item[InputIterator]\ccc{value_type} is \ccc{Point_3}. \item[FT]number type.\end{description}
\end{description}
\begin{description}
\item[Returns:]average spacing (scalar) \end{description}
\begin{description}
\item[Parameters: ]
\begin{description}
\item[first]input points \item[KNN]number of neighbors \end{description}
\end{description}
\ccGlue

%END-AUTO(\ccDefinition)
  
\ccInclude{Surface_reconstruction_3/average_spacing_3.h}

\ccParameters

The full template declarations are:

% The section below is automatically generated. Do not edit!
%START-AUTO(\ccParameters)

template$<$  \\
typename Kernel,   \\
typename Tree$>$  \\
\ccc{Kernel::FT}  \\
\ccc{average_spacing_3} (const typename \ccc{Kernel::Point_3}\& query, Tree\& tree, unsigned int KNN);  \\
  \\
template$<$  \\
typename InputIterator,   \\
typename Kernel$>$  \\
\ccc{Kernel::FT}  \\
\ccc{average_spacing_3} (InputIterator first, InputIterator beyond, unsigned int KNN, const Kernel\& );  \\
  \\
template$<$  \\
typename InputIterator,   \\
typename FT$>$  \\
FT  \\
\ccc{average_spacing_3} (InputIterator first, InputIterator beyond, unsigned int KNN);  \\

%END-AUTO(\ccParameters)

\ccExample

A short example program.
Instead of a short program fragment, a full running program can be
included using the 
\verb|\ccIncludeExampleCode{Surface_reconstruction_3/average_spacing_3.C}| 
macro. The program example would be part of the source code distribution and
also part of the automatic test suite.

\begin{ccExampleCode}
void your_example_code() {
}
\end{ccExampleCode}

%% \ccIncludeExampleCode{Surface_reconstruction_3/average_spacing_3.C}

\end{ccRefFunction}

% +------------------------------------------------------------------------+
%%RefPage: end of main body, begin of footer
\ccRefPageEnd
% EOF
% +------------------------------------------------------------------------+

