% +------------------------------------------------------------------------+
% | Reference manual page: pca_normal_estimation.tex
% +------------------------------------------------------------------------+
% | 14.02.2008   Pierre Alliez, Laurent Saboret, Gael Guennebaud
% | Package: Surface_reconstruction_3
% |
\RCSdef{\RCSpcanormalestimationRev}{$Id$}
\RCSdefDate{\RCSpcanormalestimationDate}{$Date$}
% |
\ccRefPageBegin
%%RefPage: end of header, begin of main body
% +------------------------------------------------------------------------+


\begin{ccRefFunction}{pca_normal_estimation}  %% add template arg's if necessary

%% \ccHtmlCrossLink{}     %% add further rules for cross referencing links
%% \ccHtmlIndexC[function]{} %% add further index entries

\ccDefinition

\ccc{CGAL::pca_normal_estimation()} estimates normals direction of a point set using linear least squares fitting of a plane on the K nearest neighbors.
The result is an unoriented normal vector for each input point.

The \ccc{CGAL::pca_normal_estimation()} function exists in two flavors. 
The complete version applies to a point set and requires the kernel to use for computations. 
The main version applies to a point set and deduces the kernel from input parameters.

\ccInclude{CGAL/pca_normal_estimation.h}

% The section below is automatically generated. Do not edit!
%START-AUTO(\ccDefinition)

\ccFunction{OutputIterator pca_normal_estimation(InputIterator first, InputIterator beyond, OutputIterator normals, unsigned int KNN, const Kernel& );}
{
Estimate normals direction using linear least squares fitting of a plane on the K nearest neighbors. This variant requires the kernel.
Precondition: KNN $>$= 2.
}
\ccGlue
\begin{description}
\item[Template Parameters:]
\begin{description}
\item[InputIterator]\ccc{value_type} is \ccc{Point_3}. \item[OutputIterator]\ccc{value_type} is \ccc{Point_3}. \item[Kernel]Geometric traits class.\end{description}
\end{description}
\begin{description}
\item[Returns:]past-the-end output iterator. \end{description}
\begin{description}
\item[Parameters: ]
\begin{description}
\item[first]input points \item[normals]output normals \item[KNN]number of neighbors \end{description}
\end{description}
\ccGlue
\ccFunction{OutputIterator pca_normal_estimation(InputIterator first, InputIterator beyond, OutputIterator normals, unsigned int KNN);}
{
Estimate normals direction using linear least squares fitting of a plane on the K nearest neighbors. This variant deduces the kernel from iterator types.
Precondition: KNN $>$= 2.
}
\ccGlue
\begin{description}
\item[Template Parameters:]
\begin{description}
\item[InputIterator]\ccc{value_type} is \ccc{Point_3}. \item[OutputIterator]\ccc{value_type} is \ccc{Point_3}.\end{description}
\end{description}
\begin{description}
\item[Returns:]past-the-end output iterator. \end{description}
\begin{description}
\item[Parameters: ]
\begin{description}
\item[first]input points \item[normals]output normals \item[KNN]number of neighbors \end{description}
\end{description}
\ccGlue

%END-AUTO(\ccDefinition)

\ccParameters

The full template declarations are:

% The section below is automatically generated. Do not edit!
%START-AUTO(\ccParameters)

template$<$  \\
typename InputIterator,   \\
typename OutputIterator,   \\
typename Kernel$>$  \\
OutputIterator  \\
\ccc{pca_normal_estimation} (InputIterator first, InputIterator beyond, OutputIterator normals, unsigned int KNN, const Kernel\& );  \\
  \\
template$<$  \\
typename InputIterator,   \\
typename OutputIterator$>$  \\
OutputIterator  \\
\ccc{pca_normal_estimation} (InputIterator first, InputIterator beyond, OutputIterator normals, unsigned int KNN);  \\

%END-AUTO(\ccParameters)

\ccSeeAlso

\ccRefIdfierPage{CGAL::jet_normal_estimation}  \\
\ccRefIdfierPage{CGAL::mst_normal_orientation}  \\

\ccExample

\ccIncludeExampleCode{Surface_reconstruction_3/pca_normal_estimation_example.cpp}

\end{ccRefFunction}

% +------------------------------------------------------------------------+
%%RefPage: end of main body, begin of footer
\ccRefPageEnd
% EOF
% +------------------------------------------------------------------------+

