% +------------------------------------------------------------------------+
% | Reference manual page: estimate_normals_jet_fitting_3.tex
% +------------------------------------------------------------------------+
% | 19.02.2008   Laurent Saboret, Pierre Alliez
% | Package: Package
% |
\RCSdef{\RCSestimatenormalsjetfittingRev}{$Id$}
\RCSdefDate{\RCSestimatenormalsjetfittingDate}{$Date$}
% |
\ccRefPageBegin
%%RefPage: end of header, begin of main body
% +------------------------------------------------------------------------+


\begin{ccRefFunction}{estimate_normals_jet_fitting_3}  %% add template arg's if necessary

%% \ccHtmlCrossLink{}     %% add further rules for cross referencing links
%% \ccHtmlIndexC[function]{} %% add further index entries

\ccDefinition
  
\ccInclude{CGAL/estimate_normals_jet_fitting_3.h}

% The section below is automatically generated. Do not edit!
%START-AUTO(\ccDefinition)

\ccFunction{void estimate_normals_jet_fitting_3(InputIterator first, InputIterator beyond, OutputIterator normals, const unsigned int K, const Kernel &, const unsigned int degre_fitting = 2);}
{
Estimate normal directions using jet fitting on the K nearest neighbors. This variant requires the kernel.
Precondition: K $>$= 2.
}
\ccGlue
\begin{description}
\item[Parameters: ]
\begin{description}
\item[first]input points \item[normals]output normals \item[K]number of neighbors \end{description}
\end{description}
\ccGlue
\ccFunction{void estimate_normals_jet_fitting_3(InputIterator first, InputIterator beyond, OutputIterator normals, const unsigned int K, const unsigned int degre_fitting = 2);}
{
Estimate normal directions using jet fitting on the K nearest neighbors. This variant deduces the kernel from iterator types.
Precondition: K $>$= 2.
}
\ccGlue
\begin{description}
\item[Parameters: ]
\begin{description}
\item[first]input points \item[normals]output normals \item[K]number of neighbors \end{description}
\end{description}
\ccGlue

%END-AUTO(\ccDefinition)

\ccParameters

The full template declarations are:

% The section below is automatically generated. Do not edit!
%START-AUTO(\ccParameters)

template$<$  \\
typename InputIterator,   \\
typename OutputIterator,   \\
typename Kernel$>$  \\
void  \\
\ccc{estimate_normals_jet_fitting_3} (InputIterator first, InputIterator beyond, OutputIterator normals, const unsigned int K, const Kernel \&, const unsigned int \ccc{degre_fitting}=2);  \\
  \\
template$<$  \\
typename InputIterator,   \\
typename OutputIterator$>$  \\
void  \\
\ccc{estimate_normals_jet_fitting_3} (InputIterator first, InputIterator beyond, OutputIterator normals, const unsigned int K, const unsigned int \ccc{degre_fitting}=2);  \\

%END-AUTO(\ccParameters)

\ccExample

TODO: add short example program.
Instead of a short program fragment, a full running program can be
included using the 
\verb|\ccIncludeExampleCode{Package/estimate_normals_jet_fitting_3.C}| 
macro. The program example would be part of the source code distribution and
also part of the automatic test suite.

\begin{ccExampleCode}
void your_example_code() {
}
\end{ccExampleCode}

%% \ccIncludeExampleCode{Package/estimate_normals_jet_fitting_3.C}

\end{ccRefFunction}

% +------------------------------------------------------------------------+
%%RefPage: end of main body, begin of footer
\ccRefPageEnd
% EOF
% +------------------------------------------------------------------------+

