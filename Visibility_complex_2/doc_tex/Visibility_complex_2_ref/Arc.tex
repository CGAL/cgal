% +------------------------------------------------------------------------+
% | Reference manual page: Arc.tex
% +------------------------------------------------------------------------+
% | Package: Visibility_complex
% +------------------------------------------------------------------------+

\ccRefPageBegin

%%RefPage: end of header, begin of main body
% +------------------------------------------------------------------------+

\begin{ccRefConcept}{VC2Arc}

\ccDefinition

The \ccRefName{} concept defines the minimal requirement for the \ccc{Arc_2}
type of the \ccc{VC2GeomTraits} concept. It models an \emph{arc} (see the
Introduction), i.e. a connected portion of a disk.

The \ccRefName{} concept is introduced for efficiency reasons, to speedup
the creation of a bitangent when an arc it is known to be tangent to is
known. When the disks have constant complexity, the implementation of
\ccRefName{} is actually trivial. The \ccc{VC2Arc} concept is used as a
base type for the concept \ccc{VC2Edge}.
% ------------------------------------------------------------------------------
  
% ------------------------------------------------------------------------------
\ccTypes
\ccThree{Disk_handle}{Disk_handle}{}
\ccThreeToTwo

\ccNestedType{Disk}{Type of disks defining the arc.}
\ccNestedType{Bitangent_2}{Type of the bitangents.}


\ccThree{Disk_handle}{a.object();}{}
\ccThreeToTwo
\ccCreation
\ccCreationVariable{a}

\ccConstructor{Arc();}{Default constructor.}
\ccGlue
\ccConstructor{Arc(Disk_handle d);}
{Creates an arc representing the complete boundary of the disk \ccc{d}.}

\ccAccessFunctions
\ccMethod{Disk_handle object(); const}{Returns the supporting object.}

\ccOperations
\ccThree{Disk_handle}{split(Arc& tmp, const Bitangent_2& p);}{}
\ccTagFullDeclarations
\ccMethod{void set_object(Disk_handle d);}
{sets \ccc{d} as the disk containing \ccVar\ .}
\ccGlue
\ccMethod{void split (Arc& tmp, const Bitangent_2& b);} {Splits the arc
\ccVar{} at the tangency point of \ccc{p}. The origin of \ccVar{} is
unchanged, while its endpoint of \ccVar{} becomes the tangency point of
\ccc{p}. The arc \ccc{tmp} receives the other half of the original arc
\ccVar{}.}
\ccMethod{void split_cw (Arc& tmp, const Bitangent_2& b);}{Same as
\ccc{split}, except that \ccc{tmp} becomes the first half, and \ccVar{} the
second.}
\ccGlue
\ccMethod{void join (Arc& y);}
{joins the two arcs \ccVar\ and \ccc{y}. That is, the target of \ccVar
becomes the target of \ccc{y}.}
\ccMethod{void update_begin(const Bitangent_2& v);}{Sets the origin of
\ccVar{} to the tangency point of \ccc{v}}
\ccMethod{void update_end(const Bitangent_2& v);}{Sets the end of
\ccVar{} to the tangency point of \ccc{v}}

\ccSeeAlso
\ccRefConceptPage{VC2GeomTraits}\\
\ccRefConceptPage{VC2Edge}
% ------------------------------------------------------------------------------

% ------------------------------------------------------------------------------
\ccTagDefaults
\end{ccRefConcept}
\ccRefPageEnd
% +------------------------------------------------------------------------+
