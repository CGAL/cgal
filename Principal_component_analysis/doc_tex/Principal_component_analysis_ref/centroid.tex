% +------------------------------------------------------------------------+
% | Reference manual page: centroid.tex
% +------------------------------------------------------------------------+
% | 
% | November 2008  Pierre Alliez and Sylvain Pion and Ankit Gupta
% | Package:   Principal Component Analysis
% |
% +------------------------------------------------------------------------+

\begin{ccRefFunction}{centroid}  
%% add template arg's if necessary

\ccDefinition
  
The function \ccRefName\ computes the uniform center of mass of a set of 2D or 3D bounded objects. In 2D these objects include points, segments, triangles, iso rectangles, circles and disks. In 3D these objects include points, segments, triangles, iso cuboids, spheres, balls and tetrahedra.

\ccInclude{CGAL/centroid.h}

There is a set of overloaded \ccc{centroid} functions for 2D and 3D objects.
The user can also optionally pass an explicit kernel, in case the default based on \ccc{Kernel_traits} is not sufficient. The dimension is deduced automatically, although the user can pass a \ccc{tag} specifying the manifold dimension of the objects. For example, the default dimension of tetrahedra is 3, but specifying a dimension 0 fits only the vertices of the tetrahedra, specifying a dimension 1 fits only the edges of the tetrahedra and specifying a dimension 2 fits only the triangle facets of the tetrahedra.

\ccFunction{template < typename InputIterator, typename Tag >
            K::Point_2
            centroid(InputIterator first, InputIterator beyond, const Tag& t);}
{ computes the centroid of a non-empty set of 2D objects having a manifold described by t.
  \ccc{K} is \ccc{Kernel_traits<std::iterator_traits<InputIterator>::value_type>::Kernel}.  The value type must be \ccc{K::Point_2}. The tag must range between \ccc{CGAL::Dimension_tag<0>} and \ccc{CGAL::Dimension_tag<2>}.
\ccPrecond{first != beyond.} }

\ccFunction{template < typename InputIterator, typename K, typename Tag >
            K::Point_2
            centroid(InputIterator first, InputIterator beyond, const K & k, const Tag& t);}
{ computes the centroid of a non-empty set of 2D objects having a manifold described by t.
  The value type must be \ccc{K::Point_2}. The tag must range between \ccc{CGAL::Dimension_tag<0>} and \ccc{CGAL::Dimension_tag<2>}.
\ccPrecond{first != beyond.} }

\ccFunction{template < typename InputIterator, typename Tag >
            K::Point_3
            centroid(InputIterator first, InputIterator beyond, const Tag& t);}
{ computes the centroid of a non-empty set of 3D objects having a manifold described by t.
  \ccc{K} is \ccc{Kernel_traits<std::iterator_traits<InputIterator>::value_type>::Kernel}.
  The value type must be \ccc{K::Point_3}. The tag must range between \ccc{CGAL::Dimension_tag<0>} and \ccc{CGAL::Dimension_tag<3>}.
\ccPrecond{first != beyond.} }

\ccFunction{template < typename InputIterator, typename K, typename Tag >
            K::Point_3
            centroid(InputIterator first, InputIterator beyond, const K & k, const Tag& t);}
{ computes the centroid of a non-empty set of 3D objects having a manifold described by t.
  The value type must be \ccc{K::Point_3}. The tag must range between \ccc{CGAL::Dimension_tag<0>} and \ccc{CGAL::Dimension_tag<3>}.
\ccPrecond{first != beyond.} }

\ccSeeAlso
\ccRefIdfierPage{CGAL::barycenter}

\end{ccRefFunction}
