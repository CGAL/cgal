
\section{Introduction}

\cgal ~is targeting towards exact computation with non-linear objects, 
in particular objects defined on algebraic curves and surfaces. 
As a consequence types representing polynomials, algebraic extensions and 
finite fields play a more important role in related implementations. 
This package has been introduced to stay abreast of these changes. 
Since in particular polynomials must be supported by the introduced framework
the package avoids the term {\em number type}. Instead the package distinguishes 
between the {\em algebraic structure} of a type and whether a type is embeddable on 
the real axis, or {\em real embeddable} for short. 
Moreover, the package introduces the notion of {\em interoperable} types which 
allows an explicit handling of mixed operations. 
%In particular it is possible to access the result type of a binary operation. 
 

\ignore{ 
Moreover, the package takes care of the problematic of mixed operations by 
introducing the notion of {\em interoperable types}. 

The later is gathered by the concept \ccc{RealEmbeddable}

Instead the concepts 
introduced within this package reflect the {\em algebraic structure} of a type. 

Instead it introduces a 
concept hierarchy for {\em algebraic structures} and an orthogonal concept
\ccc{RealEmbeddable}. 

The most important change is, that the new concept \ccc{RealEmbeddable} is now 
orthogonal to the old {\em number type} concepts which are now called {\em algebraic structures}. 

The package avoids the term {\em number type} and introduces the  notion of an 
{\em algebraic structure}, i.e. it introduces an concepts hierarchy 
of algebraic structures such as  \ccc{IntegralDomain}, 
\ccc{UniqueFactorizationDomain}, \ccc{EuclideanRing}, \ccc{Field} etc. 
Orthogonal to these concepts it introduces the concept \ccc{RealEmbeddable}
for number types that are embeddable on the real axis. 
This allows type representing polynomials, finite fields, complex numbers etc.
to be valid models of algebraic structure concepts. 

The package also introduces the notion of interoperability of types, 
by introducing the concepts \ccc{ExplicitInteroperable} and 
\ccc{ImplictInteroparable}. 
}



